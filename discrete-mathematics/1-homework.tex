\documentclass[12pt,letterpaper]{article}

\newcommand{\IncludePath}{../include}
\usepackage{extsizes}
\usepackage{titling}

\usepackage{amssymb,amsmath,amsthm}
\usepackage{enumerate}
\usepackage[margin=1in]{geometry}
\usepackage{graphicx,ctable,booktabs}
\usepackage{fancyhdr}
\usepackage[utf8]{inputenc}

\makeatletter
\newenvironment{problem}{\@startsection
       {section}
       {1}
       {-.2em}
       {-3.5ex plus -1ex minus -.2ex}
       {2.3ex plus .2ex}
       {\pagebreak[3]
       \large\bf\noindent{Problem }
       }
       }
\makeatother

\pagestyle{fancy}
\lhead{\thetitle}
\chead{}
\rhead{\thepage}
\lfoot{\small\scshape Grade 4 Olympic Math}
\cfoot{}
\rfoot{}
\renewcommand{\headrulewidth}{.3pt}
\renewcommand{\footrulewidth}{.3pt}
\setlength\voffset{-0.25in}
\setlength\textheight{648pt}
\setlength\headheight{15pt}


% \answerstrue % turn on for answers
\defauthor

\title{Vectors}
\date{September 28, 2019}

\begin{document}

\maketitle

\thispagestyle{empty}

\begin{problem}{Scalar Multiplication}
  Recall \(\mathbf{Q}\) is the set of rational numbers.

  If \(\mathbf{v} = [x, y]\) is a vector, and \(q\in\mathbf{Q}\), then we
  define \[
    q\mathbf{v} = [qx, qy]
  \] and for sake of example, if \(\mathbf{w} = [3, 4]\), then
  \(2\mathbf{w} = [2\times3, 2\times4] = [6, 8]\). (This is called \emph{scalar
  multiplication}, because we're multiplying a vector by a scalar [non-vector] number.)

  \begin{enumerate}
    \item For any vector \([x, y]\), \(0[x, y] = [0, 0]\). \hfill \TFTrue
    \item For any vector \([x, y]\), \(1[x, y] = [x, y]\). \hfill \TFTrue
    \item For any vector \([x, y]\), and rational numbers \(a\) and \(b\),
    \(a[x, y] + b[x, y] = (a+b)[x, y]\). \hfill \TFTrue
    \item For any rational numbers \(a\), \(x\), and \(y\),
    \(a[x, x] + a[y, y] = a[x, y]\). \hfill \TFFalse
    \item Compute \[
      \frac{7}{12} [60, 72] = \SAF{35, 45}
    \]
  \end{enumerate}
\end{problem}

\end{document}
