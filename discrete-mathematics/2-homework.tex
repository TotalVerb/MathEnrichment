\documentclass[12pt,letterpaper]{article}

\newcommand{\IncludePath}{../include}
\usepackage{extsizes}
\usepackage{titling}

\usepackage{amssymb,amsmath,amsthm}
\usepackage{enumerate}
\usepackage[margin=1in]{geometry}
\usepackage{graphicx,ctable,booktabs}
\usepackage{fancyhdr}
\usepackage[utf8]{inputenc}

\makeatletter
\newenvironment{problem}{\@startsection
       {section}
       {1}
       {-.2em}
       {-3.5ex plus -1ex minus -.2ex}
       {2.3ex plus .2ex}
       {\pagebreak[3]
       \large\bf\noindent{Problem }
       }
       }
\makeatother

\pagestyle{fancy}
\lhead{\thetitle}
\chead{}
\rhead{\thepage}
\lfoot{\small\scshape Grade 4 Olympic Math}
\cfoot{}
\rfoot{}
\renewcommand{\headrulewidth}{.3pt}
\renewcommand{\footrulewidth}{.3pt}
\setlength\voffset{-0.25in}
\setlength\textheight{648pt}
\setlength\headheight{15pt}

\usepackage{tikz}
\usetikzlibrary{arrows, decorations.markings}

\tikzset{
vertex/.style={
  circle,
  draw,
  inner sep=0pt,
  minimum size=15pt
  }
}

% \answerstrue % turn on for answers
\defauthor

\title{Functions}
\date{September 29, 2018}

\begin{document}
\thispagestyle{empty}

\maketitle

\noindent Recall:

\noindent
\(\mathbf{N} = \{0, 1, 2, \dots\}\) is the set of natural numbers.

\noindent
\(\mathbf{Z} = \{\dots, -2, 1, 0, 1, 2, \dots\}\) is the set of integers (whole numbers).

\noindent
\(\mathbf{Q} = \operatorname{Quot}(\mathbf{Z})\) is the set of rational numbers (fractions).

\noindent
\(\mathbf{R} = \overline{\mathbf{Q}}\) is the set of real numbers.

\begin{problem}{Find the Solutions}
  Find all possible values for \(x\) given each of the constraint sets below.

  \begin{itemize}
    \item \(x \in \mathbf{Z}\), \(x^2 = 9\). \hfill \(x \in \{\ESAB{-3}, \ESAB{3}\}\).
    \item \(x \in \mathbf{R}\), \(3x = 2\). \hfill \(x \in \Big\{\ESAB{\frac{2}{3}}\Big\}\).
    \item \(x \in \mathbf{N}\), \(x + 3 \in \{2, 4\}\). \hfill \(x \in \{\ESAB{1}\}\).
  \end{itemize}
\end{problem}

\begin{problem}{A Function on the Real Numbers}
  Let \(f: \mathbf{R} \to \mathbf{R}\) be defined by \(f(x) = 1 + 3x\).

  \begin{itemize}
    \item Evaluate \(f(f(3)) = \SAB{31}\).
    \item What is the domain of \(f\)? \hfill \SAB{\mathbf{R}}
    \item What is the codomain of \(f\)? \hfill \SAB{\mathbf{R}}
    \item What is the range of \(f\)? Is it the same as the codomain? Why or why not?
    \Switch{The range is the same as the codomain because we can get every possible real
    number \(y\) by using the input \(x = -1 + y/3\).}{\vspace{4em}}
    \item Draw a small graph for \(f\) with \(x\) values from \(-3\) to \(3\). Make sure you
    label your axes!
  \end{itemize}
\end{problem}

\begin{problem}{Challenge: Higher Order Functions}
  Recall that \(A \to B\) is the type of all functions with domain \(A\) and codomain \(B\).

  \noindent Consider \(\operatorname{applytwice}: (\mathbf{N} \to \mathbf{N}) \to
  (\mathbf{N} \to \mathbf{N})\) given by
  \(\operatorname{applytwice}(f) = x \mapsto f(f(x))\).
  That is, \(\operatorname{applytwice}\) takes as input a function on the natural numbers,
  and returns as output a new function which applys the original function twice.

  \begin{itemize}
    \item Evaluate \(\operatorname{applytwice}(x\mapsto 1+x)(0)\).
    \item Evaluate \(\operatorname{applytwice}(x\mapsto 1+x)(10)\).
    \item Let \(g = \operatorname{applytwice}(x\mapsto 1+x): \mathbf{N} \to \mathbf{N}\).
    What does \(g\) do? Write \(g(x) = \SAC{2+x}\) directly in terms of \(x\) without using
    \(\operatorname{applytwice}\).
  \end{itemize}
\end{problem}

\begin{problem}{Challenge: Graph Reachability}
  The picture below shows a map of several cities and some one-way roads connecting them.
  (The road between B and D is an exception, as it goes both ways.)

  \begin{center}
  \begin{tikzpicture}
  \node[vertex] (A) at (0, 2) [label = left:\(A\)] {};
  \node[vertex] (B) at (10, 4) [label = right:\(B\)] {};
  \node[vertex] (C) at (5, 1) [label = below:\(C\)] {};
  \node[vertex] (D) at (8, 2) [label = below:\(D\)] {};
  \node[vertex] (E) at (3, 5) [label = above:\(E\)] {};
  \node[vertex] (F) at (6, 5) [label = right:\(F\)] {};

  \draw[->, line width=1.5mm] (A) -- (C);
  \draw[->, line width=1.5mm] (C) -- (D);
  \draw[->, line width=1.5mm] (C) -- (E);
  \draw[<->, line width=1.5mm] (B) -- (D);
  \draw[->, line width=1.5mm] (D) -- (B);
  \draw[->, line width=1.5mm] (E) -- (A);
  \draw[->, line width=1.5mm] (F) -- (E);

  \end{tikzpicture}
  \end{center}

  \begin{itemize}
    \item If I am in city \(A\), can I reach (get to) city \(B\)? \Switch{\AnsT{Yes}}{}
    \item If I am in city \(B\), can I reach city \(A\)? \Switch{\AnsT{No}}{}
    \item Starting at city \(C\), which cities can I reach? Which cities are impossible for
    me to reach? \Switch{\Ans{A,C,E,D,B}}{}
    \item Which city is able to reach every city? (Starting from this city, you can get to
    any other city.) \Switch{\Ans{F}}{}
    \item Which cities are reachable from every city? (Starting from any city, you can get
    to these cities.) \Switch{\Ans{B, D}}{}
    \item Can you add a new one-way road so that it is possible to reach any city from any
    other city? \Switch{\AnsT{Yes, from B or D to F.}}{}
  \end{itemize}
\end{problem}

\end{document}
