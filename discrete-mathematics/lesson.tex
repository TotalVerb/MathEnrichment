\documentclass[a4paper,10pt]{report}

\newcommand{\IncludePath}{../include}
\newcommand{\ProjectName}{Grade 7/8 Olympic Math}
\usepackage{extsizes}
\usepackage{titling}

\usepackage{tikz}
\usetikzlibrary{shapes}

\usepackage{amssymb,amsmath,amsthm}
\usepackage{enumerate}
\usepackage{graphicx,ctable,booktabs}
\usepackage{fancyhdr}
\usepackage[utf8]{inputenc}
\usepackage{gensymb}

\usepackage[toc]{glossaries}

\makeatletter
\newenvironment{problem}{\@startsection
       {subsection}
       {1}
       {-.2em}
       {-3.5ex plus -1ex minus -.2ex}
       {2.3ex plus .2ex}
       {\pagebreak[3]
       \large\bf\noindent{Problem }
       }
       }
\makeatother

\makeatletter
\g@addto@macro\@floatboxreset\centering
\makeatother

\newenvironment{solution}
{ \vspace{1em} \noindent \textbf{Solution:} }
{  }

\pagestyle{fancy}
\lhead{\thetitle}
\chead{}
\rhead{\thepage}
\lfoot{\small\scshape \ProjectName}
\cfoot{}
\rfoot{}
\renewcommand{\headrulewidth}{.3pt}
\renewcommand{\footrulewidth}{.3pt}
\setlength\voffset{-0.25in}
\setlength\textheight{648pt}
\setlength\headheight{15pt}

\newcommand{\Ans}[1]{\framebox{$#1$}}
\newcommand{\SAA}[1]{\Switch{\Ans{#1}}{\blankA}}
\newcommand{\SAB}[1]{\Switch{\Ans{#1}}{\blankB}}
\newcommand{\SAC}[1]{\Switch{\Ans{#1}}{\blankC}}
\newcommand{\SAD}[1]{\Switch{\Ans{#1}}{\blankD}}
\newcommand{\SAE}[1]{\Switch{\Ans{#1}}{\blankE}}
\newcommand{\SAF}[1]{\Switch{\Ans{#1}}{\blankF}}
\newcommand{\STA}[1]{\Switch{\AnsT{#1}}{\blankA}}
\newcommand{\STB}[1]{\Switch{\AnsT{#1}}{\blankB}}
\newcommand{\STC}[1]{\Switch{\AnsT{#1}}{\blankC}}
\newcommand{\STD}[1]{\Switch{\AnsT{#1}}{\blankD}}
\newcommand{\STE}[1]{\Switch{\AnsT{#1}}{\blankE}}
\newcommand{\STF}[1]{\Switch{\AnsT{#1}}{\blankF}}
\newcommand{\AnsT}[1]{\framebox{#1}}
\newif\ifanswers
\newcommand{\Switch}[2]{\ifanswers#1\else#2\fi}
\newcommand{\MCSelect}[1]{\Switch{\AnsT{#1}}{#1}}
\newcommand{\TFTrue}{\MCSelect{True}~~False}
\newcommand{\TFFalse}{True~~\MCSelect{False}}

\newcommand{\blankA}{\underline{\hspace{1em}}}
\newcommand{\blankB}{\underline{\hspace{2em}}}
\newcommand{\blankC}{\underline{\hspace{3em}}}
\newcommand{\blankD}{\underline{\hspace{4em}}}
\newcommand{\blankE}{\underline{\hspace{5em}}}
\newcommand{\blankF}{\underline{\hspace{6em}}}



% use glossaries for this document
\newglossaryentry{algorithm}
{
  name=algorithm,
  description={step-by-step procedure for performing a calculation
  according to well-defined rules (Wikipedia)}
}

\newglossaryentry{acute angle}
{
  name=acute angle,
  description={an angle measuring less than a right angle (\SI{90}{\degree} or a
  quarter of a turn); such angles are typically characterized as being sharp}
}

\newglossaryentry{angle}
{
  name=angle,
  description={the figure formed by two rays, called the sides of the angle,
  sharing a common endpoint, called the vertex of the angle (Wikipedia)}
}

\newglossaryentry{coincident}
{
  name=coincident,
  description={a description of two objects which occupy exactly the same space}
}

\newglossaryentry{domain}
{
  name=domain,
  description={the set of values that are possible inputs for a function}
}

\newglossaryentry{endpoint}
{
  name=endpoint,
  description={an extreme point of a line segment or ray; line segments have two
  endpoints whereas rays have just one}
}

\newglossaryentry{function}
{
  name=function,
  description={object that may take any allowed input and will produce a
  single associated output for that input; alternatively, relation for
  which each input value has exactly one related output value}
}

\newglossaryentry{integer}
{
  name=integer,
  description={positive or negative whole number, or $0$; for example,
  $-8$, $2000$}
}

\newglossaryentry{line}
{
  name=line,
  description={a straight one-dimensional object that extends forever in both
  directions}
}

\newglossaryentry{line segment}
{
  name=line segment,
  description={a straight one-dimensional object terminated at both ends}
}

\newglossaryentry{multiplicand}
{
  name=multiplicand,
  description={the number that is being multiplied; for instance, in
  $2\times3=6$, the multiplicand is $2$}
}

\newglossaryentry{multiplier}
{
  name=multiplier,
  description={the factor to multiply a number by; for instance, in
  $2\times3=6$, the multiplier is $3$}
}

\newglossaryentry{obtuse angle}
{
  name=obtuse angle,
  description={an angle measuring more than a right angle (\SI{90}{\degree} or a
  quarter of a turn) but less than a straight angle (\SI{180}{\degree} or a half
  of a turn)}
}

\newglossaryentry{parallel}
{
  name=parallel,
  description={a description of two lines that never intersect at any point}
}

\newglossaryentry{parity}
{
  name=parity,
  description={decribes whether an integer is even or odd}
}

\newglossaryentry{plane}
{
  name=plane,
  description={a two-dimensional flat surface}
}

\newglossaryentry{plane geometry}
{
  name=plane geometry,
  description={the study of figures on a plane (a two-dimensional flat surface)}
}

\newglossaryentry{product}
{
  name=product,
  description={the result of a multiplication; for instance, in
  $2\times3=6$, the product is $6$}
}

\newglossaryentry{range}
{
  name=range,
  description={the set of values that are possible outputs for a function}
}

\newglossaryentry{ray}
{
  name=ray,
  description={a straight one-dimensional object terminated at one end and
  extending forever in the other direction}
}

\newglossaryentry{right angle}
{
  name=right angle,
  description={an angle measuring \SI{90}{\degree} or a quarter of a turn; angle
  between two rays intersecting in an L shape}
}

\newglossaryentry{right triangle}
{
  name=right triangle,
  description={an triangle with one \SI{90}{\degree} (right) angle}
}

\newglossaryentry{summand}
{
  name=summand,
  description={something which is being added; for instance, in $1+2=3$,
  the two summands are $1$ and $2$}
}

\newglossaryentry{reflex angle}
{
  name=reflex angle,
  description={an angle measuring more than \SI{180}{\degree} or a half of a
  turn, but less than \SI{360}{\degree} or a full turn}
}

\newglossaryentry{straight angle}
{
  name=straight angle,
  description={an angle measuring \SI{180}{\degree} or a half of a turn; angle
  between two rays in opposite directions}
}

\newglossaryentry{transitivity}
{
  name=transitivity,
  description={the property of certain relations that specifies if an
  element $a$ is related to $b$, and the element $b$ is related to $c$,
  then $a$ is similarly related to $c$}
}

\makeglossaries

\title{Discrete Mathematics}
\author{Fengyang Wang}
\date{November 21, 2018}

\begin{document}

\begin{abstract}
 This is a rough sketch of lessons in discrete mathematics suitable for an advanced Grade
 7/8 audience. These notes were prepared for the Grand River Chinese School. Each chapter
 may take two to four hours to deliver entirely, depending on the level of detail.

 These notes are intended to be a very rough outline of what is taught, and not a rigorous
 and complete reference. I do not necessarily cover all the material written in these notes
 in any particular year, and I may occasionally cover material beyond that written in the
 notes.

 Although the notes are intended to be presented to a young audience, they are written for a
 teacher and not for a student. Many of the terms used will not be familiar to the students,
 and will need to be explained differently.
\end{abstract}

\maketitle

\tableofcontents

\chapter{Functions of One Variable}

\section{Mathematical Objects}

What is a mathematical object? We are familiar with everyday objects: things we can feel.
Mathematical objects are like those, but more abstract. For instance, a number is a
mathematical object.

\subsection{Pairs}

An \gls{ordered pair} is two things written in an order. For example, \((3,
4)\) is an ordered pair of numbers. Ordered pairs frequently represent a single
concept that is made of two components.

A simple ordered pair like \((3, 4)\) does not itself have much meaning, aside
from being a collection of two numbers. However, we may assign an
interpretation to particular ordered pairs to give them a meaning.

\subsection{Fractions}

\Glspl{fraction} are a common example of ordered pairs with an assigned
interpretation. The fraction \(\frac{a}{b}\) is itself an ordered pair \((a,
b)\), with the first element of this ordered pair representing the number to
divide, and the second element representing the number to be divided by.

\begin{problem}{Fractions}
 Compute each of the following:

 \[
         \frac{3}{8} \times \frac{2}{7} = \frac{6}{56} =
         \Ans{\displaystyle\frac{3}{28}}
        \]
 \[
  \frac{5}{9} \times \frac{2}{5} = \frac{10}{45} =
  \Ans{\displaystyle\frac{2}{9}}
  \]
\end{problem}

It happens to be the case with fractions that distinct ordered pairs might
represent the same quantity. For instance, \(\frac{3}{6}\) and \(\frac{7}{14}\)
are different pairs of numbers, but they represent the same fraction: one half.
All the fractions that represent a particular quantity form a so-called
\emph{equivalence class}. All numbers that can be formed from fractions of
integers are called \glspl{rational number}.

\subsection{Sets and Variables}

A set is an unordered collection of mathematical objects. For our purposes, we will use sets
as a convenient notation to describe the concept of ``one of these kinds of things''.

A variable is a letter that represents a mathematical object whose value may be unknown. We
say ``may be unknown'' because it is possible we do know the value of a variable. For
instance, if I write \(x := 1\), this means that I define the variable \(x\) to refer to the
number \(1\). But I might also say ``Let \(x\) be an integer (whole number).''; here, we do
not know the exact value of \(x\), but we have a constraint on it: it must be a whole
number.

We can express certain kinds of constraint with set-membership notation, as in
Constraint~\ref{fov:setmember}, which states that the value of \(x\) must be \(1\) or \(4\)
or \(100\):
\begin{equation}
  x \in \{1, 4, 100\}
  \label{fov:setmember}
\end{equation}

How would we express the idea that \(x\) is an integer with this notation? We obviously
cannot list out all the integers, since there are infinitely many. Instead we will adopt a
notation for an infinite set of all whole numbers: \(\mathbf{Z}\) (a boldface Z). The reason
for the choice of the letter Z comes from the German word Zahlen, which means ``number''.
Thus we can express the constraint ``\(x\) is an integer'' using the notation of
Constraint~\ref{fov:isinteger}.
\begin{equation}
  x \in \mathbf{Z}
  \label{fov:isinteger}
\end{equation}

Another kind of constraint we often see is an equation. A series of constraints is seen in
Constraints~\ref{fov:equation1} and \ref{fov:equation2}, where we are given that \(x\) is a
real number (a positive or negative number that can be a fraction or can even be an
irrational number), and further that \(x^2 = 4\). This series of constraints is actually
equivalent to \(x \in \{-2, 2\}\), since these are the only two real numbers whose square
is \(4\).
\begin{align}
  x &\in \mathbf{R}
  \label{fov:equation1}
  \\
  x^2 &= 4
  \label{fov:equation2}
\end{align}

We have notation for some important sets that we see frequently:

\begin{itemize}
  \item \(\mathbf{N} = \{0, 1, 2, \dots\}\) is the set of natural numbers.
  \item \(\mathbf{Z} = \{\dots, -2, 1, 0, 1, 2, \dots\}\) is the set of integers (whole
  numbers).
  \item \(\mathbf{Q} = \operatorname{Quot}(\mathbf{Z})\) is the set of rational numbers
  (fractions).
  \item \(\mathbf{R} = \overline{\mathbf{Q}}\) is the set of real numbers.
\end{itemize}

\subsection{Two-Dimensional Vector Spaces}

Another interpretation of ordered pairs is as vectors in a two-dimensional plane. The
components of the vector \([a, b]\) represent the displacement in two directions. For
example, the first component might represent displacement to the right, and the second
component displacement toward top of the page. (Such an interpretation is called a
\emph{vector space}, and the choices of directions are collectively called a \emph{basis}.)

We have notation for the set of two-dimensional vectors where both components are real
numbers: \(\mathbf{R}^2\). The superscript \(^2\) denotes that the vector space is two
dimensional, i.e. has two components.

\section{Introduction}

Imagine a factory where raw materials are converted into finished products.
Each different kind of raw material is converted into a different kind of
product. At our factory, wood is converted into furniture, glass into windows,
silk into clothing, and cotton also into clothing. We can create a table
describing which raw materials are accepted, and which goods are produced from
those raw materials (Figure~\ref{fov:factory}).

\begin{figure}
 \renewcommand{\arraystretch}{1.2}
 \begin{tabular}{|c|c|}
  \hline
  \textbf{In} & \textbf{Out} \\
  \hline
  wood & furniture \\
  glass & windows \\
  silk & clothing \\
  cotton & clothing \\
  \hline
 \end{tabular}

 \caption{Raw inputs accepted by a factory and the outputs associated with
 them}
 \label{fov:factory}
\end{figure}

The factory that we have described is a simple example of a \gls{function}. For each input
that a function accepts, it has a particular output. The set of possible inputs that a
function accepts is called the \gls{domain} of the function, and the set of possible outputs
is called the \gls{range}. For the function described in Figure~\ref{fov:factory}, the
domain consists of wood, glass, silk, and cotton, and the range consists of furniture,
windows, and clothing.

Another way to think about a function, then, is an object that takes any element in its
domain, and associates it with an element of the range. In mathematics, we often (but not
always) deal with function that associate numbers with numbers. For instance, the table of
values in Figure~\ref{fov:numberfunction} is a function which maps numbers in the set \(\{1,
2, 3, 4\}\) to numbers in the set \(\{3, 4, 5\}\).

\begin{figure}
  \renewcommand{\arraystretch}{1.2}
  \begin{tabular}{|c|c|}
    \hline
    \textbf{In} & \textbf{Out} \\
    \hline
    \(1\) & \(3\) \\
    \(2\) & \(3\) \\
    \(3\) & \(5\) \\
    \(4\) & \(4\) \\
    \hline
  \end{tabular}

  \caption{A function over the domain \(\{1, 2, 3, 4\}\)}
  \label{fov:numberfunction}
\end{figure}

We see examples of functions all the time in mathematics. A simple example is the successor
function, which takes each whole number to the next number. This function can be written in
a ``closed form'' as Equation~\ref{fov:successor}, which gives a formula that can be used to
evaluate the function:
\begin{equation}
  S(n) := 1+n
  \label{fov:successor}
\end{equation}

Note that we used \(:=\) instead of the usual \(=\), as we are emphasizing that this is a
definition; that is, we are not only saying that \(S(n) = 1+n\) always, but in fact that
the meaning of \(S(n)\) is defined to be \(1+n\).

We also now have the notation \(S(n)\) for the output of applying function \(S\), the
successor function, to input \(n\). We can use this notation to say things like \(S(0) = 1\)
or \(S(10) = 11\). Since \(S(n)\) is itself a natural number, we can even say things like
\(S(S(2)) = 4\).

However, it does not make sense to say \(S(\{1, 2\})\); we cannot add one to a set. That is,
this function is only meaningful on a certain type of mathematical objects.

The set of objects that a function can accept as inputs is called the \gls{domain} of the
function. A function can only be applied to an input in its domain. All other inputs are
invalid and are therefore meaningless. Similarly, we can call the set of objects that a
function could produce as an output the \gls{range} of the function.

When we define a function by a closed form, as in Equation~\ref{fov:successor}, we also
define the domain that that closed form applies to. A more complete definition of the
function \(S\) would thus also mention what values of \(n\) are allowed.
Equation~\ref{fov:successorany} includes a function annotation that gives the domain of the
function in addition to a new concept, called the codomain. The codomain, which is
\(\mathbf{N}\) in this example, is a promise as to the kind of values the function can
output. What is the difference between a codomain and a range? The range includes all values
that the function actually does output, but the codomain is just a loose promise, and can
include values that are not actually achieved. For instance, \(0\) is a natural number, but
is not actually the successor of any natural number. Thus, the range of this function is
not the same as the given codomain --- the range is missing the number \(0\).

\begin{equation}
  \begin{array}{l}
    S: \mathbf{N} \to \mathbf{N} \\
    S(n) := 1 + n
  \end{array}
  \label{fov:successorany}
\end{equation}

\begin{problem}{A Function on the Natural Numbers}
  Let \(f: \mathbf{N} \to \mathbf{N}\) be defined by \(f(n) = n^2 = n\times n\).

  \begin{itemize}
    \item Evaluate \(f(f(3)) = \Ans{81}\)
    \item What is the domain of \(f\)? \hfill \Ans{\mathbf{N}}
    \item What is the codomain of \(f\)? \hfill \Ans{\mathbf{N}}
    \item What is the range of \(f\)? (Describe it with words and write out the smallest few
    numbers in the range). Note that it is not the same
    as the codomain!
    \Switch{The range consists of \(\{0, 1, 4, 9, \dots\}\); all numbers that are perfect
    squares.}{\vspace{4em}}
  \end{itemize}
\end{problem}

\section{Graphing Functions}

A function is a mathematical object. Just like other mathematical objects, there are often
many ways to present or write the function down. The fraction \(1/2\) can be presented as
\(1/2\) or \(0.5\) or even a picture. We have already seen two ways to describe a function:

\begin{itemize}
  \item As a table of values, for example \(f(0) = 1\) and \(f(1) = 0\).
  \item As a closed form, for example \(f(n) = 1 - n\).
\end{itemize}

We will see another way to present functions defined on the real numbers by drawing
pictures. TK.

\section{Lambda Abstractions}

Functions are themselves mathematical objects, and mathematical objects can have names. In
fact, so far, all the functions we've seen have had names. But just like numbers, it is not
required for a function to have a name. We can use lambda notation to talk about functions
directly, without giving them names.

The notation \begin{equation}
  x \mapsto x + 3
  \label{fov:lambdap3}
\end{equation} (which can be read as \(x\) maps to \(x+3\)) is a function which takes a
number \(x\) as input, and produces \(3\) more than that number as output. The domain and
range of the function are implied by context --- how we use the function.

\begin{problem}{Lambda Abstractions}
  Let \(f: A \to B\) and \(g: B \to C\) be functions. Recall that the domain of \(f\) is
  \(A\), and the codomain of \(f\) is \(B\). What is the domain and codomain of each of the
  following functions?

  \begin{itemize}
    \item \(x \mapsto f(x)\) \hfill \Ans{A \to B}
    \item \(x \mapsto g(x)\) \hfill \Ans{B \to C}
    \item \(x \mapsto g(f(x))\) \hfill \Ans{A \to C}
  \end{itemize}

  Does the function \(x \mapsto f(g(x))\) make sense? Why or why not?

  \begin{solution}
    No, it does not make sense. \(g\) takes an input of type \(B\) and returns an
    output of type \(C\), but \(f\) accepts inputs of type \(A\), not of type \(C\).
  \end{solution}
\end{problem}

There is notation \(g \circ f\), read as ``g compose f'', for the function \(x \mapsto
g(f(x))\).) In some sense, this composition puts multiple functions together. TK.

\section{Sequences}

You may have heard of sequences before. Sequences are a collection of mathematical objects
with a certain order. For example, sequence~\ref{fov:sequenceeven} below is a sequence of
natural numbers:

\begin{equation}
  2, 4, 6, 8, 10, 12, \dots
  \label{fov:sequenceeven}
\end{equation}

This particular sequence goes on forever; there are always more even numbers. If we are
interested in the 100th term of this sequence, it may take a very long time to compute all
100 even numbers. Luckily, we do not need to calculate all those numbers! There is a pattern
in this sequence: the \(n\)th term of the sequence is given by the value \(2n\). This gives
a closed form for the terms of the sequence.

Here, we say that \(n\) is the term number of the sequence, or the index, and \(2n\) is the
term corresponding to that index. If we give the sequence a name --- for example, \(E\) ---
we can further give the term at index \(n\) the name \(E_n\), with a subscript \(n\). Then
we can write our closed form for this sequence as follows:

\begin{equation*}
  E_n = 2n
\end{equation*}

The closed form equation above should look familiar. Indeed, it looks almost like \(E\) can
be thought of as a function; the index \(n\) can be thought of as the input, and the value
\(E_n\) as the output. Thus, in these cases, sequences correspond exactly to functions with
the natural numbers as their domain.

\section{Relations}

TK



\chapter{Probability}

Probability is the study of how likely various events are to occur.

Many of you have heard of probability and have some intuition on what it means and why it is
important. Before we can soundly speak of probability, we should reconcile this intuition
with some philosophical justification for what and why we are studying this subject.

\section{Philosophy of Probability}

In fact, even experts do not agree on what the philosophical foundations of probability are.
The first question we might want to ask is: what if the universe is deterministic? That is,
what if the laws of physics govern all aspects of the universe's evolution. Does that make
all talk of probability useless? Under this assumption, every possible event either happens
always or never. So is there still any meaning to probability?

There are two words frequently used to describe probabilistic thought: objective and
subjective. In objective probabilistic thought, there is a real probability that an event
occurs. This probability is independent of your frame of reference, but it is perhaps hard
to calculate without full information. For instance, an objectivist would say there's a real
probability \(p\) that it will rain tomorrow, but while weather forecasters may try to guess
it, they probably do not have enough information to make a perfect guess.

In subjective probabilistic thought, the probability that an event occurs is inherently
dependent on the information used to calculate that probability. Depending on your frame of
reference (what information you have available to you), you might calculate a different
probability for an event as someone else. If your frame of reference changes (for example,
you receive new information), then the probability of that event in this frame of reference
also changes.

We will see that, even if you believe in an objective probability for all events, it is
sometimes useful to think subjectively.

\section{Simple Probability}

For many random processes, we tend to believe (unless there is information to the contrary)
that all possible outcomes (possibilities) are around equally likely. For instance, when we
roll a six-sided die shaped like a cube, we might make the reasonable assumption that it is
probably fair (not weighted toward any sides). This assmption is not necessarily valid, so
we must be careful. However, we can start by looking at processes (like fair dice) where the
assumption is correct.

It is easy to calculate the probability of events in this case: we can list all possible
outcomes and generate the probability of one particular outcome happening.

\begin{equation}
 P(A) = \frac{\text{Number of possibilities where \(A\) occurs}}
 {\text{Total possibilities}}
\end{equation}

\begin{problem}{Simple Probability I}
 What's the probability of flipping a fair coin and seeing heads?

 \begin{solution}
  There are two possible possibilities, and in one of them, heads is shown. Thus
  the probability is \Ans{\frac{1}{2}}.
 \end{solution}
\end{problem}

\begin{problem}{Simple Probability II}
 What's the probability of rolling a fair \(6\)-sided die and seeing an odd
 number?

 \begin{solution}
  There are six possible possibilities, and in three of them (\(1\), \(3\), and \(5\)), an
  odd number is shown. We have \[
   \frac{3}{6} = \frac{1}{2}
  \]

  Thus the probability is \Ans{\frac{1}{2}}.
 \end{solution}
\end{problem}

\begin{problem}{Simple Probability III}
 What's the probability of picking a red marble from a bag that has \(5\) red marbles and
 \(7\) blue marbles, if the probability of picking any marble is the same?

 \begin{solution}
  There are \(5+7=12\) total possibilities, and in \(5\) of them, a red marble is drawn.
  Thus the probability is \Ans{\frac{5}{12}}.
 \end{solution}
\end{problem}

\section{The Binomial Distribution}

TK

\section{Independent Events}

Sometimes we have more than one event, and the outcome of one event does not
change the next. These events would be called ``independent''. For instance,
in each of the following cases, we have independent events:

\begin{enumerate}
 \item I have two rabbits and a dog. I pick a pet at random; then I pick a
 different pet at random.
 \item I roll two six-sided dice.
 \item I roll one six-sided die twice.
\end{enumerate}

\section{Joint Probabilities}

When two events $A$ and $B$ are independent, we have the following rule for the
chances that both happen:

\begin{equation}
 P(A \cap B) = P(A) \times P(B)
\end{equation}

We can see this identity by drawing a tree and counting the number of
possibilities where both events come true.

\begin{problem}{Joint Probability I}
 A bag contains $5$ red marbles and $7$ blue marbles. I draw a marble from the
 bag and then flip a coin. What's the chance of flipping heads and drawing a red
 marble?
\end{problem}

\begin{problem}{Joint Probability II}
 A bag contains $10$ red marbles and $5$ blue marbles. I draw a marble from the
 bag at random, then I put it back. Then I draw another marble from the bag at
 random. What's the probability of drawing two blue marbles?
\end{problem}

\section{Dependent Events}

When two events $A$ and $B$ are not independent, we can still calculate the
probability nevertheless. We just need to be careful to update the probability.

\begin{problem}{Drawing Cards I}

 A deck of cards (without jokers) has $13$ cards of each of the $4$ suits ($52$
 cards total). What's the probability of drawing two spades without replacement?

 \begin{solution}
  \begin{align*}
   \frac{13}{52} \times \frac{12}{51}
   &= \frac{1}{4} \times \frac{12}{51} \\
   &= \frac{12}{204} \\
   &= \frac{1}{17}
  \end{align*}
  so the probability is \Ans{\frac{1}{17}}.
 \end{solution}

\end{problem}

\begin{problem}{Drawing Cards II}
 What about the probability of drawing two aces (without replacement) from a
 deck of cards (without jokers)?

 \begin{solution}
  \begin{align*}
   \frac{4}{52} \times \frac{3}{51}
   &= \frac{1}{13} \times \frac{1}{17} \\
   &= \frac{1}{221}
  \end{align*}
  so the probability is \Ans{\frac{1}{221}}.
 \end{solution}
\end{problem}

\section{Bayes' Theorem}

TK

\section{Expected Value}

The expected value is the mean of all possibilities. For instance, the expected
value of rolling a six-sided die is $3.5$.

\chapter{Logic}

\begin{problem}{true}
  If Alice starts her homework, she will finish it. If Alice finishes her homework, she will
  receive a good mark.

  \textit{True or false:}
  With the information above, we can conclude that if Alice does not receive a good mark,
  then Alice did not start her homework.

  (Write ``true'' or ``false'' on the blank.)
\end{problem}

\begin{problem}{false}
  If Alice starts her homework, she will finish it with probability \(3/4\). If Alice
  finishes her homework, she will receive a good mark with probability \(3/4\).

  Obviously, if Alice does not start her homework, then she cannot finish it. Also, if she
  does not finish her homework, then she cannot receive a good mark.

  \textit{True or false:}
  With the information above, we can conclude that if Alice does not receive a good mark,
  then Alice did not start her homework with probability \(9/16\).

  (Write ``true'' or ``false'' on the blank.)
\end{problem}


Logical thinking is the kind of reasoning necessary to make logical and correct
conclusions from givens. Much of logic can be thought of as building up
statements using rules.

\begin{problem}{Contrapositive}
If it is snowing, then it is cold. Which of the below statements is a logical
conclusion?

\begin{enumerate}[\hspace{.5cm}a.]
\item If it is cold, then it is snowing.
\item If it is not cold, then it is not snowing.
\item If it is not snowing, then it is not cold.
\end{enumerate}
\end{problem}

\begin{problem}{Logical Weather}
Select the statements where the left side logically implies the right side.

\begin{enumerate}[\hspace{.5cm}a.]
\item It is raining and it is cloudy $\to$ It is cloudy.
\item It is sunny or it is snowing. $\to$ It is snowing or it is sunny.
\item It is cold or it is not foggy. $\to$ It is foggy and it is cold.
\item It is not hot and it is not humid. $\to$ It is hot or it is humid.
\end{enumerate}
\end{problem}

\begin{problem}{Alvin in the Rain}

 If it is not raining, then Alvin is happy. Which of the following statements is
 a logical conclusion?

 \begin{enumerate}[\hspace{.5cm}a.]
  \item It is not raining and Alvin is happy.
  \item It is raining or Alvin is happy.
  \item If Alvin is happy, then it is not raining.
 \end{enumerate}
\end{problem}

\begin{problem}{Treasure Chest}

The key to a treasure chest is hidden in one of box A, box B, or box C. It is
either yellow, blue, or green. Where is the key given the following statements?
Is it possible to logically determine the colour of the key, or is there not
enough information to determine that?

 \begin{enumerate}
  \item If the key is in box C, then the key is yellow.
  \item If the key is blue, then the key is not in box B.
  \item If the key is not green, then the key is in box A.
  \item If the key is not in box C, then the key is not green.
 \end{enumerate}
\end{problem}

\begin{problem}{Inconclusive}

 Identify the statements that are not logical conclusions. (Assume the left hand
 side is true.) That is, the left side alone does not logically imply the right
 side.

 \begin{enumerate}[\hspace{.5cm}a.]
  \item If it is cold and it is snowing, then it is winter. It is winter. $\to$
  Either it is cold or it is snowing, or both.
  \item If the sun shines at 7:00, then it is summer. It is not summer. $\to$
  The sun is not shining at 7:00.
  \item All birds are animals. Some animals are mammals. $\to$ Some birds are
  mammals.
  \item All frogs eat flies. Some frogs eat mosquitoes. $\to$ Some frogs eat
  both mosquitoes and flies.
 \end{enumerate}

\end{problem}


\begin{problem}{Challenge}

You find three treasure boxes. Only one of them has treasure; the other two have
nothing. Each treasure box has a message written on it. Only one of these
messages is true; the other two are false. Which treasure box has the treasure?

\begin{enumerate}
\item The treasure is not here.
\item The treasure is not here.
\item The treasure is in box 2.
\end{enumerate}
\end{problem}

\section{Propositions}

A proposition is any statement. Example: ``It is sunny outside.''

\subsection{Negation}

We can negate this proposition like this: ``It is not sunny outside.''

\subsection{Conjunction}

We can create a proposition that is true only when both components are true by
using the word ``and''. For example: ``It is sunny outside and it is warm
outside.''

\subsection{Disjunction}

We can create a proposition that is true when either component is true, or if
both components are true. For example: ``It is sunny outside or it is warm
outside.''

\section{Arguments}

An argument is a list of propositions ending in a statement that is a logical
conclusion of the other propositions. For instance:

\begin{align*}
 1.~&\text{It is daytime.} \\
 2.~&\text{There are no clouds.} \\
 3.~&\text{If it is daytime and there are no clouds, then it is sunny.} \\
 \hline
 4.~&(1), (2) \to \text{It is daytime and there are no clouds.} \\
 \hline
 \therefore~&(4), (3) \to \text{It is sunny.}
\end{align*}

\section{Rules}

There are rules to make sure all statements are logical. We'll cover them below.

\subsection{Double negative}
\begin{equation*}
 \lnot\lnot p \implies p
\end{equation*}

\subsection{Conditional elimination}
\begin{equation*}
 \{p, p \implies q\} \implies q
\end{equation*}

\subsection{Conjunction introduction}

\begin{equation*}
 \{p, q\} \implies p \land q
\end{equation*}

\subsection{Conjunction elimination}

\begin{equation*}
 p \land q \implies p \\
 p \land q \implies q
\end{equation*}

\subsection{Disjunction introduction}

\begin{equation*}
 p \implies p \lor q \\
 q \implies p \lor q
\end{equation*}

\subsection{Disjunction elimination}

\begin{equation*}
 \{p \implies r, q \implies r, p \lor r\} \implies r
\end{equation*}

\subsection{Modus tollens}

Denying the consequent.

\begin{equation*}
 \{p \implies q, \lnot q\} \implies \lnot p
\end{equation*}

\section{Not logical}
These are not rules:

\begin{itemize}
 \item Denying the antecedent.
 \item Affirming the consequent.
\end{itemize}

\chapter{Optimization}

\begin{problem}{3}
  Ashley is a treasure hunter! She is on the island of Fidra off the Scottish coast. She
  has \(9\) minutes to find and dig up as much treasure as she can!

  Ashley needs to decide how long to spend looking for treasure, and how long to spend
  digging it up. Every minute she spends looking for treasure, she finds two buried gold
  coins. Every minute she spends digging up treasure, she can dig up a buried gold coin she
  has already found.

  For instance, if Ashley spends \(5\) minutes looking for treasure, she would have found
  \(10\) buried gold coins. But then she would only have \(4\) minutes left to dig them up,
  so she would leave the island with only \(4\) gold coins. The other \(6\) that she found
  have to be left behind!

  How much time should Ashley spend looking for treasure, to leave the island with as many
  gold coins as possible?
\end{problem}


% Glossaries and List of Figures
\printglossaries

\cleardoublepage
\addcontentsline{toc}{chapter}{\listfigurename}
\listoffigures
\end{document}
