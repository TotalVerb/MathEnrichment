\documentclass[12pt,letterpaper]{article}

\newcommand{\IncludePath}{../include}
\usepackage{extsizes}
\usepackage{titling}

\usepackage{tikz}
\usetikzlibrary{shapes}

\usepackage{amssymb,amsmath,amsthm}
\usepackage{enumerate}
\usepackage[margin=0.8in]{geometry}
\usepackage{graphicx,ctable,booktabs}
\usepackage{fancyhdr}
\usepackage[utf8]{inputenc}
\usepackage{gensymb}

\makeatletter
\newenvironment{problem}{\@startsection
       {section}
       {1}
       {-.2em}
       {-3.5ex plus -1ex minus -.2ex}
       {2.3ex plus .2ex}
       {\pagebreak[3]
       \large\bf\noindent{Problem }
       }
       }
\makeatother

\pagestyle{fancy}
\lhead{\thetitle}
\chead{}
\rhead{\thepage}
\lfoot{\small\scshape Olympic Math}
\cfoot{}
\rfoot{}
\renewcommand{\headrulewidth}{.3pt}
\renewcommand{\footrulewidth}{.3pt}
\setlength\voffset{-0.25in}
\setlength\textheight{648pt}
\setlength\headheight{15pt}

\newcommand{\blankA}{\underline{\hspace{1em}}}
\newcommand{\blankB}{\underline{\hspace{2em}}}
\newcommand{\blankC}{\underline{\hspace{3em}}}
\newcommand{\blankD}{\underline{\hspace{4em}}}
\newcommand{\blankE}{\underline{\hspace{5em}}}
\newcommand{\blankF}{\underline{\hspace{6em}}}



\title{Function Golf}
\author{Name: \underline{\hspace{5cm}}}

\begin{document}

\HomeworkTitle

\thispagestyle{empty}

In function golf, the objective is simple. Given a list of functions, and two
numbers $n$ and $m$, apply functions to $n$ until you reach $m$. For example,
if:
\begin{align*}
 A(x) &= x^2 \\
 B(x) &= x-1 \\
 n &= 10 \\
 m &= 36 && \text{golf for $m$ starting with $n$}
\end{align*}
then one solution is:
\begin{align*}
 m = n \rhd B \rhd B \rhd B \rhd B \rhd A
\end{align*}

\begin{problem}{Function Golf I}
 Solve the following puzzle.
 \begin{align*}
  A(x) &= x + 3 \\
  n &= 10 \\
  m &= 21 && \text{golf for $m$ starting with $n$}
 \end{align*}
\end{problem}

\begin{problem}{Function Golf II}
 Solve the following puzzle.
 \begin{align*}
  A(x) &= x \times 2 \\
  n &= 3 \\
  m &= 24 && \text{golf for $m$ starting with $n$}
 \end{align*}
\end{problem}

\begin{problem}{Inverse Functions}
A \emph{function} is an object that takes in a value and produces a value.
For example, we can write \[
  x \mapsto x + 2
\] to describe the operation of adding two. Sometimes we give this operation
a name. For instance, we might say that \[
  f = x \mapsto x + 2
\] which means that we name the operation of adding two \(f\).

To apply an operation, we have the notation \[
  f(5) = 5 + 2 = 7
\] which can be read as ``\(f\) applied to \(5\)''. We compute this number by
replacing the \(x\) in \(x + 2\) with \(5\). Then

\begin{itemize}
  \item Compute \(f(10) = \blankE\)
  \item Compute \(f(5) + f(2) = \blankE\)
\end{itemize}

When we want to reverse addition, we use subtraction. For instance, let's say
we have a number \(10\), and we add \(2\) to get \(10 + 2 = 12\). How can we
get \(10\) back? We need to subtract \(2\) from \(12\): \(12 - 2 = 10\).

\begin{itemize}
  \item Let \(g\) be the inverse operation of \(f\). Then \[
    g = x \mapsto \blankE
  \]
\end{itemize}
\end{problem}


\end{document}
