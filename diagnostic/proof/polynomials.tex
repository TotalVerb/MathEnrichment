\documentclass[12pt,letterpaper]{article}

\newcommand{\IncludePath}{../../include}
\usepackage{extsizes}
\usepackage{titling}

\usepackage{tikz}
\usetikzlibrary{shapes}

\usepackage{amssymb,amsmath,amsthm}
\usepackage{enumerate}
\usepackage[margin=0.8in]{geometry}
\usepackage{graphicx,ctable,booktabs}
\usepackage{fancyhdr}
\usepackage[utf8]{inputenc}
\usepackage{gensymb}

\makeatletter
\newenvironment{problem}{\@startsection
       {section}
       {1}
       {-.2em}
       {-3.5ex plus -1ex minus -.2ex}
       {2.3ex plus .2ex}
       {\pagebreak[3]
       \large\bf\noindent{Problem }
       }
       }
\makeatother

\pagestyle{fancy}
\lhead{\thetitle}
\chead{}
\rhead{\thepage}
\lfoot{\small\scshape Olympic Math}
\cfoot{}
\rfoot{}
\renewcommand{\headrulewidth}{.3pt}
\renewcommand{\footrulewidth}{.3pt}
\setlength\voffset{-0.25in}
\setlength\textheight{648pt}
\setlength\headheight{15pt}

\newcommand{\blankA}{\underline{\hspace{1em}}}
\newcommand{\blankB}{\underline{\hspace{2em}}}
\newcommand{\blankC}{\underline{\hspace{3em}}}
\newcommand{\blankD}{\underline{\hspace{4em}}}
\newcommand{\blankE}{\underline{\hspace{5em}}}
\newcommand{\blankF}{\underline{\hspace{6em}}}



\title{Polynomials}
\author{By Fengyang Wang for Saturday Sessions}
\date{March 4, 2017}

\newcommand{\im}{i}

\begin{document}
\HomeworkTitle
\begin{problem}{Factorization in \(\mathbf{Z}[x]\)}
  Define \(\mathbf{Z}[x]\) to be the set of polynomials with integer
  coefficients. In \(\mathbf{Z}[x]\), fully factor the following.

  \begin{enumerate}
    \item \(16 - x^2\)
    \item \(2 + 5x + 3x^2\)
    \item \(1 + 6x + 12x^2 + 8x^3\)
    \item \(1 - 4x + 6x^2 - 4x^3 + x^4\)
    \item \(1 - x^4\)
  \end{enumerate}
\end{problem}

\begin{problem}{Factorization in \(\mathbf{Q}[x]\)}
  Define \(\mathbf{Q}[x]\) to be the set of polynomials with rational
  coefficients. In \(\mathbf{Q}[x]\), fully factor the following.

  \begin{enumerate}
    \item \(\frac{1}{4} + x + x^2\)
    \item \(\frac{4}{9} - \frac{4}{9}x + \frac{1}{9}x^2\)
    \item \(-x + 2x^2\)
  \end{enumerate}
\end{problem}

\begin{problem}{Factorization in \(\mathbf{R}[x]\)}
  Define \(\mathbf{R}[x]\) to be the set of polynomials with real coefficients.
  In \(\mathbf{R}[x]\), fully factor the following.

  \begin{enumerate}
    \item \(1 - 5x^2\)
    \item \(-1 + x + x^2\)
    \item \(3 - 3x - x^2 + x^3\)
  \end{enumerate}
\end{problem}

\begin{problem}{Finite Geometric Series}
  \begin{enumerate}
    \item Expand \(\frac{1-x^2}{1-x}\) as a polynomial.
    \item Expand \(\frac{1-x^5}{1-x}\) as a polynomial.
    \item Find a fraction of the form \(\frac{p(x)}{q(x)}\), where \(p\) and
    \(q\) are polynomials, such that \[
      \frac{p(x)}{q(x)} = \sum_{k=0}^n x^k = 1 + x + \dots + x^n
    \]
  \end{enumerate}
\end{problem}

\begin{problem}{Binomial Coefficients}
  Recall that \(n! = 1\cdot2 \cdot \dots \cdot (n-1) \cdot n\) is the product
  of all integers from \(1\) to \(n\).

  \begin{enumerate}
    \item Define \[
      \binom{x}{n} := \frac{x\cdot(x-1)\cdot\dots\cdot(x-n+1)}{n!}
    \]

    Find the constant term of \(\binom{x}{n}\).
    \item Find the leading coefficient (that is, the coefficient of the \(x^n\)
    term) of \(\binom{x}{n}\).
    \item Compute each of the following. \[
      \binom{5}{2} = \hspace{4ex}
      \binom{-1}{10} = \hspace{4ex}
      \binom{1/2}{3} = \hspace{4ex}
    \]
    \item Prove that \[
      \binom{n}{n} = \binom{n}{0} = 1
    \]
  \end{enumerate}
\end{problem}

\begin{problem}{\(q\)-analogs of Binomial Coefficients}
  \begin{enumerate}
    \item Define for non-negative integers \(n > m\) \[
      \binom{n}{m}_x := \prod_{k=1}^{m} \frac{1-x^{n-k-1}}{1-x^{k}}
    \]

    Prove that \(\binom{n}{m}_x\) is a polynomial in \(x\).
    \item Show that \[
      \binom{n+1}{m}_x = x^m \binom{n}{m}_x + \binom{n}{m-1}_x
    \]
  \end{enumerate}
\end{problem}

\end{document}
