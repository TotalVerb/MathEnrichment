\documentclass[12pt,letterpaper]{article}

\newcommand{\IncludePath}{../../include}
\usepackage{extsizes}
\usepackage{titling}

\usepackage{tikz}
\usetikzlibrary{shapes}

\usepackage{amssymb,amsmath,amsthm}
\usepackage{enumerate}
\usepackage[margin=0.8in]{geometry}
\usepackage{graphicx,ctable,booktabs}
\usepackage{fancyhdr}
\usepackage[utf8]{inputenc}
\usepackage{gensymb}

\makeatletter
\newenvironment{problem}{\@startsection
       {section}
       {1}
       {-.2em}
       {-3.5ex plus -1ex minus -.2ex}
       {2.3ex plus .2ex}
       {\pagebreak[3]
       \large\bf\noindent{Problem }
       }
       }
\makeatother

\pagestyle{fancy}
\lhead{\thetitle}
\chead{}
\rhead{\thepage}
\lfoot{\small\scshape Olympic Math}
\cfoot{}
\rfoot{}
\renewcommand{\headrulewidth}{.3pt}
\renewcommand{\footrulewidth}{.3pt}
\setlength\voffset{-0.25in}
\setlength\textheight{648pt}
\setlength\headheight{15pt}

\newcommand{\blankA}{\underline{\hspace{1em}}}
\newcommand{\blankB}{\underline{\hspace{2em}}}
\newcommand{\blankC}{\underline{\hspace{3em}}}
\newcommand{\blankD}{\underline{\hspace{4em}}}
\newcommand{\blankE}{\underline{\hspace{5em}}}
\newcommand{\blankF}{\underline{\hspace{6em}}}



\title{Complex Numbers --- Examples}
\author{By Fengyang Wang for Saturday Sessions}
\date{November 4, 2016}

\newcommand{\im}{i}

\begin{document}

\HomeworkTitle

\begin{problem}{Polar Form}
  \begin{itemize}
    \item \(1 = \exp\left(\blankB + \blankB\im\right)\)
    \item \(\sqrt{2}-\sqrt{2}\im = \blankB\exp\blankC\im\)
    \item \(\sqrt{2}\exp\frac{\im\pi}{4} = \blankB + \blankB\im\)
  \end{itemize}
\end{problem}

\begin{problem}{Measurement of Length}

\end{problem}

\begin{problem}{Measurement of Angles}

\end{problem}

\begin{problem}{Collinearity of Points}
  Let \(a, b, c\in\mathbf{C}\) represent points \(A, B, C\) in the 2D Euclidean
  plane, all distinct.

  \begin{itemize}
    \item When are these three points collinear (that is, they lie on the same
    line)? Again, express your answer as a single equation involving complex
    numbers \(a, b, c, d\), and free parameter \(t\in\mathbf{R}\).
    \item Hence, derive that a condition for \(A, B, C\) collinear is \[
      \frac{c-a}{c-b} = \overline{\left(\frac{c-a}{c-b}\right)}
    \]
  \end{itemize}
\end{problem}

\end{document}
