\documentclass[12pt,letterpaper]{article}

\newcommand{\IncludePath}{../../include}
\usepackage{extsizes}
\usepackage{titling}

\usepackage{amssymb,amsmath,amsthm}
\usepackage{enumerate}
\usepackage[margin=1in]{geometry}
\usepackage{graphicx,ctable,booktabs}
\usepackage{fancyhdr}
\usepackage[utf8]{inputenc}

\makeatletter
\newenvironment{problem}{\@startsection
       {section}
       {1}
       {-.2em}
       {-3.5ex plus -1ex minus -.2ex}
       {2.3ex plus .2ex}
       {\pagebreak[3]
       \large\bf\noindent{Problem }
       }
       }
\makeatother

\pagestyle{fancy}
\lhead{\thetitle}
\chead{}
\rhead{\thepage}
\lfoot{\small\scshape Grade 4 Olympic Math}
\cfoot{}
\rfoot{}
\renewcommand{\headrulewidth}{.3pt}
\renewcommand{\footrulewidth}{.3pt}
\setlength\voffset{-0.25in}
\setlength\textheight{648pt}
\setlength\headheight{15pt}


\title{Complex Numbers}
\author{By Fengyang Wang for Saturday Sessions}
\date{November 5, 2016}

\newcommand{\im}{i}

\begin{document}
\HomeworkTitle

\begin{problem}{Complex Logarithm}
 Compute the logarithm of each complex number, given in Cartesian form. Because
 the complex logarithm is multivalued, here you need simply fill in the blanks
 so that the equation is true. Recall that \(\exp{x} = e^x\). Make all angles
 between \(-\pi\) and \(\pi\), keeping the sign suggested in the problem. Write
 answers in exact form.

 \begin{enumerate}[\hspace{.5cm}a.]
  \item \(e = 2.718\ldots = \exp\left(\blankB + \blankB\im\right)\)
  \item \(-1 = \exp\left(\blankB + \blankB\im\right)\)
  \item \(-\im = \exp\left(\blankB - \blankC\im\right)\)
 \end{enumerate}
\end{problem}

\begin{problem}{Polar Form}
 Write each complex number, given in Cartesian form, in polar form. Recall
 that polar form is \(r\exp{\im\theta}\), where \(-\pi < \theta \le \pi\).

 \begin{enumerate}[\hspace{.5cm}a.]
  \item \(1 + \sqrt{3}\im = \blankB\exp\blankC\im\)
  \item \(-4 = \blankB\exp\blankC\im\)
  \item \(5\im = \blankB\exp\blankC\im\)
 \end{enumerate}

 \noindent Write the following complex number, given in polar form, in
 Cartesian form.

 \begin{enumerate}[\hspace{.5cm}a.]
  \item \(2\sqrt{3}\exp\frac{2\im\pi}{3} = \blankB + \blankB\im\)
 \end{enumerate}
\end{problem}

\begin{problem}{Orthogonality}
  Let \(a,b,c,d\in\mathbf{C}\) represent points \(A, B, C, D\) in the 2D
  Euclidean plane, all distinct.

  \begin{itemize}
    \item The line \(AB\) can be defined by \(a + t(b-a)\) where
    \(t\in\mathbf{R}\). What equation describes this line, rotated
    \SI{90}{\degree}, pivoted at \(A\)?
    \item When is \(AB\) orthogonal to \(CD\)? Express your answer as a single
    equation involving complex numbers \(a, b, c, d\), the imaginary unit
    \(\im\), and free parameter \(t\in\mathbf{R}\), without directly involving
    complex conjugates.
    \item Hence, derive that a condition for \(AB \perp CD\) is \[
      \frac{d-c}{b-a} = -\overline{\left(\frac{d-c}{b-a}\right)}
    \]
  \end{itemize}
\end{problem}

\begin{problem}{Characterization of Cyclic Quadrilaterals}
  Let \(a,b,c,d\in\mathbf{C}\) represent points \(A, B, C, D\) in the 2D
  Euclidean plane, all distinct.

  \begin{itemize}
    \item There is a theorem that says that a quadrilateral is cyclic (i.e. all
    four vertices lie on a circle) if the opposite angles are equal. Give an
    equation that characterizes when \(\angle ABC + \angle ADC =
    \SI{180}{\degree}\). Be careful, as angles computed with \(\arg\) are
    directed.
    \item Hence, derive that \(A, B, C, D\) lie on the same circle if and only
    if \[
      \frac{b-a}{b-c} = t\frac{d-a}{d-c}
    \] for some real parameter \(t\in\mathbf{R}\).
  \end{itemize}
\end{problem}

\begin{problem}{Reflection}
  Let \(a, b, c\in\mathbf{C}\) represent points \(A, B, C\) in the 2D Euclidean
  plane, all distinct.

  \begin{itemize}
    \item Write the equation of the line \(AB\).
    \item Write the equation of the line perpendicular to \(AB\) that passes
    through \(C\).
    \item Let \(C'\) be the reflection of point \(C\) about line \(AB\). Show
    that the complex number \(c'\) representing point \(C'\) is given by \[
      c' = \frac{(a-b)\overline{c} + \overline{a}b -
      a\overline{b}}{\overline{a} - \overline{b}}
    \]
  \end{itemize}
\end{problem}

\begin{problem}{Complex Trivia}
  \begin{itemize}
    \item Let \(z\in\mathbf{C}\). Show that \(z\overline{z} = {\|z\|}^2\).
    \item Let \(z\in\mathbf{C}\). Show that \({z}^{-1} = \overline{z} \iff
    \|z\| = 1\).
    \item Let \(a, b, c\in\mathbf{C}\) represent points \(A, B, C\) in the 2D
    Euclidean plane, all distinct. Show that \[
      \frac{a-b}{b-c} \in \mathbf{R} \iff A, B, C\text{ are collinear}
    \]
    \item Find the area of triangle \(\triangle ABC\) in terms of complex
    numbers \(a, b, c\in\mathbf{C}\).
    \item (IMO 2004/5) In the convex quadrilateral \(ABCD\), the diagonal
    \(BD\) is not the bisector of any of the angles \(\angle ABC\) and \(\angle
    CDA\). Let \(P\) be the point in the interior of \(ABCD\) such that
    \(\angle PBC\) = \(\angle DBA\) and \(\angle PDC = \angle BDA\). Prove that
    the quadrilateral \(ABCD\) is cyclic if and only if \(AP = CP\).
  \end{itemize}
\end{problem}

\end{document}
