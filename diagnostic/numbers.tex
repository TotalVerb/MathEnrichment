\documentclass[12pt,letterpaper]{article}

\usepackage{extsizes}
\usepackage{titling}

\usepackage{amssymb,amsmath,amsthm}
\usepackage{enumerate}
\usepackage[margin=1in]{geometry}
\usepackage{graphicx,ctable,booktabs}
\usepackage{fancyhdr}
\usepackage[utf8]{inputenc}

\makeatletter
\newenvironment{problem}{\@startsection
       {section}
       {1}
       {-.2em}
       {-3.5ex plus -1ex minus -.2ex}
       {2.3ex plus .2ex}
       {\pagebreak[3]
       \large\bf\noindent{Problem }
       }
       }
\makeatother

\pagestyle{fancy}
\lhead{\thetitle}
\chead{}
\rhead{\thepage}
\lfoot{\small\scshape Grade 4 Olympic Math}
\cfoot{}
\rfoot{}
\renewcommand{\headrulewidth}{.3pt}
\renewcommand{\footrulewidth}{.3pt}
\setlength\voffset{-0.25in}
\setlength\textheight{648pt}
\setlength\headheight{15pt}


\title{Numbers \& Operations}
\author{Name: \underline{\hspace{5cm}}}
\date{December 22, 2015}

\begin{document}

\maketitle

\thispagestyle{empty}

\begin{problem}{Discrete Exponents}
 Evaluate each expression. Write your answer as an integer in simplest form
 using the place value system.

 \begin{enumerate}[\hspace{.5cm}a.]
  \item $2^4=$ \hfill\underline{\hspace{3em}}
  \item $3^2=$ \hfill\underline{\hspace{3em}}
  \item $10^6=$ \hfill\underline{\hspace{6em}}
 \end{enumerate}
\end{problem}

\begin{problem}{Scientific Notation}
 Express in scientific notation.

 \begin{enumerate}[\hspace{.5cm}a.]
  \item $1234=$ \hfill\underline{\hspace{6em}}
  \item $0.000987=$ \hfill\underline{\hspace{6em}}
 \end{enumerate}
\end{problem}

\begin{problem}{Fractions, Exponents \& Radicals}
 Evaluate each expression. Write your answer in simplest form as a fraction, or
 as an integer using the place value system.

 \begin{enumerate}[\hspace{.5cm}a.]
  \item $4^{\frac{1}{2}}=$ \hfill\underline{\hspace{3em}}
  \item $9^{\frac{3}{2}}=$ \hfill\underline{\hspace{3em}}
  \item ${\left(\frac{2}{3}\right)}^3=$ \hfill\underline{\hspace{3em}}
  \item $\sqrt{\frac{16}{25}}=$ \hfill\underline{\hspace{3em}}
  \item $\sqrt[4]{\frac{256}{81}}=$ \hfill\underline{\hspace{3em}}
 \end{enumerate}
\end{problem}

\begin{problem}{Irrational Numbers}
 Classify each number as rational or irrational.

 \begin{enumerate}[\hspace{.5cm}a.]
  \item $8.25$ \hfill Rational~~Irrational
  \item $\sqrt{2}$ \hfill Rational~~Irrational
  \item $\sqrt{9}$ \hfill Rational~~Irrational
  \item $\pi$ \hfill Rational~~Irrational
 \end{enumerate}
\end{problem}

\begin{problem}{Negative and Zero Exponents}
 Evaluate each expression. Write your answer in simplest form as a fraction, or
 as an integer using the place value system.

 \begin{enumerate}[\hspace{.5cm}a.]
  \item ${-1}^{-1}=$ \hfill\underline{\hspace{3em}}
  \item $4^{-2}=$ \hfill\underline{\hspace{3em}}
  \item ${999}^0=$ \hfill\underline{\hspace{3em}}
  \item ${\left(\frac{-17}{4}\right)}^0=$ \hfill\underline{\hspace{3em}}
 \end{enumerate}
\end{problem}

\begin{problem}{Associativity}
 Call an operation $\blacksquare$ ``associative'' if we have for all $a$, $b$,
 and $c$: $(a \blacksquare b) \blacksquare c = a \blacksquare (b \blacksquare
 c)$.

 \begin{enumerate}[\hspace{.5cm}a.]
  \item \emph{Apropos} (with regard to) positive integers, is $+$ associative?
  (That is, is $(a+b)+c=a+(b+c)$ for all $a$, $b$ and $c$?)
  \item Apropos positive integers, is $\times$ associative?
  \item Let $\uparrow$ represent exponentiation; that is, $2\uparrow4=2^4=16$.
  Apropos positive integers, is $\uparrow$ associative?
 \end{enumerate}
\end{problem}

\begin{problem}{Commutativity}
 Call an operation $\blacksquare$ ``commutative'' if we have for all $a$ and
 $b$: $a \blacksquare b = b \blacksquare a$.

 \begin{enumerate}[\hspace{.5cm}a.]
  \item Apropos positive integers, is $+$ commutative?
  \item Apropos positive integers, is $\times$ commutative?
  \item Apropos positive integers, is $\uparrow$ commutative?
 \end{enumerate}
\end{problem}

\end{document}
