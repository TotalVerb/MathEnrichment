\documentclass[12pt,a4paper]{report}

\newcommand{\IncludePath}{../../include}
\newcommand{\ProjectName}{Enrichment Math}
\usepackage{extsizes}
\usepackage{titling}

\usepackage{tikz}
\usetikzlibrary{shapes}

\usepackage{amssymb,amsmath,amsthm}
\usepackage{enumerate}
\usepackage{graphicx,ctable,booktabs}
\usepackage{fancyhdr}
\usepackage[utf8]{inputenc}
\usepackage{gensymb}

\usepackage[toc]{glossaries}

\makeatletter
\newenvironment{problem}{\@startsection
       {subsection}
       {1}
       {-.2em}
       {-3.5ex plus -1ex minus -.2ex}
       {2.3ex plus .2ex}
       {\pagebreak[3]
       \large\bf\noindent{Problem }
       }
       }
\makeatother

\makeatletter
\g@addto@macro\@floatboxreset\centering
\makeatother

\newenvironment{solution}
{ \vspace{1em} \noindent \textbf{Solution:} }
{  }

\pagestyle{fancy}
\lhead{\thetitle}
\chead{}
\rhead{\thepage}
\lfoot{\small\scshape \ProjectName}
\cfoot{}
\rfoot{}
\renewcommand{\headrulewidth}{.3pt}
\renewcommand{\footrulewidth}{.3pt}
\setlength\voffset{-0.25in}
\setlength\textheight{648pt}
\setlength\headheight{15pt}

\newcommand{\Ans}[1]{\framebox{$#1$}}
\newcommand{\SAA}[1]{\Switch{\Ans{#1}}{\blankA}}
\newcommand{\SAB}[1]{\Switch{\Ans{#1}}{\blankB}}
\newcommand{\SAC}[1]{\Switch{\Ans{#1}}{\blankC}}
\newcommand{\SAD}[1]{\Switch{\Ans{#1}}{\blankD}}
\newcommand{\SAE}[1]{\Switch{\Ans{#1}}{\blankE}}
\newcommand{\SAF}[1]{\Switch{\Ans{#1}}{\blankF}}
\newcommand{\STA}[1]{\Switch{\AnsT{#1}}{\blankA}}
\newcommand{\STB}[1]{\Switch{\AnsT{#1}}{\blankB}}
\newcommand{\STC}[1]{\Switch{\AnsT{#1}}{\blankC}}
\newcommand{\STD}[1]{\Switch{\AnsT{#1}}{\blankD}}
\newcommand{\STE}[1]{\Switch{\AnsT{#1}}{\blankE}}
\newcommand{\STF}[1]{\Switch{\AnsT{#1}}{\blankF}}
\newcommand{\AnsT}[1]{\framebox{#1}}
\newif\ifanswers
\newcommand{\Switch}[2]{\ifanswers#1\else#2\fi}
\newcommand{\MCSelect}[1]{\Switch{\AnsT{#1}}{#1}}
\newcommand{\TFTrue}{\MCSelect{True}~~False}
\newcommand{\TFFalse}{True~~\MCSelect{False}}

\newcommand{\blankA}{\underline{\hspace{1em}}}
\newcommand{\blankB}{\underline{\hspace{2em}}}
\newcommand{\blankC}{\underline{\hspace{3em}}}
\newcommand{\blankD}{\underline{\hspace{4em}}}
\newcommand{\blankE}{\underline{\hspace{5em}}}
\newcommand{\blankF}{\underline{\hspace{6em}}}



\title{Operations}
\author{Name: \underline{\hspace{6cm}}}
\date{March 24, 2016}

\begin{document}

\maketitle

\chapter{Algebra}

Various forms of numbers arise from the desire to solve algebraic equations. But
contrary to popular belief, this desire was not to be able to solve equations
that have no solution, but rather to solve equations that do have a solution,
but whose solution can be arrived at easier by introducing a new kind of number.

The first kinds of numbers discovered were the (positive) natural numbers: \[ 1,
2, 3, 4, 5, 6, 7, 8, 9, 10, \dots \]

There are many positive natural numbers, and there are many things that we can
do with them. For example, we can add any two positive natural numbers to obtain
another. It does not matter how big the numbers become: their sum is always
still a natural number.

We may also multiply two positive natural numbers to obtain another. Again, it
does not matter how big or small the numbers are: multiplication is something
we can do with \emph{any} positive natural numbers.

Positive natural numbers are great for representing ``things''. In particular,
they're great for representing ``some things''. But they fall short when we want
to represent ``no things'' (or ``nothing''). Even worse, they fall apart
completely when we want to represent \emph{less} than nothing. But why would we
ever want that?

The desire for a more general kind of number comes from the desire to represent
\emph{change}. If yesterday there were $20$ students in attendance, and today
there are $24$ students in attendance, then we can say $4$ more people attended
today than did yesterday. But what if tomorrow there will again be $24$ students
in attendance? Then how many more students will have attended? Even worse, what
if the day after tomorrow, there will be $20$ students in attendance again. Then
what is the change in the number of students?

It is not that these problems cannot be solved with positive natural numbers. In
fact, they certainly can. The number of students could increase by $4$, and we
have a number for that. The number of students could also remain unchanged. We
don't need a number to express lack of change. Finally, the number of students
could \emph{decrease} by $4$. We already have a number for that.

So to express a change in a quantity, we must convey several pieces of
information: firstly, whether there is a change at all; secondly, in what
direction is the change; and thirdly, by how much the quantity is changed. This
seems quite complicated. It would be simpler to introduce a new kind of
quantity that could represent these changes more conveniently and more
compactly.

\section{Exponents}

\begin{problem}{Discrete Exponents}
 Evaluate each expression. Write your answer as an integer in simplest form
 using the place value system.

 \begin{enumerate}[\hspace{.5cm}a.]
  \item $2^4=$ \hfill\blankC
  \item $3^2=$ \hfill\blankC
  \item $10^6=$ \hfill\blankF
 \end{enumerate}
\end{problem}

\begin{problem}{Scientific Notation}
 Express in scientific notation.

 \begin{enumerate}[\hspace{.5cm}a.]
  \item $1234=$ \hfill\blankF
  \item $0.000987=$ \hfill\blankF
 \end{enumerate}
\end{problem}

\begin{problem}{Fractions, Exponents \& Radicals}
 Evaluate each expression. Write your answer in simplest form as a fraction, or
 as an integer using the place value system.

 \begin{enumerate}[\hspace{.5cm}a.]
  \item $4^{\frac{1}{2}}=$ \hfill\blankC
  \item $9^{\frac{3}{2}}=$ \hfill\blankC
  \item ${\left(\frac{2}{3}\right)}^3=$ \hfill\blankC
  \item $\sqrt{\frac{16}{25}}=$ \hfill\blankC
  \item $\sqrt[4]{\frac{256}{81}}=$ \hfill\blankC
 \end{enumerate}
\end{problem}

\begin{problem}{Irrational Numbers}
 Classify each number as rational or irrational.

 \begin{enumerate}[\hspace{.5cm}a.]
  \item $8.25$ \hfill Rational~~Irrational
  \item $\sqrt{2}$ \hfill Rational~~Irrational
  \item $\sqrt{9}$ \hfill Rational~~Irrational
  \item $\pi$ \hfill Rational~~Irrational
 \end{enumerate}
\end{problem}

\begin{problem}{Negative and Zero Exponents}
 Evaluate each expression. Write your answer in simplest form as a fraction, or
 as an integer using the place value system.

 \begin{enumerate}[\hspace{.5cm}a.]
  \item ${-1}^{-1}=$ \hfill\blankC
  \item $4^{-2}=$ \hfill\blankC
  \item ${999}^0=$ \hfill\blankC
  \item ${\left(\frac{-17}{4}\right)}^0=$ \hfill\blankC
 \end{enumerate}
\end{problem}

\begin{problem}{Associativity}
 Call an operation $\blacksquare$ ``associative'' if we have for all $a$, $b$,
 and $c$: $(a \blacksquare b) \blacksquare c = a \blacksquare (b \blacksquare
 c)$.

 \begin{enumerate}[\hspace{.5cm}a.]
  \item \emph{Apropos} (with regard to) positive integers, is $+$ associative?
  (That is, is $(a+b)+c=a+(b+c)$ for all $a$, $b$ and $c$?)
  \item Apropos positive integers, is $\times$ associative?
  \item Let $\uparrow$ represent exponentiation; that is, $2\uparrow4=2^4=16$.
  Apropos positive integers, is $\uparrow$ associative?
 \end{enumerate}
\end{problem}

\begin{problem}{Commutativity}
 Call an operation $\blacksquare$ ``commutative'' if we have for all $a$ and
 $b$: $a \blacksquare b = b \blacksquare a$.

 \begin{enumerate}[\hspace{.5cm}a.]
  \item Apropos positive integers, is $+$ commutative?
  \item Apropos positive integers, is $\times$ commutative?
  \item Apropos positive integers, is $\uparrow$ commutative?
 \end{enumerate}
\end{problem}

\chapter{Sums}

Adding things is a very important part of mathematics. When we have a large
number of things to add, it helps to use algebra to simplify the problem. The
notation for finite sums is \[
 \sum_{k = 1}^n a_k
\] where $a_1, a_2, \dots, a_n$ represent some numbers. Such a sum is called a
series, but we will use this term with caution, as it is often used to denote
infinite sums, which we will not cover. The numbers themselves, $(a_1, a_2,
\dots a_n)$, form an $n$-tuple.

\section{Arithmetic Series}

The first kind of series is a called an arithmetic series, in which the
$n$-tuple satisfies the property that all consecutive pairwise differences are
equal. That is, if for all $1 \le k < n$, we have \[ a_{k+1} - a_k = c \] where
$c$ is a constant, then we say that $(a_1, a_2, \dots, a_n)$ is in arithmetic
progression.

The sum of an arithmetic series can be computed by using the method of averages.
The idea here is quite simple: in any arithmetic progression, the average term
is $\frac{a_1 + a_n}{2}$. There is a simple proof of this fact, but it is not
too hard to think of intuitively. A result fundamental to statistics tells us
that, letting $\mu$ represent the average of the terms, and $s$ their sum, \[
 s = n \mu
\]

Hence, the formula for the sum of an arithmetic series \begin{equation}
 \sum_{k=1}^n a_k = \frac{n}{2} (a_1 + a_n)
\end{equation}

\section{Geometric Series}

The second kind of series is called a geometric series, in which the $n$-tuple
satisfies the property that all consecutive pairwise ratios are equal. This is,
if for all $1 \le k < n$, we have \[ \frac{a_{k+1}}{a_k} = r \] where $r$ is
again a constant, then we say that $(a_1, a_2, \dots, a_n)$ is in geometric
progression.

The sum of a geometric series can be computed by polynomial multiplication.
First, let $\alpha = a_1$. Then we rewrite the series: \begin{align*}
 \sum_{k=1}^n a_n
 &= \sum_{k=1}^n \alpha r^{k-1} \\
 &= \alpha \sum_{k=1}^n r^{k-1} \\
 &= \alpha (1 + r + r^2 + r^3 + \dots + r^{n-1}) \\
 &= \alpha \frac{(1 + r + r^2 + r^3 + \dots + r^{n-1})(1-r)}{1-r} \\
 &= \alpha \frac{1-r^n}{1-r}
\end{align*} which leads us to the formula for the sum of a geometric series
with common ratio $r$,
\begin{equation}
 \sum_{k=1}^n a_n = a_1 \frac{1 - r^n}{1 - r}
\end{equation}

\end{document}
