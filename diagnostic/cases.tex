\documentclass[12pt,letterpaper]{article}

\usepackage{extsizes}
\usepackage{titling}

\usepackage{amssymb,amsmath,amsthm}
\usepackage{enumerate}
\usepackage[margin=1in]{geometry}
\usepackage{graphicx,ctable,booktabs}
\usepackage{fancyhdr}
\usepackage[utf8]{inputenc}

\makeatletter
\newenvironment{problem}{\@startsection
       {section}
       {1}
       {-.2em}
       {-3.5ex plus -1ex minus -.2ex}
       {2.3ex plus .2ex}
       {\pagebreak[3]
       \large\bf\noindent{Problem }
       }
       }
\makeatother

\pagestyle{fancy}
\lhead{\thetitle}
\chead{}
\rhead{\thepage}
\lfoot{\small\scshape Grade 4 Olympic Math}
\cfoot{}
\rfoot{}
\renewcommand{\headrulewidth}{.3pt}
\renewcommand{\footrulewidth}{.3pt}
\setlength\voffset{-0.25in}
\setlength\textheight{648pt}
\setlength\headheight{15pt}


\title{Proof by Cases}
\author{By Anton Borissov, Billy Jin, Fengyang Wang\\
For Saturday Sessions}
\date{March 5, 2016}

\begin{document}

\maketitle

\thispagestyle{empty}

\begin{problem}{Proof by Cases I}
 Prove each statement. Where one exists, an algebraic solution may be quicker.

 \begin{enumerate}[\hspace{.5cm}a.]
  \item For all $x\in\mathbb{R}$, $1 + x + x^2 > 0$
  \item For all $a\in\mathbb{Z}$, $2^a \ge a + 1$
 \end{enumerate}
\end{problem}

\begin{problem}{Proof by Cases II}
 Prove the following statements regarding the set $\mathbb{Z}_2=\{0, 1\}$ under
 some operations with special properties.

 \begin{enumerate}[\hspace{.5cm}a.]
  \item Let's define the following binary operations for all $\alpha$, $\beta$
  in the set $\mathbb{Z}_2=\{0, 1\}$: \[
    \alpha \lor \beta = \begin{cases}
     0 & \text{if }\alpha=\beta=0 \\
     1 & \text{otherwise}
    \end{cases}
   \] and \[
    \alpha \land \beta = \begin{cases}
     1 & \text{if }\alpha=\beta=1 \\
     0 & \text{otherwise}
    \end{cases}
   \]

   Prove for all $\alpha$, $\beta$, $\gamma\in\mathbb{Z}_2$, the following
   identity holds: \[
    \alpha \lor \left(\beta \land \gamma\right) = \left( \alpha \lor \beta
    \right) \land \left( \alpha \lor \gamma \right)
   \]
  \item Let's use the same definitions as in the last problem. Additionally
  define the following unary operation for $\alpha\in\mathbb{Z}_2$: \[
   \lnot\alpha = \begin{cases}
    0 & \text{if }\alpha=1 \\
    1 & \text{otherwise}
   \end{cases}
  \]
  Prove for all $\alpha$, $\beta\in\mathbb{Z}_2$, the following
  identity holds: \hfill (De Morgan's) \[
   \lnot \left(\alpha \land \beta\right) = \lnot\alpha \lor \lnot\beta
  \]
 \end{enumerate}

 The set $\mathbb{Z}_2$ under the operations $\land$, $\lor$, and $\lnot$ is
 very useful in logic and computer science. The above two identities are two
 important logical principles. Combined with several others, they form the
 foundation of a topic known as ``propositional logic''. In computer science,
 the set $\mathbb{Z}_2$ and the operations defined above form the basis for what
 are called ``booleans''.
\end{problem}

\begin{problem}{Proof by Cases III}
 Prove each statement.

 \begin{enumerate}[\hspace{.5cm}a.]
  \item For all $a\in\mathbb{Z}$, $a^2\ge 0$
  \item For all $x\in\mathbb{R}$, if $x^3-x=0$, then $\left|x\right|=x^2$
  \item For all $n\in\mathbb{Z}$, if $2^{18} - 2^9 = 2^{10}n - n - n^2$,
  then $n$ is composite
 \end{enumerate}
\end{problem}

\begin{problem}{Proof by Cases IV}
 Prove each statement.

 \begin{enumerate}[\hspace{.5cm}a.]
  \item For all $a$, $b\in\mathbb{R}$,
  $\left|ab\right|=\left|a\right|\left|b\right|$
  \item For all $x\in\mathbb{R}$, $x+\left|x-7\right|\ge7$
  \item For all $a$, $b\in\mathbb{R}$,
  $\left|a+b\right|\le\left|a\right|+\left|b\right|$ \hfill
  (Triangle Inequality)
  \item For all $a$, $b\in\mathbb{R}$, $\max{\{a,b\}} = \frac{1}{2} \left(
  a + b + \left|a-b\right| \right)$
  \item For all $a\in\mathbb{R}$, $a^2+a^4\ge a^3$
 \end{enumerate}
\end{problem}

\begin{problem}{Proof by Cases V}
 Prove each statement.

 \begin{enumerate}[\hspace{.5cm}a.]
  \item For all $n\in\mathbb{Z}$, $n(n-1)$ is even
  \item For all $n\in\mathbb{Z}$, $n^2 \equiv n^4 \pmod{4}$
  \item For all $n\in\mathbb{Z}$, $n^6 \equiv n^{24} \pmod{9}$
 \end{enumerate}
\end{problem}

\end{document}
