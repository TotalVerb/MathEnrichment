\documentclass[12pt,letterpaper]{article}

\newcommand{\IncludePath}{../../include}
\usepackage{extsizes}
\usepackage{titling}

\usepackage{amssymb,amsmath,amsthm}
\usepackage{enumerate}
\usepackage[margin=1in]{geometry}
\usepackage{graphicx,ctable,booktabs}
\usepackage{fancyhdr}
\usepackage[utf8]{inputenc}

\makeatletter
\newenvironment{problem}{\@startsection
       {section}
       {1}
       {-.2em}
       {-3.5ex plus -1ex minus -.2ex}
       {2.3ex plus .2ex}
       {\pagebreak[3]
       \large\bf\noindent{Problem }
       }
       }
\makeatother

\pagestyle{fancy}
\lhead{\thetitle}
\chead{}
\rhead{\thepage}
\lfoot{\small\scshape Grade 4 Olympic Math}
\cfoot{}
\rfoot{}
\renewcommand{\headrulewidth}{.3pt}
\renewcommand{\footrulewidth}{.3pt}
\setlength\voffset{-0.25in}
\setlength\textheight{648pt}
\setlength\headheight{15pt}


\title{Number Theory}
\author{By Fengyang Wang for Saturday Sessions}
\date{September 23, 2017}

\newcommand{\im}{i}

\begin{document}
\HomeworkTitle
\begin{problem}{Factorization I}
  \begin{enumerate}
    \item
    List all the (positive integer) factors of \(20\):
   \blankB, \blankB,
   \blankB, \blankB,
   \blankB, \blankB
   \item
    A number is \emph{perfect} if the sum of its factors is equal to twice the
    number itself. For example, \(6\) is perfect because
    \(1+2+3+6=12=6\times2\). Which of the following is a perfect number:
    \(10\), \(17\), \(28\)?
   \item
   Between \(1\) and \(1000\) (inclusive), how many numbers are divisible by
   all of \(2\), \(3\), \(4\), and \(6\)? \Switch{\Ans{83}}{}
  \item \(2^{10}=1024\). How many (positive integer) factors does \(1024\)
  have?
 \end{enumerate}
\end{problem}

\begin{problem}{Factorization II}

The notation $5!$ (pronounced ``five factorial'') means the product of all
integers from $1$ to $5$; that is, $1 \times 2 \times 3 \times 4 \times 5$.
Similarly, $3! = 1 \times 2 \times 3$ and $100! = 1 \times 2 \times 3
\times\ldots\times 99 \times 100$.

\begin{enumerate}
 \item Find the prime factorization of $4!$. \hfill
 $\blankB \times \blankB \times \blankB \times \blankB$
 \item In the prime factorization of $100!$, how many $2$'s are there? (Hint:
 Do not write out the entire prime factorization! Which numbers in the product
 are composite? How many of those composite numbers contain a prime factor of
 $2$? Which of those contain more than one $2$ in their prime factorization?)
\end{enumerate}

\end{problem}

\begin{problem}{Palindromes}
 A \emph{palindrome} is a number that reads the same forwards and backwards in
 decimal notation. For example, $12321$ is a palindrome, as is $7227$ or
 $888888$.

 \begin{enumerate}
  \item How many two-digit palindromes are there? ($00=0$ does not have two
  digits.)
  \item What two numbers are factors of all two-digit palindromes?
  \item $1221$ and $9999$ are two examples of four-digit palindromes. How many
  four-digit palindromes are there? (Do not include palindromes starting with
  $0$.)
  \item How many four-digit palindromes are even?
  \item How many four-digit palindromes are divisible by $11$?
\end{enumerate}
\end{problem}

\begin{problem}{B\'ezout's Lemma}
 There is a very interesting property of the GCD called B\'ezout's Lemma. It
 says that if $a$ and $b$ are integers, and $\gcd(a, b) = d$, then there are
 some other integers $x$ and $y$ so that $ax + by = d$. Here are two examples.

 \begin{enumerate}
  \item Compute $\gcd(18, 16) = \blankC$
  \item Fill in the blanks with integers so that the following equation is true:
  \[ 18 \times \blankC + 16 \times (- \blankC) = 2 \]
  \item Fill in the blanks with integers so that the following equation is true:
  \[ 20 \times \blankC + 28 \times (- \blankC) = 4 \]
 \end{enumerate}
\end{problem}

\begin{problem}{Coprimeness}

 We say that two numbers $a$ and $b$ are relatively prime (or co-prime for
 short) if $\gcd(a, b) = 1$. That is, if two numbers have no common factors
 except $1$, then they are co-prime.

 \begin{enumerate}
  \item Are $3$ and $6$ co-prime? \(8\) and \(13\)? \(9\) and \(16\)?
  \item Fill in the blanks with integers so that the following equation is true:
  \[ 22331115 \times 9 \times \blankC + 22331115 \times 16 \times (-\blankC) =
  22331115 \]
  \item Given that $9 \mid 16\times22331115$, prove that $9\mid22331115$?
 \end{enumerate}
\end{problem}

\begin{problem}{Challenge}
  \begin{enumerate}
    \item
 A \emph{semiprime} number is a number which is the product of exactly two
 prime numbers. For example, $9=3\times3$ is semiprime, as is $38=2\times19$.
 Say $9409$ and $8633$ are two semiprime numbers that share a prime factor.
 Find their shared prime factor.
 \item How many numbers between $1$ and $100000$ (including both endpoints)
 are there that yield a remainder of $1$ when divided by $2$, $3$, $4$, $5$,
 $6$, $7$, and $8$?
 \item (NCC 2016 F6) Find the smallest positive integer \(n\) that satisfies \[
 \gcd(n^2 + 4n - 5, 2n^2 + 9n - 5) = 2016 \] where \(\gcd(a, b)\) denotes the
 greatest common divisor of \(a\) and \(b\).
 \item (NCC 2016 F8) Determine the number of \(4\)-tuples \((a, b, c, d)\) of
(not necessarily distinct) positive integers whose product, \(abcd\), is
\(2016\). You may use the fact that \(2016 = 2^5 \cdot 3^2 \cdot 7\).
  \item What is the unit's digit of \(7^7\)?
 \end{enumerate}
\end{problem}

\end{document}
