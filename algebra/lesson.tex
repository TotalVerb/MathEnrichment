\documentclass[a4paper,10pt]{report}

\newcommand{\IncludePath}{../include}
\newcommand{\ProjectName}{Grade 7/8 Olympic Math}
\usepackage{extsizes}
\usepackage{titling}

\usepackage{tikz}
\usetikzlibrary{shapes}

\usepackage{amssymb,amsmath,amsthm}
\usepackage{enumerate}
\usepackage{graphicx,ctable,booktabs}
\usepackage{fancyhdr}
\usepackage[utf8]{inputenc}
\usepackage{gensymb}

\usepackage[toc]{glossaries}

\makeatletter
\newenvironment{problem}{\@startsection
       {subsection}
       {1}
       {-.2em}
       {-3.5ex plus -1ex minus -.2ex}
       {2.3ex plus .2ex}
       {\pagebreak[3]
       \large\bf\noindent{Problem }
       }
       }
\makeatother

\makeatletter
\g@addto@macro\@floatboxreset\centering
\makeatother

\newenvironment{solution}
{ \vspace{1em} \noindent \textbf{Solution:} }
{  }

\pagestyle{fancy}
\lhead{\thetitle}
\chead{}
\rhead{\thepage}
\lfoot{\small\scshape \ProjectName}
\cfoot{}
\rfoot{}
\renewcommand{\headrulewidth}{.3pt}
\renewcommand{\footrulewidth}{.3pt}
\setlength\voffset{-0.25in}
\setlength\textheight{648pt}
\setlength\headheight{15pt}

\newcommand{\Ans}[1]{\framebox{$#1$}}
\newcommand{\SAA}[1]{\Switch{\Ans{#1}}{\blankA}}
\newcommand{\SAB}[1]{\Switch{\Ans{#1}}{\blankB}}
\newcommand{\SAC}[1]{\Switch{\Ans{#1}}{\blankC}}
\newcommand{\SAD}[1]{\Switch{\Ans{#1}}{\blankD}}
\newcommand{\SAE}[1]{\Switch{\Ans{#1}}{\blankE}}
\newcommand{\SAF}[1]{\Switch{\Ans{#1}}{\blankF}}
\newcommand{\STA}[1]{\Switch{\AnsT{#1}}{\blankA}}
\newcommand{\STB}[1]{\Switch{\AnsT{#1}}{\blankB}}
\newcommand{\STC}[1]{\Switch{\AnsT{#1}}{\blankC}}
\newcommand{\STD}[1]{\Switch{\AnsT{#1}}{\blankD}}
\newcommand{\STE}[1]{\Switch{\AnsT{#1}}{\blankE}}
\newcommand{\STF}[1]{\Switch{\AnsT{#1}}{\blankF}}
\newcommand{\AnsT}[1]{\framebox{#1}}
\newif\ifanswers
\newcommand{\Switch}[2]{\ifanswers#1\else#2\fi}
\newcommand{\MCSelect}[1]{\Switch{\AnsT{#1}}{#1}}
\newcommand{\TFTrue}{\MCSelect{True}~~False}
\newcommand{\TFFalse}{True~~\MCSelect{False}}

\newcommand{\blankA}{\underline{\hspace{1em}}}
\newcommand{\blankB}{\underline{\hspace{2em}}}
\newcommand{\blankC}{\underline{\hspace{3em}}}
\newcommand{\blankD}{\underline{\hspace{4em}}}
\newcommand{\blankE}{\underline{\hspace{5em}}}
\newcommand{\blankF}{\underline{\hspace{6em}}}



% use glossaries for this document
\newglossaryentry{algorithm}
{
  name=algorithm,
  description={step-by-step procedure for performing a calculation
  according to well-defined rules (Wikipedia)}
}

\newglossaryentry{acute angle}
{
  name=acute angle,
  description={an angle measuring less than a right angle (\SI{90}{\degree} or a
  quarter of a turn); such angles are typically characterized as being sharp}
}

\newglossaryentry{angle}
{
  name=angle,
  description={the figure formed by two rays, called the sides of the angle,
  sharing a common endpoint, called the vertex of the angle (Wikipedia)}
}

\newglossaryentry{coincident}
{
  name=coincident,
  description={a description of two objects which occupy exactly the same space}
}

\newglossaryentry{domain}
{
  name=domain,
  description={the set of values that are possible inputs for a function}
}

\newglossaryentry{endpoint}
{
  name=endpoint,
  description={an extreme point of a line segment or ray; line segments have two
  endpoints whereas rays have just one}
}

\newglossaryentry{function}
{
  name=function,
  description={object that may take any allowed input and will produce a
  single associated output for that input; alternatively, relation for
  which each input value has exactly one related output value}
}

\newglossaryentry{integer}
{
  name=integer,
  description={positive or negative whole number, or $0$; for example,
  $-8$, $2000$}
}

\newglossaryentry{line}
{
  name=line,
  description={a straight one-dimensional object that extends forever in both
  directions}
}

\newglossaryentry{line segment}
{
  name=line segment,
  description={a straight one-dimensional object terminated at both ends}
}

\newglossaryentry{multiplicand}
{
  name=multiplicand,
  description={the number that is being multiplied; for instance, in
  $2\times3=6$, the multiplicand is $2$}
}

\newglossaryentry{multiplier}
{
  name=multiplier,
  description={the factor to multiply a number by; for instance, in
  $2\times3=6$, the multiplier is $3$}
}

\newglossaryentry{obtuse angle}
{
  name=obtuse angle,
  description={an angle measuring more than a right angle (\SI{90}{\degree} or a
  quarter of a turn) but less than a straight angle (\SI{180}{\degree} or a half
  of a turn)}
}

\newglossaryentry{parallel}
{
  name=parallel,
  description={a description of two lines that never intersect at any point}
}

\newglossaryentry{parity}
{
  name=parity,
  description={decribes whether an integer is even or odd}
}

\newglossaryentry{plane}
{
  name=plane,
  description={a two-dimensional flat surface}
}

\newglossaryentry{plane geometry}
{
  name=plane geometry,
  description={the study of figures on a plane (a two-dimensional flat surface)}
}

\newglossaryentry{product}
{
  name=product,
  description={the result of a multiplication; for instance, in
  $2\times3=6$, the product is $6$}
}

\newglossaryentry{range}
{
  name=range,
  description={the set of values that are possible outputs for a function}
}

\newglossaryentry{ray}
{
  name=ray,
  description={a straight one-dimensional object terminated at one end and
  extending forever in the other direction}
}

\newglossaryentry{right angle}
{
  name=right angle,
  description={an angle measuring \SI{90}{\degree} or a quarter of a turn; angle
  between two rays intersecting in an L shape}
}

\newglossaryentry{right triangle}
{
  name=right triangle,
  description={an triangle with one \SI{90}{\degree} (right) angle}
}

\newglossaryentry{summand}
{
  name=summand,
  description={something which is being added; for instance, in $1+2=3$,
  the two summands are $1$ and $2$}
}

\newglossaryentry{reflex angle}
{
  name=reflex angle,
  description={an angle measuring more than \SI{180}{\degree} or a half of a
  turn, but less than \SI{360}{\degree} or a full turn}
}

\newglossaryentry{straight angle}
{
  name=straight angle,
  description={an angle measuring \SI{180}{\degree} or a half of a turn; angle
  between two rays in opposite directions}
}

\newglossaryentry{transitivity}
{
  name=transitivity,
  description={the property of certain relations that specifies if an
  element $a$ is related to $b$, and the element $b$ is related to $c$,
  then $a$ is similarly related to $c$}
}

\makeglossaries

\title{Algebra}
\author{Fengyang Wang}
\date{November 21, 2018}
\newcommand{\im}{i}

\usepackage{amsthm}
\newtheorem{bigidea}{Big Idea}

\begin{document}

\maketitle

\tableofcontents

\chapter{Modelling with Mathemtical Objects}

What is a mathematical object? We are familiar with everyday objects: things we can see,
feel, or otherwise interact with. Mathematical objects are like those, but more abstract.
For instance, a number is a mathematical object. In the real world, a mathematical object
might correspond to a quantity, a location, a process, or any number of other things. A
particular object might be known, or it might not be known. In mathematics, we want to use
the simplest possible objects to model the real problem we are trying to solve.

When we make a mathematical computation, or a mathematical argument, we often need to refer
to mathematical objects, sometimes known and sometimes not. While these are not the only
ways to refer to mathematical objects, three common ways are:

\begin{itemize}
  \item A ``literal'': for example, just a number, like \(123\).
  \item A ``variable'': a letter that we understand the meaning of, but we may or may not
  know the value of, like \(x\).
  \item An ``expression'': a combination of literals, variables, and operations, like \(x -
  123\).
\end{itemize}

For instance, a real world problem is as follows:

\begin{problem}{More Lumber Is Required}
  Yahui and Zhen want to build a wooden treehouse. \(100\) planks are required. Yahui has
  \(33\) in her shed, and Zhen has \(25\) in his shed. How many additional planks must they
  buy to complete the treehouse?

  \begin{solution}
    We can use whole numbers to model this problem. Let \(y\) be the number of planks that
    Yahui has, \(z\) be the number of planks that Zhen has, \(r\) be the total number of
    planks required, and \(x\) be the number of additional planks they must buy. The
    equation is: \[
      y + z + x = r
    \]

    This equation is the \emph{governing equation} of our \emph{model}. We are given the
    values of \(y\), \(z\), and \(r\), and we're asked to find the value of \(x\). (This is
    not always possible!) To do so, we can substitute the known values: \[
      33 + 25 + x = 100
    \]

    To find the unknown value we must add to \(33 + 25 = 58\) to arrive at \(100\), we can
    use subtraction. That is, \[
      x = 100 - 58 = \Ans{42}
    \]
  \end{solution}
\end{problem}

In this example, all of the variables we used represent specific mathematical objects. Three
of them were immediately given to us in the question. The other, \(x\), still represented a
specific mathematical object, but we had to figure it out.

It is not always the case that variables represent specific mathematical objects ---
sometimes, we can attach a \emph{quantifier} to a variable, to say that a statement is true
of all mathematical objects of a certain type at once.

\begin{problem}{Properties of Whole Number Addition}
  Give a concrete example for each of the following properties of addition of whole numbers:

  \begin{enumerate}
    \item For all integers \(a\), \(a + 0 = a\).
    \item For all integers \(a\) and \(b\), \(a + b = b + a\).
    \item For all integers \(a\), \(b\) and \(c\), \((a + b) + c = a + (b + c)\).
  \end{enumerate}

  \begin{solution}
    Solutions may vary.

    \begin{enumerate}
      \item \(2 + 0 = 2\)
      \item \(3 + 7 = 10 = 7 + 3\)
      \item \((1 + 2) + 3 = 3 + 3 = 6 = 1 + 5 = 1 + (2 + 3)\)
    \end{enumerate}
  \end{solution}
\end{problem}

Note that these concrete examples are applications of, not justifications for, the
properties in question. It can be difficult to give a formal justification for true
properties that involve quantifiers. However, if a statement is false, it is often much
easier: we can simply give a single concrete example which does not satisfy the statement.

\begin{problem}{Counterexamples}
  Give a counterexample for each of the following incorrect statements about whole numbers:

  \begin{enumerate}
    \item For all integers \(a\), \(a + a > a\).
    \item For all integers \(a\) and \(b\), \(a - b = b - a\).
    \item For all integers \(a\), \(b\) and \(c\), \(a + b = a + c\).
  \end{enumerate}

  \begin{solution}
    Solutions may vary.

    \begin{enumerate}
      \item Take \(a = 0\). Then \(a + a = 0 + 0 = 0 \not > 0 = a\).
      \item Take \(a = 1\) and \(b = 0\). Then \(a - b = 1 - 0 = 1 \ne -1 = 0 - 1 = b - a\).
      \item Take \(a = 0\), \(b = 0\), and \(c = 1\). Then \(a + b = 0 + 0 = 0 \ne 1 = 0 + 1
      = a + c\).
    \end{enumerate}
  \end{solution}
\end{problem}

Note that even though each of the statements is false, they all have certain cases where
they do hold. In the future, we will see techniques to find out exactly which cases the
statement is true in.

\section{Pairs}

An \gls{ordered pair} is two things written in an order. For example, \((3, 4)\) is an
ordered pair of numbers. Ordered pairs frequently represent a single concept that is made of
two components --- though keep in mind that these components are not always written as in
this example.

A simple ordered pair like \((3, 4)\) does not itself have much meaning, aside from being a
collection of two numbers. However, we may assign an interpretation to particular ordered
pairs to give them a meaning.

\section{Sets and Variables}

A set is an unordered collection of mathematical objects. For our purposes, we will use sets
as a convenient notation to describe the concept of ``one of these kinds of things''.

A variable is a letter that represents a mathematical object whose value may be unknown. We
say ``may be unknown'' because it is possible we do know the value of a variable. For
instance, if I write \(x := 1\), this means that I define the variable \(x\) to refer to the
number \(1\). But I might also say ``Let \(x\) be an integer (whole number).''; here, we do
not know the exact value of \(x\), but we have a constraint on it: it must be a whole
number.

We can express certain kinds of constraint with set-membership notation, as in
Constraint~\ref{mmo:setmember}, which states that the value of \(x\) must be \(1\) or \(4\)
or \(100\):
\begin{equation}
  x \in \{1, 4, 100\}
  \label{mmo:setmember}
\end{equation}

How would we express the idea that \(x\) is an integer with this notation? We obviously
cannot list out all the integers, since there are infinitely many. Instead we will adopt a
notation for an infinite set of all whole numbers: \(\mathbf{Z}\) (a boldface Z). The reason
for the choice of the letter Z comes from the German word Zahlen, which means ``number''.
Thus we can express the constraint ``\(x\) is an integer'' using the notation of
Constraint~\ref{mmo:isinteger}.
\begin{equation}
  x \in \mathbf{Z}
  \label{mmo:isinteger}
\end{equation}

Another kind of constraint we often see is an equation. A series of constraints is seen in
Constraints~\ref{mmo:equation1} and \ref{mmo:equation2}, where we are given that \(x\) is a
real number (a positive or negative number that can be a fraction or can even be an
irrational number), and further that \(x^2 = 4\). This series of constraints is actually
equivalent to \(x \in \{-2, 2\}\), since these are the only two real numbers whose square
is \(4\).
\begin{align}
  x &\in \mathbf{R}
  \label{mmo:equation1}
  \\
  x^2 &= 4
  \label{mmo:equation2}
\end{align}

We have notation for some important sets that we see frequently:

\begin{itemize}
  \item \(\mathbf{N} = \{0, 1, 2, \dots\}\) is the set of natural numbers.
  \item \(\mathbf{Z} = \{\dots, -2, 1, 0, 1, 2, \dots\}\) is the set of integers (whole
  numbers).
  \item \(\mathbf{Q} = \operatorname{Quot}(\mathbf{Z})\) is the set of rational numbers
  (fractions), which we will discuss in a later section.
  \item \(\mathbf{R} = \overline{\mathbf{Q}}\) is the set of real numbers, which we will
  discuss in a later section.
\end{itemize}

\section{Two-Dimensional Vector Spaces}

Another interpretation of ordered pairs is as vectors in a two-dimensional plane. The
components of the vector \([a, b]\) represent the displacement in two directions. For
example, the first component might represent displacement to the right, and the second
component displacement toward top of the page. (Such an interpretation is called a
\emph{vector space}, and the choices of directions are collectively called a \emph{basis}.)

We have notation for the set of two-dimensional vectors where both components are real
numbers: \(\mathbf{R}^2\). The superscript \(^2\) denotes that the vector space is two
dimensional, i.e. has two components.

\chapter{Rational Numbers}

\section{Fractions}

\Glspl{fraction} are a common example of ordered pairs with an assigned interpretation. The
fraction \(\frac{a}{b}\) is itself an ordered pair \((a, b)\), with the first element of
this ordered pair representing the number of fractional pieces, and the second element
representing the size of a whole relative to a single fractional piece.

\begin{problem}{Fractions}
 Compute each of the following:

 \[
         \frac{3}{8} \times \frac{2}{7} = \frac{6}{56} =
         \Ans{\displaystyle\frac{3}{28}}
        \]
 \[
  \frac{5}{9} \times \frac{2}{5} = \frac{10}{45} =
  \Ans{\displaystyle\frac{2}{9}}
  \]
\end{problem}

It happens to be the case with fractions that distinct ordered pairs might represent the
same quantity. For instance, \(\frac{3}{6}\) and \(\frac{7}{14}\) are different pairs of
numbers, but they represent the same fraction: one half. All the fractions that represent
the same particular quantity form a so-called \emph{equivalence class}. All numbers that can
be formed from fractions of integers are called \glspl{rational number}.

How can we decide whether two fractions represent the same quantity? That is, suppose that
\(\frac{a}{b}\) and \(\frac{c}{d}\) are rational numbers. Are they equal? In the case where
the denominator is the same, this is easy to answer: just compare the numerators. Rational
numbers with the same denominator are equal if and only if the numerators are equal.

If the denominators are different, one technique is to rewrite the fractions with a common
denominator. We see that \(\frac{a}{b} = \frac{a\times d}{b\times d}\) and \(\frac{c}{d} =
\frac{c\times b}{b\times d}\). Now we can compare the numerators. Thus, in general, we
obtain the result Equation~\ref{rat:fractionequalityalgorithm}. Visually, we are multiplying
the top-left with the bottom-right, and the top-right with the bottom-left. This makes a
cross shape, so one way to remember this technique is that it is often called
``cross-multiplication''.

\begin{equation}
  \frac{a}{b} = \frac{c}{d} \iff a\times d = b\times c
  \label{rat:fractionequalityalgorithm}
\end{equation}

Here is a fun example of fraction multiplication:

\begin{problem}{A Telescoping Product}
  Find the product:
  \[
    \frac{1}{2} \times \frac{2}{3} \times \frac{3}{4} \times
    \dots \times \frac{99}{100}
   \]

  \begin{solution}
    Each fraction in this product, except for the last one, has a numerator which is the
    same as the denominator of the following fraction. These will cancel out if we multiply
    the fractions. For instance, \(\frac{1}{2} \times \frac{2}{3} = \frac{1\times 2}{2
    \times 3}\), and we can divide \(2\) from both the numerator and the denominator to get
    \(\frac{1}{3}\).

    In this manner, all the numbers except for the first \(1\) in the numerator and the last
    \(100\) in the denominator will get cancelled out. So we are left with
    \Ans{\displaystyle \frac{1}{100}}.
  \end{solution}
\end{problem}

\chapter{Real Numbers}

\section{From Natural To Real}

Various forms of numbers arise from the desire to solve algebraic equations. But
contrary to popular belief, this desire was not to be able to solve equations
that have no solution, but rather to solve equations that do have a solution,
but whose solution can be arrived at easier by introducing a new kind of number.

The first kinds of numbers discovered were the (positive) natural numbers: \[ 1,
2, 3, 4, 5, 6, 7, 8, 9, 10, \dots \]

There are many positive natural numbers, and there are many things that we can
do with them. For example, we can add any two positive natural numbers to obtain
another. It does not matter how big the numbers become: their sum is always
still a natural number.

We may also multiply two positive natural numbers to obtain another. Again, it
does not matter how big or small the numbers are: multiplication is something
we can do with \emph{any} positive natural numbers.

Positive natural numbers are great for representing ``things''. In particular,
they're great for representing ``some things''. But they fall short when we want
to represent ``no things'' (or ``nothing''). Even worse, they fall apart
completely when we want to represent \emph{less} than nothing. But why would we
ever want that?

The desire for a more general kind of number comes from the desire to represent
\emph{change}. If yesterday there were $20$ students in attendance, and today
there are $24$ students in attendance, then we can say $4$ more people attended
today than did yesterday. But what if tomorrow there will again be $24$ students
in attendance? Then how many more students will have attended? Even worse, what
if the day after tomorrow, there will be $20$ students in attendance again. Then
what is the change in the number of students?

It is not that these problems cannot be solved with positive natural numbers. In
fact, they certainly can. The number of students could increase by $4$, and we
have a number for that. The number of students could also remain unchanged. We
don't need a number to express lack of change. Finally, the number of students
could \emph{decrease} by $4$. We already have a number for that.

So to express a change in a quantity, we must convey several pieces of information: firstly,
whether there is a change at all; secondly, in what direction is the change; and thirdly, by
how much the quantity is changed. This seems quite complicated. It would be simpler to
introduce a new kind of quantity that could represent these changes more conveniently and
more compactly. This, of course, is the motivation for negative numbers.

\begin{bigidea}[Negative numbers]
  Negative numbers do not ``exist'' in the real world. We cannot have a negative number of
  people in a class. But they provide an easy way to talk about \textbf{change}: if positive
  numbers represent \textbf{increase}, then negative numbers represent \textbf{decrease}.
\end{bigidea}

The introduction of \(0\) and negative numbers allows us to expand our set of mathematical
objects to the \textbf{integers}, \(\dots, -3, -2, -1, 0, 1, 2, 3, \dots\). The integers are
very well-behaved with respect to two fundamental operations: addition and multiplication.

TK (talk about division, rationals)

TK (talk about limits, reals)

\section{Exponents}

Recall that a positive exponent represents repeated multiplication, much like how a positive
multiplier represents repeated addition. We can express this rule recursively using the
following identity: \[
  x^{n+1} = xx^n
\] which says that if you increase the exponent by \(1\) it is the same as multiply one more
copy of the base.

\begin{problem}{Discrete Exponents}
 Evaluate each expression. Write your answer as an integer in simplest form
 using the place value system.

 \begin{enumerate}[\hspace{.5cm}a.]
  \item \(2^4=\Ans{16}\)
  \item \(3^2=\Ans{9}\)
  \item \(10^6=\Ans{1000000}\)
 \end{enumerate}
\end{problem}

It is frequently useful to extend the system of exponents to non-positive numbers, which can
be done by applying the recursive rule in the other direction. Thus we can derive that \(x^0
= 1\) and that \(x^{-1} = \frac{1}{x}\) for all non-zero values of \(x\).

\begin{problem}{Negative and Zero Exponents}
 Evaluate each expression. Write your answer in simplest form as a fraction, or
 as an integer using the place value system.

 \begin{enumerate}[\hspace{.5cm}a.]
  \item \({-1}^{-1}=\Ans{-1}\)
  \item \(4^{-2}=\Ans{\frac{1}{16}}\)
  \item \({999}^0=\Ans{1}\)
  \item \({\left(\frac{-17}{4}\right)}^0=\Ans{1}\)
 \end{enumerate}
\end{problem}

A useful application of exponents is in shrinking large numbers to an more humanly
understandable format. Indeed, we have a poor conception of how large certain numbers are.
In science, it's common to see numbers way too large to count or way too small to visualize.
Scientists have developed notation using exponents to make comparing such numbers easier. In
scientific notation, a number \(x\) is written as \(y\times 10^n\), where \(y\) is a number
with exactly one non-zero decimal digit before the decimal point, and \(n\) is a (positive,
negative, or zero) exponent.

\begin{problem}{Scientific Notation}
 Express in scientific notation.

 \begin{enumerate}[\hspace{.5cm}a.]
  \item \(1234=\Ans{1.234\times 10^3}\)
  \item \(0.000987= \Ans{9.87\times 10^{-4}}\)
 \end{enumerate}
\end{problem}

With rational exponents, TK (we generalize a different law)

\begin{problem}{Fractions, Exponents \& Radicals}
 Evaluate each expression. Write your answer in simplest form as a fraction, or
 as an integer using the place value system.

 \begin{enumerate}[\hspace{.5cm}a.]
  \item \(4^{\frac{1}{2}}=\Ans{2}\)
  \item \(9^{\frac{3}{2}}=\Ans{27}\)
  \item \({\left(\frac{2}{3}\right)}^3=\Ans{\frac{8}{27}}\)
  \item \(\sqrt{\frac{16}{25}}=\Ans{\frac{4}{5}}\)
  \item \(\sqrt[4]{\frac{256}{81}}=\Ans{\frac{4}{3}}\)
 \end{enumerate}
\end{problem}

Of course, as we saw above (TK) some numbers are not rational

\begin{problem}{Irrational Numbers}
 Classify each number as rational or irrational.

 \begin{enumerate}[\hspace{.5cm}a.]
  \item $8.25$ \hfill \AnsT{Rational}~~Irrational
  \item $\sqrt{2}$ \hfill Rational~~\AnsT{Irrational}
  \item $\sqrt{9}$ \hfill \AnsT{Rational}~~Irrational
  \item $\pi$ \hfill Rational~~\AnsT{Irrational}
 \end{enumerate}
\end{problem}

\begin{problem}{Associativity}
 Call an operation $\blacksquare$ ``associative'' if we have for all $a$, $b$,
 and $c$: $(a \blacksquare b) \blacksquare c = a \blacksquare (b \blacksquare
 c)$.

 \begin{enumerate}[\hspace{.5cm}a.]
  \item \emph{Apropos} (with regard to) positive integers, is $+$ associative?
  (That is, is $(a+b)+c=a+(b+c)$ for all $a$, $b$ and $c$?) \AnsT{Yes}
  \item Apropos positive integers, is $\times$ associative? \AnsT{Yes}
  \item Let $\uparrow$ represent exponentiation; that is, $2\uparrow4=2^4=16$.
  Apropos positive integers, is $\uparrow$ associative? \AnsT{No}
 \end{enumerate}
\end{problem}

\begin{problem}{Commutativity}
 Call an operation $\blacksquare$ ``commutative'' if we have for all $a$ and
 $b$: $a \blacksquare b = b \blacksquare a$.

 \begin{enumerate}[\hspace{.5cm}a.]
  \item Apropos positive integers, is $+$ commutative? \AnsT{Yes}
  \item Apropos positive integers, is $\times$ commutative? \AnsT{Yes}
  \item Apropos positive integers, is $\uparrow$ commutative? \AnsT{No}
 \end{enumerate}
\end{problem}

\section{Sums}

Adding things is a very important part of mathematics. When we have a large
number of things to add, it helps to use algebra to simplify the problem. The
notation for finite sums is \[
 \sum_{k = 1}^n a_k
\] where $a_1, a_2, \dots, a_n$ represent some numbers. Such a sum is called a
series, but we will use this term with caution, as it is often used to denote
infinite sums, which we will not cover. The numbers themselves, $(a_1, a_2,
\dots a_n)$, form an $n$-tuple.

\subsection{Arithmetic Series}

The first kind of series is a called an arithmetic series, in which the
$n$-tuple satisfies the property that all consecutive pairwise differences are
equal. That is, if for all $1 \le k < n$, we have \[ a_{k+1} - a_k = c \] where
$c$ is a constant, then we say that $(a_1, a_2, \dots, a_n)$ is in arithmetic
progression.

The sum of an arithmetic series can be computed by using the method of averages.
The idea here is quite simple: in any arithmetic progression, the average term
is $\frac{a_1 + a_n}{2}$. There is a simple proof of this fact, but it is not
too hard to think of intuitively. A result fundamental to statistics tells us
that, letting $\mu$ represent the average of the terms, and $s$ their sum, \[
 s = n \mu
\]

Hence, the formula for the sum of an arithmetic series \begin{equation}
 \sum_{k=1}^n a_k = \frac{n}{2} (a_1 + a_n)
\end{equation}

\subsection{Geometric Series}

The second kind of series is called a geometric series, in which the $n$-tuple
satisfies the property that all consecutive pairwise ratios are equal. This is,
if for all $1 \le k < n$, we have \[ \frac{a_{k+1}}{a_k} = r \] where $r$ is
again a constant, then we say that $(a_1, a_2, \dots, a_n)$ is in geometric
progression.

The sum of a geometric series can be computed by polynomial multiplication.
First, let $\alpha = a_1$. Then we rewrite the series: \begin{align*}
 \sum_{k=1}^n a_n
 &= \sum_{k=1}^n \alpha r^{k-1} \\
 &= \alpha \sum_{k=1}^n r^{k-1} \\
 &= \alpha (1 + r + r^2 + r^3 + \dots + r^{n-1}) \\
 &= \alpha \frac{(1 + r + r^2 + r^3 + \dots + r^{n-1})(1-r)}{1-r} \\
 &= \alpha \frac{1-r^n}{1-r}
\end{align*} which leads us to the formula for the sum of a geometric series
with common ratio $r$,
\begin{equation}
 \sum_{k=1}^n a_n = a_1 \frac{1 - r^n}{1 - r}
\end{equation}

\chapter{Complex Numbers}

\begin{problem}{Polar Form}
  \begin{enumerate}[\hspace{.5cm}a.]
    \item \(1 = \exp\left(\blankB + \blankB\im\right)\)
    \item \(\sqrt{2}-\sqrt{2}\im = \blankB\exp\blankC\im\)
    \item \(\sqrt{2}\exp\frac{\im\pi}{4} = \blankB + \blankB\im\)
  \end{enumerate}
\end{problem}

\begin{problem}{Collinearity of Points}
  Let \(a, b, c\in\mathbf{C}\) represent points \(A, B, C\) in the 2D Euclidean
  plane, all distinct.

  \begin{enumerate}[\hspace{.5cm}a.]
    \item When are these three points collinear (that is, they lie on the same
    line)? Again, express your answer as a single equation involving complex
    numbers \(a, b, c, d\), and free parameter \(t\in\mathbf{R}\).
    \item Hence, derive that a condition for \(A, B, C\) collinear is \[
      \frac{c-a}{c-b} = \overline{\left(\frac{c-a}{c-b}\right)}
    \]
  \end{enumerate}
\end{problem}

\end{document}
