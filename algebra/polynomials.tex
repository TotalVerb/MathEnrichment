\documentclass[12pt,letterpaper]{article}

\newcommand{\IncludePath}{../include}
\usepackage{extsizes}
\usepackage{titling}

\usepackage{tikz}
\usetikzlibrary{shapes}

\usepackage{amssymb,amsmath,amsthm}
\usepackage{enumerate}
\usepackage[margin=0.8in]{geometry}
\usepackage{graphicx,ctable,booktabs}
\usepackage{fancyhdr}
\usepackage[utf8]{inputenc}
\usepackage{gensymb}

\makeatletter
\newenvironment{problem}{\@startsection
       {section}
       {1}
       {-.2em}
       {-3.5ex plus -1ex minus -.2ex}
       {2.3ex plus .2ex}
       {\pagebreak[3]
       \large\bf\noindent{Problem }
       }
       }
\makeatother

\pagestyle{fancy}
\lhead{\thetitle}
\chead{}
\rhead{\thepage}
\lfoot{\small\scshape Olympic Math}
\cfoot{}
\rfoot{}
\renewcommand{\headrulewidth}{.3pt}
\renewcommand{\footrulewidth}{.3pt}
\setlength\voffset{-0.25in}
\setlength\textheight{648pt}
\setlength\headheight{15pt}

\newcommand{\blankA}{\underline{\hspace{1em}}}
\newcommand{\blankB}{\underline{\hspace{2em}}}
\newcommand{\blankC}{\underline{\hspace{3em}}}
\newcommand{\blankD}{\underline{\hspace{4em}}}
\newcommand{\blankE}{\underline{\hspace{5em}}}
\newcommand{\blankF}{\underline{\hspace{6em}}}



% \answerstrue % turn on for answers
\defauthor

\title{Polynomials}
\date{November 30, 2019}

\begin{document}

\maketitle

\thispagestyle{empty}

\Switch{}{For answers see class website: \url{https://grcs.uwseminars.com/}}

\begin{problem}{Degrees, Leading Coefficients, and Constant Terms}
  \textbf{Recall}: The degree of a polynomial is the highest exponent of $x$ with a non-zero
  coefficient. The leading coefficient is the coefficient of the highest exponent term. The
  constant term is the coefficient of the $x^0$ term, i.e. the monomial which does not
  depend on $x$ (hence, constant).

  \vspace{1em}
  \noindent Find the degree, leading coefficient, and constant term for each of the
  following polynomials.

  \vspace{1em}
  \noindent
  Hint: The polynomials in this homework are not written in the same order as we did in
  class. Don't let that confuse you! The leading coefficient is always in the term with the
  highest exponent, regardless of what order it is written in.

  \begin{enumerate}
    \item \(-12 + x + x^2\)
    \hfill Degree: \SAC{2}~~Leading coefficient: \SAC{1}~~Constant term: \SAC{-12}
    \item \(-x^2 + x^{11} - x^{17} + x^{26}\)
    \hfill Degree: \SAC{26}~~Leading coefficient: \SAC{1}~~Constant term: \SAC{0}
  \end{enumerate}
\end{problem}

\begin{problem}{Long Division}
  Use long division to find the following quotients.

  \begin{enumerate}
    \item \(\displaystyle\frac{-12 + x + x^2}{4 + x} = \SAF{-3 + x}\)
    \item \(\displaystyle\frac{-x^2 + x^{11} - x^{17} + x^{26}}{-1 + x^9} = \SAF{x^2 +
    x^17}\) \end{enumerate}
\end{problem}

\begin{problem}{Factorization in \(\mathbf{Q}[x]\) with Integer Coefficients}
  Define \(\mathbf{Q}[x]\) to be the set of polynomials with rational coefficients. In
  \(\mathbf{Q}[x]\), fully factor the following. Use the rational root theorem, which states
  that any rational root $\frac{m}{n}$ (in lowest terms) must have $m$ be a factor the
  constant term and $n$ be a factor of the leading coefficient. You may use the fact that
  any degree $2$ or $3$ rational polynomial with no rational root is irreducible.

  \begin{enumerate}
    \item \(16 - x^2 = \SAF{(4 + x)(-4 + x)}\)
    \item \(-x + 2x^2 = \SAF{x(-1 + 2x)}\)
    \item \(2 + 5x + 3x^2 = \SAF{(2 + 3x)(1 + x)}\)
    \item \(1 + 6x + 12x^2 + 8x^3 = \SAF{(1 + 4x + 4x^2)(1 + 2x)}\)
    \item \(1 - 4x + 6x^2 - 4x^3 + x^4 = \SAF{(-1 + x)^4}\)
    \item \(1 - x^4 = \SAF{(1 - x)(1 + x)(1 + x^2)}\)
  \end{enumerate}
\end{problem}

\begin{problem}{Factorization in \(\mathbf{Q}[x]\)}
  Define \(\mathbf{Q}[x]\) to be the set of polynomials with rational coefficients. In
  \(\mathbf{Q}[x]\), fully factor the following.

  \vspace{1em}
  \noindent
  Hint: The rational root theorem requires all the coefficients to be integers. To get the
  polynomials to have integer coefficients, first factor out an appropriate fraction, e.g.
  $\frac{1}{2}x + \frac{3}{2} = \frac{1}{2}(x + 3)$. Then use the rational root theorem
  ignoring the fraction you factored out. Any root of $p(x)$ is still a root of $q p(x)$,
  where $q$ is any non-zero rational number!

  \begin{enumerate}
    \item \(\frac{1}{4} + x + x^2 = \SAF{\frac{1}{4}{(1 + 2x)}^2}\)
    \item \(\frac{4}{9} - \frac{4}{9}x + \frac{1}{9}x^2 = \SAF{\frac{1}{9}{(-2 + x)}^2}\)
  \end{enumerate}
\end{problem}

\begin{problem}{Finite Geometric Series}
  \begin{enumerate}
    \item Expand \(\frac{1-x^2}{1-x}\) as a polynomial. Hint: Use long division.
    \Switch{\Ans{\frac{1-x^2}{1-x} = 1 + x}}{}
    \item Expand \(\frac{1-x^5}{1-x}\) as a polynomial.
    \Switch{\Ans{\frac{1-x^5}{1-x} = 1 + x + x^2 + x^3 + x^4}}{}
    \item Define \[
      \sum_{k=0}^n x^k := 1 + x + \dots + x^n
    \] (this is just a shorthand notation; you can read it as the ``sum with $k$ ranging
    from $0$ to $n$ of $x^k$). Find a fraction of the form \(\frac{p(x)}{q(x)}\), where
    \(p\) and \(q\) are polynomials, such that \[
      \frac{p(x)}{q(x)} = \sum_{k=0}^n x^k
    \]
    \Switch{\Ans{\frac{1 - x^{k+1}}{1 - x}}}{}
    \item Calculate \[
      \sum_{k=0}^{20} 2^k := 1 + 2 + 2^2 + \dots + 2^{20} = \SAF{2097151}
    \]

    You may use the fact that $2^{21} = 2097152$.
  \end{enumerate}
\end{problem}

\end{document}
