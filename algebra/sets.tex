\documentclass[12pt,letterpaper]{article}

\newcommand{\IncludePath}{../include}
\usepackage{extsizes}
\usepackage{titling}

\usepackage{amssymb,amsmath,amsthm}
\usepackage{enumerate}
\usepackage[margin=1in]{geometry}
\usepackage{graphicx,ctable,booktabs}
\usepackage{fancyhdr}
\usepackage[utf8]{inputenc}

\makeatletter
\newenvironment{problem}{\@startsection
       {section}
       {1}
       {-.2em}
       {-3.5ex plus -1ex minus -.2ex}
       {2.3ex plus .2ex}
       {\pagebreak[3]
       \large\bf\noindent{Problem }
       }
       }
\makeatother

\pagestyle{fancy}
\lhead{\thetitle}
\chead{}
\rhead{\thepage}
\lfoot{\small\scshape Grade 4 Olympic Math}
\cfoot{}
\rfoot{}
\renewcommand{\headrulewidth}{.3pt}
\renewcommand{\footrulewidth}{.3pt}
\setlength\voffset{-0.25in}
\setlength\textheight{648pt}
\setlength\headheight{15pt}


% \answerstrue % turn on for answers
\defauthor

\title{Sets}
\date{November 2, 2019}

\begin{document}

\maketitle

\thispagestyle{empty}

\Switch{}{For answers see class website: \url{https://grcs.uwseminars.com/}}

\begin{problem}{Set Membership}
  Determine whether each statement is true or false. Remember that $\varnothing$ is the
  empty set, which has no members.

  \begin{multicols}{2}
  \begin{enumerate}[\hspace{.5cm}a.]
    \item \( 1 \in \{1, 2, 3\} \) \hfill \TFTrue
    \item \( 0 \in \varnothing \) \hfill \TFFalse
    \item \( \{1, 2\} \in \{\{2, 1\}, \{3, 4\}\} \) \hfill \TFTrue
    \item \( \{\varnothing\} \in \{\varnothing\} \) \hfill \TFFalse
    \item \( 0.5 \in \left\{\frac{1}{2}\right\} \) \hfill \TFTrue
    \item \( 1 \in \left\{\frac{1}{2}, \frac{3}{2}\right\} \) \hfill \TFFalse
  \end{enumerate}
  \end{multicols}
\end{problem}

\begin{problem}{Set Equality}
  Determine whether each statement is true or false. Remember that two sets are equal if
  they contain exactly the same members, regardless of order. Also remember that there is no
  concept of multiple membership in a set: \(\{0, 0\}\) is the same set as \(\{0\}\).

  \begin{multicols}{2}
  \begin{enumerate}[\hspace{.5cm}a.]
    \item \( \varnothing = \varnothing \) \hfill \TFTrue
    \item \( \varnothing = \{\varnothing\} \) \hfill \TFFalse
    \item \( \{1, 2\} = \{2, 1\} \) \hfill \TFTrue
    \item \( \{1, 2, 3\} = \{3, 1, 2\} \) \hfill \TFTrue
    \item \( \{1, 2\} = \{1, 2, 3\} \) \hfill \TFFalse
    \item \( \{1, 1\} = \{1\} \) \hfill \TFTrue
  \end{enumerate}
  \end{multicols}
\end{problem}

\begin{problem}{Subset}
  Determine whether each statement is true or false. Remember that $A \subseteq B$ if all
  members of $A$ are also members of $B$. $A \subsetneq B$ if $A \subseteq B$ but $A \ne B$,
  i.e. all members of $A$ are also members of $B$ but at least one member of $B$ is not a
  member of $A$.

  \begin{multicols}{2}
  \begin{enumerate}[\hspace{.5cm}a.]
    \item \( \varnothing \subseteq \{1, 2\} \) \hfill \TFTrue
    \item \( \{1, 2\} \subseteq \{1, 2\} \) \hfill \TFTrue
    \item \( \{1, 2\} \subseteq \varnothing \) \hfill \TFFalse
    \item \( \{1, 2\} \subseteq \{\{1, 2\}\} \) \hfill \TFFalse
    \item \( \{1, 2\} \subsetneq \{1, 2, 3\} \) \hfill \TFTrue
    \item \( \{1, 2\} \subsetneq \{2, 3\} \) \hfill \TFFalse
  \end{enumerate}
  \end{multicols}
\end{problem}

\begin{problem}{Set Unions and Intersections}
  List all the members of the following sets using set notation. The first two are done for
  you.

  \begin{enumerate}[\hspace{.5cm}a.]
    \item \( \{0, 1\} \cup \{1, 2\} = \{0, 1, 2\} \)
    \item \( \{0, 1\} \cap \{1, 2\} = \{1\} \)
    \item \( \{0, 1, 2\} \cup \{2, 1, 0\} = \{\SAF{0, 1, 2}\} \)
    \item \( \{0, 1, 2, 3, 4\} \cap \{0, 1, 4, 9, 16\} = \{\SAF{0, 1, 4}\} \)
  \end{enumerate}
\end{problem}

\begin{problem}{True}
  Write a brief explanation for why each of the following statements is true.

  \begin{enumerate}
    \item For all sets $A$ and $B$, $A \subseteq A \cup B$.
    \Switch{\AnsT{By definition, every member of $A$ is in the union}}{}
    \item For all sets $A$ and $B$, $A \cap B \subseteq A$.
    \Switch{\AnsT{By definition, every member of $A \cap B$ is in $A$}}{}
    \item For all sets $A$, $B$, and $C$, if $A \subseteq B$ and $B \subseteq C$ then $A
    \subseteq C$.
    \Switch{\newline\AnsT{If all $A$s are $B$s and all $B$s are $C$s, then all $A$s are $C$s.}}{}
    \item For all sets $A$ and $B$, $A \cap B \subseteq A \cup B$.
    \Switch{\AnsT{Combine parts a, b, and c}}{}
  \end{enumerate}
\end{problem}

\begin{problem}{False}
  Find a counterexample for each of the false statements below.

  \begin{enumerate}
    \item For all sets $A$ and $B$, either $A \subseteq B$ or $B \subseteq A$ (or both).
    \Switch{\Ans{A = \{1\}}, \Ans{B = \{2\}}}{}
    \item For all sets $A$, $\varnothing \subsetneq A$. \Switch{\Ans{A = \varnothing}}{}
  \end{enumerate}
\end{problem}

\begin{problem}{Infinite Sets}
  Recall that:

  \begin{itemize}
    \item $\mathbf{N} = \{0, 1, 2, \dots\}$ is the set of natural numbers.
    \item $\mathbf{Z} = \{\dots, -2, -1, 0, 1, 2, \dots\}$ is the set of integers.
    \item $\mathbf{Q}$ is the set of rational numbers (fractions).
    \item $\mathbf{R}$ is the set of real numbers.
  \end{itemize}

  \noindent We saw in class that $\mathbf{N} \subseteq \mathbf{Z} \subseteq \mathbf{Q}
  \subseteq \mathbf{R}$. Demonstrate that each of these inclusions is proper, i.e.
  $\mathbf{N} \subsetneq \mathbf{Z} \subsetneq \mathbf{Q} \subsetneq \mathbf{R}$.

  \Switch{\textbf{Solution.} For $\mathbf{N} \subsetneq \mathbf{Z}$, an example is $-1$. For
  $\mathbf{Z} \subsetneq \mathbf{Q}$, an example is $\frac{1}{2}$. For $\mathbf{Q}
  \subsetneq \mathbf{R}$, an example is $\sqrt{2}$.}{}
\end{problem}

\end{document}
