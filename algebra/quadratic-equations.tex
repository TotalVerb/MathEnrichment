\documentclass[12pt,letterpaper]{article}

\newcommand{\IncludePath}{../include}
\usepackage{extsizes}
\usepackage{titling}

\usepackage{amssymb,amsmath,amsthm}
\usepackage{enumerate}
\usepackage[margin=1in]{geometry}
\usepackage{graphicx,ctable,booktabs}
\usepackage{fancyhdr}
\usepackage[utf8]{inputenc}

\makeatletter
\newenvironment{problem}{\@startsection
       {section}
       {1}
       {-.2em}
       {-3.5ex plus -1ex minus -.2ex}
       {2.3ex plus .2ex}
       {\pagebreak[3]
       \large\bf\noindent{Problem }
       }
       }
\makeatother

\pagestyle{fancy}
\lhead{\thetitle}
\chead{}
\rhead{\thepage}
\lfoot{\small\scshape Grade 4 Olympic Math}
\cfoot{}
\rfoot{}
\renewcommand{\headrulewidth}{.3pt}
\renewcommand{\footrulewidth}{.3pt}
\setlength\voffset{-0.25in}
\setlength\textheight{648pt}
\setlength\headheight{15pt}


% \answerstrue % turn on for answers
\defauthor

\title{Quadratic Equations}
\date{November 16, 2019}

\begin{document}

\maketitle

\thispagestyle{empty}

\Switch{}{For answers see class website: \url{https://grcs.uwseminars.com/}}

\begin{problem}{Quadratic Equations with a Unique Solution}
  Determine the unique possible real value of \(x\). Why do these equations have only one
  real solution?

  \begin{enumerate}[\hspace{.5cm}a.]
    \item \( x^2 - 2x + 1 = 0 \)
    (Hint: $x^2 - 2x + 1 = (x-1)^2$)
    \hfill \(x = \SAB{1}\)
    \item \( x^2 + \frac{2}{3}x + \frac{1}{9} = 0 \)
    (Hint: $x^2 + \frac{2}{3}x + \frac{1}{9} = \left(x+\frac{1}{3}\right)^2$)
    \hfill \(x = \SAB{-\frac{1}{3}}\)
  \end{enumerate}
\end{problem}

\begin{problem}{Quadratic Equations with No Real Solution}
  Explain why the following quadratic equations do not have any real solutions.

  \begin{enumerate}[\hspace{.5cm}a.]
    \item \( x^2 = -1 \)
    (Hint: Is the square of a real number ever negative?)
    \Switch{\textbf{Solution:} The square of a real number is never negative. Therefore
    this equation has no solution.}{}
    \item \( x^2 - \frac{5}{4}x + \frac{367}{448} = 0 \)
    (Hint: $x^2 - \frac{5}{4}x + \frac{367}{448} = {\left(x-\frac{5}{8}\right)}^2 +
    \frac{3}{7}$)
    \Switch{\textbf{Solution:} The hint tells us that this equation, if it had a real
    solution $x$, would mean that ${\left(x-\frac{5}{8}\right)}^2 = -\frac{3}{7}$ is
    negative, but that is not possible!}{}
  \end{enumerate}
\end{problem}

\begin{problem}{A Tale of Two Cities}
  Phitsanulok and Sukhothhai are two small cities in Thailand. Bangkok is the capital of
  Thailand.  Assume, for simplicity, that the Earth is flat for this question. It's not, of
  course! But when we are at relatively small scales, this assumption is not so terrible.

  \begin{enumerate}
    \item Phitsanulok is \SI{400}{\kilo\meter} north of Bangkok. Draw this information on
    the map below, where Bangkok is labelled for you already. We will say that Bangkok is at
    $(0, 0)$ and Phitsanulok is at $(0, 400)$.
    \item Sukhothhai is \SI{430}{\kilo\meter} away from Bangkok, but we do not know the
    direction. Draw a circle around Bangkok representing the possible locations of
    Sukhothhai, based on this information.
    \item The equation of a circle centred at $(0, 0)$ is $x^2 + y^2 = r^2$, where $r$ is
    the radius. If Sukhothhai is at $(x, y)$, write down an equation that is satisfied by
    Sukhothhai: $x^2 + y^2 = \SAF{176400}$. (Hint: $430^2 = 184900$)
    \item Sukhothhai is \SI{150}{\kilo\meter} away from Phitsanulok, but we do not know the
    direction. Draw a circle around Phitsanulok representing the possible locations of
    Sukhothhai, based on this information.
    \item The equation of a circle centred at $(0, 400)$ is $x^2 + {(y - 400)}^2 = r^2$, where
    $r$ is the radius. Write down another equation that is satisfied by Sukhothhai: $x^2 +
    {(y - 400)}^2 = \SAF{22500}$. (Hint: $150^2 = 22500$)
    \item We have narrowed down the possible locations of Sukhothhai! At how many points do
    the circles intersect? \SAC{2}
    \item Do you think this is related to the fact that the equations we wrote are
    quadratic? \hfill \TFTrue
    \item To solve this system of equations, we can subtract the first equation from the
    second. This will make the $x^2$ term disappear (why?). We are left with $y^2 - {(y -
    400)}^2 = 800y - 160000 = 162400$. This is actually a linear equation for $y$, so it
    should have only one solution. That is, wherever Sukhothhai is, we know exactly how far
    north or south it is, even if we do not know for sure how far to the east or west it is.
    Does this match your picture? \hfill \TFTrue
    \item We can then substitute the value for $y$ from the above linear equation, which is
    $y = 403$, into one of the quadratic equations. It does not matter which one (why?). We
    will get, after some simplification, the quadratic equation $x^2 = 22491$. How many real
    solutions $x$ are there? How can we solve this equation? (The solutions are not
    integers.)
    \Switch{\textbf{Solution:} There are two real solutions. We can solve this equation by
    taking negative and positive square roots.}{}
  \end{enumerate}

  \begin{center}
    \begin{tikzpicture}
      \draw[help lines, color=black, dashed] (-5.9,-5.9) grid (5.9,5.9);
      \draw[->,ultra thick] (-6,0)--(6,0) node[right]{$x$ (east)};
      \draw[->,ultra thick] (0,-6)--(0,6) node[above]{$y$ (north)};
      \draw[ultra thick] (1, -0.1) -- (1, 0.1);
      \node at (1, 0) [above right]{\SI{100}{\kilo\meter}};
      \draw[ultra thick] (-0.1, 1) -- (0.1, 1);
      \node at (0, 1) [above right]{\SI{100}{\kilo\meter}};
      \draw[fill=black] (0,0) circle[radius=4pt] node[below right] {Bangkok};

      \Switch{
      \draw[fill=black] (0,4) circle[radius=4pt] node[below right] {Phitsanulok};
      \draw (0,0) circle[radius=4.3];
      \draw (0,4) circle[radius=1.5];
      }{}
    \end{tikzpicture}
  \end{center}
\end{problem}

\end{document}
