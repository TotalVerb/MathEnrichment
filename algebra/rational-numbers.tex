\documentclass[12pt,letterpaper]{article}

\newcommand{\IncludePath}{../include}
\usepackage{extsizes}
\usepackage{titling}

\usepackage{amssymb,amsmath,amsthm}
\usepackage{enumerate}
\usepackage[margin=1in]{geometry}
\usepackage{graphicx,ctable,booktabs}
\usepackage{fancyhdr}
\usepackage[utf8]{inputenc}

\makeatletter
\newenvironment{problem}{\@startsection
       {section}
       {1}
       {-.2em}
       {-3.5ex plus -1ex minus -.2ex}
       {2.3ex plus .2ex}
       {\pagebreak[3]
       \large\bf\noindent{Problem }
       }
       }
\makeatother

\pagestyle{fancy}
\lhead{\thetitle}
\chead{}
\rhead{\thepage}
\lfoot{\small\scshape Grade 4 Olympic Math}
\cfoot{}
\rfoot{}
\renewcommand{\headrulewidth}{.3pt}
\renewcommand{\footrulewidth}{.3pt}
\setlength\voffset{-0.25in}
\setlength\textheight{648pt}
\setlength\headheight{15pt}


% \answerstrue % turn on for answers
\defauthor

\title{Rational Numbers}
\date{September 28, 2019}

\begin{document}

\maketitle

\thispagestyle{empty}

\begin{problem}{Equivalent Fractions}
  Determine, using cross-multiplication or otherwise, whether the fractions are equivalent.

  \begin{multicols}{2}
  \begin{enumerate}[\hspace{.5cm}a.]
    \item \( \displaystyle \frac{2}{4} = \frac{1}{2} \)     \hfill \TFTrue
    \item \( \displaystyle \frac{3}{-12} = \frac{-1}{4} \)  \hfill \TFTrue
    \item \( \displaystyle \frac{8}{7} = \frac{9}{8} \)     \hfill \TFTrue
    \item \( \displaystyle \frac{-1}{-2} = \frac{-5}{10} \) \hfill \TFTrue
  \end{enumerate}
  \end{multicols}
\end{problem}

\begin{problem}{Simplify Fractions}
  Find the equivalent fraction with the smallest (positive) denominator. In other words,
  simplify the fraction. If the denominator is \(1\), you may leave it as a fraction, even
  though it is equal to an integer.

  \begin{multicols}{2}
  \begin{enumerate}[\hspace{.5cm}a.]
    \item \( \displaystyle \frac{2}{4} = \fracans{1}{2} \)
    \item \( \displaystyle \frac{3}{12} = \fracans{1}{4} \)
    \item \( \displaystyle \frac{10}{-15} = \fracans{-2}{3} \)
    \item \( \displaystyle \frac{-3}{-3} = \fracans{1}{1} \)
  \end{enumerate}
  \end{multicols}
\end{problem}

\begin{problem}{Fraction Operations}
  Compute the result of each expression and write it as a fraction with the smallest
  possible positive denominator. If the denominator is \(1\), you may leave it as a
  fraction, even though it is equal to an integer.

  \begin{multicols}{2}
  \begin{enumerate}[\hspace{.5cm}a.]
  \item \( \displaystyle \frac{1}{2} + \frac{1}{2} = \fracans{1}{1} \)
  \item \( \displaystyle \frac{1}{2} + \frac{-1}{3} = \fracans{1}{6} \)
  \item \( \displaystyle \frac{1}{-2} + \frac{-3}{-4} = \fracans{1}{4} \)
  \item \( \displaystyle \frac{1}{2} \times \frac{2}{3} = \fracans{1}{3} \)
  \item \( \displaystyle \frac{3}{8} \times \frac{5}{9} = \fracans{5}{24} \)
  \item \( \displaystyle \frac{1}{2} \times \frac{1}{2} \times \frac{1}{2} \times
  \frac{1}{2} = \fracans{1}{16} \)
  \end{enumerate}
  \end{multicols}
\end{problem}

\begin{problem}{True or False}
  Recall that rational numbers include all fractions of integers, including those with denominator \(1\), so all integers are rational numbers.

  \begin{enumerate}
    \item The sum of two rational numbers is always a rational number.
    \hfill \TFTrue
    \item The product of two rational numbers is always a rational number.
    \hfill \TFTrue
    \item There are multiple ways to write every rational number as a fraction.
    \hfill \TFTrue
    \item For all integers \(a, b, c, d\), if \(b \ne 0 \ne d\), \[
      \frac{a}{b} + \frac{c}{d} = \frac{ac}{bd}
    \] \hfill \TFFalse
  \end{enumerate}
\end{problem}

\begin{problem}{Pizza Confusion}
  Melek and Zahari each have a pizza, but they are not the same size. \(\frac{2}{5}\) of
  Melek's pizza has the same mass as \(\frac{3}{8}\) of Zahari's pizza. If Zahari's pizza is
  \(\SI{320}{\gram}\), then what is the mass of Melek's pizza? \hfill \SAC{\SI{300}{\gram}}
\end{problem}

\begin{problem}{Integral Root Theorem}
  The \textbf{integral root theorem} says that if \(x: \mathbf{Z}\) is an integer satisfying
  a \textbf{polynomial equation} constraint \[
    a_0 + a_1 x + \dots + a_{n-1} x^{n-1} + x^n = 0
  \] where \(n\) is a positive integer and each \(a_0, a_1, \dots, a_{n-1}\) is an integer,
  then \(x\) is a (positive or negative) factor of \(a_0\).  For example, if \(3 - 4x + x^2 =
  0\), the only possible solutions for \(x\) are the factors of \(3\): \(-3\), \(-1\),
  \(1\), and \(3\). We can check that \(1\) and \(3\) are solutions for \(x\) that satisfy
  the constraint, while \(-1\) and \(-3\) do not work.

  Using the integral root theorem, determine the integer solutions \(x\) to the polynomial
  equation: \[
    -5 - x + 5x^2 + x^3 = 0
  \]

  Answer: \(x \in \{\SAB{-5}, \SAB{-1}, \SAB{1}\}\).
\end{problem}

\begin{problem}{A Telescoping Sum}
 Find the sum:
 \[
  \frac{1}{1 \times 2} + \frac{1}{2 \times 3} + \frac{1}{3 \times 4}
  + \dots + \frac{1}{99 \times 100} = \fracans{99}{100}
 \]
\end{problem}

\end{document}
