\documentclass[12pt,letterpaper]{article}

\newcommand{\IncludePath}{../include}
\usepackage{extsizes}
\usepackage{titling}

\usepackage{amssymb,amsmath,amsthm}
\usepackage{enumerate}
\usepackage[margin=1in]{geometry}
\usepackage{graphicx,ctable,booktabs}
\usepackage{fancyhdr}
\usepackage[utf8]{inputenc}

\makeatletter
\newenvironment{problem}{\@startsection
       {section}
       {1}
       {-.2em}
       {-3.5ex plus -1ex minus -.2ex}
       {2.3ex plus .2ex}
       {\pagebreak[3]
       \large\bf\noindent{Problem }
       }
       }
\makeatother

\pagestyle{fancy}
\lhead{\thetitle}
\chead{}
\rhead{\thepage}
\lfoot{\small\scshape Grade 4 Olympic Math}
\cfoot{}
\rfoot{}
\renewcommand{\headrulewidth}{.3pt}
\renewcommand{\footrulewidth}{.3pt}
\setlength\voffset{-0.25in}
\setlength\textheight{648pt}
\setlength\headheight{15pt}


% \answerstrue % turn on for answers
\defauthor

\title{Linear Equations}
\date{October 5, 2019}

\begin{document}

\maketitle

\thispagestyle{empty}

\Switch{}{For answers see class website: \url{https://grcs.uwseminars.com/}}

\begin{problem}{Linear Equations with a Unique Solution}
  Using division, determine the unique possible value of \(x\). Express any fractions in
  simplest form.

  \begin{multicols}{2}
  \begin{enumerate}[\hspace{.5cm}a.]
    \item \( 2x = 6 \)     \hfill \(x = \SAB{3}\)
    \item \( -3x = 12 \)   \hfill \(x = \SAB{-4}\)
    \item \( \displaystyle \frac{1}{2}x = \frac{1}{4} \)
    \hfill \(x = \SAB{\frac{1}{2}}\)
    \item \( 5x = 3 \)     \hfill \(x = \SAB{\frac{3}{5}}\)
  \end{enumerate}
  \end{multicols}
\end{problem}

\begin{problem}{All Roads Lead To Rome}
  Khulan, Alessa, and Clara are on vacation in Italy. Read Khulan's journal. For each
  question, decide whether we have enough information to answer it. If we do, use a linear
  equation or otherwise to solve the problem.

  \textit{Rome is the capital of Italy. I arrived in Venice,~Italy and took the train to
  Rome. Alessa and Clara arrived in Naples,~Italy and also took the train to Rome. The train
  from Venice takes \(4\) times as long as the train from Naples. Clara told me she spent
  \(40\)€ on her train ticket. If you add up the amount of time that each of us spent on the
  train, we spent a total of \(6\) hours combined, which is not bad considering the
  distance! Naples is \SI{185}{\kilo\metre} away from Rome. For comparison, Paris and Rome
  are \SI{1100}{\kilo\meter}) apart. While in Rome, we tried a lot of coffee. Alessa spent
  twice as much on coffee as me, and Clara spent twice as much on coffee as Alessa! We spent
  a total of \(14\)€ on coffee.}

  \begin{enumerate}
    \item How much time does the Naples--Rome train take?
    \Switch{Yes. \(t + t + 4t = \SI{6}{\hour} \Rightarrow t = \SI{1}{\hour}\)}{}
    \item How much time does the Venice--Rome train take? (Did we have to write a new linear
    equation to solve this problem?)
    \Switch{Yes. We can just use \(4t = \SI{4}{\hour}\).}
    \item How much time does the Paris--Rome train take?
    \Switch{No. While we know the ratio of distances, we cannot assume that trains travel at
    the same speed.}{}
    \item How much did Alessa spend on coffee?
    \Switch{Yes. \(c + 2c + 4c = 14 \Rightarrow c = 2\), so Alessa spent \(2\)€ on coffee.}{}
    \item How much does a cup of coffee cost in Italy?
    \Switch{No. We do not know how many units of coffee any of them consumed.}{}
    \item How much money did Alessa spend on her train ticket?
    \Switch{No. While we know the ratio of times, we cannot assume that train ticket cost is
    only based on time.}{}
  \end{enumerate}
\end{problem}

\begin{problem}{Number of Solutions to a Linear Equation}
  Suppose that \(a\) and \(b\) are known rational numbers. Which of the below statements
  about the number of rational solutions to \(ax = b\) are true?

  \begin{enumerate}
    \item If \(a=0\), then \(ax = b\) always has no rational solutions.
    \hfill \TFFalse
    \item If \(a\ne 0\), then \(ax = b\) always has a unique rational solution.
    \hfill \TFTrue
    \item If \(b=0\), then \(ax = b\) always has multiple rational solutions.
    \hfill \TFFalse
    \item If \(a=b=0\), then \(ax = b\) always has multiple rational solutions.
    \hfill \TFTrue
    \item If \(a\ne0\) but \(b=0\), then \(ax = b\) always has no rational solutions.
    \hfill \TFFalse
    \item If \(a=0\) but \(b\ne0\), then \(ax = b\) always has no rational solutions.
    \hfill \TFTrue
  \end{enumerate}
\end{problem}

\begin{problem}{Extra Known Variables}
  Suppose that \(a\), \(b\), and \(c\) are known rational numbers, with \(a \ne b\). Find an
  expression that gives \(x\) in terms of these known rational numbers, if \(ax = bx + c\).
  Is the solution unique?
\end{problem}

\begin{problem}{A Homogeneous Linear System}
  Up until now, we have been focusing on a single unknown value, \(x\). Consider the
  following system of equations (that is, all the equations are true):

  \begin{align*}
    x + y + z &= 0 \\
    x + 2y + 3z &= 0 \\
  \end{align*}

  where \(x\), \(y\), \(z\) are unknown rational numbers.

  \Switch{\textbf{Answers may vary for this problem.}}{}

  \begin{enumerate}
    \item Find one solution. \hfill \(x_1 = \SAB{0}\), \(y_1 = \SAB{0}\), \(z_1 = \SAB{0}\)
    \item Find a different solution. \hfill \(x_2 = \SAB{-1}\), \(y_2 = \SAB{-1}\), \(z_2 = \SAB{1}\)
    \item Take your solutions and add them together. That is, let \(x := x_1 + x_2\), \(y :=
    y_1 + y_2\), \(z := z_1 + z_2\). Is this again a solution? Why or why not?
    \Switch{It is also a solution. This is because we can add the equations for \(x_1\),
    \(y_1\), and \(z_1\) together with the equations for \(x_2\), \(y_2\) and \(z_2\). Since
    the right hand sides are \(0\), this results in the same system!}{}
  \end{enumerate}
\end{problem}

\end{document}
