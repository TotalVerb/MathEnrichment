\documentclass[12pt,letterpaper]{article}

\newcommand{\IncludePath}{../../include}
\usepackage{extsizes}
\usepackage{titling}

\usepackage{amssymb,amsmath,amsthm}
\usepackage{enumerate}
\usepackage[margin=1in]{geometry}
\usepackage{graphicx,ctable,booktabs}
\usepackage{fancyhdr}
\usepackage[utf8]{inputenc}

\makeatletter
\newenvironment{problem}{\@startsection
       {section}
       {1}
       {-.2em}
       {-3.5ex plus -1ex minus -.2ex}
       {2.3ex plus .2ex}
       {\pagebreak[3]
       \large\bf\noindent{Problem }
       }
       }
\makeatother

\pagestyle{fancy}
\lhead{\thetitle}
\chead{}
\rhead{\thepage}
\lfoot{\small\scshape Grade 4 Olympic Math}
\cfoot{}
\rfoot{}
\renewcommand{\headrulewidth}{.3pt}
\renewcommand{\footrulewidth}{.3pt}
\setlength\voffset{-0.25in}
\setlength\textheight{648pt}
\setlength\headheight{15pt}


\title{Enumeration}
\author{By Fengyang Wang for Saturday Sessions}
\date{March 25, 2017}

\newcommand{\im}{i}

\begin{document}
\HomeworkTitle
\begin{problem}{Cardinality of Finite Sets}
  For each finite set \(S\), find \(|S|\).

  \begin{enumerate}
    \item \(S := \{1, 2, 3\}\)
    \item \(S := \{n \in \mathbf{Z} : n^2 = 4\}\)
    \item \(S := \{x \in \mathbf{R} : x(x+1)(x+2)(x+3) = 0\}\)
    \item \(S := \mathbf{Z} \cap \left\{\frac{1}{n} : n \in \mathbf{Z}\right\}\)
  \end{enumerate}
\end{problem}

\begin{problem}{Power Sets}
  Recall \(\mathcal{P}(S)\) denotes the set of subsets of \(S\), including
  \(\varnothing\) and \(S\) itself. Find all the elements of

  \begin{enumerate}
    \item \(\mathcal{P}(\{1, 2\})\)
    \item \(\mathcal{P}(\{1, 2, 3\} \cap \mathcal{P}(\{3, 4\})\)
    \item \(\{S \in \mathcal{P}(\{1, 2, 3, 4\}) : |S| = 2\}\)
  \end{enumerate}
\end{problem}

\begin{problem}{Polynomials}
  Write each of the following polynomials in standard form.

  \begin{enumerate}
    \item Let \(S = \mathcal{P}(\{1, 2, 3, 4, 5\})\). Find \[
      \sum_{A\in S} x^{|A|}
    \]
    \item Find \((1+x)^5\)
    \item Find \[
      \sum_{n=0}^5 \binom{5}{n} x^n
    \]
  \end{enumerate}
\end{problem}

\begin{problem}{Symmetric Group}
  Let \(S_n\) be the set of all permutations of \(\{1, \dots, n\}\).

  \begin{enumerate}
    \item Find \(|S_5|\)
    \item Find \[
      \sum_{\sigma\in S_3} x^{\sigma(1)}
    \]
    \item Find \[
      \sum_{\sigma\in S_4} x^{\sigma(1) + \sigma(2)}
    \]
    \item Substitute \(x=1\) in the above polynomial to obtain \(|S_4| = 24\).
    Explain why this is the case.
  \end{enumerate}
\end{problem}

\begin{problem}{More Binomial Coefficients}
  \begin{enumerate}
    \item Prove that for all \(n\ge1\), \[
      \sum_{k=0}^n 2^k \binom{n}{k} = 3^n
    \]
    Hint: Use the binomial theorem on \((1+2)^n\).
    \item Prove that for all \(n\ge1\), \[
      \sum_{k=0}^n (-1)^k \binom{n}{k} = 0
    \]
  \end{enumerate}
\end{problem}

\begin{problem}{Triangular Numbers}
  \begin{enumerate}
    \item Show that for \(n\ge1\), \[
      \sum_{k=1}^n k = 1 + 2 + \dots + n = \frac{n(n+1)}{2} = \binom{n+1}{2}
    \]
    There are many ways to do this problem but I encourage you think think
    about \(2\)-element subsets of \(\{1, \dots, n+1\}\). How would you list
    those subsets?
    \item Show that \[
      \sum_{k=2}^n \binom{k}{2} = \binom{n+1}{3}
    \]
    \item Let \(n\ge m\ge 1\) be integers. Find an expression for \[
      \sum_{k=m}^n \binom{k}{m}
    \]
  \end{enumerate}
\end{problem}

\end{document}
