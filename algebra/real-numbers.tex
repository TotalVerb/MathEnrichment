\documentclass[12pt,letterpaper]{article}

\newcommand{\IncludePath}{../include}
\usepackage{extsizes}
\usepackage{titling}

\usepackage{tikz}
\usetikzlibrary{shapes}

\usepackage{amssymb,amsmath,amsthm}
\usepackage{enumerate}
\usepackage[margin=0.8in]{geometry}
\usepackage{graphicx,ctable,booktabs}
\usepackage{fancyhdr}
\usepackage[utf8]{inputenc}
\usepackage{gensymb}

\makeatletter
\newenvironment{problem}{\@startsection
       {section}
       {1}
       {-.2em}
       {-3.5ex plus -1ex minus -.2ex}
       {2.3ex plus .2ex}
       {\pagebreak[3]
       \large\bf\noindent{Problem }
       }
       }
\makeatother

\pagestyle{fancy}
\lhead{\thetitle}
\chead{}
\rhead{\thepage}
\lfoot{\small\scshape Olympic Math}
\cfoot{}
\rfoot{}
\renewcommand{\headrulewidth}{.3pt}
\renewcommand{\footrulewidth}{.3pt}
\setlength\voffset{-0.25in}
\setlength\textheight{648pt}
\setlength\headheight{15pt}

\newcommand{\blankA}{\underline{\hspace{1em}}}
\newcommand{\blankB}{\underline{\hspace{2em}}}
\newcommand{\blankC}{\underline{\hspace{3em}}}
\newcommand{\blankD}{\underline{\hspace{4em}}}
\newcommand{\blankE}{\underline{\hspace{5em}}}
\newcommand{\blankF}{\underline{\hspace{6em}}}



% \answerstrue % turn on for answers
\defauthor

\title{Real Numbers}
\date{October 25, 2019}

\begin{document}

\maketitle

\thispagestyle{empty}

\Switch{}{For answers see class website: \url{https://grcs.uwseminars.com/}}

\begin{problem}{Graphing a Function}
  In this problem, we will draw a graph for the function \(y = (x-1)(x+1)\). First, fill out
  this table of values. The first few entries are filled out for you.

  \begin{center}
    \begin{tabular}{|c|c|l|}
      \hline
      $x$ & $y = (x-1)(x+1)$ & Rough work \\
      \hline
      $-2$ & $3$ & $(-2-1) \times (-2+1) = -3 \times (-1) = 3$ \\
      $-1$ & $0$ & $(-1-1) \times (-1+1) = -1 \times 0 = 0$ \\
      $0$ & \Switch{\Ans{-1}}{} & \\
      $1$ & \Switch{\Ans{0}}{} & \\
      $2$ & \Switch{\Ans{3}}{} & \\
      \hline
    \end{tabular}
  \end{center}

  On the grid paper below, axes have been drawn for you. Draw the five points above onto the
  grid. Again, the first two points are already done for you.

  \begin{center}
    \begin{tikzpicture}[x=1cm, y=1cm]
      \draw[line width=1mm, black, ->] (4, 0) -- (4, 8);
      \node[above] at (4, 8) {$y$};
      \node[below left] at (4, 0) {$-2$};
      \node[below left] at (4, 1) {$-1$};
      \node[below left] at (4, 3) {$1$};
      \node[below left] at (4, 4) {$2$};
      \node[below left] at (4, 5) {$3$};
      \node[below left] at (4, 6) {$4$};
      \node[below left] at (4, 7) {$5$};
      \node[below left] at (4, 8) {$6$};
      \draw[line width=1mm, black, ->] (0, 2) -- (8, 2);
      \node[right] at (8, 2) {$x$};
      \node[below left] at (0, 2) {$-4$};
      \node[below left] at (1, 2) {$-3$};
      \node[below left] at (2, 2) {$-2$};
      \node[below left] at (3, 2) {$-1$};
      \node[below left] at (4, 2) {$0$};
      \node[below left] at (5, 2) {$1$};
      \node[below left] at (6, 2) {$2$};
      \node[below left] at (7, 2) {$3$};
      \node[below left] at (8, 2) {$4$};
      \draw[step=1cm, line width=0.2mm, black] (0,0) grid (8cm, 8cm);

      \node[fill=black] at (2, 5) {};
      \node[fill=black] at (3, 2) {};
      \Switch{
      \node[fill=black] at (4, 1) {};
      \node[fill=black] at (5, 2) {};
      \node[fill=black] at (6, 5) {};
      \draw(1.5,7.25) parabola bend(4,1)(6.5,7.25);}{}
    \end{tikzpicture}
  \end{center}

  Now try to connect the five points using a smooth curve. Extrapolate (guess how the curve
  will continue) past $-2$ and $2$. You can calculate the correct values of $y$ for $x=-3$
  or $x=3$ if this helps you guess how the curve will continue. Answer the following
  questions using your graph:

  \begin{enumerate}
    \item What does the minimum value of $y$ look like? \SAB{-1}
    \item If $x$ becomes very large (say, $x > 10$), what happens to $y$? \Switch{\AnsT{It
    becomes very large!}}{}
    \item If $x$ becomes very small (say, $x < -10$), what happens to $y$? \Switch{\AnsT{It
    becomes very large!}}{}
    \item If $y = 3$, what are the possible values of $x$? \SAB{-2}, \SAB{2}
    \item If $y = 4$, is it possible for $x$ to be an integer? Do you think $x$ could be a
    rational number? Why or why not? \Switch{\AnsT{No, it is not possible for $x$ to be an
    integer. $x$ cannot be a rational number either because if $4 = (x-1)(x+1) = x^2-1$,
    then $x^2 = 3$, and we have noted that $\sqrt{3}$ is not rational.}}{}
  \end{enumerate}
\end{problem}

\begin{problem}{Linear Equations with a Unique Solution}
  Using division, determine the unique possible value of \(x\). Simplify your answers if
  possible.

  \begin{multicols}{2}
  \begin{enumerate}[\hspace{.5cm}a.]
    \item \( \sqrt{3}x = 3 \)           \hfill \(x = \SAB{\sqrt{3}}\)
    \item \( \sqrt{2}x = \sqrt{8} \)    \hfill \(x = \SAB{2}\)
    \item \( \displaystyle \frac{1}{2}x = \frac{1}{\sqrt{2}} \)
    \hfill \(x = \SAB{\sqrt{2}}\)
    \item \( \sqrt{12}x = -2\sqrt{3} \) \hfill \(x = \SAB{-1}\)
  \end{enumerate}
  \end{multicols}
\end{problem}

\begin{problem}{Exponentiation}
  Simplify as much as possible. Classify the answer as an integer, rational number, and/or
  real number. (Remember that integers are still rational numbers, and rational numbers ar
  still real numbers, so be sure to circle all the correct answers.)

  \begin{enumerate}
    \item \({\left(\sqrt{2}^3\right)}^2 = \SAB{8}\)
    \hfill \MCSelect{Integer} ~~ \MCSelect{Rational} ~~ \MCSelect{Real}
    \item \(\sqrt[3]{3^9} = \SAB{27}\)
    \hfill \MCSelect{Integer} ~~ \MCSelect{Rational} ~~ \MCSelect{Real}
    \item \(\sqrt{\frac{2}{98}} = \SAB{\frac{1}{7}}\)
    \hfill Integer ~~ \MCSelect{Rational} ~~ \MCSelect{Real}
    \item \(\sqrt{\frac{1 + \sqrt{2}}{\sqrt{2} - 1} - \sqrt{8}} = \SAB{\sqrt{3}}\)
    \hfill Integer ~~ Rational ~~ \MCSelect{Real}
  \end{enumerate}
\end{problem}

\begin{problem}{Real Problems}
  In class, we saw that some real numbers can be obtained through using rational exponents
  (radicals), and adding, subtracting, and multiplying these numbers.

  In 1824, Niels Henrick Abel proved that these numbers are not even enough to describe all
  real solutions to polynomials. An example of such a polynomial is $x^5 - x + 1 = 0$. We
  say that this polynomial does not have a solution in radicals.

  Show that the polynomial $x^5 - x + 30 = 0$, by contrast, does have a solution in
  radicals. This is in fact the unique real solution for this polynomial. \Switch{\AnsT{The
  solution is $-2$. This is an integer, and so is a solution in radicals of course.}}{}
\end{problem}

\end{document}
