\documentclass[12pt,letterpaper]{article}

\usepackage{extsizes}

\usepackage{amssymb,amsmath,amsthm}
\usepackage{enumerate}
\usepackage[margin=1.25in]{geometry}
\usepackage{graphicx,ctable,booktabs}
\usepackage{fancyhdr}
\usepackage[utf8]{inputenc}

\makeatletter
\newenvironment{problem}{\@startsection
       {section}
       {1}
       {-.2em}
       {-3.5ex plus -1ex minus -.2ex}
       {2.3ex plus .2ex}
       {\pagebreak[3]
       \large\bf\noindent{Problem }
       }
       }
\makeatother

\title{Arithmetic Sequences}
\author{Name: \underline{\hspace{5cm}}}
\date{November 8, 2014}

\pagestyle{fancy}
\lhead{Arithmetic Sequences}
\chead{} 
\rhead{\thepage} 
\lfoot{\small\scshape Grade 4 Olympic Math} 
\cfoot{} 
\rfoot{} 
\renewcommand{\headrulewidth}{.3pt} 
\renewcommand{\footrulewidth}{.3pt}
\setlength\voffset{-0.25in}
\setlength\textheight{648pt}
\setlength\headheight{15pt}

\begin{document}

\maketitle

\thispagestyle{empty}

\begin{problem}{Continue the sequence}
 Write the next three terms of each sequence.
 
\begin{enumerate}[\hspace{.5cm}a.]
\item $1, 2, 3, 4, \underline{\hspace{2em}}, \underline{\hspace{2em}}, \underline{\hspace{2em}}$
\item $2, 4, 6, 8, \underline{\hspace{2em}}, \underline{\hspace{2em}}, \underline{\hspace{2em}}$
\item $10, 15, 20, 25, \underline{\hspace{2em}}, \underline{\hspace{2em}}, \underline{\hspace{2em}}$
\item $30, 28, 26, 24, \underline{\hspace{2em}}, \underline{\hspace{2em}}, \underline{\hspace{2em}}$
\item $100, 91, 82, 73, \underline{\hspace{2em}}, \underline{\hspace{2em}}, \underline{\hspace{2em}}$
\end{enumerate}
\end{problem}

\begin{problem}{Fill in the blank}
 Fill in the blanks for each arithmetic sequence.
 
\begin{enumerate}[\hspace{.5cm}a.]
\item $1, 4, \underline{\hspace{2em}}, \underline{\hspace{2em}}, \underline{\hspace{2em}}, 16$
\item $100, \underline{\hspace{2em}}, \underline{\hspace{2em}}, \underline{\hspace{2em}}, 60, 50, 40$
\item $2, \underline{\hspace{2em}}, 14, \underline{\hspace{2em}}, 26$
\end{enumerate}
\end{problem}

\begin{problem}{General terms 1}
 Find the general term. (The general term looks like: $t_k = 10 \times k + 2$.)
 
\begin{enumerate}[\hspace{.5cm}a.]
\item $1, 2, 3, 4, 5, \ldots$ \hfill $t_k = \underline{\hspace{6em}}$
\item $10, 12, 14, 16, 18, \ldots$ \hfill $t_k = \underline{\hspace{6em}}$
\item $99, 96, 93, 90, 87, \ldots$ \hfill $t_k = \underline{\hspace{6em}}$
\end{enumerate}
\end{problem}

\begin{problem}{General terms 2}
 Write the first four terms of the sequence using the general term.
 
\begin{enumerate}[\hspace{.5cm}a.]
\item $t_k = 5 \times k + 2$ \hfill
$\underline{\hspace{2em}}, \underline{\hspace{2em}}, \underline{\hspace{2em}}, \underline{\hspace{2em}}$
\item $t_k = 30 - k$ \hfill
$\underline{\hspace{2em}}, \underline{\hspace{2em}}, \underline{\hspace{2em}}, \underline{\hspace{2em}}$
\item $t_k = 10 \times k - 5$ \hfill
$\underline{\hspace{2em}}, \underline{\hspace{2em}}, \underline{\hspace{2em}}, \underline{\hspace{2em}}$
\end{enumerate}
\end{problem}

\begin{problem}{Specific terms}
 Find the general term, then calculate the 100th term. (The general term looks like: $t_k = 10 \times k + 2$.)
 
\begin{enumerate}[\hspace{.5cm}a.]
\item $20, 40, 60, 80, 100, \ldots$ \hfill $t_k = \underline{\hspace{6em}}$; $t_{100} = \underline{\hspace{3em}}$
\item $500, 499, 498, 497, 496, \ldots$ \hfill $t_k = \underline{\hspace{6em}}$; $t_{100} = \underline{\hspace{3em}}$
\item $100, 103, 106, 109, 112, \ldots$ \hfill $t_k = \underline{\hspace{6em}}$; $t_{100} = \underline{\hspace{3em}}$
\end{enumerate}
\end{problem}

\begin{problem}{Word problem}
 On day $1$, I had $28$ pencils. Every day, I lose $2$ pencils. How many pencils do I have on day $5$?
 On what day will I have no pencils?
\end{problem}

\begin{problem}{Challenge problems}
  \begin{enumerate}[\hspace{.5cm}a.]
  \item In an arithmetic sequence, $t_1 = 5$ and $t_2 = 10$. Then $t_{100} = \underline{\hspace{3em}}$.
  \item In an arithmetic sequence, $t_1 = 1$ and $t_5 = 17$. Then $t_{100} = \underline{\hspace{3em}}$.
  \item In an arithmetic sequence, $t_1 = 1000$ and $t_{100} = 802$. Then $t_{5} = \underline{\hspace{3em}}$.
  \item In an arithmetic sequence, $t_{100} = 1000$ and $t_{500} = 5000$. Then $t_{1} = \underline{\hspace{3em}}$.
  \item In an arithmetic sequence, $t_1$ is odd and $t_2$ is odd.
  Is $t_{100}$ odd, even, or is there not enough information to say?
  \item In an arithmetic sequence, $t_1$ is odd and $t_2$ is even.
  Is $t_{100}$ odd, even, or is there not enough information to say?
  \end{enumerate}
\end{problem}

\end{document}