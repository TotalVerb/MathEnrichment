\documentclass[letterpaper,10pt]{article}
\usepackage[utf8]{inputenc}
\usepackage{amsmath}
\usepackage{amssymb}

\title{Fibonacci Sequence}
\author{Fengyang Wang}

\begin{document}
\maketitle

Begin with an introduction by means of rabbits. Start with one young rabbit.
Every generation, each mature rabbit has one kid, who becomes a new young rabbit.
Every year, young rabbits grow into mature rabbits. Rabbits never die.
Keep track of the total rabbit population every month.

\begin{center}
 \begin{tabular}{|c|c|c|c|}
  \hline
  Month \# & Young rabbits & Mature rabbits & Total rabbits \\
  \hline
  1 & 1 & 0 & 1 \\
  2 & 0 & 1 & 1 \\
  3 & 1 & 1 & 2 \\
  4 & 1 & 2 & 3 \\
  5 & 2 & 3 & 5 \\
  6 & 3 & 5 & 8 \\
  \vdots & \vdots & \vdots & \vdots \\
  \hline
 \end{tabular}
\end{center}

Proceed to highlight the sequence of numbers at right. Introduce the Italian
mathematician \textsc{Leonardo~Bonacci} (c. 1170 -- c. 1250). Discuss how the
sequence at right is called the ``Fibonacci sequence''. Discuss the recurrence
rule where each Fibonacci number is the sum of the last two Fibonacci numbers.
Introduce the $F_n$ notation for the $n$th Fibonacci number.

Begin with a simple Fibonacci-related problem: given $F_{21}=10946$ and $F_{22}=17711$, find
$F_{23}=28657$. Proceed to a backwards such problem; given $F_8=21$ and $F_7=13$ find $F_6=8$.
Finally, do a ``monkey in the middle'' problem; given $F_5=5$ and $F_3=2$ find $F_4=3$.

Discuss the odd-even parity pattern of the Fibonacci numbers. Figure out
whether the $1000$th Fibonacci number is odd or even.

Discuss the partial sums of the Fibonacci sequence:

\begin{center}
 \begin{tabular}{|c|c|c|c|}
  \hline
  $n$ & $\displaystyle \sum_{k=1}^n F_n$ & Value & $F_{n+2}$ \\
  \hline
  $1$ & $1$ & $1$ & $2$ \\
  $2$ & $1+1$ & $2$ & $3$ \\
  $3$ & $1+1+2$ & $4$ & $5$ \\
  $4$ & $1+1+2+3$ & $7$ & $8$ \\
  $5$ & $1+1+2+3+5$ & $12$ & $13$ \\
  \vdots & \vdots & \vdots & \vdots \\
  \hline
 \end{tabular}
\end{center}

\end{document}