\documentclass[a4paper,10pt]{report}

\newcommand{\IncludePath}{../include}
\newcommand{\ProjectName}{Grade 4 Olympic Math}
\usepackage{extsizes}
\usepackage{titling}

\usepackage{tikz}
\usetikzlibrary{shapes}

\usepackage{amssymb,amsmath,amsthm}
\usepackage{enumerate}
\usepackage{graphicx,ctable,booktabs}
\usepackage{fancyhdr}
\usepackage[utf8]{inputenc}
\usepackage{gensymb}

\usepackage[toc]{glossaries}

\makeatletter
\newenvironment{problem}{\@startsection
       {subsection}
       {1}
       {-.2em}
       {-3.5ex plus -1ex minus -.2ex}
       {2.3ex plus .2ex}
       {\pagebreak[3]
       \large\bf\noindent{Problem }
       }
       }
\makeatother

\makeatletter
\g@addto@macro\@floatboxreset\centering
\makeatother

\newenvironment{solution}
{ \vspace{1em} \noindent \textbf{Solution:} }
{  }

\pagestyle{fancy}
\lhead{\thetitle}
\chead{}
\rhead{\thepage}
\lfoot{\small\scshape \ProjectName}
\cfoot{}
\rfoot{}
\renewcommand{\headrulewidth}{.3pt}
\renewcommand{\footrulewidth}{.3pt}
\setlength\voffset{-0.25in}
\setlength\textheight{648pt}
\setlength\headheight{15pt}

\newcommand{\Ans}[1]{\framebox{$#1$}}
\newcommand{\SAA}[1]{\Switch{\Ans{#1}}{\blankA}}
\newcommand{\SAB}[1]{\Switch{\Ans{#1}}{\blankB}}
\newcommand{\SAC}[1]{\Switch{\Ans{#1}}{\blankC}}
\newcommand{\SAD}[1]{\Switch{\Ans{#1}}{\blankD}}
\newcommand{\SAE}[1]{\Switch{\Ans{#1}}{\blankE}}
\newcommand{\SAF}[1]{\Switch{\Ans{#1}}{\blankF}}
\newcommand{\STA}[1]{\Switch{\AnsT{#1}}{\blankA}}
\newcommand{\STB}[1]{\Switch{\AnsT{#1}}{\blankB}}
\newcommand{\STC}[1]{\Switch{\AnsT{#1}}{\blankC}}
\newcommand{\STD}[1]{\Switch{\AnsT{#1}}{\blankD}}
\newcommand{\STE}[1]{\Switch{\AnsT{#1}}{\blankE}}
\newcommand{\STF}[1]{\Switch{\AnsT{#1}}{\blankF}}
\newcommand{\AnsT}[1]{\framebox{#1}}
\newif\ifanswers
\newcommand{\Switch}[2]{\ifanswers#1\else#2\fi}
\newcommand{\MCSelect}[1]{\Switch{\AnsT{#1}}{#1}}
\newcommand{\TFTrue}{\MCSelect{True}~~False}
\newcommand{\TFFalse}{True~~\MCSelect{False}}

\newcommand{\blankA}{\underline{\hspace{1em}}}
\newcommand{\blankB}{\underline{\hspace{2em}}}
\newcommand{\blankC}{\underline{\hspace{3em}}}
\newcommand{\blankD}{\underline{\hspace{4em}}}
\newcommand{\blankE}{\underline{\hspace{5em}}}
\newcommand{\blankF}{\underline{\hspace{6em}}}



\title{Sequences}
\author{Fengyang Wang}
\date{April 12, 2016}

\begin{document}

\begin{abstract}

 This is an introduction to sequences suitable for an advanced Grade 4 audience.
 These notes were prepared for the Grand River Chinese School.

 These notes are intended to be a rough outline of what is taught, and not a
 rigorous and complete reference. In particular, I typically cover much more
 content than is written in these notes.

 Although the notes are intended to be presented to a young audience, they are
 written for a teacher and not for a student. Many of the terms used will not be
 familiar to the students, and will need to be explained differently.

\end{abstract}

\maketitle

\tableofcontents

\chapter{The Fibonacci Sequence}

We'll begin with an introduction by means of rabbits. Start with one young
rabbit. Every generation, each mature rabbit has one kid, who becomes a new
young rabbit. Every year, young rabbits grow into mature rabbits. Rabbits never
die. We can keep track of the total rabbit population every month (See Figure
~\ref{fib:rabbit}).

\begin{figure}
 \begin{tabular}{|c|c|c|c|}
  \hline
  Month \# & Young rabbits & Mature rabbits & Total rabbits \\
  \hline
  1 & 1 & 0 & 1 \\
  2 & 0 & 1 & 1 \\
  3 & 1 & 1 & 2 \\
  4 & 1 & 2 & 3 \\
  5 & 2 & 3 & 5 \\
  6 & 3 & 5 & 8 \\
  \vdots & \vdots & \vdots & \vdots \\
  \hline
 \end{tabular}

 \caption{Rabbit population in the first few months}

 \label{fib:rabbit}
\end{figure}

The sequence of numbers at the right hand side is interesting, not least because
it occurs in the other sequences written down also. It was discovered by the
Italian mathematician \textsc{Leonardo~Bonacci} (c. 1170 -- c. 1250), and this
sequence is therefore called the ``Fibonacci sequence''.

There is a so-called recurrence relation for the Fibonacci numbers. Each
Fibonacci number is the sum of the last two Fibonacci numbers. We use the
notation $F_n$ for the $n$th Fibonacci number.

\section{Example Problems}

\begin{problem}{Find $F_{23}$}
 Given $F_{21}=10946$ and $F_{22}=17711$, find $F_{23}=\Ans{28657}$.
\end{problem}

\begin{problem}{Find $F_6$}
 Given $F_8=21$ and $F_7=13$ find $F_6=\Ans{8}$.
\end{problem}

\begin{problem}{Find $F_4$}
 Given $F_5=5$ and $F_3=2$ find $F_4=\Ans{3}$.
\end{problem}

\section{Patterns}

There is an interesting odd-even parity pattern of the Fibonacci numbers. Think
about whether the $1000$th Fibonacci number is odd or even.

\section{Partial Sums}

Observe the partial sums of the Fibonacci sequence (Figure
~\ref{fib:partialsums}).

\begin{figure}
 \begin{tabular}{|c|c|c|c|}
  \hline
  $n$ & $\displaystyle \sum_{k=1}^n F_n$ & Value & $F_{n+2}$ \\
  \hline
  $1$ & $1$ & $1$ & $2$ \\
  $2$ & $1+1$ & $2$ & $3$ \\
  $3$ & $1+1+2$ & $4$ & $5$ \\
  $4$ & $1+1+2+3$ & $7$ & $8$ \\
  $5$ & $1+1+2+3+5$ & $12$ & $13$ \\
  \vdots & \vdots & \vdots & \vdots \\
  \hline
 \end{tabular}

 \caption{Partial sums of the Fibonacci sequence}

 \label{fib:partialsums}
\end{figure}

% Glossaries and List of Figures
\printglossaries

\cleardoublepage
\addcontentsline{toc}{chapter}{\listfigurename}
\listoffigures

\end{document}
