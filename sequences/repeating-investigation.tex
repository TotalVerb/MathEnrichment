\documentclass[12pt,letterpaper]{article}

\newcommand{\IncludePath}{../include}
\usepackage{extsizes}
\usepackage{titling}

\usepackage{amssymb,amsmath,amsthm}
\usepackage{enumerate}
\usepackage[margin=1in]{geometry}
\usepackage{graphicx,ctable,booktabs}
\usepackage{fancyhdr}
\usepackage[utf8]{inputenc}

\makeatletter
\newenvironment{problem}{\@startsection
       {subsection}
       {1}
       {-.2em}
       {-3.5ex plus -1ex minus -.2ex}
       {2.3ex plus .2ex}
       {\pagebreak[3]
       \large\bf\noindent{Problem }
       }
       }
\makeatother

\makeatletter
\newenvironment{step}{\@startsection
       {section}
       {1}
       {-.2em}
       {-3.5ex plus -1ex minus -.2ex}
       {2.3ex plus .2ex}
       {\pagebreak[3]
       \large\bf\noindent{Step }
       }
       }
\makeatother

\pagestyle{fancy}
\lhead{\thetitle}
\chead{}
\rhead{\thepage}
\lfoot{\small\scshape Grade 4 Olympic Math}
\cfoot{}
\rfoot{}
\renewcommand{\headrulewidth}{.3pt}
\renewcommand{\footrulewidth}{.3pt}
\setlength\voffset{-0.25in}
\setlength\textheight{648pt}
\setlength\headheight{15pt}


\title{Patterns \& Sequences}
\author{Name: \underline{\hspace{5cm}}}
\date{December 3, 2016}

\begin{document}
\HomeworkTitle

\thispagestyle{empty}

A \textbf{sequence} is like a line at a movie theatre: someone is first in line,
then second, then third, and so forth. For instance, $1, 2, 3, 4, \ldots$ is a
sequence. $\uparrow, \leftarrow, \uparrow, \leftarrow, \ldots$ is also a
sequence. The difference is that a sequence goes on forever!

Some sequences have patterns. In this unit, we will investigate different kinds
of patterns.

\begin{step}{Repetition}
Probably the simplest type of pattern is a repeating pattern, where some part of
the sequence repeats over and over again. The repeating subunit (the part of the
sequence that repeats) can be short or long. Today we will focus on these
patterns.

For example, consider these sequences:

\begin{enumerate}
\item $2, 2, 2, 2, 2, \ldots$
\item A, B, C, D, E, F, G, H, I, J, K, L, M, N, O, P, Q, R, S, T, U, V, W, X,
Y, Z, A, B, C, \ldots
\end{enumerate}

In the first sequence, the repeating subunit is only one item long---the number
$2$. In the second sequence, the repeating subunit is $26$ items long---the
entire alphabet!
\end{step}

\begin{problem}{Finding Repetition}
Find the repeating subunit:

\begin{enumerate}[\hspace{.5cm}a.]
\item red, green, blue, red, green, blue, \ldots
\item $1, 4, 2, 8, 5, 7, 1, 4, 2, 8, 5, 7, \ldots$
\end{enumerate}
\end{problem}

\begin{problem}{Continue the Pattern}
Write the next three terms (items) of the sequence:

\begin{enumerate}[\hspace{.5cm}a.]
\item $2, 4, 8, 6, 2, 4, 8, 6,$ \underline{\hspace{1em}},
\underline{\hspace{1em}}, \underline{\hspace{1em}}
\item $\uparrow, \downarrow, \downarrow, \uparrow, \downarrow, \downarrow,$
\underline{\hspace{1em}}, \underline{\hspace{1em}}, \underline{\hspace{1em}}
\end{enumerate}
\end{problem}

\begin{step}{Revisiting Remainders}
Consider the sequence of letters on the last page. What's the 500th letter? One
way to solve this problem is to write the pattern out, but this will take a very
long time. Notice that if the first letter (\#$1$) is A, then the 26th letter is
Z. The 27th letter would be A again. Every $26$ letters, we skip past the whole
alphabet and end back where we started.

Suppose the 8th letter is H. Then, since $8+26=34$, the 34th letter is also H.
This goes the other way too: if the 28th letter is $B$, then since $28-26=2$,
the 2nd letter is also $B$.

When we do a division, we have a \textbf{quotient} and a \textbf{remainder}. For
instance, $13 \div 4 = 3\,\mathrm{R}\,1$. The quotient ($3$) stands for how many
groups of $4$ can be made from $13$ items, and the remainder ($1$) stands for
how many items are left over.

One way we can solve the problem of what the 500th letter is to find the
remainder when we divide $500$ by $26$. It turns out that this remainder is $6$,
and the quotient is $19$. This means that if we start at the $6$th letter, and
we go forward $26$ letters nineteen times, we'll end up at the $500$th
letter---but this must be the same letter we started at, since going forward
$26$ letters will see the sequence repeat. Since the $6$th letter is F, the
$500$th letter must also be F.
\end{step}

\begin{problem}{The 100th Term}
Find the 100th term for each sequence.

\begin{enumerate}[\hspace{.5cm}a.]
\item $0, 9, 0, 9, 0, 9, \ldots$
\item $1, 3, 5, 1, 3, 5, \ldots$
\item \(\framebox{\circ}, \framebox{\bullet}, \circ, \bullet,
        \framebox{\circ}, \framebox{\bullet}, \circ, \bullet, \dots\)
\end{enumerate}
\end{problem}

\begin{problem}{The 1000th Term}
Find the 1000th term for each sequence.

\begin{enumerate}[\hspace{.5cm}a.]
\item R, E, D, R, E, D, \ldots
\item M, I, S, S, I, S, S, I, P, P, I, M, I, S, S, I, S, S, I, P, P, I, \ldots
\item $3, 9, 7, 1, 3, 9, 7, 1 \ldots$
\end{enumerate}
\end{problem}

\begin{problem}{Challenge}
A sequence has a repeating subunit that is $14$ items long. It starts $1, 2, 3,
4, \ldots$. What's the 100th term?
\end{problem}

\end{document}
