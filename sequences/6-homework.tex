\documentclass[12pt,letterpaper]{article}

\newcommand{\IncludePath}{../include}
\usepackage{extsizes}
\usepackage{titling}

\usepackage{amssymb,amsmath,amsthm}
\usepackage{enumerate}
\usepackage[margin=1in]{geometry}
\usepackage{graphicx,ctable,booktabs}
\usepackage{fancyhdr}
\usepackage[utf8]{inputenc}

\makeatletter
\newenvironment{problem}{\@startsection
       {section}
       {1}
       {-.2em}
       {-3.5ex plus -1ex minus -.2ex}
       {2.3ex plus .2ex}
       {\pagebreak[3]
       \large\bf\noindent{Problem }
       }
       }
\makeatother

\pagestyle{fancy}
\lhead{\thetitle}
\chead{}
\rhead{\thepage}
\lfoot{\small\scshape Grade 4 Olympic Math}
\cfoot{}
\rfoot{}
\renewcommand{\headrulewidth}{.3pt}
\renewcommand{\footrulewidth}{.3pt}
\setlength\voffset{-0.25in}
\setlength\textheight{648pt}
\setlength\headheight{15pt}


\title{Sequences Review}
\author{Name: \underline{\hspace{5cm}}}
\date{January 16, 2016}

\begin{document}

\maketitle

\thispagestyle{empty}

\begin{problem}{Repeating sequences}
\begin{enumerate}
 \item Complete the sequence:
 $7, 2, 4, 7, 2, 4, \underline{\hspace{2em}}, \underline{\hspace{2em}},
 \underline{\hspace{2em}}$.
 \item  Find the $30$th term:
 $7, 2, 4, 7, 2, 4, \ldots$ \hfill $t_{30} = \underline{\hspace{2em}}$
\end{enumerate}
\end{problem}

\begin{problem}{Arithmetic sequences}
\begin{enumerate}
 \item
 Find the common difference:
 $1, 4, 7, 10, 13, \ldots$. \hfill
 $\text{common difference} = \underline{\hspace{2em}}$

 \item
 Find the $30$th term:
 $1, 4, 7, 10, 13, \ldots$. \hfill
 $t_{30} = \underline{\hspace{2em}}$
\end{enumerate}
\end{problem}

\begin{problem}{Geometric sequences}
\begin{enumerate}
 \item
 Complete the sequence:
 $4, 16, 64, 256, \underline{\hspace{4em}}, \underline{\hspace{4em}}$.

 \item
 Find the common ratio:
 $4, 16, 64, 256, \ldots$. \hfill
 $\text{common ratio} = \underline{\hspace{2em}}$
\end{enumerate}
\end{problem}

\begin{problem}{Fibonacci sequence}
\begin{enumerate}
 \item
 If $F_9=34$ and $F_{10}=55$, what is $F_8$? \hfill
 $F_8 = \underline{\hspace{2em}}$

 \item
 If $F_5=5$ and $F_6=8$, what is $F_7$? \hfill
 $F_7 = \underline{\hspace{2em}}$
\end{enumerate}
\end{problem}

\begin{problem}{Challenge}
 Look at this pattern:
 \[
  0, 1, 4, 9, 16, 25, \ldots
 \]

 Let's look at the differences between the terms. $1-0=1$, $4-1=3$, $9-4=5$,
 $16-9=7$, and $25-16=9$. This is the result:
 \[
  1, 3, 5, 7, 9, \ldots
 \]

 What are the rules for each of the patterns?
\end{problem}

\begin{problem}{Complete the sequence}
 Write the next few terms (fill in all the blanks) of each sequence. Circle all
 types of sequences that apply to the sequence in question. (For all questions,
 at least one option should be circled. For exactly one of the questions, more
 than one option will be circled---be careful!)

 \begin{enumerate}
  \item $1, 1, 1, 1, \underline{\hspace{2em}}, \underline{\hspace{2em}},
  \underline{\hspace{2em}}$
  \hfill Repeating Arithmetic Geometric Fibonacci
  \item $1, 0, 1, 0, 1, 0, \underline{\hspace{2em}}, \underline{\hspace{2em}},
  \underline{\hspace{2em}}$
  \hfill Repeating Arithmetic Geometric Fibonacci
  \item $1, 2, 3, 4, \underline{\hspace{2em}}, \underline{\hspace{2em}},
  \underline{\hspace{2em}}$
  \hfill Repeating Arithmetic Geometric Fibonacci
  \item $50, 48, 46, 44, \underline{\hspace{2em}}, \underline{\hspace{2em}},
  \underline{\hspace{2em}}$
  \hfill Repeating Arithmetic Geometric Fibonacci
  \item $3, 9, 27, 81, \underline{\hspace{2em}}, \underline{\hspace{2em}},
  \underline{\hspace{2em}}$
  \hfill Repeating Arithmetic Geometric Fibonacci
  \item $2048, 1024, 512, 256, \underline{\hspace{2em}},
  \underline{\hspace{2em}}$
  \hfill Repeating Arithmetic Geometric Fibonacci
  \item $1, 1, 2, 3, 5, 8, 13, \underline{\hspace{2em}},
  \underline{\hspace{2em}}, \underline{\hspace{2em}}$
  \hfill Repeating Arithmetic Geometric Fibonacci
 \end{enumerate}
\end{problem}

\begin{problem}{Cassini's Identity}
 Which of the following identities did Jean-Dominique~Cassini discover? (Circle
 one)

 \begin{enumerate}[\hspace{1em}a)]
  \item The sum of two consecutive Fibonacci numbers is the next Fibonacci
  number.
  \item The square of a Fibonacci number is one greater or one less than the
  product of the Fibonacci numbers to its left and right.
  \item The sum of the first few Fibonacci numbers is always one less than
  another Fibonacci number.
 \end{enumerate}
\end{problem}


\end{document}
