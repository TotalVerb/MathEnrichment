\documentclass[12pt,letterpaper]{article}

\usepackage{extsizes}

\usepackage{amssymb,amsmath,amsthm}
\usepackage{enumerate}
\usepackage[margin=1.25in]{geometry}
\usepackage{graphicx,ctable,booktabs}
\usepackage{fancyhdr}
\usepackage[utf8]{inputenc}

\makeatletter
\newenvironment{problem}{\@startsection
       {section}
       {1}
       {-.2em}
       {-3.5ex plus -1ex minus -.2ex}
       {2.3ex plus .2ex}
       {\pagebreak[3]
       \large\bf\noindent{Problem }
       }
       }
\makeatother

\title{The Fibonacci Sequence}
\author{Name: \underline{\hspace{5cm}}}
\date{December 6, 2014}

\pagestyle{fancy}
\lhead{The Fibonacci Sequence}
\chead{} 
\rhead{\thepage} 
\lfoot{\small\scshape Grade 4 Olympic Math} 
\cfoot{} 
\rfoot{} 
\renewcommand{\headrulewidth}{.3pt} 
\renewcommand{\footrulewidth}{.3pt}
\setlength\voffset{-0.25in}
\setlength\textheight{648pt}
\setlength\headheight{15pt}

\begin{document}

\maketitle

\thispagestyle{empty}

\begin{problem}{Continue the sequence}
 Write the next three terms of the Fibonacci sequence. Remember: each term in the Fibonacci sequence
 is the sum of the previous two terms.
 
 \[
  1, 1, 2, 3, 5, \underline{\hspace{2em}}, \underline{\hspace{2em}}, \underline{\hspace{2em}}
 \]
\end{problem}

\begin{problem}{Going forwards}
 The 11th Fibonacci number is $89$ and the 12th Fibonacci number is $144$. What's the 13th Fibonacci number?
 \hfill Answer: \underline{\hspace{4em}}
\end{problem}

\begin{problem}{Going backwards}
 The 15th Fibonacci number is $610$ and the 16th Fibonacci number is $987$. What's the 14th Fibonacci number?
 \hfill Answer: \underline{\hspace{4em}}
\end{problem}

\begin{problem}{Monkey in the middle}
 The 17th Fibonacci number is $1597$ and the 19th Fibonacci number is $4181$. What's the 18th Fibonacci number?
\end{problem}

\begin{problem}{Odd or even?}
 Recall $F_1 = 1$ and $F_2 = 1$. Is $F_{20}$ odd or even? \hfill Circle the answer: Odd\hspace{1em}Even
\end{problem}

\begin{problem}{A curious pattern}
 Below is a list of Fibonacci numbers. Fill in the empty row with the square of each Fibonacci
 number (the first few and last few are done for you). The square of a number is the number multiplied
 by itself. For example, the square of $3$ (denoted $3^2$) is $9$ because $3 \times 3=9$.
 
 \begin{center}
 \begin{tabular}{|c|c|c|c|c|c|c|c|c|c|c|c|}
 \hline
  Term \# & $F_1$ & $F_2$ & $F_3$ & $F_4$ & $F_5$ & $F_6$ & $F_7$ & $F_8$ & $F_9$ & $F_{10}$ & $\ldots$ \\ \hline
  Value & $1$ & $1$ & $2$ & $3$ & $5$ & $8$ & $13$ & $21$ & $34$ & $55$ & $\ldots$ \\ \hline
  Square & $1$ & $1$ & $4$ & & & & & & $1156$ & $3025$ & $\ldots$ \\ \hline
 \end{tabular}
 \end{center}
 
 Using this table, compute or look up the following:
 
 \begin{enumerate}
  \item $\left(F_3\right)^2$ (square of third Fibonacci number) \hfill Answer: \underline{\hspace{4em}}
  \item $F_2 \times F_4$ (product of 2nd and 4th Fibonacci numbers) \hfill Answer: \underline{\hspace{4em}}
  \item $\left(F_4\right)^2$ \hfill Answer: \underline{\hspace{4em}}
  \item $F_3 \times F_5$ \hfill Answer: \underline{\hspace{4em}}
  \item $\left(F_5\right)^2$ \hfill Answer: \underline{\hspace{4em}}
  \item $F_4 \times F_6$ \hfill Answer: \underline{\hspace{4em}}
  \item $\left(F_6\right)^2$ \hfill Answer: \underline{\hspace{4em}}
  \item $F_5 \times F_7$ \hfill Answer: \underline{\hspace{4em}}
 \end{enumerate}
 
 Do you notice a pattern? This is known as ``Cassini's Identity''.

 Mathematician Jean-Dominique Cassini (1625--1712) was the first to discover this interesting property of these squares.
 A Fibonacci number's square is really close to the product of the Fibonacci numbers to its left and to its right.
 In fact, it is always exactly one off: either bigger by $1$ or smaller by $1$!
 This is true for all Fibonacci numbers. For example, consider
 $\left(F_9\right)^2=1156$, which is just one greater than $F_8 \times F_{10} = 21 \times 55 = 1155$.
\end{problem}

\end{document}