\documentclass[12pt,letterpaper]{article}

\usepackage{extsizes}
\usepackage{titling}

\usepackage{tikz}
\usetikzlibrary{shapes}

\usepackage{amssymb,amsmath,amsthm}
\usepackage{enumerate}
\usepackage[margin=0.8in]{geometry}
\usepackage{graphicx,ctable,booktabs}
\usepackage{fancyhdr}
\usepackage[utf8]{inputenc}
\usepackage{gensymb}

\makeatletter
\newenvironment{problem}{\@startsection
       {section}
       {1}
       {-.2em}
       {-3.5ex plus -1ex minus -.2ex}
       {2.3ex plus .2ex}
       {\pagebreak[3]
       \large\bf\noindent{Problem }
       }
       }
\makeatother

\pagestyle{fancy}
\lhead{\thetitle}
\chead{}
\rhead{\thepage}
\lfoot{\small\scshape Olympic Math}
\cfoot{}
\rfoot{}
\renewcommand{\headrulewidth}{.3pt}
\renewcommand{\footrulewidth}{.3pt}
\setlength\voffset{-0.25in}
\setlength\textheight{648pt}
\setlength\headheight{15pt}

\newcommand{\blankA}{\underline{\hspace{1em}}}
\newcommand{\blankB}{\underline{\hspace{2em}}}
\newcommand{\blankC}{\underline{\hspace{3em}}}
\newcommand{\blankD}{\underline{\hspace{4em}}}
\newcommand{\blankE}{\underline{\hspace{5em}}}
\newcommand{\blankF}{\underline{\hspace{6em}}}



\title{Challenging Sequence Problems}
\author{Name: \underline{\hspace{5cm}}}
\date{December 6, 2014}

\begin{document}

\maketitle

\thispagestyle{empty}

\begin{problem}{Warm-up}
Fill in the blanks with the appropriate numbers to complete the arithmetic or
geometric sequence.

\begin{enumerate}
 \item $2, 6, \underline{\hspace{2em}}, \underline{\hspace{2em}}, 18$
 \item $5, 10, \underline{\hspace{2em}}, \underline{\hspace{2em}}, 80$
\end{enumerate}
\end{problem}

\begin{problem}{Binary strings}
 Consider finite sequences of circles and stars. Each element of the sequence is either a
 circle ($\circ$) or a star ($\star$). Two stars are not allowed to be next to each other, but circles
 are allowed to be next to each other.

 If the sequence has one shape, the possibilities are $\circ$ and $\star$. If it has two shapes, the
 possibilities are $\circ \star$, $\circ \circ$, or $\star \circ$. (Note that two stars can't be next
 to each other, so star-star is \emph{not} a possibility.)

 \begin{enumerate}
  \item There are $5$ possible sequences with three shapes. What are they?
  \item How many possible sequences have four shapes? Five?
  \item Is there a pattern for how many possible sequences there are?
 \end{enumerate}
\end{problem}

\begin{problem}{Odd number sums}
 Carl has discovered an interesting property of odd number sums.
 The sum of the first $n$ odd numbers ($n$ is any positive whole number) is $n^2$!
 (Recall that $n^2=n \times n$ and is pronounced ``$n$ squared'').
 Look at the table below.

 \begin{center}
  \begin{tabular}{|c|c|c|}
  \hline
   Sum & Value & Square \\
  \hline
   $1$ & $1$ & $1^2$ \\
   $1 + 3$ & $4$ & $2^2$ \\
   $1 + 3 + 5$ & $9$ & $3^2$ \\
   $1 + 3 + 5 + 7$ & $16$ & $4^2$ \\
   \hline
  \end{tabular}
 \end{center}

 Using this pattern, Carl predicts that $5^2=25$ will be equal to the sum of the first $5$ odd numbers ($1+3+5+7+9$).

 \begin{enumerate}
  \item Is Carl right? Try adding the numbers together.
  \item Using Carl's rule, predict what the sum of the first $10$ odd numbers will be.
  \item Why does Carl's rule work? Try drawing a diagram to solve this problem. For a hint, see below.
 \end{enumerate}

 \begin{center}
  \includegraphics[width=100px]{hint.png}
 \end{center}

 What do you see? If you split each individual shade pictured into small squares,
 how many are of each shade? What do you notice about the total number of small
 squares in each larger square?

\end{problem}

\begin{problem}{Binary}
 Figure out the rule of this pattern. Fill in the blanks.

 \[
  0000, 0001, 0010, 0011, 0100, 0101, 0110, \underline{\hspace{2em}}, \underline{\hspace{2em}}, \ldots
 \]

 Hint: The last digit is always changing between $0$ and $1$. The second-last digit changes only when
 the last digit changes from $1$ to $0$. Try figuring out the rule of the other digits.
\end{problem}

\end{document}
