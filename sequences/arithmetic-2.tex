\documentclass[12pt,letterpaper]{article}

\newcommand{\IncludePath}{../include}
\usepackage{extsizes}
\usepackage{titling}

\usepackage{amssymb,amsmath,amsthm}
\usepackage{enumerate}
\usepackage[margin=1in]{geometry}
\usepackage{graphicx,ctable,booktabs}
\usepackage{fancyhdr}
\usepackage[utf8]{inputenc}

\makeatletter
\newenvironment{problem}{\@startsection
       {section}
       {1}
       {-.2em}
       {-3.5ex plus -1ex minus -.2ex}
       {2.3ex plus .2ex}
       {\pagebreak[3]
       \large\bf\noindent{Problem }
       }
       }
\makeatother

\pagestyle{fancy}
\lhead{\thetitle}
\chead{}
\rhead{\thepage}
\lfoot{\small\scshape Grade 4 Olympic Math}
\cfoot{}
\rfoot{}
\renewcommand{\headrulewidth}{.3pt}
\renewcommand{\footrulewidth}{.3pt}
\setlength\voffset{-0.25in}
\setlength\textheight{648pt}
\setlength\headheight{15pt}


\title{Arithmetic Sequences II}
\author{Name: \underline{\hspace{5cm}}}
\date{December 16, 2016}

\begin{document}

\maketitle

\thispagestyle{empty}

\begin{problem}{Fill in the Blank}
 Fill in the blanks for each arithmetic sequence.

\begin{enumerate}[\hspace{.5cm}a.]
\item \(1, 4, \blankB, \blankB, \blankB, 16\)
\item \(100, \blankB, \blankB, \blankB, 60, 50, 40\)
\item \(2, \blankB, 14, \blankB, 26\)
\end{enumerate}
\end{problem}

\begin{problem}{Closed Forms I}
 Find the closed form. (The closed form looks like: \(t_n = 10 \times n + 2\).)

\begin{enumerate}[\hspace{.5cm}a.]
\item $1, 2, 3, 4, 5, \ldots$ \hfill $t_n = \blankF$
\item $10, 12, 14, 16, 18, \ldots$ \hfill $t_n = \blankF$
\item $99, 96, 93, 90, 87, \ldots$ \hfill $t_n = \blankF$
\end{enumerate}
\end{problem}

\begin{problem}{Closed Forms II}
 Using the closed form, write the first four terms of the sequence.

\begin{enumerate}[\hspace{.5cm}a.]
\item $t_n = 5 \times n + 2$ \hfill
$\blankB, \blankB, \blankB, \blankB$
\item $t_n = 30 - n$ \hfill
$\blankB, \blankB, \blankB, \blankB$
\item $t_n = 10 \times n - 5$ \hfill
$\blankB, \blankB, \blankB, \blankB$
\end{enumerate}
\end{problem}

\begin{problem}{Specific Terms}
 Find the closed form, then calculate the 100th term. (The closed form looks
 like: $t_n = 10 \times n + 2$.)

\begin{enumerate}[\hspace{.5cm}a.]
\item $20, 40, 60, 80, 100, \ldots$ \hfill $t_n = \blankF$;
$t_{100} = \blankC$
\item $500, 499, 498, 497, 496, \ldots$ \hfill $t_n = \blankF$;
$t_{100} = \blankC$
\item $100, 103, 106, 109, 112, \ldots$ \hfill $t_n = \blankF$;
$t_{100} = \blankC$
\end{enumerate}
\end{problem}

\begin{problem}{Word Problem}
 On day $1$, I had $28$ pencils. Every day, I lose $2$ pencils. How many
 pencils do I have on day $5$? On what day will I have no pencils?
\end{problem}

\begin{problem}{True or False}
  Decide whether each statement is true or false.

  \begin{enumerate}
    \item Suppose that \((t_n)\) is an arithmetic sequence of integers (whole
    numbers). If \((t_n)\) contains both even and odd numbers, then the common
    difference must be odd.
    \hfill \MCSelect{True}~~False
    \item Suppose that \((t_n)\) is an arithmetic sequence of integers. If
    \((t_n)\) contains only odd numbers, then the common difference must be
    even.
    \hfill \MCSelect{True}~~False
    \item Suppose that \((t_n)\) is an arithmetic sequence of integers. If
    \(t_1 + t_2 = 4\), then the common difference must be \(2\).
    \hfill True~~\MCSelect{False}
    \item The following sequence is an arithmetic sequence of fractions: \[
      \frac{1}{2}, 1, \frac{3}{2}, 2, \frac{5}{2}, \dots
    \]
    \hfill \MCSelect{True}~~False
    \item The following sequence is an arithmetic sequence, and every term is a positive integer: \[
      10, 9, 8, 7, \dots
    \]
    \hfill True~~\MCSelect{False}
    \item Suppose that \((t_n)\) is an arithmetic sequence. If \(t_1 = 5\) and
    \(t_2 = 10\), then \(t_{100} = 500\).
    \hfill \MCSelect{True}~~False
    \item Suppose that \((t_n)\) is an arithmetic sequence with common
    difference \(-2\) and initial value \(t_1 = 0\). Then \(t_{100} = 198\).
    \hfill True~~\MCSelect{False}
    \item The sum of the first \(100\) terms of an arithmetic sequence is
    always positive.
    \hfill True~~\MCSelect{False}
    \item All arithmetic sequences contain at least one positive number as a
    term.
    \hfill True~~\MCSelect{False}
    \item Suppose that \((t_n)\) is an arithmetic sequence. Then \(t_1 + t_9 =
    t_3 + t_7\).
    \hfill \MCSelect{True}~~False
    \item Suppose that \((t_n)\) is an arithmetic sequence with closed form
    \(t_n = 10 \times n + 3\). Then every term in \(t_n\) is odd.
    \hfill True~~\MCSelect{False}
    \item Let \((t_n)\) be the following sequence is an arithmetic sequence,
    with common difference \(0\): \[
      9, 9, 9, 9, \dots
    \]
    Its closed form is \(t_n = 9\).
    \hfill \MCSelect{True}~~False
  \end{enumerate}
\end{problem}

\end{document}
