\documentclass[a4paper,10pt]{report}

\newcommand{\IncludePath}{../include}
\newcommand{\ProjectName}{Grade 4 Olympic Math}
\usepackage{extsizes}
\usepackage{titling}

\usepackage{tikz}
\usetikzlibrary{shapes}

\usepackage{amssymb,amsmath,amsthm}
\usepackage{enumerate}
\usepackage{graphicx,ctable,booktabs}
\usepackage{fancyhdr}
\usepackage[utf8]{inputenc}
\usepackage{gensymb}

\usepackage[toc]{glossaries}

\makeatletter
\newenvironment{problem}{\@startsection
       {subsection}
       {1}
       {-.2em}
       {-3.5ex plus -1ex minus -.2ex}
       {2.3ex plus .2ex}
       {\pagebreak[3]
       \large\bf\noindent{Problem }
       }
       }
\makeatother

\makeatletter
\g@addto@macro\@floatboxreset\centering
\makeatother

\newenvironment{solution}
{ \vspace{1em} \noindent \textbf{Solution:} }
{  }

\pagestyle{fancy}
\lhead{\thetitle}
\chead{}
\rhead{\thepage}
\lfoot{\small\scshape \ProjectName}
\cfoot{}
\rfoot{}
\renewcommand{\headrulewidth}{.3pt}
\renewcommand{\footrulewidth}{.3pt}
\setlength\voffset{-0.25in}
\setlength\textheight{648pt}
\setlength\headheight{15pt}

\newcommand{\Ans}[1]{\framebox{$#1$}}
\newcommand{\SAA}[1]{\Switch{\Ans{#1}}{\blankA}}
\newcommand{\SAB}[1]{\Switch{\Ans{#1}}{\blankB}}
\newcommand{\SAC}[1]{\Switch{\Ans{#1}}{\blankC}}
\newcommand{\SAD}[1]{\Switch{\Ans{#1}}{\blankD}}
\newcommand{\SAE}[1]{\Switch{\Ans{#1}}{\blankE}}
\newcommand{\SAF}[1]{\Switch{\Ans{#1}}{\blankF}}
\newcommand{\STA}[1]{\Switch{\AnsT{#1}}{\blankA}}
\newcommand{\STB}[1]{\Switch{\AnsT{#1}}{\blankB}}
\newcommand{\STC}[1]{\Switch{\AnsT{#1}}{\blankC}}
\newcommand{\STD}[1]{\Switch{\AnsT{#1}}{\blankD}}
\newcommand{\STE}[1]{\Switch{\AnsT{#1}}{\blankE}}
\newcommand{\STF}[1]{\Switch{\AnsT{#1}}{\blankF}}
\newcommand{\AnsT}[1]{\framebox{#1}}
\newif\ifanswers
\newcommand{\Switch}[2]{\ifanswers#1\else#2\fi}
\newcommand{\MCSelect}[1]{\Switch{\AnsT{#1}}{#1}}
\newcommand{\TFTrue}{\MCSelect{True}~~False}
\newcommand{\TFFalse}{True~~\MCSelect{False}}

\newcommand{\blankA}{\underline{\hspace{1em}}}
\newcommand{\blankB}{\underline{\hspace{2em}}}
\newcommand{\blankC}{\underline{\hspace{3em}}}
\newcommand{\blankD}{\underline{\hspace{4em}}}
\newcommand{\blankE}{\underline{\hspace{5em}}}
\newcommand{\blankF}{\underline{\hspace{6em}}}



\title{Problem Solving Skills}
\author{Fengyang Wang}

\begin{document}
\maketitle

\tableofcontents

\chapter{Introduction}

At your day school, you focus on only a small subset of mathematics. In Grade 1,
you learned about whole numbers, the place value system, and basic addition and
subtraction. You recognized two-dimensional figures such as triangles,
rectangles, or circles. You learned how to make and interpret pictographs.

In Grade 2, you learned how to do multi-digit addition and subtraction, as well
as multiplication and division. You learned about area and perimeter and how to
measure them. Finally, you also learned about symmetry and patterning.

Next, in Grade 3, you learned about angles, three-dimensional solids, and more
two-dimensional figures. You learned how to tell time and what units we use to
measure things like time, distance, etc. Finally, you learned a bit about
fractions.

That sounds like a lot of math, doesn't it? But in fact, that's just the tip of
the iceberg. For those of you who took the Grade 3 Olympic Math course, you will
know that there's a lot more to it than numbers, figures, and measurement. Those
are very important parts of math, and that's why you do them at school.

What we will do in this interest course is extend your mathematical knowledge to
more fields of math. Before we can do that, we will investigate just what else
there is in math:

\section{Fields of mathematics}

Mathematics is a broad field. Grown-up mathematicians investigate all kinds of
math, and a lot of that math is really complicated. Unfortunately, we can't do
any of that stuff yet. But there are a lot of fields of mathematics that we can
do, including:

\begin{itemize}
 \item Algebra
 \item Logic
 \item Number theory
 \item Probability
 \item Combinatorics
\end{itemize}

Don't worry if you don't recognize these words. We'll end up covering these
topics throughout the year.

We'll also do some work in topics that you do cover in your day school. But
we'll focus on things that you won't learn in school.

\begin{itemize}
 \item Arithmetic
 \item Geometry
 \item Measurement
\end{itemize}

\section{Problem solving}
A major focus of this interest course is on problem solving. Sometimes we will
solve problems individually, and sometimes we'll work as a class or in small
groups. Just to get you started, we'll do a few example problems.

These problems will range from easy ones to harder ones that you'll have to
think about.

\begin{problem}{The Hundredth Rock}
 Henry has numbered 100 rocks with numbers from 1 to 100. Starting from the
 first rock, the colours are: red, green, blue, red, green, blue, and so on and
 so forth. If this pattern continues, what's the colour of the hundredth rock?
\end{problem}

\begin{problem}{A Big Sum}
 What is $1+2+3+4+\ldots+100$?
\end{problem}

\begin{problem}{Treasure}
 Henrietta has found a treasure chest containing a large sum of money. She
 spends half of it on a new bike, and then spends \$50 more on a new backpack.
 Finally she spends half of the remaining money to buy a pizza. Now she only has
 \$15 left. How much money did Henrietta find in the treasure chest?
\end{problem}

\begin{problem}{Handshakes}
 Four people attend a meeting. Each shakes the hand of every person (except
 themselves) exactly once. How many handshakes occur?
\end{problem}

\chapter{Problem Solving Skills}

\section{Measures of centrality}

Before we begin the course, we will cover an important topic so that we are all
on the same page.

Suppose we have a list of numerical data. What is data? Data  consists of the
results of some kind of experiment. For example, if I measure the height of
everyone in this room in centimetres, I would get some data.

Let's say I conducted that experiment, and I got the following answers: $[105,
115, 115, 115, 125, 125, 145]$. This is data, because it consists of the results
of my experiment. It's also \emph{numerical} data because it's made up of
numbers.

Suppose I wanted to know what the \emph{typical} student's height was. I want a
single number that represents the results of the whole experiment. It turns out
that there is not just a single way to do this!

The first way is to find which numbers show up most often. In our example, the
number $115$ shows up most. This means that more students had a height of $115$
than any other height. We call this the \emph{mode}. The mode is usually easy to
find, but it has several disadvantages. Firstly, there could possibly be more
than one mode, if more than one number shows up most often. For example, if the
$145$ here were $125$, then both $115$ and $125$ would be modes. Secondly, the
mode is very susceptible to error from incorrect measurements. A single mistake,
such as a $110$ being recorded as $115$, can cause a large change in the mode.

The second way is to put the numbers in order and look at which one is in the
middle. That number is called the \emph{median}. In our example, the median is
$115$. When there are an even number of data, the median is harder to calculate.
Usually we take the two in the middle, sum them, and divide by two.

Finally, the third way is to add all the numbers up and divide by the total
number of numbers. This is called the \emph{average} (sometimes also called the
\emph{mean}). The average is not always a nice number to calculate. In are
example, the sum is $845$, and when we divide by $7$ we get
$120.\overline{714285}$.

\begin{problem}{The Rest}
 The average of ten numbers is $100$. The smallest of the ten numbers is $19$.
 What is the average of the nine remaining numbers?
\end{problem}

\section{Working backwards}

It is useful in a lot of situations to work backwards to arrive at an answer to
a question.

\end{document}
