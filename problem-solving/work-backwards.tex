\documentclass[14pt,letterpaper]{article}

\newcommand{\IncludePath}{../include}
\usepackage{extsizes}
\usepackage{titling}

\usepackage{amssymb,amsmath,amsthm}
\usepackage{enumerate}
\usepackage[margin=1in]{geometry}
\usepackage{graphicx,ctable,booktabs}
\usepackage{fancyhdr}
\usepackage[utf8]{inputenc}

\makeatletter
\newenvironment{problem}{\@startsection
       {section}
       {1}
       {-.2em}
       {-3.5ex plus -1ex minus -.2ex}
       {2.3ex plus .2ex}
       {\pagebreak[3]
       \large\bf\noindent{Problem }
       }
       }
\makeatother

\pagestyle{fancy}
\lhead{\thetitle}
\chead{}
\rhead{\thepage}
\lfoot{\small\scshape Grade 4 Olympic Math}
\cfoot{}
\rfoot{}
\renewcommand{\headrulewidth}{.3pt}
\renewcommand{\footrulewidth}{.3pt}
\setlength\voffset{-0.25in}
\setlength\textheight{648pt}
\setlength\headheight{15pt}


\title{Working Backward}
\author{Name: \underline{\hspace{5cm}}}
\date{November 26, 2016}

\begin{document}
\HomeworkTitle

\thispagestyle{empty}

\begin{problem}{Fruit}
 Ming bought a bag of fruit to share among her friends. She kept half the
 fruits for herself, and gave $3$ fruits each to $5$ friends. How many fruits
 did Ming buy?
\end{problem}

\begin{problem}{Groceries}
 Nick is buying groceries. He spent $\$10$ on fruit, $\$20$ on vegetables, and
 $\$30$ on meat. Nick then spent half his remaining money on fuel for his car.
 Now, Nick has $\$35$. How much money did he start with?
\end{problem}

\begin{problem}{Bookstore}
 Octavio went to the bookstore to buy some books. In one week, he finished
 reading half the books. The next week, he finished reading $2$ more books. One
 week later, he finished reading half the remaining books. Now, he has just one
 book left to read. How many books did Octavio buy?
\end{problem}

\begin{problem}{Spelling Bee}
 Each round in a spelling bee eliminates $4$ of every $5$ people. For example,
 if there were $125$ people, $100$ people will be eliminated. Right now, there
 are $5$ people left. How many people were there $2$ rounds ago?
\end{problem}

\begin{problem}{Problem Set}
 Tamara is doing a page of $5$ math puzzles. Every question she solves takes
 $1$ minute longer than the previous question. For example, if she solves one
 question in $5$ minutes, then it will take her $6$ minutes to solve the next
 question. The last question on the page takes her $10$ minutes to solve. How
 much time did it take Tamara to finish the page?
\end{problem}

\begin{problem}{Reading Problem}
  Reading problems are about reading new concepts, trying to understand them
  through examples, and working through some exercises.

  A \emph{set} of numbers is an unordered collection of different numbers. For
  example, \(\{1, 2, 3\}\) is a set of numbers. \(\{1, 3, 5, 9\}\) is also a
  set of numbers. We allow sets to be \emph{empty}; that is, \(\{\}\) is a set
  of numbers, just with nothing in it. Sets can also contain only a single
  element. For instance, \(\{5\}\) is a set, containing only the number \(5\).

  We write \(3 \in \{3, 5, 10\}\) to mean that \(3\) \emph{belongs to} the set
  \(\{3, 5, 10\}\), and \(4 \notin \{10\}\) to mean that \(4\) \emph{does not
  belong to} the set \(\{10\}\). Fill out each of the following with the right
  symbol, either \(\in\) or \(\notin\). \begin{itemize}
    \item \(8 \blankB \{8\}\)
    \item \(13 \blankB \{\}\)
    \item \(2 \blankB \{0, 1, 2, 3, 4, 5, 6, 22\}\)
  \end{itemize}

  Given the definitions above, solve the following problem. Find a set (name
  it) \(S\) so that all of the following are true: \begin{itemize}
    \item \(1 \in S\)
    \item \(2 \notin S\)
    \item \(S\) has \(3\) elements
  \end{itemize}
  noting that there are multiple solutions. \[
    S = \blankF
  \]
\end{problem}

\end{document}
