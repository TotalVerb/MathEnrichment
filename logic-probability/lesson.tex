\documentclass[a4paper,10pt]{report}

\newcommand{\IncludePath}{../include}
\newcommand{\ProjectName}{Grade 6 Olympic Math}
\usepackage{extsizes}
\usepackage{titling}

\usepackage{tikz}
\usetikzlibrary{shapes}

\usepackage{amssymb,amsmath,amsthm}
\usepackage{enumerate}
\usepackage{graphicx,ctable,booktabs}
\usepackage{fancyhdr}
\usepackage[utf8]{inputenc}
\usepackage{gensymb}

\usepackage[toc]{glossaries}

\makeatletter
\newenvironment{problem}{\@startsection
       {subsection}
       {1}
       {-.2em}
       {-3.5ex plus -1ex minus -.2ex}
       {2.3ex plus .2ex}
       {\pagebreak[3]
       \large\bf\noindent{Problem }
       }
       }
\makeatother

\makeatletter
\g@addto@macro\@floatboxreset\centering
\makeatother

\newenvironment{solution}
{ \vspace{1em} \noindent \textbf{Solution:} }
{  }

\pagestyle{fancy}
\lhead{\thetitle}
\chead{}
\rhead{\thepage}
\lfoot{\small\scshape \ProjectName}
\cfoot{}
\rfoot{}
\renewcommand{\headrulewidth}{.3pt}
\renewcommand{\footrulewidth}{.3pt}
\setlength\voffset{-0.25in}
\setlength\textheight{648pt}
\setlength\headheight{15pt}

\newcommand{\Ans}[1]{\framebox{$#1$}}
\newcommand{\SAA}[1]{\Switch{\Ans{#1}}{\blankA}}
\newcommand{\SAB}[1]{\Switch{\Ans{#1}}{\blankB}}
\newcommand{\SAC}[1]{\Switch{\Ans{#1}}{\blankC}}
\newcommand{\SAD}[1]{\Switch{\Ans{#1}}{\blankD}}
\newcommand{\SAE}[1]{\Switch{\Ans{#1}}{\blankE}}
\newcommand{\SAF}[1]{\Switch{\Ans{#1}}{\blankF}}
\newcommand{\STA}[1]{\Switch{\AnsT{#1}}{\blankA}}
\newcommand{\STB}[1]{\Switch{\AnsT{#1}}{\blankB}}
\newcommand{\STC}[1]{\Switch{\AnsT{#1}}{\blankC}}
\newcommand{\STD}[1]{\Switch{\AnsT{#1}}{\blankD}}
\newcommand{\STE}[1]{\Switch{\AnsT{#1}}{\blankE}}
\newcommand{\STF}[1]{\Switch{\AnsT{#1}}{\blankF}}
\newcommand{\AnsT}[1]{\framebox{#1}}
\newif\ifanswers
\newcommand{\Switch}[2]{\ifanswers#1\else#2\fi}
\newcommand{\MCSelect}[1]{\Switch{\AnsT{#1}}{#1}}
\newcommand{\TFTrue}{\MCSelect{True}~~False}
\newcommand{\TFFalse}{True~~\MCSelect{False}}

\newcommand{\blankA}{\underline{\hspace{1em}}}
\newcommand{\blankB}{\underline{\hspace{2em}}}
\newcommand{\blankC}{\underline{\hspace{3em}}}
\newcommand{\blankD}{\underline{\hspace{4em}}}
\newcommand{\blankE}{\underline{\hspace{5em}}}
\newcommand{\blankF}{\underline{\hspace{6em}}}



% use glossaries for this document
\newglossaryentry{algorithm}
{
  name=algorithm,
  description={step-by-step procedure for performing a calculation
  according to well-defined rules (Wikipedia)}
}

\newglossaryentry{acute angle}
{
  name=acute angle,
  description={an angle measuring less than a right angle (\SI{90}{\degree} or a
  quarter of a turn); such angles are typically characterized as being sharp}
}

\newglossaryentry{angle}
{
  name=angle,
  description={the figure formed by two rays, called the sides of the angle,
  sharing a common endpoint, called the vertex of the angle (Wikipedia)}
}

\newglossaryentry{coincident}
{
  name=coincident,
  description={a description of two objects which occupy exactly the same space}
}

\newglossaryentry{domain}
{
  name=domain,
  description={the set of values that are possible inputs for a function}
}

\newglossaryentry{endpoint}
{
  name=endpoint,
  description={an extreme point of a line segment or ray; line segments have two
  endpoints whereas rays have just one}
}

\newglossaryentry{function}
{
  name=function,
  description={object that may take any allowed input and will produce a
  single associated output for that input; alternatively, relation for
  which each input value has exactly one related output value}
}

\newglossaryentry{integer}
{
  name=integer,
  description={positive or negative whole number, or $0$; for example,
  $-8$, $2000$}
}

\newglossaryentry{line}
{
  name=line,
  description={a straight one-dimensional object that extends forever in both
  directions}
}

\newglossaryentry{line segment}
{
  name=line segment,
  description={a straight one-dimensional object terminated at both ends}
}

\newglossaryentry{multiplicand}
{
  name=multiplicand,
  description={the number that is being multiplied; for instance, in
  $2\times3=6$, the multiplicand is $2$}
}

\newglossaryentry{multiplier}
{
  name=multiplier,
  description={the factor to multiply a number by; for instance, in
  $2\times3=6$, the multiplier is $3$}
}

\newglossaryentry{obtuse angle}
{
  name=obtuse angle,
  description={an angle measuring more than a right angle (\SI{90}{\degree} or a
  quarter of a turn) but less than a straight angle (\SI{180}{\degree} or a half
  of a turn)}
}

\newglossaryentry{parallel}
{
  name=parallel,
  description={a description of two lines that never intersect at any point}
}

\newglossaryentry{parity}
{
  name=parity,
  description={decribes whether an integer is even or odd}
}

\newglossaryentry{plane}
{
  name=plane,
  description={a two-dimensional flat surface}
}

\newglossaryentry{plane geometry}
{
  name=plane geometry,
  description={the study of figures on a plane (a two-dimensional flat surface)}
}

\newglossaryentry{product}
{
  name=product,
  description={the result of a multiplication; for instance, in
  $2\times3=6$, the product is $6$}
}

\newglossaryentry{range}
{
  name=range,
  description={the set of values that are possible outputs for a function}
}

\newglossaryentry{ray}
{
  name=ray,
  description={a straight one-dimensional object terminated at one end and
  extending forever in the other direction}
}

\newglossaryentry{right angle}
{
  name=right angle,
  description={an angle measuring \SI{90}{\degree} or a quarter of a turn; angle
  between two rays intersecting in an L shape}
}

\newglossaryentry{right triangle}
{
  name=right triangle,
  description={an triangle with one \SI{90}{\degree} (right) angle}
}

\newglossaryentry{summand}
{
  name=summand,
  description={something which is being added; for instance, in $1+2=3$,
  the two summands are $1$ and $2$}
}

\newglossaryentry{reflex angle}
{
  name=reflex angle,
  description={an angle measuring more than \SI{180}{\degree} or a half of a
  turn, but less than \SI{360}{\degree} or a full turn}
}

\newglossaryentry{straight angle}
{
  name=straight angle,
  description={an angle measuring \SI{180}{\degree} or a half of a turn; angle
  between two rays in opposite directions}
}

\newglossaryentry{transitivity}
{
  name=transitivity,
  description={the property of certain relations that specifies if an
  element $a$ is related to $b$, and the element $b$ is related to $c$,
  then $a$ is similarly related to $c$}
}

\makeglossaries

\title{Logic \& Probability}
\author{Fengyang Wang}
\date{April 12, 2016}

\begin{document}

\begin{abstract}

 This is an introduction to logic and probability. Although it was originally
 intended for Grade 4 students, I found that the content was too abstract and
 hard to understand for many. Hence, I've rated it at the Grade 6 level, and am
 no longer actively developing this curriculum.

 These notes are intended to be a rough outline of what is taught, and not a
 rigorous and complete reference. I covered more material than is written in the
 notes.

 Although the notes are intended to be presented to a young audience, they are
 written for a teacher and not for a student. Many of the terms used will not be
 familiar to the students, and will need to be explained differently.

\end{abstract}

\maketitle

\tableofcontents

\chapter{Logic}

Logical thinking is the kind of reasoning necessary to make logical and correct
conclusions from givens. Much of logic can be thought of as building up
statements using rules.

\section{Propositions}

A proposition is any statement. Example: ``It is sunny outside.''

\subsection{Negation}

We can negate this proposition like this: ``It is not sunny outside.''

\subsection{Conjunction}

We can create a proposition that is true only when both components are true by
using the word ``and''. For example: ``It is sunny outside and it is warm
outside.''

\subsection{Disjunction}

We can create a proposition that is true when either component is true, or if
both components are true. For example: ``It is sunny outside or it is warm
outside.''

\section{Arguments}

An argument is a list of propositions ending in a statement that is a logical
conclusion of the other propositions. For instance:

\begin{align*}
 1.~&\text{It is daytime.} \\
 2.~&\text{There are no clouds.} \\
 3.~&\text{If it is daytime and there are no clouds, then it is sunny.} \\
 \hline
 4.~&(1), (2) \to \text{It is daytime and there are no clouds.} \\
 \hline
 \therefore~&(4), (3) \to \text{It is sunny.}
\end{align*}

\section{Rules}

There are rules to make sure all statements are logical. We'll cover them below.

\subsection{Double negative}
\begin{equation*}
 \lnot\lnot p \implies p
\end{equation*}

\subsection{Conditional elimination}
\begin{equation*}
 \{p, p \implies q\} \implies q
\end{equation*}

\subsection{Conjunction introduction}

\begin{equation*}
 \{p, q\} \implies p \land q
\end{equation*}

\subsection{Conjunction elimination}

\begin{equation*}
 p \land q \implies p \\
 p \land q \implies q
\end{equation*}

\subsection{Disjunction introduction}

\begin{equation*}
 p \implies p \lor q \\
 q \implies p \lor q
\end{equation*}

\subsection{Disjunction elimination}

\begin{equation*}
 \{p \implies r, q \implies r, p \lor r\} \implies r
\end{equation*}

\subsection{Modus tollens}

Denying the consequent.

\begin{equation*}
 \{p \implies q, \lnot q\} \implies \lnot p
\end{equation*}

\section{Not logical}
These are not rules:

\begin{itemize}
 \item Denying the antecedent.
 \item Affirming the consequent.
\end{itemize}

\chapter{Probability}

\section{Review of Fractions}

\begin{problem}{Fractions}
 Compute each of the following:

 \begin{itemize}
  \item \(
         \frac{3}{8} \times \frac{2}{7} = \frac{6}{56} = \Ans{\frac{3}{28}}
        \)
  \item \(
  \frac{5}{9} \times \frac{2}{5} = \frac{10}{45} = \Ans{\frac{2}{9}}
  \)
 \end{itemize}
\end{problem}

\section{Simple Probability}

We can list all possible outcomes and generate the probability of one particular
outcome happening.

\begin{equation}
 P(A) = \frac{\text{Possibilities where $A$ occurs}}
 {\text{Total possible possibilities}}
\end{equation}


\begin{problem}{Simple Probability I}
 What's the probability of flipping a fair coin and seeing heads?

 \begin{solution}
  There are two possible possibilities, and in one of them, heads is shown. Thus
  the probability is \Ans{\frac{1}{2}}.
 \end{solution}
\end{problem}

\begin{problem}{Simple Probability II}
 What's the probability of rolling a fair $6$-sided die and seeing an odd
 number?

 \begin{solution}
  There are six possible possibilities, and in three of them ($1$, $3$, and
  $5$), an odd number is shown. We have \[
   \frac{3}{6} = \frac{1}{2}
  \]

  Thus the probability is \Ans{\frac{1}{2}}.
 \end{solution}
\end{problem}

\begin{problem}{Simple Probability III}
 What's the probability of picking a red marble from a bag that has $5$ red
 marbles and $7$ blue marbles?

 \begin{solution}
  There are $5+7=12$ total possibilities, and in $5$ of them, a red marble is
  drawn. Thus the probability is \Ans{\frac{5}{12}}.
 \end{solution}
\end{problem}

\section{Independent Events}

Sometimes we have more than one event, and the outcome of one event does not
change the next. These events would be called ``independent''. For instance,
in each of the following cases, we have independent events:

\begin{enumerate}
 \item I have two rabbits and a dog. I pick a pet at random; then I pick a
 different pet at random.
 \item I roll two six-sided dice.
 \item I roll one six-sided die twice.
\end{enumerate}

\section{Joint Probabilities}

When two events $A$ and $B$ are independent, we have the following rule for the
chances that both happen:

\begin{equation}
 P(A \cap B) = P(A) \times P(B)
\end{equation}

We can see this identity by drawing a tree and counting the number of
possibilities where both events come true.

\begin{problem}{Joint Probability I}
 A bag contains $5$ red marbles and $7$ blue marbles. I draw a marble from the
 bag and then flip a coin. What's the chance of flipping heads and drawing a red
 marble?
\end{problem}

\begin{problem}{Joint Probability II}
 A bag contains $10$ red marbles and $5$ blue marbles. I draw a marble from the
 bag at random, then I put it back. Then I draw another marble from the bag at
 random. What's the probability of drawing two blue marbles?
\end{problem}

\section{Dependent Events}

When two events $A$ and $B$ are not independent, we can still calculate the
probability nevertheless. We just need to be careful to update the probability.

\begin{problem}{Drawing Cards I}

 A deck of cards (without jokers) has $13$ cards of each of the $4$ suits ($52$
 cards total). What's the probability of drawing two spades without replacement?

 \begin{solution}
  \begin{align*}
   \frac{13}{52} \times \frac{12}{51}
   &= \frac{1}{4} \times \frac{12}{51} \\
   &= \frac{12}{204} \\
   &= \frac{1}{17}
  \end{align*}
  so the probability is \Ans{\frac{1}{17}}.
 \end{solution}

\end{problem}

\begin{problem}{Drawing Cards II}
 What about the probability of drawing two aces (without replacement) from a
 deck of cards (without jokers)?

 \begin{solution}
  \begin{align*}
   \frac{4}{52} \times \frac{3}{51}
   &= \frac{1}{13} \times \frac{1}{17} \\
   &= \frac{1}{221}
  \end{align*}
  so the probability is \Ans{\frac{1}{221}}.
 \end{solution}
\end{problem}

\section{Expected Value}

The expected value is the mean of all possibilities. For instance, the expected
value of rolling a six-sided die is $3.5$.


% Glossaries and List of Figures
\printglossaries

\cleardoublepage
\addcontentsline{toc}{chapter}{\listfigurename}
\listoffigures
\end{document}
