\documentclass[a4paper,10pt]{report}

\newcommand{\IncludePath}{../include}
\newcommand{\ProjectName}{Grade 6 Olympic Math}
\usepackage{extsizes}
\usepackage{titling}

\usepackage{tikz,tkz-euclide}
\usetikzlibrary{quotes,angles,shapes,calc,through,intersections}
\usetkzobj{all}
\usepackage{forest}

\usepackage{amssymb,amsmath,amsthm}
\usepackage{enumerate}
\usepackage{graphicx,booktabs}
\usepackage{fancyhdr}
\usepackage{gensymb}
\usepackage{siunitx}
\usepackage{framed}
\usepackage{minted}

\usepackage[toc]{glossaries}
\makeglossaries

\newcounter{exampleproblem}[chapter]
\newcommand{\theproblem}{\thechapter\alph{exampleproblem}}
\newenvironment{problem}[1]{
 \addtocounter{exampleproblem}{1}
 \begin{framed}
  \begin{center} \underline{Example \theproblem. \textbf{#1}} \end{center}
}{
 \end{framed}
}
\def\checkmark{\tikz\fill[scale=0.4](0,.35) -- (.25,0) -- (1,.7) -- (.25,.15) -- cycle;}

\renewcommand{\arraystretch}{1.5}

\makeatletter
\g@addto@macro\@floatboxreset\centering
\makeatother

\newenvironment{solution}
{ \vspace{1em} \noindent \textbf{Solution:} }
{  }

\pagestyle{fancy}
\lhead{\thetitle}
\chead{}
\rhead{\thepage}
\lfoot{\small\scshape \ProjectName}
\cfoot{}
\rfoot{}
\renewcommand{\headrulewidth}{.3pt}
\renewcommand{\footrulewidth}{.3pt}
\setlength\voffset{-0.25in}
\setlength\textheight{648pt}
\setlength\headheight{15pt}

\newcommand{\Ans}[1]{\framebox{$#1$}}
\newcommand{\AnsT}[1]{\framebox{#1}}
\newif\ifanswers
\newcommand{\Switch}[2]{\ifanswers#1\else#2\fi}
\newcommand{\MCSelect}[1]{\Switch{\AnsT{#1}}{#1}}
\newcommand{\TFTrue}{\MCSelect{True}~~False}
\newcommand{\TFFalse}{True~~\MCSelect{False}}

\newcommand{\blankA}{\underline{\hspace{1em}}}
\newcommand{\blankB}{\underline{\hspace{2em}}}
\newcommand{\blankC}{\underline{\hspace{3em}}}
\newcommand{\blankD}{\underline{\hspace{4em}}}
\newcommand{\blankE}{\underline{\hspace{5em}}}
\newcommand{\blankF}{\underline{\hspace{6em}}}

\newcommand{\fgLabelledCycle}[1]{
\begin{tikzpicture}
\def \n {#1}
\def \r {2cm}
\def \sp {14}
\def \tt {360/\n}

\foreach \s in {0,...,\numexpr#1-1\relax}
{
\node[draw, circle] at ({\tt * \s}:\r) {$[\s]$};
\draw[->, >=latex] ({\tt * \s + \sp}:\r)
arc ({\tt * \s + \sp}:{\tt * (\s + 1) - \sp}:\r);
}
\end{tikzpicture}}

\newcommand{\fgClock}{
\begin{tikzpicture}
% draw clock border
\draw (0,0) circle [radius=1.6cm];

% draw clock label
\foreach \angle [count=\i] in {60,30,...,-270}
{
\draw (\angle:1.5cm) -- (\angle:1.6cm);
\node at (\angle:1.2cm) {\i};
}

% draw hands
\draw[line width=2pt] (60:0) -- (60:0.6cm);
\draw[line width=2pt] (90:0) -- (90:0.9cm);
\end{tikzpicture}}

\newcommand{\fgNumberLine}[2]{
\begin{tikzpicture}
\foreach \x in {#1,...,#2}
{
\draw (\x, -0.1) -- (\x, 0.1);
\node at (\x, -0.4) {$\x$};
}
\draw (#1, 0) -- (#2, 0);

% bold line at zero
\draw[line width=2pt] (0, -0.15) -- (0, 0.15);
\end{tikzpicture}}

\newcommand{\fgLineSegment}{
\begin{tikzpicture}
\draw[line width=2pt] (-5, 0) -- (5, 0);
\end{tikzpicture}}

\newcommand{\fgLine}{
\begin{tikzpicture}
\draw[->, =>latex, line width=2pt] (0, 0) -- (5, 0);
\draw[->, =>latex, line width=2pt] (0, 0) -- (-5, 0);
\end{tikzpicture}}

\newcommand{\fgRay}{
\begin{tikzpicture}
\draw[->, =>latex, line width=2pt] (-5, 0) -- (5, 0);
\end{tikzpicture}}

\newcommand{\fgTwoRays}{
\begin{tikzpicture}
\draw[->, =>latex, line width=2pt] (0, 0) -- (2.5, 0);
\draw[->, =>latex, line width=2pt] (0, 0) -- (2, 1.5);
\end{tikzpicture}}

\newcommand{\fgTwoRaysFar}{
\begin{tikzpicture}
\draw[->, =>latex, line width=2pt] (0, 0) -- (2.5, 0);
\draw[->, =>latex, line width=2pt] (0, 0) -- (1.5, 2);
\end{tikzpicture}}

\newcommand{\fgTwoRaysRight}{
\begin{tikzpicture}
\draw[->, =>latex, line width=2pt] (0, 0) -- (3, 0);
\draw[->, =>latex, line width=2pt] (0, 0) -- (0, 3);
\draw[line width=2pt] (0, 1) -- (1, 1) -- (1, 0);
\end{tikzpicture}}

\newcommand{\fgAsterisk}{
\begin{tikzpicture}
\draw[->, =>latex, line width=2pt] (0, 0) -- (1, 0);
\draw[->, =>latex, line width=2pt] (0, 0) -- (0.5, 0.87);
\draw[->, =>latex, line width=2pt] (0, 0) -- (-0.5, 0.87);
\draw[->, =>latex, line width=2pt] (0, 0) -- (-1, 0);
\draw[->, =>latex, line width=2pt] (0, 0) -- (-0.5, -0.87);
\draw[->, =>latex, line width=2pt] (0, 0) -- (0.5, -0.87);
\end{tikzpicture}}

\newcommand{\fgAngleN}[1]{
\begin{tikzpicture}
\coordinate (A) at (0,0);
\coordinate (B) at (0:3);
\coordinate (C) at (#1:3);
\draw[->, =>latex, line width=2pt] (A) -- (B);
\draw[->, =>latex, line width=2pt] (A) -- (C);
\pic [draw, line width=2pt, "{\small\SI{#1}{\degree}}", angle radius=1.5cm] {angle = B--A--C};
%\draw[line width=2pt] ++(#1:1) arc (#1:0:1) node[midway] ;
\end{tikzpicture}}



% use glossaries for this document
\newglossaryentry{integer}
{
  name=integer,
  description={a positive or negative whole number, or $0$; for example, $-8$,
  $2000$, or $0$}
}

\newglossaryentry{parity}
{
  name=parity,
  description={decribes whether an integer is even or odd}
}

\newglossaryentry{transitivity}
{
  name=transitivity,
  description={the property of certain relations that specifies if an element
  $a$ is related to $b$, and the element $b$ is related to $c$, then $a$ is
  similarly related to $c$}
}

\makeglossaries

\title{Logic \& Probability}
\author{Fengyang Wang}
\date{April 12, 2016}

\begin{document}

\begin{abstract}

 This is an introduction to logic and probability. Although it was originally
 intended for Grade 4 students, I found that the content was too abstract and
 hard to understand for many. Hence, I've rated it at the Grade 6 level, and am
 no longer actively developing this curriculum.

 These notes are intended to be a rough outline of what is taught, and not a
 rigorous and complete reference. I covered more material than is written in the
 notes.

 Although the notes are intended to be presented to a young audience, they are
 written for a teacher and not for a student. Many of the terms used will not be
 familiar to the students, and will need to be explained differently.

\end{abstract}

\maketitle

\tableofcontents

\chapter{Logic}

Logical thinking is the kind of reasoning necessary to make logical and correct
conclusions from givens.

\section{Propositions}

A proposition is any statement. Example: ``It is sunny outside.''

\subsection{Negation}

We can negate this proposition like this: ``It is not sunny outside.''

\subsection{Conjunction}

We can create a proposition that is true only when both components are true by
using the word ``and''. For example: ``It is sunny outside and it is warm
outside.''

\subsection{Disjunction}

We can create a proposition that is true when either component is true, or if
both components are true. For example: ``It is sunny outside or it is warm
outside.''

\section{Arguments}

An argument is a list of propositions ending in a statement that is a logical
conclusion of the other propositions. For instance:

\begin{align*}
 1.~&\text{It is daytime.} \\
 2.~&\text{There are no clouds.} \\
 3.~&\text{If it is daytime and there are no clouds, then it is sunny.} \\
 \hline
 4.~&(1), (2) \to \text{It is daytime and there are no clouds.} \\
 \hline
 \therefore~&(4), (3) \to \text{It is sunny.}
\end{align*}

\section{Rules}

There are rules to make sure all statements are logical. We'll cover them below.

\subsection{Double negative}
\begin{equation*}
 \lnot\lnot p \implies p
\end{equation*}

\subsection{Conditional elimination}
\begin{equation*}
 \{p, p \implies q\} \implies q
\end{equation*}

\subsection{Conjunction introduction}

\begin{equation*}
 \{p, q\} \implies p \land q
\end{equation*}

\subsection{Conjunction elimination}

\begin{equation*}
 p \land q \implies p \\
 p \land q \implies q
\end{equation*}

\subsection{Disjunction introduction}

\begin{equation*}
 p \implies p \lor q \\
 q \implies p \lor q
\end{equation*}

\subsection{Disjunction elimination}

\begin{equation*}
 \{p \implies r, q \implies r, p \lor r\} \implies r
\end{equation*}

\subsection{Modus tollens}

Denying the consequent.

\begin{equation*}
 \{p \implies q, \lnot q\} \implies \lnot p
\end{equation*}

\section{Not logical}
These are not rules:

\begin{itemize}
 \item Denying the antecedent.
 \item Affirming the consequent.
\end{itemize}

\chapter{Probability}

\section{Review of Fractions}

\begin{problem}{Fractions}
 Compute each of the following:

 \begin{itemize}
  \item \(
         \frac{3}{8} \times \frac{2}{7} = \frac{6}{56} = \Ans{\frac{3}{28}}
        \)
  \item \(
  \frac{5}{9} \times \frac{2}{5} = \frac{10}{45} = \Ans{\frac{2}{9}}
  \)
 \end{itemize}
\end{problem}

\section{Simple Probability}

We can list all possible outcomes and generate the probability of one particular
outcome happening.

\begin{equation}
 P(A) = \frac{\text{Possibilities where $A$ occurs}}
 {\text{Total possible possibilities}}
\end{equation}


\begin{problem}{Simple Probability I}
 What's the probability of flipping a fair coin and seeing heads?

 \begin{solution}
  There are two possible possibilities, and in one of them, heads is shown. Thus
  the probability is \Ans{\frac{1}{2}}.
 \end{solution}
\end{problem}

\begin{problem}{Simple Probability II}
 What's the probability of rolling a fair $6$-sided die and seeing an odd
 number?

 \begin{solution}
  There are six possible possibilities, and in three of them ($1$, $3$, and
  $5$), an odd number is shown. We have \[
   \frac{3}{6} = \frac{1}{2}
  \]

  Thus the probability is \Ans{\frac{1}{2}}.
 \end{solution}
\end{problem}

\begin{problem}{Simple Probability III}
 What's the probability of picking a red marble from a bag that has $5$ red
 marbles and $7$ blue marbles?

 \begin{solution}
  There are $5+7=12$ total possibilities, and in $5$ of them, a red marble is
  drawn. Thus the probability is \Ans{\frac{5}{12}}.
 \end{solution}
\end{problem}

\section{Independent Events}

Sometimes we have more than one event, and the outcome of one event does not
change the next. These events would be called ``independent''. For instance,
in each of the following cases, we have independent events:

\begin{enumerate}
 \item I have two rabbits and a dog. I pick a pet at random; then I pick a
 different pet at random.
 \item I roll two six-sided dice.
 \item I roll one six-sided die twice.
\end{enumerate}

\section{Joint Probabilities}

When two events $A$ and $B$ are independent, we have the following rule for the
chances that both happen:

\begin{equation}
 P(A \cap B) = P(A) \times P(B)
\end{equation}

We can see this identity by drawing a tree and counting the number of
possibilities where both events come true.

\begin{problem}{Joint Probability I}
 A bag contains $5$ red marbles and $7$ blue marbles. I draw a marble from the
 bag and then flip a coin. What's the chance of flipping heads and drawing a red
 marble?
\end{problem}

\begin{problem}{Joint Probability II}
 A bag contains $10$ red marbles and $5$ blue marbles. I draw a marble from the
 bag at random, then I put it back. Then I draw another marble from the bag at
 random. What's the probability of drawing two blue marbles?
\end{problem}

\section{Dependent Events}

When two events $A$ and $B$ are not independent, we can still calculate the
probability nevertheless. We just need to be careful to update the probability.

\begin{problem}{Drawing Cards I}

 A deck of cards (without jokers) has $13$ cards of each of the $4$ suits ($52$
 cards total). What's the probability of drawing two spades without replacement?

 \begin{solution}
  \begin{align*}
   \frac{13}{52} \times \frac{12}{51}
   &= \frac{1}{4} \times \frac{12}{51} \\
   &= \frac{12}{204} \\
   &= \frac{1}{17}
  \end{align*}
  so the probability is \Ans{\frac{1}{17}}.
 \end{solution}

\end{problem}

\begin{problem}{Drawing Cards II}
 What about the probability of drawing two aces (without replacement) from a
 deck of cards (without jokers)?

 \begin{solution}
  \begin{align*}
   \frac{4}{52} \times \frac{3}{51}
   &= \frac{1}{13} \times \frac{1}{17} \\
   &= \frac{1}{221}
  \end{align*}
  so the probability is \Ans{\frac{1}{221}}.
 \end{solution}
\end{problem}

\section{Expected Value}

The expected value is the mean of all possibilities. For instance, the expected
value of rolling a six-sided die is $3.5$.


% Glossaries and List of Figures
\printglossaries

\cleardoublepage
\addcontentsline{toc}{chapter}{\listfigurename}
\listoffigures
\end{document}
