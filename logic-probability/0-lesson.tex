\documentclass[letterpaper,10pt]{article}
\usepackage[utf8]{inputenc}
\usepackage{amsmath}
\usepackage{amssymb}

\title{Propositional Calculus 1: Logical Thinking}
\author{Fengyang Wang}

\begin{document}
\maketitle

Logical thinking is the ability to make logical and correct conclusions
from givens.

\section{Propositions}

A proposition is any statement. Example: ``It is sunny outside.''

\subsection{Negation}

We can negate this proposition like this: ``It is not sunny outside.''

\subsection{Conjunction}

We can create a proposition that is true only when both components
are true by using the word ``and''. For example: ``It is sunny outside
and it is warm outside.''

\subsection{Disjunction}

We can create a proposition that is true when either component is
true, or if both components are true. For example: ``It is sunny outside
or it is warm outside.''

\section{Arguments}

An argument is a list of propositions ending in a statement that is
a logical conclusion of the other propositions. For instance:

\begin{align*}
 1.~&\text{It is daytime.} \\
 2.~&\text{There are no clouds.} \\
 3.~&\text{If it is daytime and there are no clouds, then it is sunny.} \\
 \hline
 4.~&(1), (2) \to \text{It is daytime and there are no clouds.} \\
 \hline
 \therefore~&(4), (3) \to \text{It is sunny.}
\end{align*}

\section{Rules}

There are rules to make sure all statements are logical. We'll cover them
below.

\subsection{Double negative}
\begin{equation*}
 \lnot\lnot p \implies p
\end{equation*}

\subsection{Conditional elimination}
\begin{equation*}
 \{p, p \implies q\} \implies q
\end{equation*}

\subsection{Conjunction introduction}

\begin{equation*}
 \{p, q\} \implies p \land q
\end{equation*}

\subsection{Conjunction elimination}

\begin{equation*}
 p \land q \implies p \\
 p \land q \implies q
\end{equation*}

\subsection{Disjunction introduction}

\begin{equation*}
 p \implies p \lor q \\
 q \implies p \lor q
\end{equation*}

\subsection{Disjunction elimination}

\begin{equation*}
 \{p \implies r, q \implies r, p \lor r\} \implies r
\end{equation*}

\subsection{Modus tollens}

Denying the consequent.

\begin{equation*}
 \{p \implies q, \not q\} \implies \not p
\end{equation*}

\section{Not logical}
These are not rules:

\begin{itemize}
 \item Denying the antecedent.
 \item Affirming the consequent.
\end{itemize}


\end{document}\grid
