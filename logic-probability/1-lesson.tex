\documentclass[letterpaper,10pt]{article}
\usepackage[utf8]{inputenc}
\usepackage{amsmath}
\usepackage{amssymb}

\title{Probability 1}
\author{Fengyang Wang}

\begin{document}
\maketitle

\section{Review of Fractions}
\begin{enumerate}
 \item \(
        \frac{3}{8} \times \frac{2}{7} = \frac{6}{56} = \frac{3}{28}
       \)
 \item \(\frac{5}{9} \times \frac{2}{5} = \frac{10}{45} = \frac{2}{9}\)

\end{enumerate}

\section{Simple Probability}

We can list all possible outcomes and generate the probability of one
particular outcome happening.

\begin{equation}
 P(A) = \frac{\text{Possibilities where $A$ occurs}}{\text{Total possible possibilities}}
\end{equation}

\subsection{Examples}

\begin{enumerate}
 \item What's the probability of flipping a fair coin and seeing heads?
 \item What's the probability of rolling a fair 6-sided die and seeing an odd number?
 \item What's the probability of picking a red marble from a bag that has $5$ red marbles and $7$ blue marbles?
\end{enumerate}

\section{Independent Events}

Sometimes we have more than one event, and the outcome of one event does not change the next. These events
would be called ``independent''.

\subsection{Examples}

\begin{enumerate}
 \item I have two rabbits and a dog. I pick a pet at random; then I pick a different pet at random.
 \item I roll two six-sided dice.
 \item I roll one six-sided die twice.
\end{enumerate}

\section{Joint Probabilities}

When two events $A$ and $B$ are independent, we have the following rule for the chances that both happen:

\begin{equation}
 P(A \cap B) = P(A) \times P(B)
\end{equation}

We can see this identity by drawing a tree and counting the number of possibilities where both events come
true.

\subsection{Examples}

\begin{enumerate}
 \item A bag contains $5$ red marbles and $7$ blue marbles. I draw a marble from the bag and then
 flip a coin. What's the chance of flipping heads and drawing a red marble?
 \item A bag contains $10$ red marbles and $5$ blue marbles. I draw a marble from the bag at random, then I put
 it back. Then I draw another marble from the bag at random. What's the probability of drawing two blue marbles?
\end{enumerate}

\section{Dependent Events}

When two events $A$ and $B$ are not independent, we can still calculate the probability nevertheless. We just
need to be careful to update the probability.

\subsection{Examples}
\begin{enumerate}
 \item A deck of cards (without jokers) has $13$ cards of each of the $4$ suits ($52$ cards total). What's
 the probability of drawing two spades without replacement?
 $\frac{13}{52} \times \frac{12}{51} = \frac{1}{4} \times \frac{12}{51} = \frac{12}{204} = \frac{1}{17}$
 \item What about the probability of drawing two aces?
 $\frac{4}{52} \times \frac{3}{51} = \frac{1}{13} \times \frac{1}{17} = \frac{1}{221}$
\end{enumerate}

\section{Expected Value}

The expected value is the mean of all possibilities. For instance, the expected value of rolling a six-sided
die is $3.5$.

\end{document}