\documentclass[letterpaper,10pt]{article}
\usepackage[utf8]{inputenc}
\usepackage{amsmath}
\usepackage{amssymb}

\title{Averages}
\author{Tony Hu \& Fengyang Wang}

\begin{document}
\maketitle

\section{Data}

Suppose we have a list of numerical data. What is data? Data  consists of the
results of some kind of experiment. For example, if I measure the height of
everyone in this room in centimetres, I would get some data.

Let's say I conducted that experiment, and I got the following answers: $[105,
115, 115, 115, 125, 125, 145]$. This is data, because it consists of the results
of my experiment. It's also \emph{numerical} data because it's made up of
numbers.

\section{Measures of centrality}

Suppose I wanted to know what the \emph{typical} student's height was. I want a
single number that represents the results of the whole experiment. It turns out
that there is not just a single way to do this!

The first way is to find which numbers show up most often. In our example, the
number $115$ shows up most. This means that more students had a height of $115$
than any other height. We call this the \emph{mode}. The mode is usually easy to
find, but it has several disadvantages. Firstly, there could possibly be more
than one mode, if more than one number shows up most often. For example, if the
$145$ here were $125$, then both $115$ and $125$ would be modes. Secondly, the
mode is very susceptible to error from incorrect measurements. A single mistake,
such as a $110$ being recorded as $115$, can cause a large change in the mode.

The second way is to put the numbers in order and look at which one is in the
middle. That number is called the \emph{median}. In our example, the median is
$115$. When there are an even number of data, the median is harder to calculate.
Usually we take the two in the middle, sum them, and divide by two.

Finally, the third way is to add all the numbers up and divide by the total
number of numbers. This is called the \emph{average} (sometimes also called the
\emph{mean}). The average is not always a nice number to calculate. In are
example, the sum is $845$, and when we divide by $7$ we get about
$120.\overline{714285}$.

\section{Example}
You can do this problem however you choose. You can work with others if you
want, or work alone.

\begin{quotation}
 The average of ten numbers is $100$. The smallest of the ten numbers is $19$.
 What is the average of the nine remaining numbers?
\end{quotation}

\end{document}
