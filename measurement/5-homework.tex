\documentclass[14pt,letterpaper]{article}

\newcommand{\IncludePath}{../include}
\usepackage{extsizes}
\usepackage{titling}

\usepackage{tikz}
\usetikzlibrary{shapes}

\usepackage{amssymb,amsmath,amsthm}
\usepackage{enumerate}
\usepackage[margin=0.8in]{geometry}
\usepackage{graphicx,ctable,booktabs}
\usepackage{fancyhdr}
\usepackage[utf8]{inputenc}
\usepackage{gensymb}

\makeatletter
\newenvironment{problem}{\@startsection
       {section}
       {1}
       {-.2em}
       {-3.5ex plus -1ex minus -.2ex}
       {2.3ex plus .2ex}
       {\pagebreak[3]
       \large\bf\noindent{Problem }
       }
       }
\makeatother

\pagestyle{fancy}
\lhead{\thetitle}
\chead{}
\rhead{\thepage}
\lfoot{\small\scshape Olympic Math}
\cfoot{}
\rfoot{}
\renewcommand{\headrulewidth}{.3pt}
\renewcommand{\footrulewidth}{.3pt}
\setlength\voffset{-0.25in}
\setlength\textheight{648pt}
\setlength\headheight{15pt}

\newcommand{\blankA}{\underline{\hspace{1em}}}
\newcommand{\blankB}{\underline{\hspace{2em}}}
\newcommand{\blankC}{\underline{\hspace{3em}}}
\newcommand{\blankD}{\underline{\hspace{4em}}}
\newcommand{\blankE}{\underline{\hspace{5em}}}
\newcommand{\blankF}{\underline{\hspace{6em}}}



\title{Theorems}
\author{Name: \underline{\hspace{5cm}}}
\date{November 5, 2016}

\begin{document}
\HomeworkTitle

\thispagestyle{empty}

\begin{problem}{Squares \& Square Roots}
  Calculate each of the following. Be careful with units.
  \begin{itemize}
    \item \(10^2 = \blankD\)
    \item \({\left(\SI{2}{\metre}\right)}^2 = \blankD\)
    \item \(\sqrt{36} = \blankD\)
    \item \(\sqrt{\SI{81}{\metre\squared}} = \blankD\)
  \end{itemize}
\end{problem}

\begin{problem}{Pythagorean Theorem}
 Find \(x\). Use the Pythagorean theorem to find unknown sides.

 \begin{center}
   \begin{tikzpicture}
     \coordinate (A) at (0, 0);
     \coordinate (B) at (6, 0);
     \coordinate (C) at (6, 2.5);
     \coordinate (D) at (7.2, 1.6);
     \draw[thick] (A) -- (B) node[midway,below] {\SI{12}{\metre}}
                      -- (C)
                      -- cycle node[midway,above] {\(x\)};
     \draw[thick] (D) -- (B) node[midway,right] {\SI{4}{\metre}}
                      -- (C)
                      -- cycle node[at start,right=3pt] {\SI{3}{\metre}};
     \tkzMarkRightAngle(A,B,C)
     \tkzMarkRightAngle(B,D,C)
   \end{tikzpicture}
 \end{center}
\end{problem}

\begin{problem}{Challenge I}
  Pythagoras thought that \(\sqrt{2}\) must be a rational number (that is, a
  fraction). It is possible to get quite close --- for example,
  \(\frac{99}{70}\) is almost \(\sqrt{2}\): \[
    {\left(\frac{99}{70}\right)}^2 \approx 2.000205
  \] but it turns out that there is no rational number (fraction) that is
  exactly \(\sqrt{2}\). Using this fact, is there a rational number that is the
  square root of \(8\)? Why or why not?
\end{problem}

\begin{problem}{Challenge II}
  In this problem, we will look at the Pythagorean theorem from the point of
  view of vectors. Let \(\mathbf{v}=(x, y)\) be a two-dimensional vector.
  Define the ``norm'' of vector \(\|\mathbf{v}\|\) to be: \[ \|\mathbf{v}\| =
  \sqrt{x^2 + y^2} \]

  There are a lot of variables in the above, so don't worry if it's confusing.
  Let's look at a few examples. If \(\mathbf{u} = (3, 4)\), then \[
    \|\mathbf{u}\| = \sqrt{3^2 + 4^2} = \sqrt{9 + 16} = \sqrt{25} = 5
  \]

  If \(\mathbf{v} = (12, 5)\), then \[
    \|\mathbf{v}\| = \sqrt{12^2 + 5^2} = \sqrt{144 + 25} = \sqrt{169} = 13
  \]

  A good way to think about the norm is that it is the length of a vector. Note
  in particular the similarity in the definition above and the Pythagorean
  theorem. The diagram below shows vectors \(\mathbf{u}\) and \(\mathbf{v}\),
  as defined above, and their norms:

  \begin{center}
    \begin{tikzpicture}
      \draw[step=0.5cm,black!30] (0,0) grid (7,4);
      \draw[line width=1mm,->] (0,0) -- (1.5,2) node[anchor=south]
      {\(\|\mathbf{u}\|=5\)};
      \draw[line width=1mm,->] (0,0) -- (6,2.5) node[anchor=north]
      {\(\|\mathbf{v}\|=13\)};
    \end{tikzpicture}
  \end{center}

  Say that \(\mathbf{w} = (8, 6)\). Then if you recall from two weeks ago, we
  defined \(2\mathbf{w} = \mathbf{w} + \mathbf{w} = (16, 12)\).

  \begin{itemize}
    \item Find \(\|2\mathbf{w}\|\). (You may use a calculator.)
    \item Find \(2\times\|\mathbf{w}\|\). If you did it correctly, this should
    be the same as your answer above. Do you think this is always going to
    happen for all vectors \(\mathbf{w}\)?
    \item Make a convincing argument why your belief is correct, one way or the
    other.
  \end{itemize}
\end{problem}

\end{document}
