\documentclass[a4paper,10pt]{report}

\newcommand{\IncludePath}{../include}
\newcommand{\ProjectName}{Grade 4 Olympic Math}
\usepackage{extsizes}
\usepackage{titling}

\usepackage{tikz}
\usetikzlibrary{shapes}

\usepackage{amssymb,amsmath,amsthm}
\usepackage{enumerate}
\usepackage{graphicx,ctable,booktabs}
\usepackage{fancyhdr}
\usepackage[utf8]{inputenc}
\usepackage{gensymb}

\usepackage[toc]{glossaries}

\makeatletter
\newenvironment{problem}{\@startsection
       {subsection}
       {1}
       {-.2em}
       {-3.5ex plus -1ex minus -.2ex}
       {2.3ex plus .2ex}
       {\pagebreak[3]
       \large\bf\noindent{Problem }
       }
       }
\makeatother

\makeatletter
\g@addto@macro\@floatboxreset\centering
\makeatother

\newenvironment{solution}
{ \vspace{1em} \noindent \textbf{Solution:} }
{  }

\pagestyle{fancy}
\lhead{\thetitle}
\chead{}
\rhead{\thepage}
\lfoot{\small\scshape \ProjectName}
\cfoot{}
\rfoot{}
\renewcommand{\headrulewidth}{.3pt}
\renewcommand{\footrulewidth}{.3pt}
\setlength\voffset{-0.25in}
\setlength\textheight{648pt}
\setlength\headheight{15pt}

\newcommand{\Ans}[1]{\framebox{$#1$}}
\newcommand{\SAA}[1]{\Switch{\Ans{#1}}{\blankA}}
\newcommand{\SAB}[1]{\Switch{\Ans{#1}}{\blankB}}
\newcommand{\SAC}[1]{\Switch{\Ans{#1}}{\blankC}}
\newcommand{\SAD}[1]{\Switch{\Ans{#1}}{\blankD}}
\newcommand{\SAE}[1]{\Switch{\Ans{#1}}{\blankE}}
\newcommand{\SAF}[1]{\Switch{\Ans{#1}}{\blankF}}
\newcommand{\STA}[1]{\Switch{\AnsT{#1}}{\blankA}}
\newcommand{\STB}[1]{\Switch{\AnsT{#1}}{\blankB}}
\newcommand{\STC}[1]{\Switch{\AnsT{#1}}{\blankC}}
\newcommand{\STD}[1]{\Switch{\AnsT{#1}}{\blankD}}
\newcommand{\STE}[1]{\Switch{\AnsT{#1}}{\blankE}}
\newcommand{\STF}[1]{\Switch{\AnsT{#1}}{\blankF}}
\newcommand{\AnsT}[1]{\framebox{#1}}
\newif\ifanswers
\newcommand{\Switch}[2]{\ifanswers#1\else#2\fi}
\newcommand{\MCSelect}[1]{\Switch{\AnsT{#1}}{#1}}
\newcommand{\TFTrue}{\MCSelect{True}~~False}
\newcommand{\TFFalse}{True~~\MCSelect{False}}

\newcommand{\blankA}{\underline{\hspace{1em}}}
\newcommand{\blankB}{\underline{\hspace{2em}}}
\newcommand{\blankC}{\underline{\hspace{3em}}}
\newcommand{\blankD}{\underline{\hspace{4em}}}
\newcommand{\blankE}{\underline{\hspace{5em}}}
\newcommand{\blankF}{\underline{\hspace{6em}}}


\newglossaryentry{algorithm}
{
  name=algorithm,
  description={step-by-step procedure for performing a calculation
  according to well-defined rules (Wikipedia)}
}

\newglossaryentry{acute angle}
{
  name=acute angle,
  description={an angle measuring less than a right angle (\SI{90}{\degree} or a
  quarter of a turn); such angles are typically characterized as being sharp}
}

\newglossaryentry{angle}
{
  name=angle,
  description={the figure formed by two rays, called the sides of the angle,
  sharing a common endpoint, called the vertex of the angle (Wikipedia)}
}

\newglossaryentry{coincident}
{
  name=coincident,
  description={a description of two objects which occupy exactly the same space}
}

\newglossaryentry{domain}
{
  name=domain,
  description={the set of values that are possible inputs for a function}
}

\newglossaryentry{endpoint}
{
  name=endpoint,
  description={an extreme point of a line segment or ray; line segments have two
  endpoints whereas rays have just one}
}

\newglossaryentry{function}
{
  name=function,
  description={object that may take any allowed input and will produce a
  single associated output for that input; alternatively, relation for
  which each input value has exactly one related output value}
}

\newglossaryentry{integer}
{
  name=integer,
  description={positive or negative whole number, or $0$; for example,
  $-8$, $2000$}
}

\newglossaryentry{line}
{
  name=line,
  description={a straight one-dimensional object that extends forever in both
  directions}
}

\newglossaryentry{line segment}
{
  name=line segment,
  description={a straight one-dimensional object terminated at both ends}
}

\newglossaryentry{multiplicand}
{
  name=multiplicand,
  description={the number that is being multiplied; for instance, in
  $2\times3=6$, the multiplicand is $2$}
}

\newglossaryentry{multiplier}
{
  name=multiplier,
  description={the factor to multiply a number by; for instance, in
  $2\times3=6$, the multiplier is $3$}
}

\newglossaryentry{obtuse angle}
{
  name=obtuse angle,
  description={an angle measuring more than a right angle (\SI{90}{\degree} or a
  quarter of a turn) but less than a straight angle (\SI{180}{\degree} or a half
  of a turn)}
}

\newglossaryentry{parallel}
{
  name=parallel,
  description={a description of two lines that never intersect at any point}
}

\newglossaryentry{parity}
{
  name=parity,
  description={decribes whether an integer is even or odd}
}

\newglossaryentry{plane}
{
  name=plane,
  description={a two-dimensional flat surface}
}

\newglossaryentry{plane geometry}
{
  name=plane geometry,
  description={the study of figures on a plane (a two-dimensional flat surface)}
}

\newglossaryentry{product}
{
  name=product,
  description={the result of a multiplication; for instance, in
  $2\times3=6$, the product is $6$}
}

\newglossaryentry{range}
{
  name=range,
  description={the set of values that are possible outputs for a function}
}

\newglossaryentry{ray}
{
  name=ray,
  description={a straight one-dimensional object terminated at one end and
  extending forever in the other direction}
}

\newglossaryentry{right angle}
{
  name=right angle,
  description={an angle measuring \SI{90}{\degree} or a quarter of a turn; angle
  between two rays intersecting in an L shape}
}

\newglossaryentry{right triangle}
{
  name=right triangle,
  description={an triangle with one \SI{90}{\degree} (right) angle}
}

\newglossaryentry{summand}
{
  name=summand,
  description={something which is being added; for instance, in $1+2=3$,
  the two summands are $1$ and $2$}
}

\newglossaryentry{reflex angle}
{
  name=reflex angle,
  description={an angle measuring more than \SI{180}{\degree} or a half of a
  turn, but less than \SI{360}{\degree} or a full turn}
}

\newglossaryentry{straight angle}
{
  name=straight angle,
  description={an angle measuring \SI{180}{\degree} or a half of a turn; angle
  between two rays in opposite directions}
}

\newglossaryentry{transitivity}
{
  name=transitivity,
  description={the property of certain relations that specifies if an
  element $a$ is related to $b$, and the element $b$ is related to $c$,
  then $a$ is similarly related to $c$}
}

\makeglossaries

\title{Measurement}
\author{Fengyang Wang}

\begin{document}

\begin{abstract}
 This is an introduction to more advanced topics in measurement suitable for an
 advanced Grade 4 audience. These notes were prepared for the Grand River
 Chinese School.

 This topic is new this year, though it adapts and incorporates content from the
 geometry unit I taught in previous years. Historically, I taught a four-class
 geometry primer discussing angles, angles of polygons, and area. Confusion has
 always arisen with units of measurement, which I suspect is due to the lack of
 education about units and dimension in school. Hence, I have substantially
 revised and expanded the curriculum to include much more discussion about
 dimension and units.

 However, the quality of most of these notes is untested, and I may not end up
 presenting all of the information. I anticipate that each chapter will take
 around two to four hours to present, depending on the level of detail.

 Although the notes are intended to be presented to a young audience, they are
 written for a teacher and not for a student. Many of the terms used will not be
 familiar to the students. They require explanation.
\end{abstract}

\maketitle

\tableofcontents

\chapter{Introduction}

This unit is about geometry and measuring the physical world. Our focus will be
on figures on a two-dimensional surface, and the quantities we can use to
measure those figures.

\section{Planes}

Often, we like to do geometry on a flat surface. Note that this is not the only
kind of geometry; we could do geometry in three dimensions, for example. Many
physicists and mathematicians investigate even stranger geometries, such as
geometries with time. However, flat geometry is much easier to understand and
visualize, and is interesting in its own right. In fact, even if we work in
three or more dimensions, we often still study interactions that occur on planes
within a higher-dimensional space.

The study of geometry on a flat surface is called \gls{plane geometry}. The flat
surfaces that we do our work on are called \glspl{plane}. For this unit, we will
be doing plane geometry.

\section{Lines}

To understand angles, first we must consider straight lines. In geometry, there
is a specific definition for what is a \gls{line}. For example, the object
depicted in Figure~\ref{an:line-segment} is not a line. Why not? In regular
geometry, a line is defined as continuing forever in both directions. Straight
objects that terminate at both ends are instead called \glspl{line segment}.

\begin{figure}
 \fgLineSegment{}

 \caption{A line segment --- not a line!}
 \label{an:line-segment}
\end{figure}

We cannot physically draw an infinitely long line. Instead, we use arrows at
both ends to denote that our line continues forever at both ends.
Figure~\ref{an:line} demonstrates how arrows can be used.

\begin{figure}
 \fgLine{}

 \caption{One way to draw a line continuing infinitely in both directions}
 \label{an:line}
\end{figure}

The existence of line segments, which are terminated at both ends, and lines,
which continue forever at both ends, raises a question. Are there geometric
objects that terminate at one end, and continue forever at the other end? Yes!
These kinds of objects are called \glspl{ray}. They behave like lines in one
direction, and like line segments in the other direction. Figure~\ref{an:ray}
shows how one might draw a ray.

\begin{figure}
 \fgRay{}

 \caption{One way to draw a ray terminating at one end and continuing forever
 in the other direction}
 \label{an:ray}
\end{figure}

A ray has a direction: the end that it terminates at is where the ray starts,
and travelling toward its infinite end is considered moving in the ray's
direction. You may have heard of a ``ray of light''. This phrase comes from the
well-known behaviour of light: if we shine a flashlight, the resulting ray goes
on forever (though perhaps with ever-decreasing intensity) in the direction the
light is shown. But there is a clear starting point. Behind the flashlight,
there is no light.

Although this could be confusing at first, the point that the ray ends at is
both a start and an end. It is a start because the ray begins at that point if
we travel along the ray in the direction it points. But it is also an end,
because equivalently we could say that the ray ends at that point if we travel
along the opposite direction. To minimize the confusion, from now on we will
consistently call that point the \gls{endpoint} of the ray.

\section{Properties of Lines}

There are a few facts about straight lines, line segments, and rays on a plane
that are useful to know:

\begin{itemize}
 \item If two lines are the same line (they occupy the same space), then they
 are called \gls{coincident}.
 \item The straight line segment is the shortest route between two points.
 \item If two lines never intersect, they are called \gls{parallel}.
 \item Two distinct (not coincident) straight lines will never intersect at two
 or more points.
\end{itemize}

\begin{problem}{Intersecting Lines I}
 Three lines are drawn on a plane, no two parallel or coincident. Is it possible
 for there to be four points on the plane that are on more than one line?

 \begin{solution}
  \AnsT{No}. There can only be up to three points, since there is only up to one
  point on both the first and second lines, up to one point on both the first
  and third lines, and up to one point on both the second and third lines.
  Therefore, there can only be up to \[ 1 + 1 + 1 = 3 \] points in total that
  are on more than one line.
 \end{solution}
\end{problem}

\begin{problem}{Intersecting Lines II}
 Three lines are drawn on a plane, no two parallel or coincident. Is it possible
 for there to be only one point that is on more than one line?

 \begin{solution}
  \AnsT{Yes}. We just require three lines that intersect together at one point.
  This is basically the design of the asterisk character *; see
  Figure~\ref{an:asterisk}.
 \end{solution}
\end{problem}

\begin{figure}
 \fgAsterisk{}

 \caption{Three lines interesting at a single point}
 \label{an:asterisk}
\end{figure}

\section{Length}

It does not make sense to talk about the length of a line or ray, because those
objects extend forever. Line segments, on the other hand, do have a well-defined
length. In the physical world, length is typically measured in units of
kilometres, metres, centimetres, or millimetres.

Length is related to distance. In fact, the distance between two points is the
length of the straight line segment connecting those two points. Therefore,
distance is also measured in units of length.

It is meaningless to say that the distance between two objects is $3$---but it
is meaningful to say that the distance is \SI{3}{\metre}. The unit is not
optional; it is a fundamental component of what the quantity means.

That is not to say, however, that units cannot be changed. Indeed, it is
possible to convert between various units of length. The SI system of units in
use around the world makes conversions especially convenient. It is useful to
remember that:

\begin{itemize}
  \item $\SI{1}{\kilo\metre}=\SI{1000}{\metre}$
  \item $\SI{1}{\metre}=\SI{100}{\centi\metre}$
  \item $\SI{1}{\centi\metre}=\SI{10}{\milli\metre}$
\end{itemize}

\begin{problem}{Unit Conversion I}
  Convert \SI{1}{\kilo\metre} to millimetres.

  \begin{solution}
    Note \(
      \SI{1}{\kilo\metre} = \SI{1000}{\metre}
    \) and \(
      \SI{1}{\metre} = \SI{1000}{\milli\metre}
    \), so \[
      \SI{1}{\kilo\metre} = 1000 \cdot \SI{1}{\metre} = 1000 \cdot
      \SI{1000}{\milli\metre} = \Ans{\SI{1000000}{\milli\metre}}
    \]
  \end{solution}
\end{problem}

\chapter{Angles}

\section{Angles between Rays}

Consider Figure~\ref{an:tworays}, which depicts two rays that share a common
endpoint. Then look at Figure~\ref{an:tworaysfar}, which also depicts two rays
with a common endpoint, but further apart. At least, to our eyes the two rays
look further apart. But what does it mean for rays to be far apart?

This is no longer a simple matter of distance. Because the rays share a common
endpoint, the distance between them is zero. That means that the rays are no
more distant from each other in Figure~\ref{an:tworaysfar} than in
Figure~\ref{an:tworays}. Rather, we need another way to measure how far two rays
are. We will use the \gls{angle} between the rays to describe this.

\begin{figure}
 \fgTwoRays{}

 \caption{Two rays sharing a common endpoint}
 \label{an:tworays}
\end{figure}

\begin{figure}
 \fgTwoRaysFar{}

 \caption{Two rays sharing a common endpoint, but further apart}
 \label{an:tworaysfar}
\end{figure}

The size (also called ``measure'') of an angle is measured in degrees. We use
\SI{90}{\degree} to denote the angle of an L. Therefore, for example,
\SI{45}{\degree} would be half the size of that angle. We will soon develop more
intuition into what angles measuring a particular number of degrees look like.

To make it clear that we are talking about the angle between two rays, we can
draw figures with part of a circle sandwiched between two rays. This part-circle
indicates that the figure's subject is the angle between the rays. Often, we can
add additional description near the part-circle, such as a measure of how large
the angle is. In Figure~\ref{an:anglethirty}, a \SI{30}{\degree} angle is
depicted.

\begin{figure}
 \fgAngleN{30}

 \caption{An angle measuring \SI{30}{\degree} with a part-circle highlighting
 it}
 \label{an:anglethirty}
\end{figure}

\section{Classifying Angles by Measure}

\subsection{Right and Straight Angles}

When two rays meet either other in an L shape, the resulting angle is called a
\gls{right angle}. Figure~\ref{an:tworaysright} depicts a right angle. Note that
we often use a square to represent a right angle instead of drawing part of a
circle. Right angles are everywhere. For instance, rectangles have four right
angles. Take a look around the classroom and note how many right angles you see.

\begin{figure}
 \fgTwoRaysRight{}

 \caption{Two rays meeting at a right angle, with the right angle labelled using
 a square shape}
 \label{an:tworaysright}
\end{figure}

Another special kind of angle occurs when two rays of opposite directions meet
at a point. Since the rays together form a straight line, such angles are called
\glspl{straight angle}. Figure~\ref{an:tworaysstraight} depicts a straight
angle.

\begin{figure}
 \fgAngleN{180}

 \caption{Two rays meeting at a straight angle}
 \label{an:tworaysstraight}
\end{figure}

\subsection{Acute and Obtuse Angles}

The right angle, as mentioned earlier, is given a size of \SI{90}{\degree}.
Right angles are so important that other angles are classified based off their
size in relation to the right angle. Angles measuring more than \SI{0}{\degree},
but less than \SI{90}{\degree}, are classified as \glspl{acute angle}. Angles
measuring more than \SI{90}{\degree}, but less than \SI{180}{\degree}, are
classified as \glspl{obtuse angle}.

\begin{problem}{Acute, Right, Obtuse}
 Circle the correct category for each angle.

 \begin{itemize}
  \item \SI{41}{\degree} \hfill Acute~~Right~~Obtuse
  \item \SI{175}{\degree} \hfill Acute~~Right~~Obtuse
 \end{itemize}

 \begin{solution}
  Because $\SI{0}{\degree} < \SI{41}{\degree} < \SI{90}{\degree}$,
  \SI{41}{\degree} is an \AnsT{acute} angle. Because $\SI{90}{\degree} <
  \SI{175}{\degree} < \SI{180}{\degree}$, \SI{175}{\degree} is an \AnsT{obtuse}
  angle.
 \end{solution}
\end{problem}

\subsection{Reflex Angles}

When two rays intersect, there are actually two angles that one can draw. One of
these angles is an acute, right, obtuse, or straight angle; the other is either
a straight angle or a so called \gls{reflex angle}. Reflex angles are greater
than \SI{180}{\degree} but less than \SI{360}{\degree}. A reflex angle is
depicted in Figure~\ref{an:tworaysreflex}; note that it combines with an acute
angle to form the full circle.

Reflex angles are not as important as acute, right, and obtuse angles; therefore
we will not focus on them much for the remainder of this course.

\begin{figure}
 \fgAngleN{300}

 \caption{Two rays meeting at an acute angle, but the reflex angle is
 highlighted}
 \label{an:tworaysreflex}
\end{figure}

\section{Angle Arithmetic}

We can add two angles placed next to each other together, forming a bigger
angle. The measure of this bigger angle is the sum of the measures of the two
components. In Figure~\ref{an:acutesum}, we see two acute angles measuring
\SI{30}{\degree} and \SI{60}{\degree}, and their sum is \SI{90}{\degree}. We can
add degrees just by adding the numbers together and leaving the unit as-is.

\begin{figure}
 \fgAngleSumNM{30}{60}

 \caption{Two acute angles summing to a right angle; the sum is highlighted in
 blue}
 \label{an:acutesum}
\end{figure}

Similarly, we can subtract two angles if one is contained within the other. The
angle left over is a smaller angle. In Figure~\ref{an:obtusediff}, we see a
\SI{60}{\degree} angle taken away from a \SI{120}{\degree} angle, leaving
another \SI{60}{\degree} angle as the difference.

\begin{figure}
 \fgAngleDiffNM{100}{60}

 \caption{An acute angle is subtracted off an obtuse angle, leaving an acute
 angle; the difference is highlighted in blue}
 \label{an:obtusediff}
\end{figure}

\begin{problem}{Adding Angles I}
  Find each sum or difference of angles.

  \begin{itemize}
    \item $\SI{20}{\degree} + \SI{30}{\degree}$
    \item $\SI{155}{\degree} - \SI{65}{\degree}$
  \end{itemize}

  \begin{solution}
    We have \[ \SI{20}{\degree} + \SI{30}{\degree} = (20+30)\si{\degree} =
    \Ans{\SI{50}{\degree}} \]

    And \[ \SI{155}{\degree} + \SI{65}{\degree} =
    (155-65)\si{\degree} = \Ans{\SI{90}{\degree}} \]
  \end{solution}
\end{problem}

\begin{problem}{Adding Angles II}
  Determine whether each case is possible. If so, give an example. Otherwise,
  give a reason why it is not possible.

  \begin{itemize}
    \item The sum of two acute angles is an acute angle.
    \item An obtuse angle minus a right angle is an obtuse angle.
  \end{itemize}

  \begin{solution}
    The first case is \AnsT{possible}; we take for example \[
      \SI{30}{\degree} + \SI{40}{\degree} = \SI{70}{\degree}
    \] which is acute.

    The second case is \AnsT{not possible}. No obtuse angle is \SI{180}{\degree}
    or larger; therefore, subtracting \SI{90}{\degree} will always result in an
    angle less than \SI{90}{\degree}---an acute angle.
  \end{solution}
\end{problem}

\subsection{Complementary and Supplementary Angles}

If two angles labelled $\alpha$ and $\beta$ sum to \SI{90}{\degree}, then we say
that $\alpha$ and $\beta$ are complementary angles. If two angles labelled
$\alpha$ and $\beta$ sum to \SI{180}{\degree}, then we say that $\alpha$ and
$\beta$ are supplementary angles. You will not be required to memorize these
terms; the quiz and homework will remind you of what they mean.

\begin{problem}{Complementary Angles}
 Find the angle complementary to each of the following.

 \begin{itemize}
  \item \SI{81}{\degree} \hfill Complementary: \blankC \si{\degree}
  \item \SI{18}{\degree} \hfill Complementary: \blankC \si{\degree}
 \end{itemize}

 \begin{solution}
   We just calculate \[
    \SI{90}{\degree}-\SI{81}{\degree}=\Ans{\SI{9}{\degree}}
   \]
   and \[
    \SI{90}{\degree}-\SI{18}{\degree}=\Ans{\SI{72}{\degree}}
   \]
 \end{solution}
\end{problem}

\begin{problem}{Supplementary Angles}
 Find the angle supplementary to each of the following.

 \begin{itemize}
  \item \SI{81}{\degree} \hfill Supplementary: \blankC \si{\degree}
  \item \SI{18}{\degree} \hfill Supplementary: \blankC \si{\degree}
 \end{itemize}

 \begin{solution}
   We just calculate \[
    \SI{180}{\degree}-\SI{81}{\degree}=\Ans{\SI{99}{\degree}}
   \]
   and \[
    \SI{180}{\degree}-\SI{18}{\degree}=\Ans{\SI{162}{\degree}}
   \]
 \end{solution}
\end{problem}

\section{Angles as Turns}

Imagine rotating an object around some pivot point. How can we measure how much
an object has rotated? One way would be to measure the number of full rotations,
defining a full rotation to be a complete turn around the pivot. After a full
rotation, the object is in the same orientation that it was in originally.

Consider a straight bar with both tips coloured differently, initially oriented
horizontally. We draw the initial position here and define this position to be
the starting position.

TK.

\section{Angles of a Polygon}

TK.

\chapter{Units of Measurement}

In the first chapter, we saw that \si{\kilo\metre}, \si{\metre},
\si{\centi\metre}, and \si{\milli\metre} are units of length. We know that \[
  \SI{1}{\kilo\metre} = \SI{1000}{\metre}
\] but of course, \(1\ne1000\). What's going on here?

In the last chapter, we saw two units of measurement for angles: the turn and
the degree. We learned that one turn is equivalent to \SI{360}{\degree}. But
from common sense, we know that \(1 \ne 360\). How can it be?

The answer is quite simple: while \si{\kilo\metre} and \si{\metre} are both
units of length, they are not the same unit of length. Similarly, although both
the turn and the degree are units of angle measure, they are not the same unit
of angle measure.

TK.

\chapter{Length and Area}

TK.

\section{Area of a Triangle}

TK.

\begin{problem}{Area of a Triangle}
 Complete the table for each triangle.

 \begin{center}
 \begin{tabular}{|c|c|c|}
  \hline
  Base & Height & Area \\
  \hline
  \SI{15}{\metre} & \SI{8}{\metre} & \\
  & \SI{14}{\metre} & \SI{49}{\metre\squared} \\
  \SI{20}{\metre} & & \SI{100}{\metre\squared} \\
  \hline
 \end{tabular}
 \end{center}

 \begin{solution}
   We use the formula for the area of a triangle to obtain:

   \begin{center}
   \begin{tabular}{|c|c|c|}
    \hline
    Base & Height & Area \\
    \hline
    \SI{15}{\metre} & \SI{8}{\metre} & \Ans{\SI{60}{\metre\squared}} \\
    \Ans{\SI{7}{\metre}} & \SI{14}{\metre} & \SI{49}{\metre\squared} \\
    \SI{20}{\metre} & \Ans{\SI{10}{\metre}} & \SI{100}{\metre\squared} \\
    \hline
   \end{tabular}
   \end{center}
 \end{solution}
\end{problem}

\chapter{Geometric Theorems}

TK.


% Glossaries and List of Figures
\printglossaries

\cleardoublepage
\addcontentsline{toc}{chapter}{\listfigurename}
\listoffigures

\end{document}
