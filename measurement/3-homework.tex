\documentclass[14pt,letterpaper]{article}

\newcommand{\IncludePath}{../include}
\usepackage{extsizes}
\usepackage{titling}

\usepackage{amssymb,amsmath,amsthm}
\usepackage{enumerate}
\usepackage[margin=1in]{geometry}
\usepackage{graphicx,ctable,booktabs}
\usepackage{fancyhdr}
\usepackage[utf8]{inputenc}

\makeatletter
\newenvironment{problem}{\@startsection
       {section}
       {1}
       {-.2em}
       {-3.5ex plus -1ex minus -.2ex}
       {2.3ex plus .2ex}
       {\pagebreak[3]
       \large\bf\noindent{Problem }
       }
       }
\makeatother

\pagestyle{fancy}
\lhead{\thetitle}
\chead{}
\rhead{\thepage}
\lfoot{\small\scshape Grade 4 Olympic Math}
\cfoot{}
\rfoot{}
\renewcommand{\headrulewidth}{.3pt}
\renewcommand{\footrulewidth}{.3pt}
\setlength\voffset{-0.25in}
\setlength\textheight{648pt}
\setlength\headheight{15pt}


\title{Units of Measurement}
\author{Name: \underline{\hspace{5cm}}}
\date{October 22, 2016}

\begin{document}
\HomeworkTitle

\thispagestyle{empty}

\begin{problem}{Garden Dimensions}
 The area of a rectangular garden is \SI{36}{\metre\squared}. What are its
 dimensions if\ldots

 \begin{enumerate}[\hspace{1cm}a.]
   \item \ldots{}the width is \SI{3}{\metre}?
   \item \ldots{}the length and width are equal?
   \item \ldots{}the perimeter is \SI{26}{\metre}?
 \end{enumerate}
\end{problem}

\begin{problem}{Units of Money}
  Money also has units. In Canada, there are two kinds of units in common use:
  the dollar and the cent. We know that $\$1=100\text{\textcent}$. The average
  car costs $\$0.0000001$ of gasoline to drive \SI{1}{\milli\metre}. How much
  would the gasoline for a car to drive \SI{1}{\kilo\metre} cost, in cents?
  \hfill \blankC\textcent
\end{problem}

\begin{problem}{Dimensional Analysis}
  Exactly one of the following four equations is correct. Which is it? (Hint:
  think about units. You can eliminate many possibilities because the units
  are incorrect.)

  \begin{itemize}
    \item \(\SI{123891}{\metre} \times \SI{23241}{\metre}
    = \SI{15348979881}{\metre}\)
    \item \(\SI{1}{\metre} + \SI{1}{\metre} = \SI{2}{\centi\metre}\)
    \item \(\SI{17}{\kilo\metre} \times \SI{17}{\centi\metre} =
    \SI{2890}{\metre\squared}\)
    \item \(\SI{1}{\metre} \times \SI{1}{\metre} = \SI{1}{\kilo\gram}\)
  \end{itemize}
\end{problem}

\begin{problem}{Challenge II}
  A \emph{Euclidean vector} is an object with a magnitude and direction. We can
  represent displacements as vectors. For instance, suppose we make a map of
  the 2D plane, with each square grid cell \SI{1}{\centi\metre\squared} in
  size. We can measure a vector by recording the distance to the right and the
  distance upward that the vector displaces.

  \begin{center}
    \begin{tikzpicture}
      \draw[step=1cm,gray] (0,0) grid (6,6);
      \draw[line width=1mm,->] (0,0) -- (3,4) node[anchor=south]
      {\(\mathbf{u}\)};
      \draw[line width=1mm,->] (3,4) -- (5,2) node[anchor=north]
      {\(\mathbf{v}\)};
    \end{tikzpicture}
  \end{center}

  The vectors \(\mathbf{u}\) and \(\mathbf{v}\) are labelled. We say that \[
    \mathbf{u} = \left( \SI{3}{\centi\metre}, \SI{4}{\centi\metre} \right)
    \hspace{2em}
    \mathbf{v} = \left( \SI{2}{\centi\metre}, \SI{-2}{\centi\metre} \right)
  \] because vector \(\mathbf{u}\) displaces \SI{3}{\centi\metre} to the right
  and \SI{4}{\centi\metre} up, and vector \(\mathbf{v}\) displaces
  \SI{2}{\centi\metre} to the right and \SI{-2}{\centi\metre} up.

  \begin{itemize}
    \item Convert \(\mathbf{u}\) to millimetres. \hfill
    \(\mathbf{u} = \left(
    \blankC\,\si{\milli\metre}, \blankC\,\si{\milli\metre} \right)\)
    \item We define \(\mathbf{u} + \mathbf{v}\) to be the total displacement
    after displacing by \(\mathbf{u}\), then further displacing by
    \(\mathbf{v}\). Note that we can calculate this by considering the
    rightward and upward components separately. Find \(\mathbf{u} +
    \mathbf{v}\). \hfill \(\mathbf{u} + \mathbf{v} = \left(
    \blankC\,\si{\centi\metre}, \blankC\,\si{\centi\metre} \right)\)
    \item We define \(2\mathbf{u} = \mathbf{u} + \mathbf{u}\). Again, we can
    calculate this by considering the rightward and upward components
    separately. Find \(2\mathbf{u}\). \hfill \(2\mathbf{u} = \left(
    \blankC\,\si{\centi\metre}, \blankC\,\si{\centi\metre} \right)\)
    \item Compute (unrelated to the diagram above), being careful about unit
    conversion, \[
      4\left( \SI{2}{\metre}, \SI{5}{\metre} \right) +
      2\left( \SI{5}{\centi\metre}, \SI{10}{\centi\metre} \right)
      = \left( \blankC\,\si{\milli\metre}, \blankC\,\si{\milli\metre} \right)
    \]
  \end{itemize}
\end{problem}

\end{document}
