\documentclass[14pt,letterpaper]{article}

\newcommand{\IncludePath}{../include}
\usepackage{extsizes}
\usepackage{titling}

\usepackage{amssymb,amsmath,amsthm}
\usepackage{enumerate}
\usepackage[margin=1in]{geometry}
\usepackage{graphicx,ctable,booktabs}
\usepackage{fancyhdr}
\usepackage[utf8]{inputenc}

\makeatletter
\newenvironment{problem}{\@startsection
       {section}
       {1}
       {-.2em}
       {-3.5ex plus -1ex minus -.2ex}
       {2.3ex plus .2ex}
       {\pagebreak[3]
       \large\bf\noindent{Problem }
       }
       }
\makeatother

\pagestyle{fancy}
\lhead{\thetitle}
\chead{}
\rhead{\thepage}
\lfoot{\small\scshape Grade 4 Olympic Math}
\cfoot{}
\rfoot{}
\renewcommand{\headrulewidth}{.3pt}
\renewcommand{\footrulewidth}{.3pt}
\setlength\voffset{-0.25in}
\setlength\textheight{648pt}
\setlength\headheight{15pt}


\title{Quiz Review}
\author{Name: \underline{\hspace{5cm}}}
\date{November 12, 2016}

\begin{document}
\HomeworkTitle

\thispagestyle{empty}

\begin{problem}{Triangle Angle Sum}
 Find the angle marked by \(\alpha\). \hfill \(\alpha = \blankC \si{\degree}\)

 \begin{center}
   \begin{tikzpicture}
     \coordinate (A) at (0:3);
     \coordinate (B) at (0:0);
     \coordinate (C) at (50:3);
     \draw[thick] (A) -- (B) -- (C) -- cycle;
     \pic [draw, thick,
           "{\small\SI{50}{\degree}}", angle radius=1cm] {angle = A--B--C};
     \pic [draw, thick,
           "{\small\SI{65}{\degree}}", angle radius=1cm] {angle = C--A--B};
     \pic [draw, thick,
           "{\alpha}", angle radius=1cm] {angle = B--C--A};
   \end{tikzpicture}
 \end{center}
\end{problem}

\begin{problem}{Area of a Triangle}
 Find the area of this triangle. \hfill \(A = \blankC \SI{\degree}\)

 \begin{center}
   \begin{tikzpicture}
     \coordinate (A) at (6,0);
     \coordinate (B) at (0,0);
     \coordinate (C) at (7,3);
     \coordinate (HC) at ($(B)!(C)!(A)$);
     \draw[thick] (A) -- node[below] {\(6\)} (B) -- (C) -- cycle;
     \draw[red,line width=.3mm]
       (C) -- node[right,black] {\(3\)} (HC);
     \tkzMarkRightAngle[draw=red](B,HC,C)
   \end{tikzpicture}
 \end{center}
\end{problem}

\begin{problem}{Area of Lattice Polygons}
  Find the area of the shape below. Each grid square is
  \SI{1}{\centi\metre\squared}.
  \hfill \(A = \blankC~\si{\centi\metre\squared}\)

  \begin{center}
    \begin{tikzpicture}
      \draw[step=1cm] (-0.3,-0.3) grid (15.3,3.3);
      \draw[line width=1mm] (0,0) -- (2,3) -- (3,1) -- (5,2) -- (7,3) -- (8,1)
      -- (11,1) -- (13,3) -- (15,0) -- cycle;
    \end{tikzpicture}
  \end{center}
\end{problem}

\begin{problem}{Challenge}
  In past weeks, we saw that if \(\mathbf{v} = (x, y)\) is a vector, then we
  define \[
    2\mathbf{v} = \mathbf{v} + \mathbf{v} = (x + x, y + y) = (2x, 2y)
  \] and again, for sake of example, if \(\mathbf{w} = (3, 4)\), then
  \(2\mathbf{w} = (6, 8)\).

  But of course, we can multiply vectors by other numbers also, not just \(2\).
  In general, if \(a\) is a number, and \(\mathbf{v} = (x, y)\) is a vector,
  then \(a\mathbf{v} = (ax, ay)\). The characters \(ax\) mean \(a \times x\);
  we are leaving out the multiplication sign \(\times\) for convenience.

  Let us do some examples first. Note \[
    6(2, 4) = (6\times2, 6\times4) = (12, 24)
  \] and \[
    3(5, 10) = (3\times5, 3\times10) = (15, 30)
  \]

  Geometrically, multiplying vectors is like stretching that vector out. For
  example, the diagram below shows \(\mathbf{u} = (3, 1)\) and a stretched-out
  \(4\mathbf{u} = (12, 4)\).

  \begin{center}
    \begin{tikzpicture}
      \draw[step=0.5cm,black!30] (0,0) grid (7,3);
      \draw[line width=1mm,->,gray] (0,0) -- (6,2.0) node[anchor=south, black]
      {\(4\mathbf{u}\)};
      \draw[line width=1mm,->] (0,0) -- (1.5,0.5) node[anchor=south]
      {\(\mathbf{u}\)};
    \end{tikzpicture}
  \end{center}

  Say that \(\mathbf{u} = (3, 1)\) and \(\mathbf{w} = (2, 3)\). Then:

  \begin{itemize}
    \item Find \(5\mathbf{u}\).
    \item Find \(5\mathbf{w}\).
    \item By adding your responses to the two problems above, find
    \(5\mathbf{u} + 5\mathbf{w}\).
    \item Find \(\mathbf{u} + \mathbf{w}\).
    \item By multiplying your response to the answer above by \(5\), find
    \(5(\mathbf{u} + \mathbf{w})\).
    \item If you did this problem correctly, your responses to the third and
    fifth parts should be the same. Draw a picture to check this geometrically.
  \end{itemize}
\end{problem}

\end{document}
