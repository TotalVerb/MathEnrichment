\documentclass[letterpaper,10pt]{article}
\usepackage[utf8]{inputenc}
\usepackage{amsmath}
\usepackage{amssymb}

\title{Verbal Arithmetic}
\author{Fengyang Wang}

\begin{document}
\maketitle

Begin with an introduction to verbal arithmetic, solving the
trivial I + DID = TOO.

Continue by solving SEND + MORE = MONEY.

\begin{enumerate}
 \item When adding two four-digit numbers and
 getting a five-digit number, the first digit must be $1$.
 So $M=1$.
 \item What are the possibilities for $O$, the second digit?
 The sum $S+1+\text{carry} \ge 10$ because it must carry.
 The carry can only be zero or one, so in either event the
 sum cannot be $12$ or greater. So $O$ is either $0$ or $1$.
 But since every letter is unique, $O=0$.
 \item The numbers are unique, so $E \ne N$. That means that
 the last sum must carry, and since $N$ also can't be zero,
 $N = E + 1$. Then this means that there is no carry to the
 $S+1$ sum and therefore $S = 9$.
 \item We have either $N+R-10=E$ or $N+R+1-10=E$. But $N$ is
 $E+1$, so that means either $R+E+1-10=E$ or $R+E+1+1-10=E$,
 hence $R+E-9=E$ or $R+E-8=E$. $R$ has to be either $8$ or $9$
 for this to be true, but $9$ is already used so $R=8$.
 \item From the above step we learned that $D+E$ must carry,
 so $D+E \ge 12$. If $D$ is $6$ or below, there would be no legal
 value of $E$, so $D$ must be $7$.
 \item As above, $E$ must be $5$ or greater. But $5$ is the only
 legal value remaining for $E$, so $E=5$ and $N=6$.
 \item $Y$ is then trivially $2$.
\end{enumerate}


\end{document}