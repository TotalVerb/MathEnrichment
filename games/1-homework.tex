\documentclass[12pt,letterpaper]{article}

\usepackage{extsizes}

\usepackage{amssymb,amsmath,amsthm}
\usepackage{enumerate}
\usepackage[margin=1.25in]{geometry}
\usepackage{graphicx,ctable,booktabs}
\usepackage{fancyhdr}
\usepackage[utf8]{inputenc}

\makeatletter
\newenvironment{problem}{\@startsection
       {section}
       {1}
       {-.2em}
       {-3.5ex plus -1ex minus -.2ex}
       {2.3ex plus .2ex}
       {\pagebreak[3]
       \large\bf\noindent{Problem }
       }
       }
\makeatother

\title{Verbal Arithmetic}
\author{Name: \underline{\hspace{5cm}}}
\date{January 10, 2015}

\pagestyle{fancy}
\lhead{Verbal Arithmetic}
\chead{} 
\rhead{\thepage} 
\lfoot{\small\scshape Grade 4 Olympic Math} 
\cfoot{} 
\rfoot{} 
\renewcommand{\headrulewidth}{.3pt} 
\renewcommand{\footrulewidth}{.3pt}
\setlength\voffset{-0.25in}
\setlength\textheight{648pt}
\setlength\headheight{15pt}

\begin{document}

\maketitle

\thispagestyle{empty}

\begin{problem}{Fool}
Solve the verbal arithmetic puzzle below. Each letter represents a unique digit.

\begin{center}
 \begin{tabular}{rr}
  & \tt ELF \\
  + & \tt ELF \\
  \hline
  & \tt FOOL
 \end{tabular}
\end{center}
\end{problem}

\begin{problem}{Number Puzzle}
Solve the verbal arithmetic puzzle below. Each letter represents a unique digit.

\begin{center}
 \begin{tabular}{rr}
  & \tt NUMBER \\
  + & \tt NUMBER \\
  \hline
  & \tt PUZZLE
 \end{tabular}
\end{center}
\end{problem}

\begin{problem}{Warning}
Solve the verbal arithmetic puzzle below. Each letter represents a unique digit.

\begin{center}
 \begin{tabular}{rr}
  & \tt CROSS \\
  + & \tt ROADS \\
  \hline
  & \tt DANGER
 \end{tabular}
\end{center}
\end{problem}

\begin{problem}{Unlucky}
Solve the verbal arithmetic puzzle below. Each letter represents a unique digit.

\begin{center}
 \begin{tabular}{rr}
  & \tt TERRIBLE \\
  + & \tt NUMBER \\
  \hline
  & \tt THIRTEEN
 \end{tabular}
\end{center}
\end{problem}

\begin{problem}{Equation}
Solve the verbal arithmetic puzzle below. Each letter represents a unique digit.

\begin{center}
 \begin{tabular}{rr}
  & \tt FORTY \\
  & \tt TEN \\
  + & \tt TEN \\
  \hline
  & \tt SIXTY
 \end{tabular}
\end{center}
\end{problem}

\begin{problem}{Easy}
Solve the verbal arithmetic puzzle below. Each letter represents a unique digit.

\begin{center}
 \begin{tabular}{rr}
  & \tt THIS \\
  & \tt IS \\
  + & \tt VERY \\
  \hline
  & \tt EASY
 \end{tabular}
\end{center}
\end{problem}

\begin{problem}{Scrabble}
Solve the verbal arithmetic puzzle below. Each letter represents a unique digit.

\begin{center}
 \begin{tabular}{rr}
  & \tt ALPHABET \\
  + & \tt LETTERS \\
  \hline
  & \tt SCRABBLE
 \end{tabular}
\end{center}
\end{problem}

\begin{problem}{Trouble}
Solve the verbal arithmetic puzzle below. Each letter represents a unique digit.

\begin{center}
 \begin{tabular}{rr}
  & \tt DOUBLE \\
  & \tt DOUBLE \\
  + & \tt TOIL \\
  \hline
  & \tt TROUBLE
 \end{tabular}
\end{center}
\end{problem}

\begin{problem}{Square}
Solve the verbal arithmetic puzzle below. Each letter represents a unique digit.
Notice that it is a product and not a sum!

\begin{center}
 \begin{tabular}{rr}
  & \tt TWO \\
  $\times$ & \tt TWO \\
  \hline
  & \tt SQUARE
 \end{tabular}
\end{center}
\end{problem}

\end{document}