\documentclass[a4paper,10pt]{report}

\newcommand{\IncludePath}{../include}
\newcommand{\ProjectName}{Grade 4 Olympic Math}
\usepackage{extsizes}
\usepackage{titling}

\usepackage{tikz}
\usetikzlibrary{shapes}

\usepackage{amssymb,amsmath,amsthm}
\usepackage{enumerate}
\usepackage{graphicx,ctable,booktabs}
\usepackage{fancyhdr}
\usepackage[utf8]{inputenc}
\usepackage{gensymb}

\usepackage[toc]{glossaries}

\makeatletter
\newenvironment{problem}{\@startsection
       {subsection}
       {1}
       {-.2em}
       {-3.5ex plus -1ex minus -.2ex}
       {2.3ex plus .2ex}
       {\pagebreak[3]
       \large\bf\noindent{Problem }
       }
       }
\makeatother

\makeatletter
\g@addto@macro\@floatboxreset\centering
\makeatother

\newenvironment{solution}
{ \vspace{1em} \noindent \textbf{Solution:} }
{  }

\pagestyle{fancy}
\lhead{\thetitle}
\chead{}
\rhead{\thepage}
\lfoot{\small\scshape \ProjectName}
\cfoot{}
\rfoot{}
\renewcommand{\headrulewidth}{.3pt}
\renewcommand{\footrulewidth}{.3pt}
\setlength\voffset{-0.25in}
\setlength\textheight{648pt}
\setlength\headheight{15pt}

\newcommand{\Ans}[1]{\framebox{$#1$}}
\newcommand{\SAA}[1]{\Switch{\Ans{#1}}{\blankA}}
\newcommand{\SAB}[1]{\Switch{\Ans{#1}}{\blankB}}
\newcommand{\SAC}[1]{\Switch{\Ans{#1}}{\blankC}}
\newcommand{\SAD}[1]{\Switch{\Ans{#1}}{\blankD}}
\newcommand{\SAE}[1]{\Switch{\Ans{#1}}{\blankE}}
\newcommand{\SAF}[1]{\Switch{\Ans{#1}}{\blankF}}
\newcommand{\STA}[1]{\Switch{\AnsT{#1}}{\blankA}}
\newcommand{\STB}[1]{\Switch{\AnsT{#1}}{\blankB}}
\newcommand{\STC}[1]{\Switch{\AnsT{#1}}{\blankC}}
\newcommand{\STD}[1]{\Switch{\AnsT{#1}}{\blankD}}
\newcommand{\STE}[1]{\Switch{\AnsT{#1}}{\blankE}}
\newcommand{\STF}[1]{\Switch{\AnsT{#1}}{\blankF}}
\newcommand{\AnsT}[1]{\framebox{#1}}
\newif\ifanswers
\newcommand{\Switch}[2]{\ifanswers#1\else#2\fi}
\newcommand{\MCSelect}[1]{\Switch{\AnsT{#1}}{#1}}
\newcommand{\TFTrue}{\MCSelect{True}~~False}
\newcommand{\TFFalse}{True~~\MCSelect{False}}

\newcommand{\blankA}{\underline{\hspace{1em}}}
\newcommand{\blankB}{\underline{\hspace{2em}}}
\newcommand{\blankC}{\underline{\hspace{3em}}}
\newcommand{\blankD}{\underline{\hspace{4em}}}
\newcommand{\blankE}{\underline{\hspace{5em}}}
\newcommand{\blankF}{\underline{\hspace{6em}}}



\title{Recreational Puzzles for Grade 4}
\author{Fengyang Wang}

\begin{document}
\maketitle

\chapter{Verbal Arithmetic}

Begin with an introduction to verbal arithmetic, solving the trivial I + DID =
TOO. Continue by solving SEND + MORE = MONEY.

\begin{enumerate}
 \item When adding two four-digit numbers and getting a five-digit number, the
 first digit must be $1$. So $M=1$.
 \item What are the possibilities for $O$, the second digit? The sum
 $S+1+\text{carry} \ge 10$ because it must carry. The carry can only be zero or
 one, so in either event the sum cannot be $12$ or greater. So $O$ is either $0$
 or $1$. But since every letter is unique, $O=0$.
 \item The numbers are unique, so $E \ne N$. That means that the last sum must
 carry, and since $N$ also can't be zero, $N = E + 1$. Then this means that
 there is no carry to the $S+1$ sum and therefore $S = 9$.
 \item We have either $N+R-10=E$ or $N+R+1-10=E$. But $N$ is $E+1$, so that
 means either $R+E+1-10=E$ or $R+E+1+1-10=E$, hence $R+E-9=E$ or $R+E-8=E$. $R$
 has to be either $8$ or $9$ for this to be true, but $9$ is already used so
 $R=8$.
 \item From the above step we learned that $D+E$ must carry, so $D+E \ge 12$. If
 $D$ is $6$ or below, there would be no legal value of $E$, so $D$ must be $7$.
 \item As above, $E$ must be $5$ or greater. But $5$ is the only legal value
 remaining for $E$, so $E=5$ and $N=6$.
 \item $Y$ is then trivially $2$.
\end{enumerate}


\end{document}
