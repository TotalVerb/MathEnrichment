\documentclass[12pt,letterpaper]{article}

\newcommand{\IncludePath}{../include}
\usepackage{extsizes}
\usepackage{titling}

\usepackage{amssymb,amsmath,amsthm}
\usepackage{enumerate}
\usepackage[margin=1in]{geometry}
\usepackage{graphicx,ctable,booktabs}
\usepackage{fancyhdr}
\usepackage[utf8]{inputenc}

\makeatletter
\newenvironment{problem}{\@startsection
       {section}
       {1}
       {-.2em}
       {-3.5ex plus -1ex minus -.2ex}
       {2.3ex plus .2ex}
       {\pagebreak[3]
       \large\bf\noindent{Problem }
       }
       }
\makeatother

\pagestyle{fancy}
\lhead{\thetitle}
\chead{}
\rhead{\thepage}
\lfoot{\small\scshape Grade 4 Olympic Math}
\cfoot{}
\rfoot{}
\renewcommand{\headrulewidth}{.3pt}
\renewcommand{\footrulewidth}{.3pt}
\setlength\voffset{-0.25in}
\setlength\textheight{648pt}
\setlength\headheight{15pt}


\title{Prime Numbers \& FTA}
\author{Answers}
\date{March 12, 2016}

\begin{document}

\maketitle

\thispagestyle{empty}

\begin{problem}{Primes}
Select True or False for each statement.

\begin{itemize}
 \item $5$ is prime \hfill \AnsT{True}~~~False
 \item $97$ is prime \hfill \AnsT{True}~~~False
 \item $10000000000$ is prime \hfill True~~~\AnsT{False}
 \item For all $k>1$, $7k$ is composite \hfill \AnsT{True}~~~False
\end{itemize}
\end{problem}

\begin{problem}{The Fundamental Theorem of Arithmetic}
 Select True or False for each statement.

 \begin{itemize}
  \item Every whole number except one has multiple prime factorizations.
  \hfill True~~~\AnsT{False}
  \item Prime factorizations must not contain the number $1$.
  \hfill \AnsT{True}~~~False
  \item Some composite numbers cannot be factored into primes.
  \hfill True~~~\AnsT{False}
  \item The prime factorization of any prime number is the number itself.
  \hfill \AnsT{True}~~~False
 \end{itemize}

\end{problem}

\begin{problem}{Find the Primes}
 Fill in the blanks with prime numbers. (Answers may vary.)

 \begin{itemize}
  \item $\Ans{5} + \Ans{47} = 52$
  \item $\Ans{3} + \Ans{3} + \Ans{67} = 73$
 \end{itemize}
\end{problem}


\begin{problem}{Prime Factor}
 Find just one prime factor for each number. Do not try to find more.
 (Answers may vary.)

 \begin{itemize}
  \item $15838$ \hfill Factor: $\Ans{2}$
  \item $19385$ \hfill Factor: $\Ans{5}$
  \item $21849$ \hfill Factor: $\Ans{3}$
 \end{itemize}
\end{problem}


\begin{problem}{Factorize}
 Find the unique prime factorization of each number.

 \begin{itemize}
  \item $18$ \hfill $\Ans{2} \times \Ans{3} \times \Ans{3}$
  \item $57$ \hfill $\Ans{3} \times \Ans{19}$
  \item $100$ \hfill $\Ans{2} \times \Ans{2} \times \Ans{5} \times \Ans{5}$
  \item $80$ \hfill $\Ans{2} \times \Ans{2} \times \Ans{2} \times \Ans{2}
  \times \Ans{5}$
 \end{itemize}
\end{problem}

\begin{problem}{Big Prime Numbers}
 Suppose you had a large number like $8644255723$. How do you figure out its
 prime factorization? One way is to use \emph{trial division}---just divide by
 all numbers starting from $1$ until you find one that leaves no remainder. It
 turns out $8644255723 = 90907\times95089$, which are both prime numbers.

 \begin{enumerate}
  \item How many divisions do you have to do before you find the first number
  that works ($90907$)?
  \item Let's say it takes $1$ minute for a person to divide $8644255723$ by a
  smaller number (it might take even longer!). How long
  (\Ans{90907} minutes) will it take a person to find the first
  prime factor?
  \item How long is that in days spent without eating, drinking, or sleeping?
  (There are $1440$ minutes in a day.) \hfill \Ans{63} days
 \end{enumerate}

\end{problem}

\begin{problem}{Challenge}

The notation $5!$ (pronounced ``five factorial'') means the product of all
integers from $1$ to $5$; that is, $1 \times 2 \times 3 \times 4 \times 5$.
Similarly, $3! = 1 \times 2 \times 3$ and $100! = 1 \times 2 \times 3
\times\ldots\times 99 \times 100$.

\begin{enumerate}
 \item Find the prime factorization of $4!$. \hfill
 $\Ans{2} \times \Ans{2} \times \Ans{2} \times \Ans{3}$
 \item In the prime factorization of $4!$, how many $2$'s are there?
 \Ans{3}
 \item In the prime factorization of $100!$, how many $2$'s are there?
 \Ans{97}
\end{enumerate}

\end{problem}



\end{document}
