\documentclass[12pt,letterpaper]{article}

\newcommand{\IncludePath}{../include}
\usepackage{extsizes}
\usepackage{titling}

\usepackage{amssymb,amsmath,amsthm}
\usepackage{enumerate}
\usepackage[margin=1in]{geometry}
\usepackage{graphicx,ctable,booktabs}
\usepackage{fancyhdr}
\usepackage[utf8]{inputenc}

\makeatletter
\newenvironment{problem}{\@startsection
       {section}
       {1}
       {-.2em}
       {-3.5ex plus -1ex minus -.2ex}
       {2.3ex plus .2ex}
       {\pagebreak[3]
       \large\bf\noindent{Problem }
       }
       }
\makeatother

\pagestyle{fancy}
\lhead{\thetitle}
\chead{}
\rhead{\thepage}
\lfoot{\small\scshape Grade 4 Olympic Math}
\cfoot{}
\rfoot{}
\renewcommand{\headrulewidth}{.3pt}
\renewcommand{\footrulewidth}{.3pt}
\setlength\voffset{-0.25in}
\setlength\textheight{648pt}
\setlength\headheight{15pt}


\title{Quiz: Parity \& Divisibility}
\author{Name: \underline{\hspace{5cm}} Mark: $\displaystyle \frac{\hspace{3em}}{50}$}
\date{October 28, 2017}

\begin{document}

\maketitle

Please write your name. While 58 marks are available, the quiz is counted out of 50, so if
you obtain more than 50 marks, you will receive 50/50. For all problems except problem 6,
you may work with one other person on this quiz, but both of you will need to write your
names and submit two separate quizzes. Please do problem 6 on your own.

\thispagestyle{empty}

\begin{problem}{Factor Pairs\hfill/6}
 Complete the factor pairs.

 \begin{itemize}
  \begin{multicols}{2}
   \item $12 = 1 \times \blankB$
   \item $12 = 2 \times \blankB$
   \item $12 = 3 \times \blankB$
   \item $12 = 4 \times \blankB$
   \item $12 = 6 \times \blankB$
   \item $12 = 12 \times \blankB$
  \end{multicols}
 \end{itemize}

\end{problem}

\begin{problem}{Divisibility\hfill/20}
 Circle factors of each number. More than one factor may be circled.

 \begin{itemize}
  \item $9$ \hfill $1$~~~$2$~~~$5$~~~$10$~~~$100$
  \item $12$ \hfill $1$~~~$2$~~~$5$~~~$10$~~~$100$
  \item $40$ \hfill $1$~~~$2$~~~$5$~~~$10$~~~$100$
  \item $100$ \hfill $1$~~~$2$~~~$5$~~~$10$~~~$100$
 \end{itemize}
\end{problem}

\begin{problem}{Terminology\hfill/5}
 Use the word ``even'', ``odd'', ``factor'', or ``multiple'' to complete each
 blank. Be careful! The same word might be used more than once.

 \begin{itemize}
  \item The product of two odd numbers is \underline{\hspace{6em}}.
  \item The sum of an odd and an even number is \underline{\hspace{6em}}.
  \item $1$ is a \underline{\hspace{6em}} of every whole number.
  \item $2$ is a factor of every \underline{\hspace{6em}} number.
  \item $0$ is a \underline{\hspace{6em}} of every whole number.
 \end{itemize}
\end{problem}

\begin{problem}{Garden\hfill/14}
 Adam wants to grow a rectangular garden in a grid layout, with a total
 of $70$ tomatoes. List all of Adam's options for garden dimensions.
 The first one is done for you. Hint: $70=2\times5\times7$, and remember
 that factors come in pairs---if $1 \times 70 = 70$ then $70 \times 1 = 70$
 also.

 \begin{itemize}
  \begin{multicols}{2}
    \item $70 = 1 \times 70$
    \item $70 = \underline{\hspace{2em}} \times \underline{\hspace{2em}}$
    \item $70 = \underline{\hspace{2em}} \times \underline{\hspace{2em}}$
    \item $70 = \underline{\hspace{2em}} \times \underline{\hspace{2em}}$
    \item $70 = \underline{\hspace{2em}} \times \underline{\hspace{2em}}$
    \item $70 = \underline{\hspace{2em}} \times \underline{\hspace{2em}}$
    \item $70 = \underline{\hspace{2em}} \times \underline{\hspace{2em}}$
    \item $70 = \underline{\hspace{2em}} \times \underline{\hspace{2em}}$
  \end{multicols}
 \end{itemize}
\end{problem}

\begin{problem}{True or False\hfill/8}
  Circle True or False for each statement.

  \begin{itemize}
    \item If \(n\) is even, then \(4 \mid n^2\). \hfill \TFTrue
    \item If \(n\) is odd, then \(n^2 = 4m + 1\) for some integer \(m\).
    \hfill \TFTrue
    \item For all integers \(n\), \(n^3 + 3n - 1\) is odd. \hfill \TFTrue
    \item There are no integer solutions \((n, m)\) to \(n^2 = 4m + 2\).
    \hfill \TFTrue
    \item \(1234 \mid 1522755\). \hfill \TFFalse
    \item If \(4 \mid n\), then \(4n \mid (n^2+n)^2\). \hfill \TFTrue
    \item If \(m \mid n\), and \(k \mid n\), then \(m \mid k\). \hfill \TFFalse
    \item If \(3 \mid n\), but \(9 \nmid n\), then \(6 \mid n\). \hfill
    \TFFalse
  \end{itemize}
\end{problem}

\begin{problem}{Learning Summary\hfill/5}
  Write two sentences to describe one concept you learned this unit. For example, you can
  write about even and odd numbers.
\end{problem}

\end{document}
