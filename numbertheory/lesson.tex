\documentclass[a4paper,10pt]{report}

\newcommand{\IncludePath}{../include}
\newcommand{\ProjectName}{Grade 4 Olympic Math}
\usepackage{extsizes}
\usepackage{titling}

\usepackage{tikz}
\usetikzlibrary{shapes}

\usepackage{amssymb,amsmath,amsthm}
\usepackage{enumerate}
\usepackage{graphicx,ctable,booktabs}
\usepackage{fancyhdr}
\usepackage[utf8]{inputenc}
\usepackage{gensymb}

\usepackage[toc]{glossaries}

\makeatletter
\newenvironment{problem}{\@startsection
       {subsection}
       {1}
       {-.2em}
       {-3.5ex plus -1ex minus -.2ex}
       {2.3ex plus .2ex}
       {\pagebreak[3]
       \large\bf\noindent{Problem }
       }
       }
\makeatother

\makeatletter
\g@addto@macro\@floatboxreset\centering
\makeatother

\newenvironment{solution}
{ \vspace{1em} \noindent \textbf{Solution:} }
{  }

\pagestyle{fancy}
\lhead{\thetitle}
\chead{}
\rhead{\thepage}
\lfoot{\small\scshape \ProjectName}
\cfoot{}
\rfoot{}
\renewcommand{\headrulewidth}{.3pt}
\renewcommand{\footrulewidth}{.3pt}
\setlength\voffset{-0.25in}
\setlength\textheight{648pt}
\setlength\headheight{15pt}

\newcommand{\Ans}[1]{\framebox{$#1$}}
\newcommand{\SAA}[1]{\Switch{\Ans{#1}}{\blankA}}
\newcommand{\SAB}[1]{\Switch{\Ans{#1}}{\blankB}}
\newcommand{\SAC}[1]{\Switch{\Ans{#1}}{\blankC}}
\newcommand{\SAD}[1]{\Switch{\Ans{#1}}{\blankD}}
\newcommand{\SAE}[1]{\Switch{\Ans{#1}}{\blankE}}
\newcommand{\SAF}[1]{\Switch{\Ans{#1}}{\blankF}}
\newcommand{\STA}[1]{\Switch{\AnsT{#1}}{\blankA}}
\newcommand{\STB}[1]{\Switch{\AnsT{#1}}{\blankB}}
\newcommand{\STC}[1]{\Switch{\AnsT{#1}}{\blankC}}
\newcommand{\STD}[1]{\Switch{\AnsT{#1}}{\blankD}}
\newcommand{\STE}[1]{\Switch{\AnsT{#1}}{\blankE}}
\newcommand{\STF}[1]{\Switch{\AnsT{#1}}{\blankF}}
\newcommand{\AnsT}[1]{\framebox{#1}}
\newif\ifanswers
\newcommand{\Switch}[2]{\ifanswers#1\else#2\fi}
\newcommand{\MCSelect}[1]{\Switch{\AnsT{#1}}{#1}}
\newcommand{\TFTrue}{\MCSelect{True}~~False}
\newcommand{\TFFalse}{True~~\MCSelect{False}}

\newcommand{\blankA}{\underline{\hspace{1em}}}
\newcommand{\blankB}{\underline{\hspace{2em}}}
\newcommand{\blankC}{\underline{\hspace{3em}}}
\newcommand{\blankD}{\underline{\hspace{4em}}}
\newcommand{\blankE}{\underline{\hspace{5em}}}
\newcommand{\blankF}{\underline{\hspace{6em}}}



% use glossaries for this document
\newglossaryentry{algorithm}
{
  name=algorithm,
  description={step-by-step procedure for performing a calculation
  according to well-defined rules (Wikipedia)}
}

\newglossaryentry{acute angle}
{
  name=acute angle,
  description={an angle measuring less than a right angle (\SI{90}{\degree} or a
  quarter of a turn); such angles are typically characterized as being sharp}
}

\newglossaryentry{angle}
{
  name=angle,
  description={the figure formed by two rays, called the sides of the angle,
  sharing a common endpoint, called the vertex of the angle (Wikipedia)}
}

\newglossaryentry{coincident}
{
  name=coincident,
  description={a description of two objects which occupy exactly the same space}
}

\newglossaryentry{domain}
{
  name=domain,
  description={the set of values that are possible inputs for a function}
}

\newglossaryentry{endpoint}
{
  name=endpoint,
  description={an extreme point of a line segment or ray; line segments have two
  endpoints whereas rays have just one}
}

\newglossaryentry{function}
{
  name=function,
  description={object that may take any allowed input and will produce a
  single associated output for that input; alternatively, relation for
  which each input value has exactly one related output value}
}

\newglossaryentry{integer}
{
  name=integer,
  description={positive or negative whole number, or $0$; for example,
  $-8$, $2000$}
}

\newglossaryentry{line}
{
  name=line,
  description={a straight one-dimensional object that extends forever in both
  directions}
}

\newglossaryentry{line segment}
{
  name=line segment,
  description={a straight one-dimensional object terminated at both ends}
}

\newglossaryentry{multiplicand}
{
  name=multiplicand,
  description={the number that is being multiplied; for instance, in
  $2\times3=6$, the multiplicand is $2$}
}

\newglossaryentry{multiplier}
{
  name=multiplier,
  description={the factor to multiply a number by; for instance, in
  $2\times3=6$, the multiplier is $3$}
}

\newglossaryentry{obtuse angle}
{
  name=obtuse angle,
  description={an angle measuring more than a right angle (\SI{90}{\degree} or a
  quarter of a turn) but less than a straight angle (\SI{180}{\degree} or a half
  of a turn)}
}

\newglossaryentry{parallel}
{
  name=parallel,
  description={a description of two lines that never intersect at any point}
}

\newglossaryentry{parity}
{
  name=parity,
  description={decribes whether an integer is even or odd}
}

\newglossaryentry{plane}
{
  name=plane,
  description={a two-dimensional flat surface}
}

\newglossaryentry{plane geometry}
{
  name=plane geometry,
  description={the study of figures on a plane (a two-dimensional flat surface)}
}

\newglossaryentry{product}
{
  name=product,
  description={the result of a multiplication; for instance, in
  $2\times3=6$, the product is $6$}
}

\newglossaryentry{range}
{
  name=range,
  description={the set of values that are possible outputs for a function}
}

\newglossaryentry{ray}
{
  name=ray,
  description={a straight one-dimensional object terminated at one end and
  extending forever in the other direction}
}

\newglossaryentry{right angle}
{
  name=right angle,
  description={an angle measuring \SI{90}{\degree} or a quarter of a turn; angle
  between two rays intersecting in an L shape}
}

\newglossaryentry{right triangle}
{
  name=right triangle,
  description={an triangle with one \SI{90}{\degree} (right) angle}
}

\newglossaryentry{summand}
{
  name=summand,
  description={something which is being added; for instance, in $1+2=3$,
  the two summands are $1$ and $2$}
}

\newglossaryentry{reflex angle}
{
  name=reflex angle,
  description={an angle measuring more than \SI{180}{\degree} or a half of a
  turn, but less than \SI{360}{\degree} or a full turn}
}

\newglossaryentry{straight angle}
{
  name=straight angle,
  description={an angle measuring \SI{180}{\degree} or a half of a turn; angle
  between two rays in opposite directions}
}

\newglossaryentry{transitivity}
{
  name=transitivity,
  description={the property of certain relations that specifies if an
  element $a$ is related to $b$, and the element $b$ is related to $c$,
  then $a$ is similarly related to $c$}
}

\makeglossaries

\title{Number Theory}
\author{Fengyang Wang}
\date{March 11, 2017}

\begin{document}

\begin{abstract}

This is an introduction to more advanced topics in number theory suitable for
an advanced Grade 4 audience. These notes were prepared for the Grand River
Chinese School. The content covered is not typically presented to elementary
school students. It is hard and designed to challenge even the strongest
students, while being accessible for everyone. Each chapter (excluding the
introduction) may take two to four hours to deliver entirely, depending on the
level of detail.

These notes are intended to be a rough outline of what is taught, and not a
rigorous and complete reference. I do not necessarily cover all the material
written in these notes in any particular year. In particular, the more abstract
algebraic concepts are often abbreviated or left out, and are included for
completeness. I may occasionally cover material beyond that written in the
notes.

Although the notes are intended to be presented to a young audience, they are
written for a teacher and not for a student. Many of the terms used will not be
familiar to the students. They require explanation.

\end{abstract}

\maketitle

\tableofcontents

\chapter{Introduction}

Number Theory is the study of \glspl{integer} and problems based on the
integers. Firstly, what is an integer? We are familiar with the whole numbers,
which can be drawn on the number line (see Figure~\ref{nt-int:pnumberline}).
The whole numbers begin with \(0\) and continue \(1\), \(2\), \(3\), and so on.

\begin{figure}
 \fgNumberLine{0}{8}

 \caption{The whole numbers on a number line}
 \label{nt-int:pnumberline}
\end{figure}

We may now add in the negative numbers, \(-1\), \(-2\), \(-3\), and so on. If
these numbers are considered together with the whole numbers, then we have a
full number line that contains all the integers (see
Figure~\ref{nt-int:inumberline}). For most of this unit, we will work with
positive integers only, which start at \(1\) and continue \(2\), \(3\), \(4\),
and so on. Note, in particular, that \(0\) is neither a positive integer nor a
negative integer.

\begin{figure}
 \fgNumberLine{-4}{4}

 \caption{The integers on a number line}
 \label{nt-int:inumberline}
\end{figure}

Many of the topics in number theory are based off the operations of
multiplication and division. Consider a multiplication expression, such as \[ 7
\times 9 = 63 \]

In this expression, \(7\) is the \gls{multiplicand}, \(9\) is the
\gls{multiplier}, and \(63\) is the \gls{product}. An important property of
multiplication is that it is commutative, which means we may switch the order.
That is, \[ 9 \times 7 = 7 \times 9 \]

Before we continue, it is important to have a good grasp of what remainders
are, and to introduce some new notation. Recall first that division is the
operation of repeated subtraction, like how multiplication is the operation of
repeated addition.

When we divide \(20\) by \(4\), we count how many groups of \(4\) are needed to
make \(20\). We know that \(4 \times 5 = 20\). This means that it takes \(5\)
groups of \(4\) to make \(20\). So \(20 \div 4 = 5\).

Sometimes the numbers might not work out perfectly. Let's try dividing \(20\)
by \(3\). If we make \(7\) groups of \(3\), the total number would be \(3
\times 7 = 21\). But this is too big. If we make \(6\) groups of \(3\), the
total number would be \(3 \times 6 = 18\). But this is too small. So we can
make six groups, but we'd then have \(20 - 18 = 2\) left over.

We say that when we divide \(20\) by \(3\), our \emph{quotient} is \(6\)
because we can make \(6\) groups of \(3\) in total. And then we say that our
\emph{remainder} is \(2\) because after making \(6\) groups of \(3\), we have
\(2\) left over. We can also write this as \(20 \div 3 = 6 \operatorname{R}
2\). The ``R'' stands for ``remainder''.

Suppose we didn't care what the quotient was; only the remainder is important.
Then we can write that as \(16 \bmod 5 = 1\). We don't care how many groups we
made, but we do care that one was left over after making those groups.

The study of remainders is very useful for several major applications, some of
which we will see later this unit. Recalling the patterning unit, we used
remainders to compute future terms in repeating sequences. Many of the
applications of remainders are based off this application.

\begin{problem}{Guess the Number}
 A number yields a remainder of \(1\) when divided by \(2\), a remainder of
 \(2\) when divided by \(3\), a remainder of \(3\) when divided by \(4\), and a
 remainder of \(4\) when divided by \(5\).

 \begin{enumerate}[\hspace{1cm}a.]
  \item What is the smallest positive number to have this property?
  \item Give two other positive numbers to have the above property.
 \end{enumerate}

 \begin{solution}
   \begin{enumerate}[\hspace{1cm}a.]
     \item The important pattern to see here is that each remainder is one less
     than the number being divided by: \(1 = 2 - 1\), \(2 = 3 - 1\), and so
     forth. That means that if we add one to a number that has this property,
     then it should be divisible by all of \(2\), \(3\), \(4\), and \(5\). The
     smallest positive number divisible by all of these is \(60\), so \(60 - 1
     = \Ans{59}\) is the smallest positive number having the property.
     \item As seen in the solution to part a., any number having the property
     is one less than a number divisible by \(2\), \(3\), \(4\), and \(5\).
     Other numbers that work therefore include \(120 - 1 = \Ans{119}\) and
     \(180 - 1 = \Ans{179}\).
   \end{enumerate}
 \end{solution}
\end{problem}

\chapter{Parity \& Divisibility}

\section{Parity}

\Gls{parity} is a property of integers that classifies integers into even
integers and odd integers. In this section, we will study the differences
between even and odd integers, and also study several important properties. We
will also approach the problem algebraically. Do not be alarmed by the presence
of letters along with numbers. Unless otherwise stated, a letter is simply a
placeholder for a number.

An \gls{even} number is a number of the form \(2n\), where \(n\) is any
integer. The expression \(2n\) means \(2\times n\); we omit the \(\times\) so
that it is faster to read and write. What are some examples? For instance,
\(6\) is an even number, because we can write \(6 = 2 \times 3\).

Is \(2\) even? Yes. We can write \(2 = 2 \times 1\), so \(2\) is an even
number. Is \(0\) even? Yes. We can write \(0 = 2 \times 0\), so \(0\) is also
an even number. What about \(-10\)? This is also an even number, because we can
write \(-10 = 2 \times -5\).

Now let's consider \(5\). Can we find an integer \(n\) so that \(5 = 2n\)? The
answer is no. If we try \(n = 2\), we get \(2 \times 2 = 4\), which is too
small. But if we try \(n = 3\), we get \(2 \times 3 = 6\), which is too big.
Since there are no integers between \(2\) and \(3\), the conclusion is that
there is no integer \(n\) with \(5 = 2n\).

Then, \(5\) is not even. We say that \(5\) is \gls{odd}: that simply means that
it is not even. All integers, then, are either even or odd, never neither and
never both.

The numbers \(-4\), \(-2\), \(0\), \(2\), and \(4\) are even, while the numbers
\(-3\), \(-1\), \(1\), and \(3\) are odd. Notice that in between every two even
numbers is an odd number, and in between every two odd numbers is an even
number. Indeed, if we add \(1\) to an even number, we will always get an odd
number; if we add \(1\) to an odd number, we will always get an even number.

Furthermore, any odd number is exactly \(1\) more than some even number. What
we conclude is that any odd number can be written as \(2n+1\), where \(n\) is
some integer. In addition, any integer of the form \(2n+1\), where \(n\) is
some integer, is odd. (The expression \(2n+1\) means \((2\times n) + 1\). We
perform the multiplication first because of the order of operations.)

\subsection{Determining Parity}

It is easy to determine whether a number is even or odd. Let's start with small
examples first. We know \(4=2\times2\), so it's even. We know \(7=2\times3+1\),
so it's odd. A number is either even or odd, but it's never both. We only need
to divide by \(2\): if it evenly divides, then it's even, and otherwise, it's
odd.

\begin{problem}{Odd or Even I}
 Are each of the following numbers even or odd?

 \begin{itemize}
  \item \(10\) \hfill \AnsT{Even}~~~Odd
  \item \(0\) \hfill \AnsT{Even}~~~Odd
  \item \(19\) \hfill Even~~~\AnsT{Odd}
  \item \(-1\) \hfill Even~~~\AnsT{Odd}
 \end{itemize}
\end{problem}

If the numbers are very big, it is convenient to simply observe the last digit.
If the last digit is even, then the whole number is even. If the last digit is
odd, then the whole number is odd. Why does this work? We will see very soon.

\subsection{Sums \& Differences}

What happens if we add two odd numbers? Two even numbers? Let's find out. Say
\(2n+1\) and \(2m+1\) are two odd numbers. We're using letters here as
placeholders; \(n\) and \(m\) can both take the value of any integer. Then \[
(2n+1) + (2m+1) = 2n + 2m + 1 + 1 = 2n + 2m + 2 = 2(n+m+1) \] but \(n+m+1\) is
an integer, and therefore the sum is even.

We can use a very similar strategy to find the sum of two even numbers. This is
even easier. Say \(2n\) and \(2m\) are two even numbers. Then \[
 2n + 2m = 2(n+m)
\] but \(n+m\) is an integer, and therefore the sum is even.

Say \(2n\) is an even number, and \(2m+1\) is an odd number. Then \[
 2n + 2m + 1 = 2(n+m) + 1
\] which is an odd number. So the sum of an even number and an odd number is
odd. And because we can rearrange the order of addition, therefore the sum of an
odd number and a even number is also odd. These results can be summarized in a
table (see Figure~\ref{pd:psumt}).

\begin{figure}
 \begin{tabular}{c|cc}
  \(+\) & Even & Odd  \\
  \hline
  Even   & Even & Odd \\
  Odd    & Odd & Even
 \end{tabular}

 \caption{Result of adding numbers, by parity}
 \label{pd:psumt}
\end{figure}

The rules for subtracting are exactly the same. The difference of two odds is
even. The difference of two evens is even. The difference of an odd and an even
is odd. These results are summarized in Figure~\ref{pd:pdifft}. Feel free to
work these out yourself.

\begin{figure}
 \begin{tabular}{c|cc}
  \(-\) & Even & Odd  \\
  \hline
  Even   & Even & Odd \\
  Odd    & Odd & Even
 \end{tabular}

 \caption{Result of subtracting numbers, by parity}
 \label{pd:pdifft}
\end{figure}

\begin{problem}{Odd or Even II}
 Are each of the following numbers even or odd?

 \begin{itemize}
  \item \(100 + 200\) \hfill \AnsT{Even}~~~Odd
  \item \(1273 + 19023\) \hfill \AnsT{Even}~~~Odd
  \item \(19082 - 1911\) \hfill Even~~~\AnsT{Odd}
  \item \(109291 + 8329 + 9107\) \hfill Even~~~\AnsT{Odd}
 \end{itemize}
\end{problem}

\subsection{Products}

We can use a similar technique to develop rules for the product of even and odd
numbers. It turns out that the product of two odd numbers is odd. Say \(2n+1\)
and \(2m+1\) are two odd numbers. Observe: \begin{align*}
 (2n+1)(2m+1)
 &= 2n(2m+1) + (2m+1) \\
 &= 4nm+2n+2m+1 \\
 &= 2(2nm+n+m) + 1
\end{align*} which is odd.

But the product of an even number with any integer is even. Say \(2n\) is an
even number, and \(k\) is any integer. Then: \[
 (2n)k = 2(nk)
\] which is even. We may summarize these results in another table (see
Figure~\ref{pd:pprodt})

\begin{figure}
 \begin{tabular}{c|cc}
  \(\times\) & Even & Odd  \\
  \hline
  Even   & Even & Even \\
  Odd    & Even & Odd
 \end{tabular}

 \caption{Result of multiplying numbers, by parity}
 \label{pd:pprodt}
\end{figure}

At this point we have enough knowledge to see why our rule for determining
parity by looking at the last digit works. In a number written in the place
value system, such as \(76103\), we can split the number into the ones' digit,
the tens' digit, hundreds' digit, and so on. Another way to think about it is,
by splitting off the ones' digit, \[
 76103 = 3 + 7610 \times 10
\]

But we know that \(10=2\times5\) is even, so by the rules covered above, we
know that anything mutliplied by \(10\) remains even. Then if we add it to an
odd number, we obtain an odd number, but if we add it to an even number, then
we obtain an even number. Therefore, just by looking at the last digit, we can
figure out whether a number is odd or even.

\section{Divisibility}

We will begin today's topic with a few definitions.

The symbol \(\mid\) means ``divides''. If \(n\) and \(m\) are two numbers, then
\(n\) divides \(m\) when \(m \div n\) has no remainder; that is, when \(m \bmod
n = 0\). We write this as \(n \mid m\).

A number is a \emph{factor} of another number if it divides that number. For
example, \(3\) is a \emph{factor} of \(9\) because \(3 \mid 9\). A number is a
\emph{multiple} of another number if the other number divides it. For example,
\(9\) is a \emph{multiple} of \(3\). Another word for factor, which we will see
later on, is \emph{divisor}.

We can see that if \(n\) and \(m\) are integers, then the following conditions
are \gls{equivalent}:

\begin{enumerate}
  \item \(n\) is a factor of \(m\)
  \item \(n\) divides \(m\)
  \item \(m\) is a multiple of \(n\)
  \item \(m \bmod n = 0\)
  \item There exists integer \(k\) such that \(m = nk\)
\end{enumerate}

Observe that \(1\) is a \emph{factor} of every whole number, and \(0\) is a
\emph{multiple} of every whole number.

We also introduce the symbol \(\nmid\) for ``does not divide''; that is, for
\(n\) and \(m\) two integers, \(n \nmid m\) whenever it is not the case that
\(n \mid m\).

\begin{problem}{Factors and Multiples}
 Let \(a\), \(b\), and \(c\) be integers such that \(c = ab\). Then
 \begin{enumerate}
  \item \(a\) must divide \(c\) \hfill \AnsT{True}~~~False
  \item \(b\) must be a multiple of \(c\) \hfill True~~~\AnsT{False}
 \end{enumerate}

 \begin{solution}
  Since \(c = a \times b\), and \(b\) is an integer, therefore \(a \mid c\), so
  the first statement is always \AnsT{true}.

  But there is no requirement that \(c \mid b\)---take \(a = b = 2\) and \(c =
  4\), for example. Here \(4 \nmid 2\), so the second statement is
  \AnsT{false}.
 \end{solution}
\end{problem}

\subsection{Finding Factors}

One class of problem that we may be interested in solving is to list the
factors for a number. When the number is small, this is easy. For example,
listing the factors of \(6\) is as simple as trying out all smaller (or equal)
positive integers, and finding that only \(1\), \(2\), \(3\), and \(6\) divide
\(6\) evenly.

Note that \(-1\), \(-2\), \(-3\), and \(-6\) are also factors of \(6\).
However, because these are redundant with the positive factors we found above,
we typically only care about finding the positive factors of a number.

For larger numbers, we may exploit the fact that factors always come in pairs.
If \(a \mid c\), then there is some number \(b\) so that \(a \times b = c\).
But then we can just flip the order of the multiplication, and we see that \(b
\times a = c\) and therefore \(b \mid c\). Therefore a convenient way to find
big factors is to divide the original number by the smaller factors.

We will revisit this topic, and obtain a faster way to do this type of problem,
in the future.

\begin{problem}{Positive Factors of \(30\)}
 Find the first \(4\) positive factors of \(30\).

 \begin{solution}
  It is clear that \(1 \mid 30\), \(2 \mid 30\), and \(3 \mid 30\). But since
  \(30 \bmod 4 = 2\), then \(4 \nmid 30\). So we check that \(30 \bmod 5 = 0\),
  so \(5 \mid 30\). Hence the first four factors are \Ans{1, 2, 3, 5}.
 \end{solution}
\end{problem}

\subsection{Revisiting Parity}

Using our new definitions, we see that the topic we covered last class---even
and odd numbers---is really just the study of numbers that are and aren't
divisible by \(2\). Even numbers are the multiples of \(2\), and all even
numbers have \(2\) as a factor. Be careful! Note that odd numbers do not
necessarily have \(3\) as a factor (for example, \(5\) is not a mutliple of
\(3\)), and even numbers might (for example, \(6\) is a multiple of \(3\) and
is also even.)

\subsection{Divisibility Rules}

Earlier we have already seen rules for divisibility by \(2\). We have a similar
rule for divisibility by \(5\). The reason that this rule is correct will be
discussed later in this unit, but please feel free to think about it. To figure
out whether a number is divisible by \(5\), simply look at the last digit. If
it is \(0\) or \(5\), then the number is divisible by \(5\).

For divisibility by \(10\), the rule is even simpler. A number is divisible by
\(10\) if and only if the last digit is \(0\). A number is divisible by \(100\)
if the last two digits are \(0\). (Note that \(0\) is a bit of a special case,
since it only has one digit, but it is still divisible by \(100\).)

For divisibility by \(3\), the rule is a little complicated. A number is
divisible by \(3\) if and only if the sum of its digits is divisible by \(3\).
For example, \(123456\) is divisible by \(3\), because the sum of digits
\(1+2+3+4+5+6=21\) is.

\subsection{Transitivity}

The \gls{transitivity} principle of divisiblity says that if \(a\), \(b\), and
\(c\) are integers, and if \(a \mid b\) and \(b \mid c\), then \(a \mid c\).
Let's do an example. We know that \(3 \mid 12\), and that \(12 \mid 36\). Then
by transitivity, \(3 \mid 36\). Here is a simple exercise that is easily done
using transitivity:

\begin{problem}{Multiples of \(12\) and \(3\)}
 Does \(3\) divide \emph{every} multiple of \(12\)? \hfill \AnsT{True}~~~False

 \begin{solution}
  Say \(a\) is some multiple of \(12\). By definition, \(12 \mid a\). Here we
  are using \(a\) as a placeholder---it can stand for \emph{any} multiple of
  \(12\).

  Since \(12 \div 3 = 4 \operatorname{R} 0\), we know that \(3 \mid 12\). But
  if \(3 \mid 12\) and \(12 \mid a\), then by transitivity, \(3 \mid a\).
  Therefore, the statement that \(3\) divides every multiple of \(12\) is
  \AnsT{true}.
 \end{solution}
\end{problem}

\section{Review}

There is a quiz for this chapter (Quiz 3: Parity and Divisibility). The
anticipated length for the quiz is \(15\) minutes. Three review questions
follow. Quiz questions will be very similar in nature.

\begin{problem}{Factor Pairs}
 Complete the factor pairs.

 \begin{itemize}
  \item \(6 = 1 \times \Ans{6}\)
  \item \(6 = 2 \times \Ans{3}\)
  \item \(6 = 3 \times \Ans{2}\)
  \item \(6 = 6 \times \Ans{1}\)
 \end{itemize}
\end{problem}

\begin{problem}{Factors and Multiples}
 Use the word ``factor'' or ``multiple'' to complete each blank.

 \begin{itemize}
  \item \(1\) is a \AnsT{factor} of every whole number.
  \item \(0\) is a \AnsT{multiple} of every whole number.
  \item \(3\) is a \AnsT{factor} of \(9\).
  \item All even numbers are \AnsT{multiple}s of \(2\).
 \end{itemize}
\end{problem}

\begin{problem}{Divisibility}
 Circle factors of each number. More than one factor may be circled.

 \begin{itemize}
  \item \(25\) \hfill \(\Ans{1}\)~~~\(2\)~~~\(\Ans{5}\)~~~\(10\)~~~\(100\)
  \item \(172\) \hfill \(\Ans{1}\)~~~\(\Ans{2}\)~~~\(5\)~~~\(10\)~~~\(100\)
  \item \(1793\) \hfill \(\Ans{1}\)~~~\(2\)~~~\(5\)~~~\(10\)~~~\(100\)
  \item \(2000\) \hfill \(\Ans{1}\)~~~\(\Ans{2}\)~~~\(\Ans{5}\)~~~\(\Ans{10}\)
  ~~~\(\Ans{100}\)
 \end{itemize}
\end{problem}

\chapter{Prime Numbers}

A \emph{prime number} is a positive integer that has exactly two positive
factors: \(1\) and itself. We care about prime numbers because they are the
basic building blocks for all other positive integers, a fact which we will
look at in more depth later on.

\section{Testing for Primality}

We can easily check whether small numbers are prime. First, note \(1\) is not
prime, because it has just one factor. However, \(2\) is prime, because it has
exactly two factors: \(1\) and itself.

\begin{problem}{The First Few Primes}
 Is each number prime or not prime?

 \begin{itemize}
  \item \(2\) \hfill \AnsT{Prime}~~~Not Prime
  \item \(3\) \hfill \AnsT{Prime}~~~Not Prime
  \item \(4\) \hfill Prime~~~\AnsT{Not Prime}
  \item \(5\) \hfill \AnsT{Prime}~~~Not Prime
  \item \(6\) \hfill Prime~~~\AnsT{Not Prime}
  \item \(7\) \hfill \AnsT{Prime}~~~Not Prime
  \item \(8\) \hfill Prime~~~\AnsT{Not Prime}
  \item \(9\) \hfill Prime~~~\AnsT{Not Prime}
  \item \(10\) \hfill Prime~~~\AnsT{Not Prime}
 \end{itemize}
\end{problem}

Don't get caught in the trap that every odd number is prime! \(9=3\times3\) is
not. Neither are \(15\), \(21\), \(25\), and many others. In fact, as the
numbers get bigger and bigger, prime numbers get rarer and rarer. However, do
note that there is only one even prime: \(2\).

\section{Composite Numbers}

A \emph{composite} number is a number that can be written as the product of two
smaller positive numbers. Thus \(4=2\times2\) is composite. It turns out that
every non-prime positive whole number except \(1\) is composite.

Composite numbers have at least three distinct factors. Usually, they have at
least four (such as \(6\), which has factors \(1\), \(2\), \(3\), and \(6\)),
but there is a special category of numbers that have exactly \(3\). These are
the squares of prime numbers. (Recall that a square is a number multiplied by
itself.) For instance, \(4\) has factors \(1\), \(2\), and \(4\), and \(9\) has
factors \(1\), \(3\) and \(9\). This happens when the only prime factor of a
number is paired with itself.

When numbers get quite large, composite numbers vastly outnumber prime numbers.
Many classes of numbers contain only one prime number and infinitely many
composite numbers. For instance, among the powers of \(2\) (\(2\), \(4\),
\(8\), \(16\), \dots), only \(2\) is prime.

\section{The Fundamental Theorem of Arithmetic}

One of the most important facts about whole numbers, one that forms the basis
for much of number theory, is the Fundamental Theorem of Arithmetic. This
theorem is a result that states that every positive whole number can be
factored uniquely into primes. What does that mean?

If we have a whole number, say, \(70\), we see that we can write it as
\(7\times10\). Furthermore, \(10=2\times5\). So we can factor
\(70=7\times2\times5\). Each of these numbers is now prime, and so we cannot
factor this further, except by introducing copies of \(1\), which is pointless.
(We could always write a number as itself times \(1\), but since multiplying by
\(1\) does not do anything, we might as well leave them out.)

But what if we had factored it a different way, first writing \(70=5\times14\)?
Then we notice that \(14=2\times7\), so we can change to
\(70=5\times2\times7\). The important thing is that this is the same
factorization as last time, except in a different order. In fact, no matter,
how we factor \(70\), we will always arrive at the same prime factorization
(with possibly a different order).

One way to quickly determine the prime factorization is to use a factor tree.
We draw the number we want to factor on top, and split it up into two children.
If any child is prime, we leave it be; otherwise, we split that up further into
two children. We repeat this procedure until all leaves are prime (see
Figure~\ref{pn:factortree70} and Figure~\ref{pn:factortree48} for examples).
The product of these leaves is the prime factorization.

\begin{figure}
 \begin{forest}
  [\(70\)
    [\(7\),circle,draw]
    [\(10\)
     [\(2\),circle,draw]
     [\(5\),circle,draw]]]
 \end{forest} \hspace{1em}
 \begin{forest}
  [\(70\)
    [\(14\)
     [\(2\),circle,draw]
     [\(7\),circle,draw]]
    [\(5\),circle,draw]]
 \end{forest}

 \caption{Two different factor trees for \(70\)}
 \label{pn:factortree70}
\end{figure}

\begin{figure}
 \begin{forest}
  [\(48\)
   [\(4\)
    [\(2\),circle,draw]
    [\(2\),circle,draw]]
   [\(12\)
    [\(2\),circle,draw]
    [\(6\)
     [\(2\),circle,draw]
     [\(3\),circle,draw]]]]
 \end{forest} \hspace{1em}
 \begin{forest}
  [\(48\)
   [\(3\),circle,draw]
   [\(16\)
    [\(4\)
     [\(2\),circle,draw]
     [\(2\),circle,draw]]
    [\(4\)
     [\(2\),circle,draw]
     [\(2\),circle,draw]]]]
 \end{forest}

 \caption{Two different factor trees for \(48\)}
 \label{pn:factortree48}
\end{figure}

\begin{problem}{Prime Factorization I}
 Find the prime factorization for each number.

 \begin{itemize}
  \item \(12 = \Ans{2\times2\times3}\)
  \item \(70 = \Ans{2\times5\times7}\)
  \item \(100 = \Ans{2\times2\times5\times5}\)
 \end{itemize}
\end{problem}

\begin{problem}{Small Prime Factors I}

 Find one prime factor for each number. Do not try to find additional factors.

 \begin{itemize}
  \item \(99959386\) \hfill Factor: \Ans{2}
  \item \(521976505\) \hfill Factor: \Ans{5}
 \end{itemize}

\end{problem}

\subsection{Factoring Large Numbers}

As we will see, it may be difficult to factor very large numbers. In fact, much
of modern cryptography, including the popular RSA encryption scheme, relies on
the fact that nobody has yet discovered a fast way to factor large numbers.

First, notice that checking whether large numbers are prime is quite hard to do
on paper. However, it is sometimes easy to see that some numbers are composite.
For example, if a number ends in \(0\), \(2\), \(4\), \(5\), \(6\), or \(8\),
then it is probably composite---the only two exceptions are \(2\) and \(5\).

But given a large number, like \(8644255723\), how do we know whether it is
prime or not? And how do we figure out its prime factorization? One way is to
use \emph{trial division}---just divide by all numbers starting from \(1\)
until you find one that leaves no remainder. It turns out \(8644255723 =
90907\times95089\), which are both prime numbers and cannot be factored
further.

Then how many divisions do we have to do before we found the first number that
works? We had to do \(90907\) divisions! That's a lot. If it took us one minute
to do each division on paper, then it would take over two months of continuous
work to find it.

Luckily, there are quicker ways to do factorization, and there are computers
that can do it much faster than humans can. But even computers have their
limits. Modern computers have a very hard time factoring numbers with thousands
of digits. But there is a fast way of checking whether a number is prime.

The implication is that it is easy to find two two big prime numbers with a
computer, and it is easy to multiply them, but it takes a long time to reverse
that operation and break the big number back down into its prime factors.

As an interesting note, which we will not explore further in this class, there
are actually infinitely many prime numbers. That means that no matter how big
our prime numbers get, we can always find another bigger prime number.

\subsection{Applications to Problem Solving}

So far in this unit, we have not done so many applications to problems of the
sort typically seen in math contests. However, this does not mean that the
Fundamental Theorem of Arithmetic is not useful for contest math. Many
questions can be simplified greatly by factorizing numbers into their prime
factorizations. It is even useful for mental arithmetic. We will see some
examples below.

\begin{problem}{Mental Arithmetic}
 Find, without using a calculator, \(1875\times48\).

 \begin{solution}
  We compute \[
   1875 = 5 \times 5 \times 5 \times 5 \times 3
  \] and \[
   48 = 2 \times 2 \times 2 \times 2 \times 3
  \] so \begin{align*}
   1875 \times 48
   &= 5 \times 5 \times 5 \times 5 \times 3 \times 2 \times 2 \times 2 \times 2
   \times 3 \\
   &= 10 \times 10 \times 10 \times 10 \times 3 \times 3 \\
   &= \Ans{90000}
  \end{align*}

  Factorizing numbers into primes before multiplying them is a common technique
  in mental math.
 \end{solution}
\end{problem}

\subsection{History}

The Fundamental Theorem of Arithmetic is known to be true. But for most people,
it is not obviously true. Why is it that all positive integers can be broken
down \emph{uniquely} into prime numbers? In mathematics, claims like the
Fundamental Theorem of Arithmetic require proof. That is, we need a convincing
argument that it is true for all numbers. The argument should depend only on
facts that are known and accepted to be true, or facts that also have proofs.
The first known proof for the Fundamental Theorem of Arithmetic was in book VII
(7) of Euclid's Elements, propositions 30 and 32. This proof is not covered in
the course.

\section{The Sieve of Eratosthenes}

Earlier, we found the first few prime numbers simply by finding factors. This
is unfortunately quite a slow way to find prime numbers. With a few
observations, we can find prime numbers much faster. Firstly, we noted earlier
that all factors come in pairs. For example, \(8 \mid 56\) (i.e. \(8\) is a
factor of \(56\)), so there is some number that we can multiply with \(8\) to
get \(56\). This number is \(7\). We have \(8 \times 7 = 7 \times 8 = 56\).

In the case of a perfect square, a factor could be paired with itself. For
example, \(10 \mid 100\), and indeed \(10 \times 10 = 100\), so \(10\) is
paired with itself. But being paired with itself still counts has being part of
a factor pair.

You may have noticed that when we search for factors of numbers, we find the
same factor pair possibly more than once. For instance, suppose we construct a
factor table for \(21\), trying out all the numbers from \(1\) to \(21\). See
Figure~\ref{pn:factor-21} for an example. This is a lot of work! It may seem at
first like all the work is necessary, because we don't find the last factor,
\(21\), until the last row. But notice that we have encounted all the factors
by the third row: some of them are just on the right.

Indeed, this is where the idea of pairs comes in. Instead of checking every
number, and try to get every pair of numbers \(a \times b = 21\), we could only
check the pairs \(a \times b = 21\) where \(a \le b\). That is to say, once the
left hand number exceeds the right hand number, then we don't need to check any
more. The right hand side of Figure~\ref{pn:factor-21} shows an abbreviated
table where we stop after checking \(4\). Why \(4\)? Well, \(5 \times 5 = 25\),
which is already bigger than \(21\). So we know that if \(5\) did divide \(21\)
(which it does not), then the factor it pairs with must be less than \(5\). But
then we would have already found it.

\begin{figure}
  \begin{tabular}{|cl|}
    \hline
    & \(21 = \) \\
    \hline
    \checkmark & \(1 \times 21\) \\
    & \(2 \times \dots\) \\
    \checkmark & \(3 \times 7\) \\
    & \(4 \times \dots\) \\
    & \(5 \times \dots\) \\
    & \(6 \times \dots\) \\
    \checkmark & \(7 \times 3\) \\
    & \(8 \times \dots\) \\
    & \(9 \times \dots\) \\
    & \(10 \times \dots\) \\
    & \(11 \times \dots\) \\
    & \(12 \times \dots\) \\
    & \(13 \times \dots\) \\
    & \(14 \times \dots\) \\
    & \(15 \times \dots\) \\
    & \(16 \times \dots\) \\
    & \(17 \times \dots\) \\
    & \(18 \times \dots\) \\
    & \(19 \times \dots\) \\
    & \(20 \times \dots\) \\
    \checkmark & \(21 \times 1\) \\
    \hline
  \end{tabular}
  \begin{tabular}{|clc|}
    \hline
    & \(21 = \) & \\
    \hline
    \checkmark & \(1 \times 21\) & \checkmark \\
    & \(2 \times \dots\) & \\
    \checkmark & \(3 \times 7\) & \checkmark \\
    & \(4 \times \dots\) & \\
    & \(5\) \textbf{stop} & \\
    \hline
  \end{tabular}

  \caption{On the left, a factor table used to find the factors of \(21\);
  while this works, it's a lot of work, not all necessary. On the right, an
  abbreviated factor table that still finds all the factors, but is much less
  work to draw.}
  \label{pn:factor-21}
\end{figure}

This significantly reduces the amount of work we have to do. But is there an
even quicker way? Over 2000 years ago, Eratosthenes of Cyrene devised an idea.
First, notice that you can find all the prime factors, then you can find all
the other factors. If we think of factorizing a number as breaking it down into
the pieces that make it up, then the prime factorization is the most broken
down state. We can recombine the prime factors to create the other factors.

For example, with \(21 = 3 \times 7\), and this is the prime factorization. We
can see that every factor of \(21\) is just \(3\), \(7\), or some combination.
(Think of \(1\), the special case, as using no pieces.) See
Figure~\ref{pn:factor-21-primes} for a diagram showing this process.

\begin{figure}
  \begin{tabular}{|ccl|}
    \hline
    Use \(3\)? & Use \(7\)? & Factor \\
    \hline
               &            & \(1\) \\
    \checkmark &            & \(3\) \\
               & \checkmark & \(7\) \\
    \checkmark & \checkmark & \(3\times7=21\) \\
    \hline
  \end{tabular}

  \caption{A table listing all factors of \(21\) by combining subsets of the
  prime numbers}
  \label{pn:factor-21-primes}
\end{figure}

TK

\section{Review}

\begin{problem}{Small Prime Factors II}
 Find one prime factor for each number. Do not try to find additional factors.

 \begin{itemize}
  \item \(999958\) \hfill Factor: \Ans{2}
  \item \(999993\) \hfill Factor: \Ans{3}
  \item \(999995\) \hfill Factor: \Ans{5}
 \end{itemize}

\end{problem}

\begin{problem}{Prime Factorization II}
 Find the unique prime factorization of each number.

 \begin{itemize}
  \item \(55\) \hfill \(\Ans{5} \times \Ans{11}\)
  \item \(30\) \hfill \(\Ans{2} \times \Ans{3} \times \Ans{5}\)
  \item \(42\) \hfill \(\Ans{2} \times \Ans{3} \times \Ans{7}\)
  \item \(24\) \hfill \(\Ans{2} \times \Ans{2} \times \Ans{2} \times \Ans{3}\)
 \end{itemize}
\end{problem}


\chapter{Greatest Common Divisor}

\section{Introduction}

In this chapter, we will discuss two very important
topics that have many applications: the greatest common divisor, and another
similar operation, the least common multiple.

\subsection{Greatest Common Divisor}

Given two positive integers, we define their GCD (greatest common divisor) to
be the largest positive integer that's a factor for both those numbers. For
instance, the factors of \(24\) are \(1\), \(2\), \(3\), \(4\), \(6\), \(8\),
\(12\), \(24\), and the factors of \(36\) are \(1\), \(2\), \(3\), \(4\),
\(6\), \(9\), \(12\), \(18\), \(36\). The greatest factor that these have in
common is \(12\). Therefore we denote \[ \gcd(24, 36) = 12 \]

\begin{problem}{Examples}
 \begin{itemize}
  \item \(\gcd(10, 20) = \Ans{10}\)
  \item \(\gcd(15, 25) = \Ans{5}\)
  \item \(\gcd(7, 11) = \Ans{1}\)
 \end{itemize}
\end{problem}

\subsection{Least Common Multiple}

Similarly, given two positive integers, we define their LCM (least common
multiple) to be the smallest positive integer that's a multiple of both those
numbers. For instance, the positive multiples of \(24\) are \(24\), \(48\),
\(72\), and so on, and the positive multiples of \(36\) are \(36\), \(72\),
\(108\), and so on. The smallest multiple that these have in common is \(72\).
Therefore we denote \[
 \operatorname{lcm}(24, 36) = 72
\]

An intriguing connection between the GCD and the LCM is given by the following
identity, which we will not justify in this class, but you may see again in the
future: \begin{equation}
 \gcd(a, b) \times \operatorname{lcm}(a, b) = a \times b
\end{equation}

Roughly speaking, this identity says that the product of two numbers is the
same as the product of their GCD and their LCM.

\section{Application to Rational Numbers}

The GCD and the LCM have many applications across mathematics. An important
application is to the study of rational numbers, or fractions. Recall that a
fraction is a pair of two integers, the numerator and the denominator.
Additionally, the denominator is not allowed to be zero, because we cannot
split any quantity into zero parts.

Note that a fraction does not have a unique representation as a numerator and
denominator. Two fractions may be equal even if the numerator and the
denominators differ. For instance, the fraction \(\frac{1}{2}\) is equal to
fractions \(\frac{2}{4}\), \(\frac{3}{6}\), and so on.

We may \emph{simplify} a fraction by writing it as an equivalent fraction with
least possible (positive) denominator. Simplifying fractions makes it easy to
determine whether two fractions are equivalent, and also makes fractions easier
to work with. There exists a simple algorithm to simplify a fraction. Given
\(\frac{n}{m}\), compute \(d=\gcd(n, m)\), and then \(\frac{n\div d}{m\div d}\)
is the simplest form.

\begin{problem}{Simplify Fraction}
 Simplify. Note that the fraction may possibly already be in simplest form.

 \begin{itemize}
  \item \(\frac{20}{15}=\Ans{\frac{4}{3}}\)
  \item \(\frac{24}{36}=\Ans{\frac{2}{3}}\)
  \item \(\frac{7}{13}=\Ans{\frac{7}{13}}\)
 \end{itemize}
\end{problem}

When we add or subtract fractions, it helps often to write the fractions with
equal denominator. To select a compatible denominator, we may use the least
common multiple. For example, to add \(\frac{a}{b}+\frac{c}{d}\), we may
compute \(\ell=\operatorname{lcm}(b, d)\) and then compute \begin{align*}
 \frac{a}{b}+\frac{c}{d}
 &= \frac{a\ell\div b}{\ell} + \frac{c\ell\div d}{\ell} \\
 &= \frac{a\ell\div b + c\ell\div d}{\ell}
\end{align*} which then we may simplify with the algorithm described above.

\begin{problem}{Adding and Subtracting Fractions}
 Add or subtract. Then simplify.

 \begin{itemize}
  \item \(\frac{1}{2} + \frac{1}{3} = \Ans{\frac{5}{6}}\)
  \item \(\frac{3}{4} + \frac{1}{6} = \Ans{\frac{11}{12}}\)
  \item \(\frac{23}{36} - \frac{11}{24} = \Ans{\frac{13}{36}}\)
 \end{itemize}
\end{problem}

\section{Euclidean Algorithm}

An \gls{algorithm} is a ``step-by-step procedure for performing a calculation
according to well-defined rules'' (Wikipedia).

In this lesson we will cover a useful algorithm that allows us to find the
greatest common divisor of two large numbers. This algorithm has been known for
thousands of years. It was discovered by the Ancient Greeks, and is named after
Euclid, one of the most famous mathematicians in history.

This algorithm is based on the fact that the greatest common denominator does
not change if the smaller number is subtracted from the larger one. That is,
\(\gcd(a, b) = \gcd(a, b - a)\) for all positive integers \(a\) and \(b\), with
\(b>a\). Proving this fact is beyond the scope of this course. However, we will
look at an example.

Consider \(\gcd(8, 12)=4\). If we subtract \(8\) from \(12\), we get \(\gcd(8,
4)=4\). If we subtract now \(4\) from \(8\), we get \(\gcd(4, 4)=4\). We notice
that the GCD is unchanged despite these subtractions.

To avoid subtracting potentially many times, we can change the repeated
subtraction into a remainder operation. Thus we arrive at the most commonly
stated version of the Euclidean algorithm:

\subsection{The Euclidean Algorithm}

To find the Greatest Common Divisor of two numbers, \(a\) and \(b\):
\begin{enumerate}
 \item If \(a=0\), then terminate. The greatest common divisor is \(b\).
 \item Otherwise, apply the Euclidean Algorithm to find the GCD of \(b \bmod
 a\) and \(a\). That is, \(\gcd(a, b) = \gcd(b \bmod a, a)\).
\end{enumerate}

\subsection{Properties}

This algorithm is an example of a recursive algorithm. More specifically, it is
a ``tail recursive'' algorithm. That means that the algorithm either produces a
result, or it requires running the algorithm another time on simpler numbers.
We may need to run the Euclidean algorithm several times on progressively
simpler numbers before finishing the computation.

Although the Euclidean algorithm may take some time to compute by hand, it is
very easy for a computer to do. In fact, the algorithm is so efficient that
computers can easily calculate the greatest common divisors of numbers with
thousands of digits, in a matter of milliseconds!

\subsection{Examples}

Here is a quick example of using the Euclidean algorithm to compute the
greatest common divisor of medium-sized numbers:

\begin{problem}{GCD of Two Numbers}
 Compute \(\gcd(60, 84) = \Ans{12}\).

 \begin{solution}
  \begin{align*}
   \gcd(60, 84)
   &= \gcd(84 \bmod 60, 60) \\
   &= \gcd(24, 60) \\
   &= \gcd(60 \bmod 24, 24) \\
   &= \gcd(12, 24) \\
   &= \gcd(24 \bmod 12, 12) \\
   &= \gcd(0, 12) \\
   &= 12
  \end{align*}
 \end{solution}
\end{problem}

We may alternatively seek to find the GCD of multiple (more than two) numbers.
This means the largest number that is a divisor of all the numbers we're
interested in. We can do this by finding the GCD of the first two, and then
finding the GCD of that number and the third, and so on. That is, we use the
formula \[
 \gcd(a, b, c, d, \dots) = \gcd(\gcd(a, b), c, d, \dots)
\]

This computation may require several long and tedious applications of the
Euclidean algorithm. Luckily, in practice, when GCDs for large numbers must be
computed, computers can typically be used. You will not be required to do
problems of this sort on your homework or quizzes.

\begin{problem}{GCD of Three Numbers}
 Compute \(\gcd(3636, 3948, 4056) = \Ans{12}\).

 \begin{solution}
  \begin{align*}
   \gcd(3636, 3948, 4056)
   &= \gcd(\gcd(3636, 3948), 4056) \\
   &= \gcd(\gcd(3948 \bmod 3636, 3636), 4056) \\
   &= \gcd(\gcd(312, 3636), 4056) \\
   &= \gcd(\gcd(3636 \bmod 312, 312), 4056) \\
   &= \gcd(\gcd(204, 312), 4056) \\
   &= \gcd(\gcd(312 \bmod 204, 204), 4056) \\
   &= \gcd(\gcd(108, 204), 4056) \\
   &= \gcd(\gcd(204 \bmod 108, 108), 4056) \\
   &= \gcd(\gcd(96, 108), 4056) \\
   &= \gcd(\gcd(108 \bmod 96, 96), 4056) \\
   &= \gcd(\gcd(12, 96), 4056) \\
   &= \gcd(\gcd(96 \bmod 12, 12), 4056) \\
   &= \gcd(\gcd(0, 12), 4056) \\
   &= \gcd(12, 4056) \\
   &= \gcd(4056 \bmod 12, 12) \\
   &= \gcd(0, 12) \\
   &= \Ans{12}
  \end{align*}
 \end{solution}
\end{problem}

We may cover a few assorted problems to reinforce your ability to compute the
GCD using the Euclidean algorithm. However, in general, this computation is not
difficult, but is brainless and long, and I do not recommend you do many of
these problems for practice.

\begin{problem}{Assorted GCD Problems}
 Use the Euclidean Algorithm to compute the following:

 \begin{itemize}
  \item \(\gcd(12, 28) = \Ans{4}\)
  \item \(\gcd(147, 392) = \Ans{49}\)
  \item \(\gcd(319, 920) = \Ans{1}\)
  \item \(\gcd(2635, 4515) = \Ans{5}\)
  \item \(\gcd(12936, 25256) = \Ans{616}\)
  \item \(\gcd(399, 1533, 1659, 2016) = \Ans{1}\)
 \end{itemize}
\end{problem}

\subsection{Computer Algorithm}

One advantage of an algorithm is that we can run them on a computer. For your
interest, in Figure~\ref{gcd:julia} we have an \gls{implementation} of the
Euclidean algorithm in the Julia \gls{programming language}. In the Julia
language, \verb|abs| represents the absolute value, and \verb|rem| represents
the remainder.

\begin{figure}
  \begin{minted}{julia}
gcd(a, b) = if a == 0
    abs(b)
else
    gcd(rem(b, a), a)
end
  \end{minted}
  \caption{An implementation of the Euclidean algorithm in Julia}

  \label{gcd:julia}
\end{figure}

As mentioned earlier, the Euclidean algorithm is quite efficient. But in fact,
there are even more efficient algorithms for finding the GCD of two large
numbers on a computer. Computers can find the GCD of two huge numbers almost
instantly---even if the numbers involved have thousands of digits.

\chapter{Modular Arithmetic}

The final chapter in this unit will be brief, and you will not be quizzed on
the material. However, the content of this chapter will certainly be
interesting.

In earlier chapters we discussed briefly remainders and some applications. In
this chapter, we will develop new powerful techniques to simplify the solution
to these problems.

\section{Generalizing Parity}

When we talked about even and odd numbers, we came up with rules about the
result of adding, subtracting, and multiplying even and odd numbers. We can
come up with similar strategies for divisibility by any number, not just \(2\).
For the purposes of discussing this with greater ease, we will consider \(7\)
as our base.

All positive integers (in fact, all integers, including zero and negative ones)
can be categorized into \(7\) distinct buckets based on their remainder when
divided by \(7\). Let us call these buckets \([0]\), \([1]\), \([2]\), \([3]\),
\([4]\), \([5]\), and \([6]\). In the following table we have categorized the
numbers from \(0\) to \(30\) into these buckets:

\begin{center}
 \begin{tabular}{|c|lllll|}
  \hline
  \([0]\) & \(0\) & \(7\) & \(14\) & \(21\) & \(28\) \\ \hline
  \([1]\) & \(1\) & \(8\) & \(15\) & \(22\) & \(29\) \\ \hline
  \([2]\) & \(2\) & \(9\) & \(16\) & \(23\) & \(30\) \\ \hline
  \([3]\) & \(3\) & \(10\) & \(17\) & \(24\) & \\ \hline
  \([4]\) & \(4\) & \(11\) & \(18\) & \(25\) & \\ \hline
  \([5]\) & \(5\) & \(12\) & \(19\) & \(26\) & \\ \hline
  \([6]\) & \(6\) & \(13\) & \(20\) & \(27\) & \\ \hline
 \end{tabular}
\end{center}

We can see a few patterns in the above table. When we add \(1\) to a number, we
move to the next bucket. But \([6]\) has no next bucket, so what happens when
we add \(1\) to a number in bucket \([6]\)? We move to bucket \([0]\)! In some
sense, these buckets form a sort of cycle. After we reach the last bucket, we
loop around and return to the first bucket (Figure~\ref{ma:equiv-class-cycle}).

\begin{figure}
 \fgLabelledCycle{7}

 \caption{When we add \(1\) to a number in a bucket, we move to the next bucket
 in this diagram.}

 \label{ma:equiv-class-cycle}
\end{figure}

This behaviour should make sense when starting at buckets \([0]\) to \([5]\).
But for bucket \([6]\), why is it that we move to bucket \([0]\) by adding one
more? We can find the solution using algebra, just like how we did when talking
about parity. Any number in bucket \([6]\) looks like \(7n+6\) for some integer
\(n\). If we add \(1\) to this, \[ 7n+6 + 1 = 7n+7 = 7(n+1) \] and we can see
that this is a multiple of \(7\), and so belongs in bucket \([0]\).

We now consider what happens if we add \(2\) to a number. In which bucket would
the new number be in? This is the same as adding \(1\) twice. So, for example,
\([0]\) goes to \([2]\), \([4]\) goes to \([6]\), \([5]\) goes to \([0]\) and
\([6]\) goes to \([1]\). Again, we are just moving around in circles. It is
similar to addition on the number line, but instead of going in a line, we are
going in a cycle.

The process of addition is moving numbers from one remainder bucket to the next
several times. What happens if we add \(7\)? Well, we would have gone one whole
cycle around, and so our result should be in the same bucket. And indeed, that
is what we would expect, since adding \(7\) should not change the remainder
when dividing by \(7\).

The same goes for any multiple of \(7\). If we add \(21\), we would go around
one whole cycle \(3\) times, and then arrive back where we started. For bigger
numbers that are not multiples of \(7\), we can think of the addition as moving
around the whole cycle several times, then moving some extra steps. How many
extra steps do we move? Why, the remainder when divided by \(7\), of course.

To recap, we have devised a system where we can predict the remainder of a sum
when divided by \(7\), knowing just the remainders of the two \glspl{summand}
when divided by \(7\). We simply start at the bucket for the first number and
advance a number of buckets in the cycle equal to the remainder of the second
number when divided by \(7\). We introduce the following notation: \[ [1] + [2] =
[3] \] to mean that the sum of a number in bucket \([1]\) and a number in
bucket \([2]\) is a number in bucket \([3]\).

We can return to algebra to provide a justification for this behaviour. Say \(n
\bmod 7 = i\), and \(m \bmod 7 = j\). Then let \(n = 7k + i\) and \(m = 7\ell +
j\). We see that \begin{align*}
 (n + m) \bmod 7
 &= (7k+i + 7\ell+j) \bmod 7 \\
 &= (7(k+\ell) + i + j) \bmod 7 \\
 &= (i + j) \bmod 7
\end{align*} which is the same as saying that the remainder of the sum of two
numbers when divided by \(7\) is the same as the remainder of the sum of the
remainders when divided by \(7\). In other words, the bucket of the result is
determined only by the buckets of the summands.

By now we have come up with a generalization of the concept of parity (for
divisibility by \(2\)) to divisibility by \(7\). Instead of categorizing
numbers into even and odd numbers, we categorize them into the \(7\) buckets
depending on their remainder when divided by \(7\). But we could have chosen
any positive integer to start with, not just \(7\). By using the same
techniques we applied above, we can create a new way of analyzing numbers using
any positive integer as our base.

\section{Introducing Multiplication}

For simplicity, we'll continue to use \(7\) as our base, but keep in mind that
the interesting results we will notice are valid in other bases also. Above, we
saw the important fact that the remainder of the sum of two numbers is the same
as the remainder of the sum of the remainders of two numbers. There is a
similar fact for multiplication. Indeed, \[ ab \bmod 7 = (a \bmod 7)(b \bmod 7)
\bmod 7 \]

That is, instead of multiplying two numbers and then taking the remainder of
the product, we can instead multiply their remainders and take the remainder of
that. While above we gave an algebraic explanation for the sum of two numbers,
for the sake of time we will not do that for the product. But a similar
algebraic explanation is possible, and if you are interested, you may try to
devise it yourself.

We are able to solve advanced problems easily by applying the techniques shown
above. Let us consider, for example, the following problems:

\begin{problem}{A Million Days}

 Today is Sunday. \(1000\) days later, it will be Saturday. \(1000000\) days
 later, what day will it be?

 \begin{solution}

  The question tells us that \(1000\) days later, it will be Saturday. Since
  Saturday is \(6\) days past Sunday, we know that \(1000 \bmod 7 = 6\).

  We need to compute \(1000000 \bmod 7\). We can use the technique developed
  above to simplify: \begin{align*}
   1000000 \bmod 7
   &= (1000 \times 1000) \bmod 7 \\
   &= ((1000 \bmod 7) \times (1000 \bmod 7)) \bmod 7 \\
   &= (6 \times 6) \bmod 7 \\
   &= 36 \bmod 7 \\
   &= 1
  \end{align*}
  so after \(1000000\) days, the day of the week will be \(1\) past Sunday,
  hence \AnsT{Monday}.

 \end{solution}
\end{problem}

\begin{problem}{A Large Power of \(2\)}

 Find \(2^{32} \bmod 7\).

 \begin{solution}

  We could find the product, but that would take a long time. Since \(16\) is a
  power of \(2\), it may be useful to consider the remainders of \(2^1\),
  \(2^2\), \(2^4\), \(2^8\), \(2^{16}\), and then finally \(2^{32}\).

  First, note that \(2 \bmod 7 = 2\). Then note \(2^2 \bmod 7 = 4 \bmod 7 =
  4\). Next, note that \begin{align*}
   2^4 \bmod 7
   &= (2 \times 2 \times 2 \times 2) \bmod 7 \\
   &= 16 \bmod 7 \\
   &= 2
  \end{align*} and so far, all the calculations have been short and easy.

  For \(2^8 \bmod 7\), instead of calculating the product, we could apply the
  above result: \begin{align*}
   2^8 \bmod 7
   &= (\underbrace{2 \times 2 \times \dots \times 2}_{8\text{ twos}}) \bmod 7 \\
   &= (2 \times 2 \times 2 \times 2)(2 \times 2 \times 2 \times 2) \bmod 7 \\
   &= 16\times16 \bmod 7 \\
   &= (16 \bmod 7)(16 \bmod 7) \bmod 7 \\
   &= 2 \times 2 \bmod 7 \\
   &= 4 \bmod 7 \\
   &= 4
  \end{align*}

  Next, for \(2^{16} \bmod 7\), we do the same thing again: \begin{align*}
   2^{16} \bmod 7
   &= (\underbrace{2 \times 2 \times \dots \times 2}_{16\text{ twos}}) \bmod 7 \\
   &= (\underbrace{2 \times 2 \times \dots \times 2}_{8\text{ twos}})
   (\underbrace{2 \times 2 \times \dots \times 2}_{8\text{ twos}}) \bmod 7 \\
   &= 2^8\times2^8 \bmod 7 \\
   &= (2^8 \bmod 7)(2^8 \bmod 7) \bmod 7 \\
   &= 4 \times 4 \bmod 7 \\
   &= 16 \bmod 7 \\
   &= 2
  \end{align*}

  And finally, one more time for \(2^{32} \bmod 7\): \begin{align*}
   2^{32} \bmod 7
   &= (\underbrace{2 \times 2 \times \dots \times 2}_{32\text{ twos}}) \bmod 7 \\
   &= (\underbrace{2 \times 2 \times \dots \times 2}_{16\text{ twos}})
   (\underbrace{2 \times 2 \times \dots \times 2}_{16\text{ twos}}) \bmod 7 \\
   &= 2^{16}\times2^{16} \bmod 7 \\
   &= (2^{16} \bmod 7)(2^{16} \bmod 7) \bmod 7 \\
   &= 2 \times 2 \bmod 7 \\
   &= 4 \bmod 7 \\
   &= \Ans{4}
  \end{align*}

 \end{solution}
\end{problem}

\section{Notation}

The kind of computation we were doing above is known as \emph{modular
arithmetic}. Sometimes it is convenient to use specialized notation to describe
operations. Now we will take look at the commonly used notation, and how it can
be used to solve problems.

First, we introduce the concept that two numbers are \emph{congruent} modulo
some base \(k\) if they have the same remainder when divided by \(k\). For
example, \(17\) and \(10\) are congruent modulo \(7\), because \[ 17 \bmod 7 =
3 = 10 \bmod 7 \].

We can write this as: \[
 17 \equiv 10 \pmod 7
\]
or alternatively without the brackets, but keeping the additional space, as \[
 17 \equiv 10 \mod 7
\]

By working with congruences, we can solve complex problems in relatively few
lines of work. For instance, consider the following.

\begin{problem}{Huge Number, Small Remainder}
 Define \(10! = 10 \times 9 \times 8 \times \dots \times 1\). Find \(10! \bmod 72\).

 \begin{solution}
  We see that \begin{align*}
   10!
   &\equiv 10 \times 9 \times 8 \times \dots \times 1 \mod {72} \\
   &\equiv 9 \times 8 \times 10 \times 7 \times 6 \times \dots \times 1
   \mod {72} \\
   &\equiv 72 \times 10 \times 7! \mod {72} \\
   &\equiv 0 \times 10 \times 7! \mod {72} \\
   &\equiv 0 \mod {72}
  \end{align*}
  so \(10! \bmod 72 = \Ans{0}\).
 \end{solution}
\end{problem}

\begin{problem}{Units Digit}
 Find the ones' digit of the following sum: \[
  40 + 7874 + 648 + 56 + 338
 \]

 \begin{solution}
  We seek \[
   (40 + 7874 + 648 + 56 + 338) \bmod 10
  \]

  It is straightforward to compute this: \begin{align*}
   40 + 7874 + 648 + 56 + 338
   &\equiv 0 + 4 + 8 + 6 + 8 \\
   &\equiv 26 \\
   &\equiv 6 \pmod {10}
  \end{align*}
  and hence the ones' digit of the desired sum is \Ans{6}.
 \end{solution}
\end{problem}

\begin{problem}{Clock Arithmetic}

 A \(12\)-hour analog clock (which are increasingly rare nowadays) is a
 time-keeping device numbered from \(1\) to \(12\) (see Figure~\ref{ma:clock}).
 There are two hands. The longer one is the minute hand, which measures
 minutes, and the shorter one is the hour hand, which measures hours. Each
 hour, the hour hand of the clock moves clockwise (that is, up one number) by
 one labelled segment. Every \(12\) hours, the clock returns to its original
 orientation. For the purposes of this question, we'll ignore the minute hand
 and focus only on the hour hand.

 If a clock's hour hand is pointing to the \(5\) right now, to where was it
 pointing \(13\times29\) hours ago?

 \begin{solution}
  We seek \(5 - 13 \times 29 \bmod 12\). It is straightforward to calculate
  this with modular arithmetic: \begin{align*}
   5 - 13 \times 29
   &\equiv 5 - 1 \times 5 \\
   &\equiv 5 - 5 \\
   &\equiv 0 \mod {12}
  \end{align*}

  Of course, there is no \(0\) on the clock---in its place is \Ans{12}, which
  was where the hour hand was pointing.
 \end{solution}

\end{problem}

\begin{figure}
 \fgClock{}

 \caption{A 12-hour clock displaying the time 1:00.}

 \label{ma:clock}
\end{figure}

\section{Further Study}

We do not have time to further discuss modular arithmetic in this course. But
this subject is very powerful, and you will see it again and again in the
future.

% Glossaries and List of Figures
\printglossaries

\cleardoublepage
\addcontentsline{toc}{chapter}{\listfigurename}
\listoffigures
\end{document}
