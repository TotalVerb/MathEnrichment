\documentclass[a4paper,10pt]{report}

\newcommand{\IncludePath}{../include}
\newcommand{\ProjectName}{Grade 4 Olympic Math}
\usepackage{extsizes}
\usepackage{titling}

\usepackage{tikz}
\usetikzlibrary{shapes}

\usepackage{amssymb,amsmath,amsthm}
\usepackage{enumerate}
\usepackage{graphicx,ctable,booktabs}
\usepackage{fancyhdr}
\usepackage[utf8]{inputenc}
\usepackage{gensymb}

\usepackage[toc]{glossaries}

\makeatletter
\newenvironment{problem}{\@startsection
       {subsection}
       {1}
       {-.2em}
       {-3.5ex plus -1ex minus -.2ex}
       {2.3ex plus .2ex}
       {\pagebreak[3]
       \large\bf\noindent{Problem }
       }
       }
\makeatother

\makeatletter
\g@addto@macro\@floatboxreset\centering
\makeatother

\newenvironment{solution}
{ \vspace{1em} \noindent \textbf{Solution:} }
{  }

\pagestyle{fancy}
\lhead{\thetitle}
\chead{}
\rhead{\thepage}
\lfoot{\small\scshape \ProjectName}
\cfoot{}
\rfoot{}
\renewcommand{\headrulewidth}{.3pt}
\renewcommand{\footrulewidth}{.3pt}
\setlength\voffset{-0.25in}
\setlength\textheight{648pt}
\setlength\headheight{15pt}

\newcommand{\Ans}[1]{\framebox{$#1$}}
\newcommand{\SAA}[1]{\Switch{\Ans{#1}}{\blankA}}
\newcommand{\SAB}[1]{\Switch{\Ans{#1}}{\blankB}}
\newcommand{\SAC}[1]{\Switch{\Ans{#1}}{\blankC}}
\newcommand{\SAD}[1]{\Switch{\Ans{#1}}{\blankD}}
\newcommand{\SAE}[1]{\Switch{\Ans{#1}}{\blankE}}
\newcommand{\SAF}[1]{\Switch{\Ans{#1}}{\blankF}}
\newcommand{\STA}[1]{\Switch{\AnsT{#1}}{\blankA}}
\newcommand{\STB}[1]{\Switch{\AnsT{#1}}{\blankB}}
\newcommand{\STC}[1]{\Switch{\AnsT{#1}}{\blankC}}
\newcommand{\STD}[1]{\Switch{\AnsT{#1}}{\blankD}}
\newcommand{\STE}[1]{\Switch{\AnsT{#1}}{\blankE}}
\newcommand{\STF}[1]{\Switch{\AnsT{#1}}{\blankF}}
\newcommand{\AnsT}[1]{\framebox{#1}}
\newif\ifanswers
\newcommand{\Switch}[2]{\ifanswers#1\else#2\fi}
\newcommand{\MCSelect}[1]{\Switch{\AnsT{#1}}{#1}}
\newcommand{\TFTrue}{\MCSelect{True}~~False}
\newcommand{\TFFalse}{True~~\MCSelect{False}}

\newcommand{\blankA}{\underline{\hspace{1em}}}
\newcommand{\blankB}{\underline{\hspace{2em}}}
\newcommand{\blankC}{\underline{\hspace{3em}}}
\newcommand{\blankD}{\underline{\hspace{4em}}}
\newcommand{\blankE}{\underline{\hspace{5em}}}
\newcommand{\blankF}{\underline{\hspace{6em}}}



\title{Number Theory}
\author{Fengyang Wang}

\begin{document}
\maketitle

\chapter{Parity \& Divisibility}

Number Theory is the study of whole numbers. That means $0$, $1$, $2$, and so
on. Number Theory also studies negative whole numbers, such as $-1$, $-2$, and
so on. Positive and negative whole numbers, and $0$, are together called
``integers''.

\section{Parity}

``Parity'' is the term used for odd and even numbers. Today, we will study their
differences and several properties. We will approach the problem algebraically.

An even number is a number of the form $2n$, where $n$ is any integer. The
expression $2n$ means $2\times n$; we omit the $\times$ so that it is faster to
read and write.

An odd number is a number of the form $2n+1$, where $n$ is any integer. The
expression $2n+1$ means $(2\times n) + 1$. We perform the multiplication first
because of the order of operations.

\subsection{Determining Parity}
It is easy to determine whether a number is even or odd. Let's start with small
examples first. We know $4=2\times2$, so it's even. We know $7=2\times3+1$, so
it's odd. A number is either even or odd, but it's never both. We only need to
divide by $2$: if it evenly divides, then it's even, and otherwise, it's odd.

\begin{problem}{Even or Odd?}
 Are each of the following numbers even or odd?

 \begin{itemize}
  \item $10$ \hfill \AnsT{Even}~~~Odd
  \item $0$ \hfill \AnsT{Even}~~~Odd
  \item $19$ \hfill Even~~~\AnsT{Odd}
  \item $-1$ \hfill Even~~~\AnsT{Odd}
 \end{itemize}
\end{problem}

If the numbers are very big, it is convenient to simply observe the last digit.
If the last digit is even, then the whole number is even. If the last digit is
odd, then the whole number is odd. Why does this work? We will see very soon.

\subsection{Sums \& Differences}
What happens if we add two odd numbers? Two even numbers? Let's find out. Say
$2n+1$ and $2m+1$ are two odd numbers. We're using letters here as placeholders;
$n$ and $m$ can both take the value of any integer. Then \[
 (2n+1) + (2m+1) = 2n + 2m + 1 + 1 = 2n + 2m + 2 = 2(n+m+1)
\] but $n+m+1$ is an integer, and therefore the sum is even.

We can use a very similar strategy to find the sum of two even numbers. This is
even easier. Say $2n$ and $2m$ are two even numbers. Then \[
 2n + 2m = 2(n+m)
\] but $n+m$ is an integer, and therefore the sum is even.

Say $2n$ is an even number, and $2m+1$ is an odd number. Then \[
 2n + 2m + 1 = 2(n+m) + 1
\] which is an odd number. So the sum of an even number and an odd number is
odd. And because we can rearrange the order of addition, therefore the sum of an
odd number and a even number is also odd.

The rules for subtracting are exactly the same. The difference of two odds is
even. The difference of two evens is even. The difference of an odd and an even
is odd. Feel free to work these out yourself.

\begin{problem}{Even or Odd?}
 Are each of the following numbers even or odd?

 \begin{itemize}
  \item $100 + 200$ \hfill \AnsT{Even}~~~Odd
  \item $1273 + 19023$ \hfill \AnsT{Even}~~~Odd
  \item $19082 - 1911$ \hfill Even~~~\AnsT{Odd}
  \item $109291 + 8329 + 9107$ \hfill Even~~~\AnsT{Odd}
 \end{itemize}
\end{problem}

\subsection{Products}
We can use a similar technique to develop rules for the product of even and odd
numbers. It turns out that the product of two odd numbers is odd. Say $2n+1$
and $2m+1$ are two odd numbers. Observe: \begin{align*}
 (2n+1)(2m+1)
 &= 2n(2m+1) + (2m+1) \\
 &= 4nm+2n+2m+1 \\
 &= 2(2nm+n+m) + 1
\end{align*} which is odd.

But the product of an even number with any integer is even. Say $2n$ is an
even number, and $k$ is any integer. Then: \[
 (2n)k = 2(nk)
\] which is even.

\section{Divisibility}

We will begin today's topic with a few definitions.

The symbol $\mid$ means ``divides''. If $A$ and $B$ are two numbers, then $A$
divides $B$ when $B \div A$ has no remainder.

A number is a \emph{factor} of another number if it divides that number. For
example, $3$ is a \emph{factor} of $9$ because $3 \mid 9$. A number is a
\emph{multiple} of another number if the other number divides it. For example,
$9$ is a \emph{multiple} of $3$. Another word for factor, which we will see
later on, is \emph{divisor}.

Observe that $1$ is a \emph{factor} of every whole number, and $0$ is a
\emph{multiple} of every whole number.

\subsection{Revisiting Parity}

Using our new definitions, we see that the topic we covered last class---even
and odd numbers---is really just the study of numbers that are and aren't
divisible by $2$.

\subsection{Transitivity}

The \emph{transitivity} principle says that if $a$, $b$, and $c$ are integers,
and if $a \mid b$ and $b \mid c$, then $a \mid c$. Let's do an example. We know
that $3 \mid 12$, and that $12 \mid 36$. Then by transitivity, $3 \mid 36$. Here
is a simple exercise that is easily done using transitivity:

\begin{problem}{True or False}
 Does $3$ divide \emph{every} multiple of $12$? \hfill \AnsT{True}~~~False
\end{problem}

\section{Quiz}

\subsection{Quiz Review}

There is a quiz today (Quiz 3: Parity and Divisibility). The allocated length
for the quiz is $15$ minutes. Three review questions follow. Quiz questions will
be very similar in nature.

\begin{problem}{Factor Pairs}
 Complete the factor pairs.

 \begin{itemize}
  \item $6 = 1 \times \Ans{6}$
  \item $6 = 2 \times \Ans{3}$
  \item $6 = 3 \times \Ans{2}$
  \item $6 = 6 \times \Ans{1}$
 \end{itemize}
\end{problem}

\begin{problem}{Factors and Multiples}
 Use the word ``factor'' or ``multiple'' to complete each blank.

 \begin{itemize}
  \item $1$ is a \AnsT{factor} of every whole number.
  \item $0$ is a \AnsT{multiple} of every whole number.
  \item $3$ is a \AnsT{factor} of $9$.
  \item All even numbers are \AnsT{multiples}s of $2$.
 \end{itemize}
\end{problem}

\begin{problem}{Divisibility}
 Circle factors of each number. More than one factor may be circled.

 \begin{itemize}
  \item $25$ \hfill $\Ans{1}$~~~$2$~~~$\Ans{5}$~~~$10$~~~$100$
  \item $172$ \hfill $\Ans{1}$~~~$\Ans{2}$~~~$5$~~~$10$~~~$100$
  \item $1793$ \hfill $\Ans{1}$~~~$2$~~~$5$~~~$10$~~~$100$
  \item $2000$ \hfill $\Ans{1}$~~~$\Ans{2}$~~~$\Ans{5}$~~~$\Ans{10}$
  ~~~$\Ans{100}$
 \end{itemize}
\end{problem}

\chapter{Prime Numbers}

Today's topic is prime numbers.

\section{Primes}
A \emph{prime number} is a positive integer that has exactly two positive
factors: $1$ and itself. We can easily check whether small numbers are prime.
First, note $1$ is not prime, because it has just one factor. However, $2$ is
prime, because it has exactly two factors: $1$ and itself.

\begin{problem}{Prime}
 Is each number prime or not prime?

 \begin{itemize}
  \item $3$ \hfill \AnsT{Prime}~~~Not Prime
  \item $4$ \hfill Prime~~~\AnsT{Not Prime}
  \item $5$ \hfill \AnsT{Prime}~~~Not Prime
  \item $6$ \hfill Prime~~~\AnsT{Not Prime}
  \item $7$ \hfill \AnsT{Prime}~~~Not Prime
  \item $8$ \hfill Prime~~~\AnsT{Not Prime}
  \item $9$ \hfill Prime~~~\AnsT{Not Prime}
  \item $10$ \hfill Prime~~~\AnsT{Not Prime}
 \end{itemize}
\end{problem}

Don't get caught in the trap that every odd number is prime! $9=3\times3$ is
not. Neither are $15$, $21$, $25$, and many others. In fact, as the numbers get
bigger and bigger, prime numbers get rarer and rarer.

However, do note that there is only one even prime: $2$.

\subsection{Composite Numbers}
A \emph{composite} number is a number that can be written as the product of two
smaller positive numbers. Thus $4=2\times2$ is composite. It turns out that
every non-prime positive whole number except $1$ is composite.

\section{The Fundamental Theorem of Arithmetic}

One of the most important facts about whole numbers, one that forms the basis
for much of number theory, is the Fundamental Theorem of Arithmetic. This
theorem is a result that states that every positive whole number can be factored
uniquely into primes. What does that mean?

If we have a whole number, say, $70$, we see that we can write it as
$7\times10$. Furthermore, $10=2\times5$. So we can factor $70=7\times2\times5$.
Each of these numbers is now prime, and so we cannot factor this further, except
by introducing copies of $1$, which is pointless.

But what if we had factored it a different way, first writing $70=5\times14$?
Then we notice that $14=2\times7$, so we can change to $70=5\times2\times7$. The
important thing is that this is the same factorization as last time, except in a
different order. In fact, no matter, how we factor $70$, we will always arrive
at the same prime factorization (with possibly a different order).

\begin{problem}{Factorize}
 Find the prime factorization for each number.

 \begin{itemize}
  \item $12 = \Ans{2\times2\times3}$
  \item $70 = \Ans{2\times5\times7}$
  \item $100 = \Ans{2\times2\times5\times5}$
 \end{itemize}
\end{problem}

\begin{problem}{Prime Factors}
 Find one (no need to find all) prime factor for each number.

 \begin{itemize}
  \item $99959386$ \hfill Factor: \Ans{2}
  \item $521976505$ \hfill Factor: \Ans{5}
 \end{itemize}
\end{problem}

As you will see on the homework, it may be difficult to factor very large
numbers. In fact, lots of modern cryptography, including the popular RSA
encryption scheme, rest on the fact that nobody has yet discovered a fast way to
factor large numbers.

\chapter{Greatest Common Divisor}

\section{GCD \& LCM}
We will discuss two very important topics that have many applications: the
greatest common divisor, and another similar operation, the least common
multiple.

\subsection{Greatest Common Divisor}
Given two positive integers, we define their GCD (greatest common divisor) to be
the largest positive integer that's a factor for both those numbers. For
instance, the factors of $24$ are $1$, $2$, $3$, $4$, $6$, $8$, $12$, $24$, and
the factors of $36$ are $1$, $2$, $3$, $4$, $6$, $9$, $12$, $18$, $36$. The
greatest factor that these have in common is $12$. Therefore we denote \[
 \gcd(24, 36) = 12
\]

\begin{problem}{GCD Examples}
 \begin{itemize}
  \item $\gcd(10, 20) = \Ans{10}$
  \item $\gcd(15, 25) = \Ans{5}$
  \item $\gcd(7, 11) = \Ans{1}$
 \end{itemize}
\end{problem}

\subsection{Least Common Multiple}
Similarly, given two positive integers, we define their LCM (least common
multiple) to be the smallest positive integer that's a multiple of both those
numbers. For instance, the positive multiples of $24$ are $24$, $48$, $72$, and
so on, and the positive multiples of $36$ are $36$, $72$, $108$, and so on.
The smallest multiple that these have in common is $72$. Therefore we
denote \[
 \operatorname{lcm}(24, 36) = 72
\]

\section{Applications}
The GCD and the LCM have many applications across mathematics. One useful
application comes from studying fractions. Two fractions are equivalent if,
despite the numerator and the denominator possibly differing, they represent
the same quantity. For instance, the fraction $\frac{1}{2}$ is equivalent to
fractions $\frac{2}{4}$, $\frac{3}{6}$, and so on.

We may \emph{simplify} a fraction by writing it as an equivalent fraction with
least possible (positive) denominator. Simplifying fractions makes it easy to
determine whether two fractions are equivalent, and also makes fractions easier
to work with. There exists a simple algorithm to simplify a fraction. Given
$\frac{n}{m}$, compute $d=\gcd(n, m)$, and then $\frac{n\div d}{m\div d}$ is the
simplest form.

\begin{problem}{Simplify Fraction}
 Simplify. Note that the fraction may possibly already be in simplest form.

 \begin{itemize}
  \item $\frac{20}{15}=\Ans{\frac{4}{3}}$
  \item $\frac{24}{36}=\Ans{\frac{2}{3}}$
  \item $\frac{7}{13}=\Ans{\frac{7}{13}}$
 \end{itemize}
\end{problem}

When we add or subtract fractions, it helps often to write the fractions with
equal denominator. To select a compatible denominator, we may use the least
common multiple. For example, to add $\frac{a}{b}+\frac{c}{d}$, we may compute
$\ell=\operatorname{lcm}(b, d)$ and then compute \begin{align*}
 \frac{a}{b}+\frac{c}{d}
 &= \frac{a\ell\div b}{\ell} + \frac{c\ell\div d}{\ell} \\
 &= \frac{a\ell\div b + c\ell\div d}{\ell}
\end{align*} which then we may simplify with the algorithm described above.

\begin{problem}{Adding and Subtracting Fractions}
 Add or subtract. Then simplify.

 \begin{itemize}
  \item $\frac{1}{2} + \frac{1}{3} = \Ans{\frac{5}{6}}$
  \item $\frac{3}{4} + \frac{1}{6} = \Ans{\frac{11}{12}}$
  \item $\frac{23}{36} - \frac{11}{24} = \Ans{\frac{13}{36}}$
 \end{itemize}
\end{problem}

\section{Remainders}
We will take a brief break from discussing 

\section{Euclid's Algorithm}
In this unit we will

\end{document}
