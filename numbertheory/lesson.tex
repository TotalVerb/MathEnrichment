\documentclass[a4paper,10pt]{report}

\newcommand{\IncludePath}{../include}
\newcommand{\ProjectName}{Grade 4 Olympic Math}
\usepackage{extsizes}
\usepackage{titling}

\usepackage{tikz}
\usetikzlibrary{shapes}

\usepackage{amssymb,amsmath,amsthm}
\usepackage{enumerate}
\usepackage{graphicx,ctable,booktabs}
\usepackage{fancyhdr}
\usepackage[utf8]{inputenc}
\usepackage{gensymb}

\usepackage[toc]{glossaries}

\makeatletter
\newenvironment{problem}{\@startsection
       {subsection}
       {1}
       {-.2em}
       {-3.5ex plus -1ex minus -.2ex}
       {2.3ex plus .2ex}
       {\pagebreak[3]
       \large\bf\noindent{Problem }
       }
       }
\makeatother

\makeatletter
\g@addto@macro\@floatboxreset\centering
\makeatother

\newenvironment{solution}
{ \vspace{1em} \noindent \textbf{Solution:} }
{  }

\pagestyle{fancy}
\lhead{\thetitle}
\chead{}
\rhead{\thepage}
\lfoot{\small\scshape \ProjectName}
\cfoot{}
\rfoot{}
\renewcommand{\headrulewidth}{.3pt}
\renewcommand{\footrulewidth}{.3pt}
\setlength\voffset{-0.25in}
\setlength\textheight{648pt}
\setlength\headheight{15pt}

\newcommand{\Ans}[1]{\framebox{$#1$}}
\newcommand{\SAA}[1]{\Switch{\Ans{#1}}{\blankA}}
\newcommand{\SAB}[1]{\Switch{\Ans{#1}}{\blankB}}
\newcommand{\SAC}[1]{\Switch{\Ans{#1}}{\blankC}}
\newcommand{\SAD}[1]{\Switch{\Ans{#1}}{\blankD}}
\newcommand{\SAE}[1]{\Switch{\Ans{#1}}{\blankE}}
\newcommand{\SAF}[1]{\Switch{\Ans{#1}}{\blankF}}
\newcommand{\STA}[1]{\Switch{\AnsT{#1}}{\blankA}}
\newcommand{\STB}[1]{\Switch{\AnsT{#1}}{\blankB}}
\newcommand{\STC}[1]{\Switch{\AnsT{#1}}{\blankC}}
\newcommand{\STD}[1]{\Switch{\AnsT{#1}}{\blankD}}
\newcommand{\STE}[1]{\Switch{\AnsT{#1}}{\blankE}}
\newcommand{\STF}[1]{\Switch{\AnsT{#1}}{\blankF}}
\newcommand{\AnsT}[1]{\framebox{#1}}
\newif\ifanswers
\newcommand{\Switch}[2]{\ifanswers#1\else#2\fi}
\newcommand{\MCSelect}[1]{\Switch{\AnsT{#1}}{#1}}
\newcommand{\TFTrue}{\MCSelect{True}~~False}
\newcommand{\TFFalse}{True~~\MCSelect{False}}

\newcommand{\blankA}{\underline{\hspace{1em}}}
\newcommand{\blankB}{\underline{\hspace{2em}}}
\newcommand{\blankC}{\underline{\hspace{3em}}}
\newcommand{\blankD}{\underline{\hspace{4em}}}
\newcommand{\blankE}{\underline{\hspace{5em}}}
\newcommand{\blankF}{\underline{\hspace{6em}}}



\title{Number Theory}
\author{Fengyang Wang}

\begin{document}
\maketitle

\chapter{Introduction to Number Theory \& Parity}

Number Theory is the study of whole numbers. That means $0$, $1$, $2$, and so
on. Number Theory also studies negative whole numbers, such as $-1$, $-2$, and
so on. Positive and negative whole numbers, and $0$, are together called
``integers''.

\section{Parity}

``Parity'' is the term used for odd and even numbers. Today, we will study their
differences and several properties. We will approach the problem algebraically.

An even number is a number of the form $2n$, where $n$ is any integer. The
expression $2n$ means $2\times n$; we omit the $\times$ so that it is faster to
read and write.

An odd number is a number of the form $2n+1$, where $n$ is any integer. The
expression $2n+1$ means $(2\times n) + 1$. We perform the multiplication first
because of the order of operations.

\section{Determining Parity}
It is easy to determine whether a number is even or odd. Let's start with small
examples first. We know $4=2\times2$, so it's even. We know $7=2\times3+1$, so
it's odd. A number is either even or odd, but it's never both. We only need to
divide by $2$: if it evenly divides, then it's even, and otherwise, it's odd.

\begin{problem}{Even or Odd?}
 Are each of the following numbers even or odd?

 \begin{itemize}
  \item $10$ \hfill \AnsT{Even}~~~Odd
  \item $0$ \hfill \AnsT{Even}~~~Odd
  \item $19$ \hfill Even~~~\AnsT{Odd}
  \item $-1$ \hfill Even~~~\AnsT{Odd}
 \end{itemize}
\end{problem}

If the numbers are very big, it is convenient to simply observe the last digit.
If the last digit is even, then the whole number is even. If the last digit is
odd, then the whole number is odd. Why does this work? We will see very soon.

\section{Sums \& Differences}
What happens if we add two odd numbers? Two even numbers? Let's find out. Say
$2n+1$ and $2m+1$ are two odd numbers. We're using letters here as placeholders;
$n$ and $m$ can both take the value of any integer. Then \[
 (2n+1) + (2m+1) = 2n + 2m + 1 + 1 = 2n + 2m + 2 = 2(n+m+1)
\] but $n+m+1$ is an integer, and therefore the sum is even.

We can use a very similar strategy to find the sum of two even numbers. This is
even easier. Say $2n$ and $2m$ are two even numbers. Then \[
 2n + 2m = 2(n+m)
\] but $n+m$ is an integer, and therefore the sum is even.

Say $2n$ is an even number, and $2m+1$ is an odd number. Then \[
 2n + 2m + 1 = 2(n+m) + 1
\] which is an odd number. So the sum of an even number and an odd number is
odd. And because we can rearrange the order of addition, therefore the sum of an
odd number and a even number is also odd.

The rules for subtracting are exactly the same. The difference of two odds is
even. The difference of two evens is even. The difference of an odd and an even
is odd. Feel free to work these out yourself.

\begin{problem}{Even or Odd?}
 Are each of the following numbers even or odd?

 \begin{itemize}
  \item $100 + 200$ \hfill \AnsT{Even}~~~Odd
  \item $1273 + 19023$ \hfill \AnsT{Even}~~~Odd
  \item $19082 - 1911$ \hfill Even~~~\AnsT{Odd}
  \item $109291 + 8329 + 9107$ \hfill Even~~~\AnsT{Odd}
 \end{itemize}
\end{problem}

\section{Products}
We can use a similar technique to develop rules for the product of even and odd
numbers. It turns out that the product of two odd numbers is odd. Say $2n+1$
and $2m+1$ are two odd numbers. Observe: \begin{align*}
 (2n+1)(2m+1)
 &= 2n(2m+1) + (2m+1) \\
 &= 4nm+2n+2m+1 \\
 &= 2(2nm+n+m) + 1
\end{align*} which is odd.

But the product of an even number with any integer is even. Say $2n$ is an
even number, and $k$ is any integer. Then: \[
 (2n)k = 2(nk)
\] which is even.

\chapter{Divisibility}

We will begin today's topic with a few definitions.

The symbol $\mid$ means ``divides''. If $A$ and $B$ are two numbers, then $A$
divides $B$ when $B \div A$ has no remainder.

A number is a \emph{factor} of another number if it divides that number. For
example, $3$ is a \emph{factor} of $9$ because $3 \mid 9$. A number is a
\emph{multiple} of another number if the other number divides it. For example,
$9$ is a \emph{multiple} of $3$. Another word for factor, which we will see
later on, is \emph{divisor}.

Observe that $1$ is a \emph{factor} of every whole number, and $0$ is a
\emph{multiple} of every whole number.

\section{Parity \& Divisibility}

Using our new definitions, we see that the topic we covered last class---even
and odd numbers---is really just the study of numbers that are and aren't
divisible by $2$.

\section{Transitivity}

The \emph{transitivity} principle says that if $a$, $b$, and $c$ are integers,
and if $a \mid b$ and $b \mid c$, then $a \mid c$. Let's do an example. We know
that $3 \mid 12$, and that $12 \mid 36$. Then by transitivity, $3 \mid 36$. Here
is a simple exercise that is easily done using transitivity:

\begin{problem}{True or False}
 Does $3$ divide \emph{every} multiple of $12$? \hfill \AnsT{True}~~~False
\end{problem}

\chapter{Prime Numbers \& The Fundamental Theorem of Arithmetic}

Today's topic is prime numbers, and a very useful and important statement about
factorizing numbers.

\section{Primes}
A \emph{prime number} is a positive integer that has exactly two positive
factors: $1$ and itself. We can easily check whether small numbers are prime.
First, note $1$ is not prime, because it has just one factor. However, $2$ is
prime, because it has exactly two factors: $1$ and itself.

\begin{problem}{Prime}
 Is each number prime or not prime?

 \begin{itemize}
  \item $3$ \hfill \AnsT{Prime}~~~Not Prime
  \item $4$ \hfill Prime~~~\AnsT{Not Prime}
  \item $5$ \hfill \AnsT{Prime}~~~Not Prime
  \item $6$ \hfill Prime~~~\AnsT{Not Prime}
  \item $7$ \hfill \AnsT{Prime}~~~Not Prime
  \item $8$ \hfill Prime~~~\AnsT{Not Prime}
  \item $9$ \hfill Prime~~~\AnsT{Not Prime}
  \item $10$ \hfill Prime~~~\AnsT{Not Prime}
 \end{itemize}
\end{problem}

Don't get caught in the trap that every odd number is prime! $9=3\times3$ is
not. Neither are $15$, $21$, $25$, and many others. In fact, as the numbers get
bigger and bigger, prime numbers get rarer and rarer.

However, do note that there is only one even prime: $2$.

\chapter{Quiz, More Primes, \& Remainders}

\section{Quiz Review}

There will be a quiz today on parity and divisibility. Three review questions
follow. Quiz questions will be very similar in nature.

\begin{problem}{Factor Pairs}
 Complete the factor pairs.

 \begin{itemize}
  \item $6 = 1 \times \Ans{6}$
  \item $6 = 2 \times \Ans{3}$
  \item $6 = 3 \times \Ans{2}$
  \item $6 = 6 \times \Ans{1}$
 \end{itemize}
\end{problem}

\begin{problem}{Factors and Multiples}
 Use the word ``factor'' or ``multiple'' to complete each blank.

 \begin{itemize}
  \item $1$ is a \AnsT{factor} of every whole number.
  \item $0$ is a \AnsT{multiple} of every whole number.
  \item $3$ is a \AnsT{factor} of $9$.
  \item All even numbers are \AnsT{factor}s of $2$.
 \end{itemize}
\end{problem}

\begin{problem}{Divisibility}
 Circle factors of each number. More than one factor may be circled.

 \begin{itemize}
  \item $25$ \hfill $\Ans{1}$~~~$2$~~~$\Ans{5}$~~~$10$~~~$100$
  \item $172$ \hfill $\Ans{1}$~~~$\Ans{2}$~~~$5$~~~$10$~~~$100$
  \item $1793$ \hfill $\Ans{1}$~~~$2$~~~$5$~~~$10$~~~$100$
  \item $2000$ \hfill $\Ans{1}$~~~$\Ans{2}$~~~$\Ans{5}$~~~$\Ans{10}$
  ~~~$\Ans{100}$
 \end{itemize}
\end{problem}

\section{QUIZ}
There is a quiz today (Quiz 3: Parity and Divisibility). The allocated length
for the quiz is $15$ minutes.

\section{Divisors \& Primes}
We will talk some more about primes today, and also about divisors.

\begin{problem}{Prime Factors}
 Find one (no need to find all) prime factor for each number.

 \begin{itemize}
  \item $99959386$ \hfill Factor: \Ans{2}
  \item $521976505$ \hfill Factor: \Ans{5}
 \end{itemize}
\end{problem}

\subsection{Greatest Common Divisor}
Given two positive integers, we define their GCD (greatest common divisor) to be
the largest positive integer that's a factor for both those numbers. For
instance, the factors of $24$ are $1$, $2$, $3$, $4$, $6$, $8$, $12$, $24$, and
the factors of $36$ are $1$, $2$, $3$, $4$, $6$, $9$, $12$, $18$, $36$. The
greatest factor that these have in common is $12$. Therefore we denote \[
 \gcd(24, 36) = 12
\]

\begin{problem}{GCD Examples}
 \begin{itemize}
  \item $\gcd(10, 20) = \Ans{10}$
  \item $\gcd(15, 25) = \Ans{5}$
  \item $\gcd(7, 11) = \Ans{1}$
 \end{itemize}
\end{problem}

\section{Homework}
Today's homework contains an investigation about remainders, and some additional
questions about prime numbers and GCD.

\end{document}
