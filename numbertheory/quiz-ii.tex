\documentclass[12pt,letterpaper]{article}

\newcommand{\IncludePath}{../include}
\usepackage{extsizes}
\usepackage{titling}

\usepackage{amssymb,amsmath,amsthm}
\usepackage{enumerate}
\usepackage[margin=1in]{geometry}
\usepackage{graphicx,ctable,booktabs}
\usepackage{fancyhdr}
\usepackage[utf8]{inputenc}

\makeatletter
\newenvironment{problem}{\@startsection
       {section}
       {1}
       {-.2em}
       {-3.5ex plus -1ex minus -.2ex}
       {2.3ex plus .2ex}
       {\pagebreak[3]
       \large\bf\noindent{Problem }
       }
       }
\makeatother

\pagestyle{fancy}
\lhead{\thetitle}
\chead{}
\rhead{\thepage}
\lfoot{\small\scshape Grade 4 Olympic Math}
\cfoot{}
\rfoot{}
\renewcommand{\headrulewidth}{.3pt}
\renewcommand{\footrulewidth}{.3pt}
\setlength\voffset{-0.25in}
\setlength\textheight{648pt}
\setlength\headheight{15pt}


\title{Quiz: Number Theory}
\author{Name: \underline{\hspace{5cm}} Mark: $\displaystyle \frac{\hspace{3em}}{50}$}
\date{November 25, 2017}

\begin{document}

\maketitle

Please write your \emph{full} name. The quiz has a maximum mark of 50.

\thispagestyle{empty}

\begin{problem}{Sieve of Eratosthenes\hfill/25}
 Follow the given instructions to create a list of all prime numbers from \(1\) to \(25\).
 The algorithm described here is known as the Sieve of Eratosthenes, and it is one of the
 fastest known ways to generate all prime numbers in an interval.

 \begin{enumerate}
  \item Cross off $1$. It is not prime.
  \item Circle $2$, because it is a prime number. Cross off all bigger multiples
  of $2$ ($4$, $6$, $8$, \dots), because they are composite.
  \item Circle the smallest number that hasn't been crossed off. Since it isn't
  divisible by any smaller prime (otherwise, it would have been crossed off),
  it must be prime. Now cross off all multiples of the number you just circled.
  \item Repeat the last step until all numbers are crossed off or circled.
 \end{enumerate}

 \begin{center}
  \def\arraystretch{1.7}
  \begin{tabular}{|c|c|c|c|c|}
   \hline
   $1$ & $2$ & $3$ & $4$ & $5$ \\
   \hline
   $6$ & $7$ & $8$ & $9$ & $10$ \\
   \hline
   $11$ & $12$ & $13$ & $14$ & $15$ \\
   \hline
   $16$ & $17$ & $18$ & $19$ & $20$ \\
   \hline
   $21$ & $22$ & $23$ & $24$ & $25$ \\
   \hline
  \end{tabular}
 \end{center}

 Check your work! There should be $9$ prime numbers circled.
\end{problem}

\begin{problem}{Learning Summary\hfill/5}
  Write one sentence to describe one concept you learned this unit. You should write about
  one of these topics: prime numbers, the divisibility lattice, GCD, LCM, Euclidean
  algorithm.
\end{problem}

\begin{problem}{Prime Factorization\hfill/12}
 Find the unique prime factorization of each number.

 \begin{itemize}
  \item $14$ \hfill $\blankB \times \blankB$
  \item $27$ \hfill $\blankB \times \blankB \times \blankB$
  \item $28$ \hfill $\blankB \times \blankB \times \blankB$
  \item $90$ \hfill $\blankB \times \blankB \times \blankB \times \blankB$
 \end{itemize}
\end{problem}

\begin{problem}{Euclidean Algorithm\hfill/20}
  We can find \(\gcd(1234, 5678)\) using the Euclidean Algorithm. Read the steps and answer
  the following questions. As an example, the first question is answered for you. Each
  question is worth 4 marks.

  \begin{align}
    \label{subtract-three}
    \gcd(1234, 5678) &= \gcd(1234, 5678 - 1234 \times 3) \\
    &= \gcd(1234, 742) \\
    \label{switch-order}
    &= \gcd(742, 1234) \\
    \label{another-step}
    &= \gcd(742, 492) \\
    &= \gcd(492, 742) \\
    &= \gcd(250, 492) \\
    &= \gcd(242, 250) \\
    &= \gcd(8, 242) \\
    \label{last-simplify}
    &= \gcd(2, 8) \\
    &= 2
  \end{align}

  \begin{itemize}
    \item Explain (one sentence) why we can substract \(1234 \times 3\) from \(5678\) in
    step~\ref{subtract-three}.

    {\fontfamily{augie}\selectfont Because the GCD stays the same if we subtract one number
    from the other.}

    \item Explain (one sentence) why we can switch the order of \(742\) and \(1234\) in
    step~\ref{switch-order}.

    \vspace{1.5em}

    \item Explain (one sentence) what we did to \(1234\) to turn it into \(492\) in
    step~\ref{another-step}.

    \vspace{1.5em}

    \item Explain (one sentence) what we did to \(8\) and \(242\) to get to \(2\) and \(8\)
    in step~\ref{last-simplify}.

    \vspace{1.5em}

    \item Now use the fact that \(\gcd(a, b) \times \operatorname{lcm}(a, b) = a \times b\)
    to find \[ \operatorname{lcm}(1234, 5678) = \blankF. \] You might find it helpful that
    \(1234 \times 5678 = 7006652\).
 \end{itemize}
\end{problem}


\end{document}
