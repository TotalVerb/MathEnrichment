\documentclass[12pt,letterpaper]{article}

\newcommand{\IncludePath}{../include}
\usepackage{extsizes}
\usepackage{titling}

\usepackage{tikz}
\usetikzlibrary{shapes}

\usepackage{amssymb,amsmath,amsthm}
\usepackage{enumerate}
\usepackage[margin=0.8in]{geometry}
\usepackage{graphicx,ctable,booktabs}
\usepackage{fancyhdr}
\usepackage[utf8]{inputenc}
\usepackage{gensymb}

\makeatletter
\newenvironment{problem}{\@startsection
       {section}
       {1}
       {-.2em}
       {-3.5ex plus -1ex minus -.2ex}
       {2.3ex plus .2ex}
       {\pagebreak[3]
       \large\bf\noindent{Problem }
       }
       }
\makeatother

\pagestyle{fancy}
\lhead{\thetitle}
\chead{}
\rhead{\thepage}
\lfoot{\small\scshape Olympic Math}
\cfoot{}
\rfoot{}
\renewcommand{\headrulewidth}{.3pt}
\renewcommand{\footrulewidth}{.3pt}
\setlength\voffset{-0.25in}
\setlength\textheight{648pt}
\setlength\headheight{15pt}

\newcommand{\blankA}{\underline{\hspace{1em}}}
\newcommand{\blankB}{\underline{\hspace{2em}}}
\newcommand{\blankC}{\underline{\hspace{3em}}}
\newcommand{\blankD}{\underline{\hspace{4em}}}
\newcommand{\blankE}{\underline{\hspace{5em}}}
\newcommand{\blankF}{\underline{\hspace{6em}}}



\answerstrue % turn on for answers
\defauthor

\title{Divisibility}
\date{October 21, 2017}

\begin{document}

\maketitle

\thispagestyle{empty}

\begin{problem}{Factors of \(20\)}
 List all the positive factors of \(20\):
 \Switch{\Ans{1}}{\blankB}, \Switch{\Ans{2}}{\blankB},
 \Switch{\Ans{4}}{\blankB}, \Switch{\Ans{5}}{\blankB},
 \Switch{\Ans{10}}{\blankB}, \Switch{\Ans{20}}{\blankB}
\end{problem}

\begin{problem}{Perfect Number}
  A number is \emph{perfect} if the sum of its positive factors is equal to twice the number
  itself. For example, \(6\) is perfect because \(1+2+3+6=12=6\times2\). Which of the
  following is a perfect number: \(10\), \(17\), \Switch{\Ans{28}}{\(28\)}?
\end{problem}

\begin{problem}{Multiples of \(3\) and \(6\)}
 \begin{itemize}
  \item Fill in the blanks in this list of multiples of \(3\): \(0\), \(3\),
  \(6\), \Switch{\Ans{9}, \Ans{12}, \Ans{15}}{\blankB, \blankB, \blankB},
  \(18\), \ldots

  \item Fill in the blanks in this list of multiples of \(6\): \(0\), \(6\),
  \(12\), \Switch{\Ans{18}, \Ans{24}, \Ans{30}}{\blankB, \blankB, \blankB},
  \(36\), \ldots
 \end{itemize}
\end{problem}

\begin{problem}{Rectangle II}
 Look at the diagram below. Each small square is one square unit.

 \begin{center}
 \begin{tikzpicture}
  \draw[step=0.5] (0,0) grid (4,3);
 \end{tikzpicture}
 \end{center}

 \begin{enumerate}
  \item The rectangle has area \Switch{\Ans{48}}{\blankD} square units.
  \item \(6\) is a \Switch{\AnsT{factor}}{\underline{\hspace{10em}}} of the
  rectangle's area.
  \item The rectangle's area is a
  \Switch{\AnsT{multiple}}{\underline{\hspace{10em}}} of \(8\).
 \end{enumerate}
\end{problem}

\begin{problem}{Linear Diophantine Equations in One Variable}
  \begin{enumerate}
    \item Find an integer \(n\) such that \(2n = 4\).
    \hfill \(n = \Switch{\Ans{2}}{\blankC}\)
    \item Find an integer \(n\) such that \(3n = -9\).
    \hfill \(n = \Switch{\Ans{-3}}{\blankC}\)
    \item Find an integer \(n\) such that \(5n = 45\).
    \hfill \(n = \Switch{\Ans{9}}{\blankC}\)
    \item Is there an integer \(n\) such that \(5n = 48\)? Explain why or why
    not. \Switch{\AnsT{No, since \(5 \nmid 48\).}}{}
  \end{enumerate}
\end{problem}

\begin{problem}{True or False}
 \begin{enumerate}
  \item \(6 \mid 2\)      \hfill \TFFalse
  \item \(4 \mid 40\)     \hfill \TFTrue
  \item \(3 \mid 99\)     \hfill \TFTrue
  \item \(100 \mid 7801\) \hfill \TFFalse
  \item Let \(a\), \(b\), and \(c\) be three integers. If \(a \mid b\), then
  \(a \mid bc\).          \hfill \TFTrue
  \item Let \(a\), \(b\), and \(c\) be three integers. If \(a \mid b\), and \(b
  \mid c\), and \(c \mid 10000\), then \(a \mid 10000\).
                          \hfill \TFTrue
 \end{enumerate}
\end{problem}

\begin{problem}{Challenge I}
 Define a number to be ``cool'' if it is divisible by all of \(2\), \(3\),
 \(4\), and \(6\). For example, \(60\) is a cool number but \(100\) is not.
 Between \(1\) and \(1000\) (including both \(1\) and \(1000\)), how many
 numbers are cool? \Switch{\Ans{83}}{}
\end{problem}

\begin{problem}{Challenge II}
 \begin{enumerate}
  \item How many positive factors does \(1\) have? \Switch{\Ans{1}}{}
  \item How many positive factors does \(2\) have? \Switch{\Ans{2}}{}
  \item How many positive factors does \(4\) have? \Switch{\Ans{3}}{}
  \item How many positive factors does \(8\) have? \Switch{\Ans{4}}{}
  \item The notation \(2^{10}\) means \(\underbrace{ 2\times2\times2\times
  \ldots\times2}_{10\text{ twos}}\); that is, \(10\) twos multiplied together.
  \(2^{10}=1024\). How many positive factors does \(1024\) have?
  \Switch{\Ans{11}}{}
 \end{enumerate}
\end{problem}

\end{document}
