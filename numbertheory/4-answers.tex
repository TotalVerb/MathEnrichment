\documentclass[12pt,letterpaper]{article}

\newcommand{\IncludePath}{../include}
\usepackage{extsizes}
\usepackage{titling}

\usepackage{amssymb,amsmath,amsthm}
\usepackage{enumerate}
\usepackage[margin=1in]{geometry}
\usepackage{graphicx,ctable,booktabs}
\usepackage{fancyhdr}
\usepackage[utf8]{inputenc}

\makeatletter
\newenvironment{problem}{\@startsection
       {section}
       {1}
       {-.2em}
       {-3.5ex plus -1ex minus -.2ex}
       {2.3ex plus .2ex}
       {\pagebreak[3]
       \large\bf\noindent{Problem }
       }
       }
\makeatother

\pagestyle{fancy}
\lhead{\thetitle}
\chead{}
\rhead{\thepage}
\lfoot{\small\scshape Grade 4 Olympic Math}
\cfoot{}
\rfoot{}
\renewcommand{\headrulewidth}{.3pt}
\renewcommand{\footrulewidth}{.3pt}
\setlength\voffset{-0.25in}
\setlength\textheight{648pt}
\setlength\headheight{15pt}


\title{Primes \& Remainders}
\author{Answers}
\date{April 2, 2016}

\begin{document}

\maketitle

\thispagestyle{empty}

The first page is about remainders. The back page is a review of prime numbers.

\begin{problem}{Fill in the Blanks I}
 When we divide $20$ by $4$, we count how many groups of $4$ are needed to make
 $20$. We know that $5 \times 4 = 20$. This means that it takes \Ans{5} groups
 of $4$ to make $20$. So $20 \div 4 = \Ans{5}$.

 When we divide $26$ by $13$, we count how many groups of \Ans{13} are needed to
 make \Ans{26}. Therefore, $26 \div 13 = \Ans{2}$.
\end{problem}

\begin{problem}{Fill in the Blanks II}
 Sometimes the numbers might not work out perfectly. Let's try dividing $20$ by
 $3$. If we make $7$ groups of $3$, the total number would be $7 \times 3 = 21$.
 But this is too big. If we make $6$ groups of $3$, the total number would be
 $\Ans{6} \times 3 = 18$. But this is too small. So we can make six groups, but
 we'd then have $20 - 18 = \Ans{2}$ left over!

 We say that when we divide $20$ by $3$, our \emph{quotient} is $6$ because we
 can make \Ans{6} groups of $3$ in total. And then we say that our
 \emph{remainder} is $2$ because after making $6$ groups of $3$, we have
 \Ans{2} left over. We can also write this as $20 \div 3 = 6 \operatorname{R}
 2$. The ``R'' stands for ``remainder''.

 When we can't make any groups, the quotient is zero ($0$). So $4 \div 10 =
 \Ans{0} \operatorname{R} \Ans{4}$. When there isn't anything left over, the
 remainder is zero ($0$). So $22 \div 2 = \Ans{11} \operatorname{R} \Ans{0}$.

 Complete the following: \begin{itemize}
  \item $17 \div 5 = \Ans{3} \operatorname{R} \Ans{2}$.
  \item $16 \div 5 = \Ans{3} \operatorname{R} \Ans{1}$.
 \end{itemize}
\end{problem}

\begin{problem}{Fill in the Blanks III}
 Suppose we didn't care what the quotient was; only the remainder is important.
 Then we can write that as $16 \bmod 5 = 1$. We don't care how many groups we
 made, but we do care that one was left over after making those groups. Complete
 the following:

 \begin{itemize}
  \item $75 \bmod 7 = \Ans{5}$.
  \item $24 \bmod 3 = \Ans{0}$.
 \end{itemize}
\end{problem}

\pagebreak

\begin{problem}{Factorize}
 Find the unique prime factorization of each number.

 \begin{itemize}
  \item $20$ \hfill $\Ans{2} \times \Ans{2} \times \Ans{5}$
  \item $4$ \hfill $\Ans{2} \times \Ans{2}$
  \item $81$ \hfill $\Ans{3} \times \Ans{3} \times \Ans{3} \times \Ans{3}$
  \item $84$ \hfill $\Ans{2} \times \Ans{2} \times \Ans{3} \times \Ans{7}$
 \end{itemize}
\end{problem}

\begin{problem}{Factor sum}
 The sum of all factors of $28$ is:

 $\Ans{1} +
 \Ans{2} +
 \Ans{4} +
 \Ans{7} +
 \Ans{14} +
 \Ans{28} = \Ans{56}$
\end{problem}

\begin{problem}{GCD of a prime}
 What is the greatest common divisor of $7$ and $10$? \hfill
 $\gcd(7, 10) = \Ans{1}$
\end{problem}

\begin{problem}{Challenge}
 A \emph{palindrome} is a number that reads the same forwards and backwards in
 decimal notation. For example, $12321$ is a palindrome, as is $7227$ or
 $888888$.

 \begin{itemize}
  \item What five-digit palindrome starts with $728$? \Ans{72827}
  \item $11$ and $55$ are two examples of two-digit palindromes.
  How many two-digit palindromes are there? ($00=0$ does not have two digits.)
  \Ans{9}
  \item What two numbers are factors of all two-digit palindromes? \Ans{1, 11}
  \item $1221$ and $9999$ are two examples of four-digit palindromes. How many
  four-digit palindromes are there? (Do not include palindromes starting with
  $0$.) \Ans{90}
  \item How many four-digit palindromes are even? \Ans{45}
  \item How many four-digit palindromes are divisible by $11$? \Ans{90}
 \end{itemize}
\end{problem}


\end{document}
