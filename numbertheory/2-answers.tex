\documentclass[12pt,letterpaper]{article}

\newcommand{\IncludePath}{../include}
\usepackage{extsizes}
\usepackage{titling}

\usepackage{tikz}
\usetikzlibrary{shapes}

\usepackage{amssymb,amsmath,amsthm}
\usepackage{enumerate}
\usepackage[margin=0.8in]{geometry}
\usepackage{graphicx,ctable,booktabs}
\usepackage{fancyhdr}
\usepackage[utf8]{inputenc}
\usepackage{gensymb}

\makeatletter
\newenvironment{problem}{\@startsection
       {section}
       {1}
       {-.2em}
       {-3.5ex plus -1ex minus -.2ex}
       {2.3ex plus .2ex}
       {\pagebreak[3]
       \large\bf\noindent{Problem }
       }
       }
\makeatother

\pagestyle{fancy}
\lhead{\thetitle}
\chead{}
\rhead{\thepage}
\lfoot{\small\scshape Olympic Math}
\cfoot{}
\rfoot{}
\renewcommand{\headrulewidth}{.3pt}
\renewcommand{\footrulewidth}{.3pt}
\setlength\voffset{-0.25in}
\setlength\textheight{648pt}
\setlength\headheight{15pt}

\newcommand{\blankA}{\underline{\hspace{1em}}}
\newcommand{\blankB}{\underline{\hspace{2em}}}
\newcommand{\blankC}{\underline{\hspace{3em}}}
\newcommand{\blankD}{\underline{\hspace{4em}}}
\newcommand{\blankE}{\underline{\hspace{5em}}}
\newcommand{\blankF}{\underline{\hspace{6em}}}



\title{Divisibility}
\author{Answers}
\date{March 5, 2016}

\begin{document}

\maketitle

\thispagestyle{empty}

\begin{problem}{Divisibility}
 Are all of the following divisibility statements correct or incorrect? If
 incorrect, change the divisibility statement so that it becomes correct.

 \begin{enumerate}
  \item $6 \mid 2$  \AnsT{Incorrect, $2 \mid 6$}
  \item $4 \mid 40$ \AnsT{Correct}
  \item $3 \mid 99$ \AnsT{Correct}
  \item $100 \mid 7801$ \AnsT{Incorrect, $100 \mid 7801$}
 \end{enumerate}
\end{problem}

\begin{problem}{Multiples of $3$ and $6$}

 \begin{itemize}
  \item Complete this list of multiples of $3$: $0$, $3$, $6$, \Ans{9},
  \Ans{12}, \Ans{15}, $18$, \ldots

  \item Complete this list of multiples of $6$: $0$, $6$, $12$, \Ans{18},
  \Ans{24}, \Ans{30}, $36$, \ldots
 \end{itemize}
\end{problem}

\begin{problem}{Transititity}
 Let $a$, $b$, and $c$ be three integers. Use transitivity to verify the
 following statements:
 \begin{itemize}
  \item If $a \mid b$, then $a \mid bc$.
  \Ans{a \mid b \land b \mid bc \implies a \mid bc}
  \item If $a \mid b$, and $b \mid c$, and $c \mid 10000$, then $a \mid 10000$.
  \Ans{a \mid b \land b \mid c \land c \mid 10000 \implies a \mid 10000}
 \end{itemize}
\end{problem}

\begin{problem}{Rectangle II}
 Look at the diagram below. Each small square is one square unit.

 \begin{center}
 \begin{tikzpicture}
  \draw[step=0.4] (0,0) grid (4,2.8);
 \end{tikzpicture}
 \end{center}

 \begin{enumerate}
  \item The rectangle has area \Ans{70} square units.
  \item $7$ is a \AnsT{factor} of the rectangle's area.
  \item The rectangle's area is a \AnsT{multiple} of $10$.
 \end{enumerate}
\end{problem}

\begin{problem}{Factors of $20$}
 List all the factors of $20$: \Ans{1, 2, 4, 5, 10, 20}
\end{problem}

\begin{problem}{Perfect Number}
 A number is \emph{perfect} if the sum of its factors is equal to twice the
 number itself. For example, $6$ is perfect because $1+2+3+6=12=6\times2$. Which
 of the following is a perfect number: $10$, $17$, \Ans{28}?
\end{problem}

\begin{problem}{Challenge I}
 Define a number to be ``cool'' if it is divisible by all of $2$, $3$, $4$, and
 $6$. For example, $60$ is a cool number but $100$ is not. Between $1$ and
 $1000$ (including both $1$ and $1000$), how many numbers are cool? \Ans{83}
\end{problem}

\begin{problem}{Challenge II}
 \begin{enumerate}
  \item How many factors does $1$ have? \Ans{1}
  \item How many factors does $2$ have? \Ans{2}
  \item How many factors does $4$ have? \Ans{3}
  \item How many factors does $8$ have? \Ans{4}
  \item The notation $2^{10}$ means $\underbrace{ 2\times2\times2\times
  \ldots\times2}_\text{$10$ twos}$; that is, $10$ twos multiplied together.
  $2^{10}=1024$. How many factors does $1024$ have? \Ans{11}
 \end{enumerate}
\end{problem}

\end{document}
