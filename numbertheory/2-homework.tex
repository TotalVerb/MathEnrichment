\documentclass[12pt,letterpaper]{article}

\newcommand{\IncludePath}{../include}
\usepackage{extsizes}
\usepackage{titling}

\usepackage{tikz}
\usetikzlibrary{shapes}

\usepackage{amssymb,amsmath,amsthm}
\usepackage{enumerate}
\usepackage[margin=0.8in]{geometry}
\usepackage{graphicx,ctable,booktabs}
\usepackage{fancyhdr}
\usepackage[utf8]{inputenc}
\usepackage{gensymb}

\makeatletter
\newenvironment{problem}{\@startsection
       {section}
       {1}
       {-.2em}
       {-3.5ex plus -1ex minus -.2ex}
       {2.3ex plus .2ex}
       {\pagebreak[3]
       \large\bf\noindent{Problem }
       }
       }
\makeatother

\pagestyle{fancy}
\lhead{\thetitle}
\chead{}
\rhead{\thepage}
\lfoot{\small\scshape Olympic Math}
\cfoot{}
\rfoot{}
\renewcommand{\headrulewidth}{.3pt}
\renewcommand{\footrulewidth}{.3pt}
\setlength\voffset{-0.25in}
\setlength\textheight{648pt}
\setlength\headheight{15pt}

\newcommand{\blankA}{\underline{\hspace{1em}}}
\newcommand{\blankB}{\underline{\hspace{2em}}}
\newcommand{\blankC}{\underline{\hspace{3em}}}
\newcommand{\blankD}{\underline{\hspace{4em}}}
\newcommand{\blankE}{\underline{\hspace{5em}}}
\newcommand{\blankF}{\underline{\hspace{6em}}}



\title{Divisibility}
\author{Name: \underline{\hspace{5cm}}}
\date{March 5, 2016}

\begin{document}

\maketitle

\thispagestyle{empty}

The symbol $\mid$ means ``divides''. If $A$ and $B$ are two numbers, then $A$
divides $B$ when $B \div A$ has no remainder.

A number is a \emph{factor} of another number if it divides that number. For
example, $3$ is a \emph{factor} of $9$ because $3 \mid 9$. A number is a
\emph{multiple} of another number if the other number divides it. For example,
$9$ is a \emph{multiple} of $3$.

$1$ is a \emph{factor} of every whole number, and $0$ is a \emph{multiple} of
every whole number. The \emph{transitivity} principle says that if $a$, $b$, and
$c$ are integers, and if $a \mid b$ and $b \mid c$, then $a \mid c$.

\begin{problem}{Divisibility}
 Are all of the following divisibility statements correct or incorrect? If
 incorrect, change the divisibility statement so that it becomes correct.

 \begin{enumerate}
  \item $6 \mid 2$
  \item $4 \mid 40$
  \item $3 \mid 99$
  \item $100 \mid 7801$
 \end{enumerate}

\end{problem}

\begin{problem}{Multiples of $3$ and $6$}

 \begin{itemize}
  \item Complete this list of multiples of $3$: $0$, $3$, $6$,
  \underline{\hspace{2em}}, \underline{\hspace{2em}}, \underline{\hspace{2em}},
  $18$, \ldots

 \item Complete this list of multiples of $6$: $0$, $6$, $12$,
 \underline{\hspace{2em}}, \underline{\hspace{2em}}, \underline{\hspace{2em}},
 $36$, \ldots
 \end{itemize}
\end{problem}

\begin{problem}{Transititity}
 Let $a$, $b$, and $c$ be three integers. Use transitivity to verify the
 following statements:
 \begin{itemize}
  \item If $a \mid b$, then $a \mid bc$.
  \item If $a \mid b$, and $b \mid c$, and $c \mid 10000$, then $a \mid 10000$.
 \end{itemize}
\end{problem}

\begin{problem}{Rectangle II}
 Look at the diagram below. Each small square is one square unit.

 \begin{center}
 \begin{tikzpicture}
  \draw[step=0.5] (0,0) grid (5,3.5);
 \end{tikzpicture}
 \end{center}

 \begin{enumerate}
  \item The rectangle has area \underline{\hspace{4em}} square units.
  \item $7$ is a \underline{\hspace{10em}} of the rectangle's area.
  \item The rectangle's area is a \underline{\hspace{10em}} of $10$.
 \end{enumerate}
\end{problem}

\begin{problem}{Factors of $20$}
 List all the factors of $20$:
 \underline{\hspace{2em}}, \underline{\hspace{2em}},
 \underline{\hspace{2em}}, \underline{\hspace{2em}},
 \underline{\hspace{2em}}, \underline{\hspace{2em}}
\end{problem}

\begin{problem}{Perfect Number}
 A number is \emph{perfect} if the sum of its factors is equal to twice the
 number itself. For example, $6$ is perfect because $1+2+3+6=12=6\times2$. Which
 of the following is a perfect number: $10$, $17$, $28$?
\end{problem}

\begin{problem}{Challenge I}
 A ``cool'' number is a number that is divisible by all of $2$, $3$, $4$, and
 $6$. For example, $60$ is a cool number but $100$ is not (since it isn't
 divisble by $3$). Between $1$ and $1000$ (and including both $1$ and $1000$),
 how many numbers are cool?
\end{problem}

\begin{problem}{Challenge II}
 \begin{enumerate}
  \item How many factors does $1$ have?
  \item How many factors does $2$ have?
  \item How many factors does $4$ have?
  \item How many factors does $8$ have?
  \item The notation $2^{10}$ means $\underbrace{ 2\times2\times2\times
  \ldots\times2}_\text{$10$ twos}$; that is, $10$ twos multiplied together.
  $2^{10}=1024$. How many factors does $1024$ have?
 \end{enumerate}

\end{problem}



\end{document}
