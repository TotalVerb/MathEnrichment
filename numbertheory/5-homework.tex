\documentclass[12pt,letterpaper]{article}

\newcommand{\IncludePath}{../include}
\usepackage{extsizes}
\usepackage{titling}

\usepackage{amssymb,amsmath,amsthm}
\usepackage{enumerate}
\usepackage[margin=1in]{geometry}
\usepackage{graphicx,ctable,booktabs}
\usepackage{fancyhdr}
\usepackage[utf8]{inputenc}

\makeatletter
\newenvironment{problem}{\@startsection
       {section}
       {1}
       {-.2em}
       {-3.5ex plus -1ex minus -.2ex}
       {2.3ex plus .2ex}
       {\pagebreak[3]
       \large\bf\noindent{Problem }
       }
       }
\makeatother

\pagestyle{fancy}
\lhead{\thetitle}
\chead{}
\rhead{\thepage}
\lfoot{\small\scshape Grade 4 Olympic Math}
\cfoot{}
\rfoot{}
\renewcommand{\headrulewidth}{.3pt}
\renewcommand{\footrulewidth}{.3pt}
\setlength\voffset{-0.25in}
\setlength\textheight{648pt}
\setlength\headheight{15pt}


\title{Euclidean Algorithm}
\author{Name: \underline{\hspace{5cm}}}
\date{April 9, 2016}

\begin{document}

\maketitle

\thispagestyle{empty}

\begin{problem}{Find the GCD I}
 Find the GCD of each pair of numbers.

 \begin{itemize}
  \item $\gcd(1, 7) = \blankC$
  \item $\gcd(2, 4) = \blankC$
  \item $\gcd(8, 12) = \blankC$
 \end{itemize}
\end{problem}

\begin{problem}{Find the GCD II}
 Find the GCD of each pair of numbers.

 \begin{itemize}
  \item $\gcd(0, 12) = \blankC$
  \item $\gcd(60, 30) = \blankC$
  \item $\gcd(1000, 100000) = \blankE$
 \end{itemize}
\end{problem}

\begin{problem}{Find the GCD III}
 Find the GCD of each triplet of numbers.

 \begin{itemize}
  \item $\gcd(9, 21, 15) = \blankC$
  \item $\gcd(30, 40, 50) = \blankC$
 \end{itemize}
\end{problem}

\begin{problem}{Find the GCD IV}
 Use the Euclidean algorithm to find the GCD.

 \begin{itemize}
  \item $\gcd(1270, 1940) = \blankC$
  \item $\gcd(2447, 1480) = \blankC$
 \end{itemize}
\end{problem}

\begin{problem}{Challenge I}
 There is a very interesting property of the GCD called B\'ezout's Lemma. It
 says that if $a$ and $b$ are integers, and $\gcd(a, b) = d$, then there are
 some other integers $x$ and $y$ so that $ax + by = d$. Here are two examples.

 \begin{enumerate}
  \item Compute $\gcd(18, 16) = \blankC$
  \item Fill in the blanks with integers so that the following equation is true:
  \[ 18 \times \blankC + 16 \times (- \blankC) = 2 \]
  (Recall that a positive number times a negative number is negative.)
  \item Compute $\gcd(20, 28) = \blankC$
  \item Fill in the blanks with integers so that the following equation is true:
  \[ 20 \times \blankC + 28 \times (- \blankC) = 4 \]
 \end{enumerate}
\end{problem}

\begin{problem}{Challenge II}

 We say that two numbers $a$ and $b$ are relatively prime (or co-prime for
 short) if $\gcd(a, b) = 1$. That is, if two numbers have no common factors
 except $1$, then they are co-prime.

 \begin{enumerate}
  \item Are $3$ and $6$ co-prime?  \hfill Yes~~No
  \item Are $8$ and $13$ co-prime? \hfill Yes~~No
  \item Are $9$ and $16$ co-prime? \hfill Yes~~No
  \item Fill in the blanks with integers so that the following equation is true:
  \[ 9 \times \blankC + 16 \times (-\blankC) = 1 \]
  \item Fill in the blanks with integers so that the following equation is true:
  \[ 22331115 \times 9 \times \blankC + 22331115 \times 16 \times (-\blankC) =
  22331115 \]
  \item Let's say that $9 \mid 16\times22331115$. Using the above equation,
  does $9\mid22331115$?
 \end{enumerate}
\end{problem}

\begin{problem}{Challenge III}
 A \emph{semiprime} number is a number which is the product of exactly two prime
 numbers. For example, $9=3\times3$ is semiprime, as is $38=2\times19$. Say
 $9409$ and $8633$ are two semiprime numbers that share a prime factor. Find
 their shared prime factor.
\end{problem}



\end{document}
