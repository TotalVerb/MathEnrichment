\documentclass[12pt,letterpaper]{article}

\newcommand{\IncludePath}{../include}
\usepackage{extsizes}
\usepackage{titling}

\usepackage{amssymb,amsmath,amsthm}
\usepackage{enumerate}
\usepackage[margin=1in]{geometry}
\usepackage{graphicx,ctable,booktabs}
\usepackage{fancyhdr}
\usepackage[utf8]{inputenc}

\makeatletter
\newenvironment{problem}{\@startsection
       {section}
       {1}
       {-.2em}
       {-3.5ex plus -1ex minus -.2ex}
       {2.3ex plus .2ex}
       {\pagebreak[3]
       \large\bf\noindent{Problem }
       }
       }
\makeatother

\pagestyle{fancy}
\lhead{\thetitle}
\chead{}
\rhead{\thepage}
\lfoot{\small\scshape Grade 4 Olympic Math}
\cfoot{}
\rfoot{}
\renewcommand{\headrulewidth}{.3pt}
\renewcommand{\footrulewidth}{.3pt}
\setlength\voffset{-0.25in}
\setlength\textheight{648pt}
\setlength\headheight{15pt}


\answerstrue % turn on for answers
\defauthor

\title{The Greatest Common Divisor}
\date{November 4, 2017}

\begin{document}

\maketitle

\thispagestyle{empty}

Here is a list of all prime numbers less than $100$: $2$, $3$, $5$, $7$, $11$,
$13$, $17$, $19$, $23$, $29$, $31$, $37$, $41$, $43$, $47$, $53$, $59$, $61$,
$67$, $71$, $73$, $79$, $83$, $89$ and $97$.

\begin{problem}{Factorize}
 Find the unique prime factorization of each number.

 \begin{itemize}
  \item $20$ \hfill $\SAB{2} \times \SAB{2} \times \SAB{5}$
  \item $4$ \hfill $\SAB{2} \times \SAB{2}$
  \item $81$ \hfill $\SAB{3} \times \SAB{3} \times \SAB{3} \times \SAB{3}$
  \item $84$ \hfill $\SAB{2} \times \SAB{2} \times \SAB{3} \times \SAB{7}$
 \end{itemize}
\end{problem}

\begin{problem}{Find the GCD I}
 Find the GCD of each pair of numbers.

 \begin{itemize}
  \item $\gcd(1, 7) = \SAB{1}$
  \item $\gcd(2, 4) = \SAB{2}$
  \item $\gcd(8, 12) = \SAB{4}$
  \item $\gcd(0, 12) = \SAB{12}$
  \item $\gcd(60, 30) = \SAB{30}$
  \item $\gcd(1000, 100000) = \SAD{1000}$
 \end{itemize}
\end{problem}

\begin{problem}{Find the GCD II}
 Find the GCD of each triplet of numbers.

 \begin{itemize}
  \item $\gcd(9, 21, 15) = \SAB{3}$
  \item $\gcd(30, 40, 50) = \SAB{10}$
 \end{itemize}
\end{problem}

\begin{problem}{Find the GCD IIII}
 Use the Euclidean algorithm to find the GCD.

 \begin{itemize}
  \item $\gcd(1270, 1940) = \SAB{10}$
  \item $\gcd(2447, 1480) = \SAB{1}$
 \end{itemize}
\end{problem}

\begin{problem}{Challenge I}
 A \emph{palindrome} is a number that reads the same forwards and backwards in
 decimal notation. For example, $12321$ is a palindrome, as is $7227$ or
 $888888$.

 \begin{itemize}
  \item What five-digit palindrome starts with $728$? \SAE{72827}
  \item $11$ and $55$ are two examples of two-digit palindromes.
  How many two-digit palindromes are there? ($00=0$ does not have two digits.)
  \Switch{\Ans{9}}{}
  \item What two numbers are factors of all two-digit palindromes?
  \Switch{\Ans{1, 11}}{}
  \item $1221$ and $9999$ are two examples of four-digit palindromes. How many
  four-digit palindromes are there? (Do not include palindromes starting with
  $0$.) \Switch{\Ans{90}}{}
  \item How many four-digit palindromes are even? \Switch{\Ans{45}}{}
  \item How many four-digit palindromes are divisible by $11$?
  \Switch{\Ans{90}}{}
 \end{itemize}
\end{problem}

\begin{problem}{Challenge II}
 A \emph{semiprime} number is a number which is the product of exactly two prime
 numbers. For example, $9=3\times3$ is semiprime, as is $38=2\times19$. Say
 $9409$ and $8633$ are two semiprime numbers that share a prime factor. Find
 their shared prime factor. \Switch{\Ans{97}}{}
\end{problem}

\begin{problem}{True or False}
  \begin{itemize}
    \item For any integers \(a\) and \(b\), \(\gcd(a, b)
    \operatorname{lcm}(a, b) = ab\). \hfill \TFTrue
    \item For any integers \(a\) and \(b\), \(\gcd(ab, 10) = \gcd(a, 10)
    \gcd(b, 10)\). \hfill \TFFalse
    \item If \(p\) is prime and \(q\) is an integer, and \(p \nmid q\), then
    \(\gcd(p, q) = 1\). \hfill \TFTrue
  \end{itemize}
\end{problem}

\end{document}
