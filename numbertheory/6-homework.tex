\documentclass[12pt,letterpaper]{article}

\newcommand{\IncludePath}{../include}
\usepackage{extsizes}
\usepackage{titling}

\usepackage{amssymb,amsmath,amsthm}
\usepackage{enumerate}
\usepackage[margin=1in]{geometry}
\usepackage{graphicx,ctable,booktabs}
\usepackage{fancyhdr}
\usepackage[utf8]{inputenc}

\makeatletter
\newenvironment{problem}{\@startsection
       {section}
       {1}
       {-.2em}
       {-3.5ex plus -1ex minus -.2ex}
       {2.3ex plus .2ex}
       {\pagebreak[3]
       \large\bf\noindent{Problem }
       }
       }
\makeatother

\pagestyle{fancy}
\lhead{\thetitle}
\chead{}
\rhead{\thepage}
\lfoot{\small\scshape Grade 4 Olympic Math}
\cfoot{}
\rfoot{}
\renewcommand{\headrulewidth}{.3pt}
\renewcommand{\footrulewidth}{.3pt}
\setlength\voffset{-0.25in}
\setlength\textheight{648pt}
\setlength\headheight{15pt}


\answerstrue % turn on for answers
\defauthor

\title{Euclidean Algorithm}
\date{November 11, 2017}

\begin{document}

\maketitle

\thispagestyle{empty}

Here is a list of all prime numbers less than $100$: $2$, $3$, $5$, $7$, $11$,
$13$, $17$, $19$, $23$, $29$, $31$, $37$, $41$, $43$, $47$, $53$, $59$, $61$,
$67$, $71$, $73$, $79$, $83$, $89$ and $97$.

\begin{problem}{Factor sum}
 The sum of all positive factors of $28$ is
 $\SAB{1} +
 \SAB{2} +
 \SAB{4} +
 \SAB{7} +
 \SAC{14} +
 \SAC{28} = \SAC{56}$
\end{problem}

\begin{problem}{Challenge I}
 There is a very interesting property of the GCD called B\'ezout's Lemma. It
 says that if $a$ and $b$ are integers, and $\gcd(a, b) = d$, then there are
 some other integers $x$ and $y$ so that $ax + by = d$. Here are two examples.

 \begin{enumerate}
  \item Compute $\gcd(18, 16) = \SAB{2}$
  \item Fill in the blanks with integers so that the following equation is true:
  \[ 18 \times \SAB{1} + 16 \times (- \SAB{1}) = 2 \]
  (Recall that a positive number times a negative number is negative.)
  \item Compute $\gcd(20, 28) = \SAB{4}$
  \item Fill in the blanks with integers so that the following equation is true:
  \[ 20 \times \SAB{3} + 28 \times (- \SAB{2}) = 4 \]
 \end{enumerate}
\end{problem}

\begin{problem}{Challenge II}

 We say that two numbers $a$ and $b$ are relatively prime (or co-prime for
 short) if $\gcd(a, b) = 1$. That is, if two numbers have no common factors
 except $1$, then they are co-prime.

 \begin{enumerate}
  \item Are $3$ and $6$ co-prime?  \hfill Yes~~\MCSelect{No}
  \item Are $8$ and $13$ co-prime? \hfill \MCSelect{Yes}~~No
  \item Are $9$ and $16$ co-prime? \hfill \MCSelect{Yes}~~No
  \item Fill in the blanks with integers so that the following equation is true:
  \[ 9 \times \SAB{9} + 16 \times (-\SAB{5}) = 1 \]
  \item Fill in the blanks with integers so that the following equation is true:
  \[ 22331115 \times 9 \times \SAB{9} + 22331115 \times 16 \times (-\SAB{5}) =
  22331115 \]
  \item Let's say that $9 \mid 16\times22331115$. Using the above equation,
  does $9\mid22331115$? \Switch{\AnsT{Yes}}{}
 \end{enumerate}
\end{problem}


\end{document}
