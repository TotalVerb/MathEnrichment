\documentclass[12pt,letterpaper]{article}

\newcommand{\IncludePath}{../include}
\usepackage{extsizes}
\usepackage{titling}

\usepackage{amssymb,amsmath,amsthm}
\usepackage{enumerate}
\usepackage[margin=1in]{geometry}
\usepackage{graphicx,ctable,booktabs}
\usepackage{fancyhdr}
\usepackage[utf8]{inputenc}

\makeatletter
\newenvironment{problem}{\@startsection
       {section}
       {1}
       {-.2em}
       {-3.5ex plus -1ex minus -.2ex}
       {2.3ex plus .2ex}
       {\pagebreak[3]
       \large\bf\noindent{Problem }
       }
       }
\makeatother

\pagestyle{fancy}
\lhead{\thetitle}
\chead{}
\rhead{\thepage}
\lfoot{\small\scshape Grade 4 Olympic Math}
\cfoot{}
\rfoot{}
\renewcommand{\headrulewidth}{.3pt}
\renewcommand{\footrulewidth}{.3pt}
\setlength\voffset{-0.25in}
\setlength\textheight{648pt}
\setlength\headheight{15pt}


\title{Modular Arithmetic}
\author{Name: \underline{\hspace{5cm}}}
\date{April 16, 2016}

\begin{document}

\maketitle

\thispagestyle{empty}

\begin{problem}{Remainders I}
 Find the remainder.

 \begin{itemize}
  \begin{multicols}{2}
   \item $10 \bmod 3 = \blankC$
   \item $17 \bmod 12 = \blankC$
   \item $0 \bmod 10 = \blankC$
   \item $119 \bmod 10 = \blankC$
  \end{multicols}
 \end{itemize}
\end{problem}

\begin{problem}{Remainders II}
 Find the remainder.

 \begin{itemize}
  \begin{multicols}{2}
   \item $(33 + 99) \bmod 3 = \blankC$
   \item $(36 + 9) \bmod 12 = \blankC$
   \item $(19 + 71) \bmod 10 = \blankC$
   \item $(100 + 99) \bmod 2 = \blankC$
  \end{multicols}
 \end{itemize}
\end{problem}

\begin{problem}{Remainders III}
 Find the remainder.

 \begin{itemize}
  \begin{multicols}{2}
   \item $(33 \times 99) \bmod 3 = \blankC$
   \item $(36 \times 9) \bmod 12 = \blankC$
   \item $(19 \times 71) \bmod 10 = \blankC$
   \item $(100 \times 99) \bmod 2 = \blankC$
  \end{multicols}
 \end{itemize}
\end{problem}

\begin{problem}{Remainders IV}
 For each congruence, determine whether it is true or false.

 \begin{itemize}
   \item $7 \equiv 19 \pmod {12}$ \hfill True~~False
   \item $-2 \equiv 2 \pmod {2}$ \hfill True~~False
   \item $99 \equiv 145 \pmod {10}$ \hfill True~~False
   \item $7 \equiv -1 \pmod {3}$ \hfill True~~False
 \end{itemize}
\end{problem}

\begin{problem}{Remainders V}

 \begin{itemize}
  \item It is 7:00 AM. Dora has an appointment in $243$ hours. At what time of
  day is Dora's appointment?
  \item It is Monday. Ethan leaves for vacation in $18$ days. On what day of the
  week does Ethan leave?
  \item It is 7:00 AM on Monday. Flora is planning to move into her new home in
  $700$ hours. At what time on what day of the week does Flora move into her new
  home?
 \end{itemize}
\end{problem}

\begin{problem}{Challenge I}
 Recall the notation used for repeated multiplication. When we write $7^4$, we
 mean $7 \times 7 \times 7 \times 7$, and there are $4$ sevens. What would it
 mean to multiply no sevens? That is, $7^0$? That would be the same as taking
 $7$ and dividing out the $7$. So we define $7^0=1$.

 Determine the units (ones') digit for each of the following integers. Hint:
 use modular arithmetic to simplify the multiplications.

 \begin{itemize}
  \item $7^0 = 1$ \hfill \blankC
  \item $7^1 = 7$ \hfill \blankC
  \item $7^2 = 7 \times 7 = 49$ \hfill \blankC
  \item $7^3 = 7 \times 7 \times 7$ \hfill \blankC
  \item $7^4$ \hfill \blankC
  \item $7^5$ \hfill \blankC
  \item $7^6$ \hfill \blankC
  \item $7^7$ \hfill \blankC
 \end{itemize}
\end{problem}

\begin{problem}{Challenge II}

 \begin{itemize}
  \item How many numbers between $1$ and $1000$, including both $1$ and $1000$,
  yield a remainder of $2$ when divided by $9$?
  \item How many numbers between $1$ and $1000$, including both $1$ and $1000$,
  yield a remainder of $2$ when divided by $9$ and a remainder of $0$ when
  divided by $2$?
  \item How many numbers between $1$ and $100000$ (including both endpoints) are
  there that yield a remainder of $1$ when divided by $2$, $3$, $4$, $5$, $6$,
  $7$, and $8$?
 \end{itemize}

 These kinds of problems involve solving so-called \emph{linear congruences}.
 You will see them often on math competitions.

\end{problem}

\end{document}
