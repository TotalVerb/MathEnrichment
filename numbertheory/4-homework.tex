\documentclass[12pt,letterpaper]{article}

\newcommand{\IncludePath}{../include}
\usepackage{extsizes}
\usepackage{titling}

\usepackage{amssymb,amsmath,amsthm}
\usepackage{enumerate}
\usepackage[margin=1in]{geometry}
\usepackage{graphicx,ctable,booktabs}
\usepackage{fancyhdr}
\usepackage[utf8]{inputenc}

\makeatletter
\newenvironment{problem}{\@startsection
       {section}
       {1}
       {-.2em}
       {-3.5ex plus -1ex minus -.2ex}
       {2.3ex plus .2ex}
       {\pagebreak[3]
       \large\bf\noindent{Problem }
       }
       }
\makeatother

\pagestyle{fancy}
\lhead{\thetitle}
\chead{}
\rhead{\thepage}
\lfoot{\small\scshape Grade 4 Olympic Math}
\cfoot{}
\rfoot{}
\renewcommand{\headrulewidth}{.3pt}
\renewcommand{\footrulewidth}{.3pt}
\setlength\voffset{-0.25in}
\setlength\textheight{648pt}
\setlength\headheight{15pt}


\title{Primes \& Remainders}
\author{Name: \underline{\hspace{5cm}}}
\date{April 2, 2016}

\begin{document}

\maketitle

\thispagestyle{empty}

The first page is about remainders. The back page is a review of prime numbers.

\begin{problem}{Fill in the Blanks I}
 When we divide $20$ by $4$, we count how many groups of $4$ are needed to make
 $20$. We know that $5 \times 4 = 20$. This means that it takes \blankC\ groups of
 $4$ to make $20$. So $20 \div 4 = \blankC$.

 When we divide $26$ by $13$, we count how many groups of \blankC\ are needed to make
 \blankC. Therefore, $26 \div 13 = \blankC$.
\end{problem}

\begin{problem}{Fill in the Blanks II}
 Sometimes the numbers might not work out perfectly. Let's try dividing $20$ by
 $3$. If we make $7$ groups of $3$, the total number would be $7 \times 3 = 21$.
 But this is too big. If we make $6$ groups of $3$, the total number would be
 $\blankC \times 3 = 18$. But this is too small. So we can make six groups, but
 we'd then have $20 - 18 = \blankC$ left over!

 We say that when we divide $20$ by $3$, our \emph{quotient} is $6$ because we
 can make \blankC\ groups of $3$ in total. And then we say that our
 \emph{remainder} is $2$ because after making $6$ groups of $3$, we have \blankC\
 left over. We can also write this as $20 \div 3 = 6 \operatorname{R} 2$. The
 ``R'' stands for ``remainder''.

 When we can't make any groups, the quotient is zero ($0$). So $4 \div 10 = \blankC
 \operatorname{R} \blankC$. When there isn't anything left over, the remainder is zero
 ($0$). So $22 \div 2 = \blankC \operatorname{R} \blankC$.

 Complete the following: \begin{itemize}
  \item $17 \div 5 = \blankC \operatorname{R} \blankC$.
  \item $16 \div 5 = \blankC \operatorname{R} \blankC$.
 \end{itemize}
\end{problem}

\begin{problem}{Fill in the Blanks III}
 Suppose we didn't care what the quotient was; only the remainder is important. Then
 we can write that as $16 \bmod 5 = 1$. We don't care how many groups we made, but we
 do care that one was left over after making those groups. Complete the following:
 \begin{itemize}
  \item $75 \bmod 7 = \blankC$.
  \item $24 \bmod 3 = \blankC$.
 \end{itemize}
\end{problem}

\pagebreak

Here is a list of all prime numbers less than $100$: $2$, $3$, $5$, $7$, $11$, $13$,
$17$, $19$, $23$, $29$, $31$, $37$, $41$, $43$, $47$, $53$, $59$, $61$, $67$,
$71$, $73$, $79$, $83$, $89$ and $97$.

\begin{problem}{Factorize}
 Find the unique prime factorization of each number.

 \begin{itemize}
  \item $20$ \hfill $\blankB \times \blankB \times \blankB$
  \item $4$ \hfill $\blankB \times \blankB$
  \item $81$ \hfill $\blankB \times \blankB \times \blankB \times \blankB$
  \item $84$ \hfill $\blankB \times \blankB \times \blankB \times \blankB$
 \end{itemize}
\end{problem}

\begin{problem}{Factor sum}
 The sum of all factors of $28$ is:

 $\blankB +
 \blankB +
 \blankB +
 \blankB +
 \blankB +
 \blankB = \underline{\hspace{3em}}$
\end{problem}

\begin{problem}{GCD of a prime}
 What is the greatest common divisor of $7$ and $10$? \hfill
 $\gcd(7, 10) = \blankB$
\end{problem}

\begin{problem}{Challenge}
 A \emph{palindrome} is a number that reads the same forwards and backwards in
 decimal notation. For example, $12321$ is a palindrome, as is $7227$ or
 $888888$.

 \begin{itemize}
  \item What five-digit palindrome starts with $728$? \underline{\hspace{5em}}
  \item $11$ and $55$ are two examples of two-digit palindromes.
  How many two-digit palindromes are there? ($00=0$ does not have two digits.)
  \item What two numbers are factors of all two-digit palindromes?
  \item $1221$ and $9999$ are two examples of four-digit palindromes. How many
  four-digit palindromes are there? (Do not include palindromes starting with
  $0$.)
  \item How many four-digit palindromes are even?
  \item How many four-digit palindromes are divisible by $11$?
 \end{itemize}
\end{problem}


\end{document}
