\documentclass[12pt,letterpaper]{article}

\newcommand{\IncludePath}{../include}
\usepackage{extsizes}
\usepackage{titling}

\usepackage{amssymb,amsmath,amsthm}
\usepackage{enumerate}
\usepackage[margin=1in]{geometry}
\usepackage{graphicx,ctable,booktabs}
\usepackage{fancyhdr}
\usepackage[utf8]{inputenc}

\makeatletter
\newenvironment{problem}{\@startsection
       {section}
       {1}
       {-.2em}
       {-3.5ex plus -1ex minus -.2ex}
       {2.3ex plus .2ex}
       {\pagebreak[3]
       \large\bf\noindent{Problem }
       }
       }
\makeatother

\pagestyle{fancy}
\lhead{\thetitle}
\chead{}
\rhead{\thepage}
\lfoot{\small\scshape Grade 4 Olympic Math}
\cfoot{}
\rfoot{}
\renewcommand{\headrulewidth}{.3pt}
\renewcommand{\footrulewidth}{.3pt}
\setlength\voffset{-0.25in}
\setlength\textheight{648pt}
\setlength\headheight{15pt}


\title{Quiz $4$: Number Theory}
\author{Answers [2]}
\date{April 23, 2016}

\begin{document}

\maketitle

Please write your \emph{first and last} name. Each part of your is worth one
bonus mark. Two additional bonus marks are available (Problem 6). Note that the
first question is worth $25$ marks (one per table cell), so please be careful
when doing that question!

\thispagestyle{empty}

\begin{problem}{Sieve of Eratosthenes [25]}
 Follow the given instructions to create a list of all prime numbers from $1$
 to $25$. The algorithm described here is known as the Sieve of Eratosthenes,
 and it is one of the fastest known ways to generate prime numbers.

 \begin{enumerate}
  \item Cross off $1$. It is not prime.
  \item Circle $2$, because it is a prime number. Cross off all bigger multiples
  of $2$ ($4$, $6$, $8$, \dots), because they are composite.
  \item Circle the smallest number that hasn't been crossed off. Since it isn't
  divisible by any smaller prime (otherwise, it would have been crossed off),
  it must be prime. Now cross off all multiples of the number you just circled.
  \item Repeat the last step until all numbers are crossed off or circled.
 \end{enumerate}

 \begin{center}
  \def\arraystretch{1.7}
  \begin{tabular}{|c|c|c|c|c|}
   \hline
   $1$ & \Ans{2} & \Ans{3} & $4$ & \Ans{5} \\
   \hline
   $6$ & \Ans{7} & $8$ & $9$ & $10$ \\
   \hline
   \Ans{11} & $12$ & \Ans{13} & $14$ & $15$ \\
   \hline
   $16$ & \Ans{17} & $18$ & \Ans{19} & $20$ \\
   \hline
   $21$ & $22$ & \Ans{23} & $24$ & $25$ \\
   \hline
  \end{tabular}
 \end{center}

 Check your work! There should be $9$ prime numbers circled.
\end{problem}

\begin{problem}{Prime Factor [3]}
 Find one prime factor for each number. Do not try to find additional factors.

 \begin{itemize}
  \item $208790602$ \hfill Factor: \Ans{2}
  \item $249898465$ \hfill Factor: \Ans{5}
  \item $299999967$ \hfill Factor: \Ans{3}
 \end{itemize}

\end{problem}

\begin{problem}{Factorize [12]}
 Find the unique prime factorization of each number.

 \begin{itemize}
  \item $14$ \hfill \Ans{2 \times 7}
  \item $27$ \hfill \Ans{3 \times 3 \times 3}
  \item $28$ \hfill \Ans{2 \times 2 \times 7}
  \item $90$ \hfill \Ans{2 \times 3 \times 3 \times 5}
 \end{itemize}
\end{problem}

\begin{problem}{GCD [6]}
 Find the GCD of each pair of numbers.

 \begin{itemize}
  \begin{multicols}{2}
   \item $\gcd(10, 20) = \Ans{10}$
   \item $\gcd(7, 11) = \Ans{1}$
   \item $\gcd(25, 25) = \Ans{25}$
   \item $\gcd(1, 19) = \Ans{1}$
   \item $\gcd(60, 80) = \Ans{20}$
   \item $\gcd(0, 128) = \Ans{128}$
  \end{multicols}
 \end{itemize}
\end{problem}

\begin{problem}{LCM [4]}
 Find the LCM of each pair of numbers.

 \begin{itemize}
  \begin{multicols}{2}
   \item $\operatorname{lcm}(10, 20) = \Ans{20}$
   \item $\operatorname{lcm}(7, 11) = \Ans{77}$
   \item $\operatorname{lcm}(1, 19) = \Ans{19}$
   \item $\operatorname{lcm}(60, 80) = \Ans{240}$
  \end{multicols}
 \end{itemize}
\end{problem}

\begin{problem}{Bonus [2]}
 \begin{itemize}
  \item Use the Euclidean algorithm to find \[
   \gcd(1234, 5678) = \Ans{2}
  \]

  \item Note that $1234 \times 5678 = 7006652$. Now use the fact that
  $\gcd(a, b) \times \operatorname{lcm}(a, b) = a \times b$ to find \[
   \operatorname{lcm}(1234, 5678) = \Ans{3503326}
  \]
 \end{itemize}
\end{problem}


\end{document}
