\documentclass[12pt,letterpaper]{article}

\newcommand{\IncludePath}{../include}
\usepackage{extsizes}
\usepackage{titling}

\usepackage{amssymb,amsmath,amsthm}
\usepackage{enumerate}
\usepackage[margin=1in]{geometry}
\usepackage{graphicx,ctable,booktabs}
\usepackage{fancyhdr}
\usepackage[utf8]{inputenc}

\makeatletter
\newenvironment{problem}{\@startsection
       {section}
       {1}
       {-.2em}
       {-3.5ex plus -1ex minus -.2ex}
       {2.3ex plus .2ex}
       {\pagebreak[3]
       \large\bf\noindent{Problem }
       }
       }
\makeatother

\pagestyle{fancy}
\lhead{\thetitle}
\chead{}
\rhead{\thepage}
\lfoot{\small\scshape Grade 4 Olympic Math}
\cfoot{}
\rfoot{}
\renewcommand{\headrulewidth}{.3pt}
\renewcommand{\footrulewidth}{.3pt}
\setlength\voffset{-0.25in}
\setlength\textheight{648pt}
\setlength\headheight{15pt}


% \answerstrue % turn on for answers
\defauthor

\title{Modular Arithmetic}
\date{April 22, 2017}

\begin{document}

\maketitle

\thispagestyle{empty}

\begin{problem}{Remainders}
 Find the remainder.

 \begin{itemize}
  \begin{multicols}{2}
   \item \(10 \bmod 3 = \SAB{1}\)
   \item \(17 \bmod 12 = \SAB{5}\)
   \item \(0 \bmod 10 = \SAB{0}\)
   \item \(119 \bmod 10 = \SAB{9}\)
   \item \((33 + 99) \bmod 3 = \SAB{0}\)
   \item \((36 + 9) \bmod 12 = \SAB{9}\)
   \item \((19 + 71) \bmod 10 = \SAB{0}\)
   \item \((100 + 99) \bmod 2 = \SAB{1}\)
   \item \((33 \times 99) \bmod 3 = \SAB{0}\)
   \item \((36 \times 9) \bmod 12 = \SAB{0}\)
   \item \((19 \times 71) \bmod 10 = \SAB{9}\)
   \item \((100 \times 99) \bmod 2 = \SAB{0}\)
  \end{multicols}
 \end{itemize}
\end{problem}

\begin{problem}{Congruences}
 For each congruence, determine whether it is true or false.

 \begin{itemize}
   \item \(7 \equiv 19 \pmod {12}\) \hfill \TFTrue
   \item \(-2 \equiv 2 \pmod {2}\) \hfill \TFTrue
   \item \(99 \equiv 145 \pmod {10}\) \hfill \TFFalse
   \item \(7 \equiv -1 \pmod {3}\) \hfill \TFFalse
 \end{itemize}
\end{problem}

\begin{problem}{Applied Problems}

 \begin{itemize}
  \item It is 7:00 AM. Dora has an appointment in \(243\) hours. At what time of
  day is Dora's appointment? \Switch{\AnsT{10:00 AM}}{}
  \item It is Monday. Ethan leaves for vacation in \(18\) days. On what day of the
  week does Ethan leave? \Switch{\AnsT{Friday}}{}
  \item It is 7:00 AM on Monday. Flora is planning to move into her new home in
  \(700\) hours. At what time on what day of the week does Flora move into her new
  home? \Switch{\AnsT{11:00 AM on Tuesday}}{}
 \end{itemize}
\end{problem}

\begin{problem}{Challenge I}
 Recall the notation used for repeated multiplication. When we write \(7^4\), we
 mean \(7 \times 7 \times 7 \times 7\), and there are \(4\) sevens. What would it
 mean to multiply no sevens? That is, \(7^0\)? That would be the same as taking
 \(7\) and dividing out the \(7\). So we define \(7^0=1\).

 Determine the units (ones') digit for each of the following integers. Hint:
 use modular arithmetic to simplify the multiplications.

 \begin{itemize}
  \item \(7^0 = 1\) \hfill \SAB{1}
  \item \(7^1 = 7\) \hfill \SAB{7}
  \item \(7^2 = 7 \times 7 = 49\) \hfill \SAB{9}
  \item \(7^3 = 7 \times 7 \times 7\) \hfill \SAB{3}
  \item \(7^4\) \hfill \SAB{1}
  \item \(7^5\) \hfill \SAB{7}
  \item \(7^6\) \hfill \SAB{9}
  \item \(7^7\) \hfill \SAB{3}
 \end{itemize}
\end{problem}

\begin{problem}{Challenge II}

 \begin{itemize}
  \item How many numbers between \(1\) and \(1000\), including both \(1\) and \(1000\),
  yield a remainder of \(2\) when divided by \(9\)? \Switch{\Ans{110}}{}
  \item How many numbers between \(1\) and \(1000\), including both \(1\) and \(1000\),
  yield a remainder of \(2\) when divided by \(9\) and a remainder of \(0\) when
  divided by \(2\)? \Switch{\Ans{55}}{}
  \item How many numbers between \(1\) and \(100000\) (including both endpoints) are
  there that yield a remainder of \(1\) when divided by \(2\), \(3\), \(4\), \(5\), \(6\),
  \(7\), and \(8\)? \Switch{\Ans{119}}{}
 \end{itemize}

 These kinds of problems involve solving so-called \emph{linear congruences}.
 You will see them often on math competitions.

\end{problem}

\end{document}
