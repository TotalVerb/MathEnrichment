\newglossaryentry{algorithm}
{
  name=algorithm,
  description={step-by-step procedure for performing a calculation
  according to well-defined rules (Wikipedia)}
}

\newglossaryentry{acute angle}
{
  name=acute angle,
  description={an angle measuring less than a right angle (\SI{90}{\degree} or a
  quarter of a turn); such angles are typically characterized as being sharp}
}

\newglossaryentry{angle}
{
  name=angle,
  description={the figure formed by two rays, called the sides of the angle,
  sharing a common endpoint, called the vertex of the angle (Wikipedia)}
}

\newglossaryentry{coincident}
{
  name=coincident,
  description={a description of two objects which occupy exactly the same space}
}

\newglossaryentry{domain}
{
  name=domain,
  description={the set of values that are possible inputs for a function}
}

\newglossaryentry{endpoint}
{
  name=endpoint,
  description={an extreme point of a line segment or ray; line segments have two
  endpoints whereas rays have just one}
}

\newglossaryentry{function}
{
  name=function,
  description={object that may take any allowed input and will produce a
  single associated output for that input; alternatively, relation for
  which each input value has exactly one related output value}
}

\newglossaryentry{integer}
{
  name=integer,
  description={positive or negative whole number, or $0$; for example,
  $-8$, $2000$}
}

\newglossaryentry{line}
{
  name=line,
  description={a straight one-dimensional object that extends forever in both
  directions}
}

\newglossaryentry{line segment}
{
  name=line segment,
  description={a straight one-dimensional object terminated at both ends}
}

\newglossaryentry{multiplicand}
{
  name=multiplicand,
  description={the number that is being multiplied; for instance, in
  $2\times3=6$, the multiplicand is $2$}
}

\newglossaryentry{multiplier}
{
  name=multiplier,
  description={the factor to multiply a number by; for instance, in
  $2\times3=6$, the multiplier is $3$}
}

\newglossaryentry{obtuse angle}
{
  name=obtuse angle,
  description={an angle measuring more than a right angle (\SI{90}{\degree} or a
  quarter of a turn) but less than a straight angle (\SI{180}{\degree} or a half
  of a turn)}
}

\newglossaryentry{parallel}
{
  name=parallel,
  description={a description of two lines that never intersect at any point}
}

\newglossaryentry{parity}
{
  name=parity,
  description={decribes whether an integer is even or odd}
}

\newglossaryentry{plane}
{
  name=plane,
  description={a two-dimensional flat surface}
}

\newglossaryentry{plane geometry}
{
  name=plane geometry,
  description={the study of figures on a plane (a two-dimensional flat surface)}
}

\newglossaryentry{product}
{
  name=product,
  description={the result of a multiplication; for instance, in
  $2\times3=6$, the product is $6$}
}

\newglossaryentry{range}
{
  name=range,
  description={the set of values that are possible outputs for a function}
}

\newglossaryentry{ray}
{
  name=ray,
  description={a straight one-dimensional object terminated at one end and
  extending forever in the other direction}
}

\newglossaryentry{right angle}
{
  name=right angle,
  description={an angle measuring \SI{90}{\degree} or a quarter of a turn; angle
  between two rays intersecting in an L shape}
}

\newglossaryentry{right triangle}
{
  name=right triangle,
  description={an triangle with one \SI{90}{\degree} (right) angle}
}

\newglossaryentry{summand}
{
  name=summand,
  description={something which is being added; for instance, in $1+2=3$,
  the two summands are $1$ and $2$}
}

\newglossaryentry{reflex angle}
{
  name=reflex angle,
  description={an angle measuring more than \SI{180}{\degree} or a half of a
  turn, but less than \SI{360}{\degree} or a full turn}
}

\newglossaryentry{straight angle}
{
  name=straight angle,
  description={an angle measuring \SI{180}{\degree} or a half of a turn; angle
  between two rays in opposite directions}
}

\newglossaryentry{transitivity}
{
  name=transitivity,
  description={the property of certain relations that specifies if an
  element $a$ is related to $b$, and the element $b$ is related to $c$,
  then $a$ is similarly related to $c$}
}
