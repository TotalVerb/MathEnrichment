\documentclass[12pt,letterpaper]{article}

\newcommand{\IncludePath}{../include}
\usepackage{extsizes}
\usepackage{titling}

\usepackage{amssymb,amsmath,amsthm}
\usepackage{enumerate}
\usepackage[margin=1in]{geometry}
\usepackage{graphicx,ctable,booktabs}
\usepackage{fancyhdr}
\usepackage[utf8]{inputenc}

\makeatletter
\newenvironment{problem}{\@startsection
       {section}
       {1}
       {-.2em}
       {-3.5ex plus -1ex minus -.2ex}
       {2.3ex plus .2ex}
       {\pagebreak[3]
       \large\bf\noindent{Problem }
       }
       }
\makeatother

\pagestyle{fancy}
\lhead{\thetitle}
\chead{}
\rhead{\thepage}
\lfoot{\small\scshape Grade 4 Olympic Math}
\cfoot{}
\rfoot{}
\renewcommand{\headrulewidth}{.3pt}
\renewcommand{\footrulewidth}{.3pt}
\setlength\voffset{-0.25in}
\setlength\textheight{648pt}
\setlength\headheight{15pt}


% \answerstrue % turn on for answers
\defauthor

\title{Ordered Pairs, Fractions, Rational Numbers, and Two-Dimensional Vector
Spaces}
\date{May 6, 2017}

\begin{document}

\maketitle

\thispagestyle{empty}

\newcommand{\fracans}[2]{
\frac{\Switch{#1}{\hspace{2em}}}{\Switch{#2}{\hspace{2em}}}}

\begin{problem}{Equivalent Fractions}

 Find the equivalent fraction with the smallest denominator. (In other words,
 simplify the fraction.)

 \begin{enumerate}[\hspace{.5cm}a.]
  \item \[ \frac{2}{4} = \fracans{1}{2} \]
  \item \[ \frac{3}{12} = \fracans{1}{4} \]
  \item \[ \frac{10}{15} = \fracans{2}{3} \]
 \end{enumerate}
\end{problem}

\begin{problem}{Multiplying Fractions}
Find the product:

\begin{enumerate}[\hspace{.5cm}a.]
\item \[ \frac{1}{2} \times \frac{2}{3} = \fracans{1}{3} \]
\item \[ \frac{3}{8} \times \frac{5}{9} = \fracans{5}{24} \]
\item \[ \frac{1}{2} \times \frac{1}{2} \times \frac{1}{2} \times \frac{1}{2}
= \fracans{1}{16} \]
\end{enumerate}
\end{problem}

\begin{problem}{True or False}
  \begin{enumerate}
    \item The sum of two rational numbers is always a rational number.
    (Remember, integers are rational numbers too!) \hfill \TFTrue
    \item The product of two rational numbers is always a rational number.
    \hfill \TFTrue
    \item There are multiple ways to write every rational number as a fraction.
    \hfill \TFTrue
    \item For all integers \(a, b, c, d\), if \(b \ne 0 \ne d\), \[
      \frac{a}{b} + \frac{c}{d} = \frac{ac}{bd}
    \] \hfill \TFFalse
  \end{enumerate}
\end{problem}

\begin{problem}{Challenge I}
 Find the product:
 \[
  \frac{1}{2} \times \frac{2}{3} \times \frac{3}{4} \times
  \dots \times \frac{99}{100} = \fracans{1}{100}
 \]
\end{problem}

\begin{problem}{Challenge II}
 Find the sum:
 \[
  \frac{1}{1 \times 2} + \frac{1}{2 \times 3} + \frac{1}{3 \times 4}
  + \dots + \frac{1}{99 \times 100} = \fracans{99}{100}
 \]
\end{problem}

\begin{problem}{Challenge III}
  A long time ago, we saw that if \(\mathbf{v} = [x, y]\) is a vector, then we
  define \[
    2\mathbf{v} = \mathbf{v} + \mathbf{v} = [x + x, y + y] = [2x, 2y]
  \] and again, for sake of example, if \(\mathbf{w} = [3, 4]\), then
  \(2\mathbf{w} = [6, 8]\).

  We can also define multiplying vectors by fractions in a similar way: if
  \(\mathbf{v} = [x, y]\) is a vector then \[
    \frac{a}{b} \mathbf{v} = \left[\frac{a}{b}x, \frac{a}{b}y\right]
  \]

  Compute \[
    \frac{7}{12} [60, 72] = \SAF{35, 45}
  \]
\end{problem}

\end{document}
