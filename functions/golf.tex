\documentclass[12pt,letterpaper]{article}

\usepackage{extsizes}
\usepackage{titling}

\usepackage{amssymb,amsmath,amsthm}
\usepackage{enumerate}
\usepackage[margin=1in]{geometry}
\usepackage{graphicx,ctable,booktabs}
\usepackage{fancyhdr}
\usepackage[utf8]{inputenc}

\makeatletter
\newenvironment{problem}{\@startsection
       {section}
       {1}
       {-.2em}
       {-3.5ex plus -1ex minus -.2ex}
       {2.3ex plus .2ex}
       {\pagebreak[3]
       \large\bf\noindent{Problem }
       }
       }
\makeatother

\pagestyle{fancy}
\lhead{\thetitle}
\chead{}
\rhead{\thepage}
\lfoot{\small\scshape Grade 4 Olympic Math}
\cfoot{}
\rfoot{}
\renewcommand{\headrulewidth}{.3pt}
\renewcommand{\footrulewidth}{.3pt}
\setlength\voffset{-0.25in}
\setlength\textheight{648pt}
\setlength\headheight{15pt}


\title{Function Golf}
\author{Name: \underline{\hspace{5cm}}}

\begin{document}

\maketitle

\thispagestyle{empty}

In function golf, the objective is simple. Given a list of functions, and two
numbers $n$ and $m$, apply functions to $n$ until you reach $m$. For example,
if:
\begin{align*}
 A(x) &= x^2 \\
 B(x) &= x-1 \\
 n &= 10 \\
 m &= 36 && \text{golf for $m$ starting with $n$}
\end{align*}
then one solution is:
\begin{align*}
 m = n \rhd B \rhd B \rhd B \rhd B \rhd A
\end{align*}

\begin{problem}{Function Golf I}
 Solve the following puzzle.
 \begin{align*}
  A(x) &= x + 3 \\
  n &= 10 \\
  m &= 21 && \text{golf for $m$ starting with $n$}
 \end{align*}
\end{problem}

\begin{problem}{Function Golf II}
 Solve the following puzzle.
 \begin{align*}
  A(x) &= x \times 2 \\
  n &= 3 \\
  m &= 24 && \text{golf for $m$ starting with $n$}
 \end{align*}
\end{problem}


\end{document}
