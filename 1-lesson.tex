\documentclass[letterpaper,10pt]{article}
\usepackage[utf8]{inputenc}

\title{The World of Math}
\author{Fengyang Wang}

\begin{document}
\maketitle

At your day school, you focus on only a small subset of mathematics.
In Grade 1, you learned about whole numbers, the place value system,
and basic addition and subtraction. You recognized two-dimensional
figures such as triangles, rectangles, or circles. You learned how to
make and interpret pictographs.

In Grade 2, you learned how to do multi-digit addition and subtraction,
as well as multiplication and division. You learned about area and
perimeter and how to measure them. Finally, you also learned about
symmetry and patterning.

Next, in Grade 3, you learned about angles, three-dimensional solids,
and more two-dimensional figures. You learned how to tell time and what
units we use to measure things like time, distance, etc. Finally, you
learned a bit about fractions.

That sounds like a lot of math, doesn't it? But in fact, that's just
the tip of the iceberg. For those of you who took the Grade 3
Olympic Math course, you will know that there's a lot more to it than
numbers, figures, and measurement. Those are very important parts of
math, and that's why you do them at school.

What we will do in this interest course is extend your mathematical knowledge
to more fields of math. Before we can do that, we will investigate
just what else there is in math:

\section{Fields of mathematics}

Mathematics is a broad field. Grown-up mathematicians investigate all kinds
of math, and a lot of that math is really complicated. Unfortunately, we
can't do any of that stuff yet. But there are a lot of fields of mathematics
that we can do, including:

\begin{itemize}
 \item Algebra
 \item Logic
 \item Number theory
 \item Probability
 \item Combinatorics
\end{itemize}

Don't worry if you don't recognize these words. We'll end up covering these
topics throughout the year.

We'll also do some work in topics that you do cover in your day school. But
we'll focus on things that you won't learn in school.

\begin{itemize}
 \item Arithmetic
 \item Geometry
 \item Measurement
\end{itemize}

\section{Problems}
A major focus of this interest course is on problem solving. Sometimes we
will solve problems individually, and sometimes we'll work as a class or
in small groups. Just to get you started, we'll do a few example problems.

These problems will range from easy ones to harder ones that you'll have
to think about.

\subsection{The Hundredth Rock}
We'll work individually on this problem. I'll give you 5 minutes to come
up with a solution. We will take up the problem after.

\begin{quotation}
 Henry has numbered 100 rocks with numbers from 1 to 100. Starting from
 the first rock, the colours are: red, green, blue, red, green, blue, 
 and so on and so forth. If this pattern continues, what's the colour of
 the hundredth rock?
\end{quotation}

\subsection{A Big Sum}
Split up into groups of two, three, or four to do this problem.

\begin{quotation}
 What is $1+2+3+4+\ldots+100$?
\end{quotation}

\subsection{Treasure}
We'll work on this problem as a class. Raise your hand if you have an
idea, and we'll finish the problem step by step.

\begin{quotation}
 Henrietta has found a treasure chest containing a large sum of money. She
 spends half of it on a new bike, and then spends \$50 more on a new
 backpack. Finally she spends half of the remaining money to buy a pizza.
 Now she only has \$15 left. How much money did Henrietta find in the
 treasure chest?
\end{quotation}


\end{document}