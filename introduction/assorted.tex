\documentclass[12pt,letterpaper]{article}

\newcommand{\IncludePath}{../include}
\usepackage{extsizes}
\usepackage{titling}

\usepackage{tikz}
\usetikzlibrary{shapes}

\usepackage{amssymb,amsmath,amsthm}
\usepackage{enumerate}
\usepackage[margin=0.8in]{geometry}
\usepackage{graphicx,ctable,booktabs}
\usepackage{fancyhdr}
\usepackage[utf8]{inputenc}
\usepackage{gensymb}

\makeatletter
\newenvironment{problem}{\@startsection
       {section}
       {1}
       {-.2em}
       {-3.5ex plus -1ex minus -.2ex}
       {2.3ex plus .2ex}
       {\pagebreak[3]
       \large\bf\noindent{Problem }
       }
       }
\makeatother

\pagestyle{fancy}
\lhead{\thetitle}
\chead{}
\rhead{\thepage}
\lfoot{\small\scshape Olympic Math}
\cfoot{}
\rfoot{}
\renewcommand{\headrulewidth}{.3pt}
\renewcommand{\footrulewidth}{.3pt}
\setlength\voffset{-0.25in}
\setlength\textheight{648pt}
\setlength\headheight{15pt}

\newcommand{\blankA}{\underline{\hspace{1em}}}
\newcommand{\blankB}{\underline{\hspace{2em}}}
\newcommand{\blankC}{\underline{\hspace{3em}}}
\newcommand{\blankD}{\underline{\hspace{4em}}}
\newcommand{\blankE}{\underline{\hspace{5em}}}
\newcommand{\blankF}{\underline{\hspace{6em}}}



\title{Assorted Problems I}
\author{Name: \underline{\hspace{5cm}}}
\date{September 26, 2015}

\usepackage[utf8]{inputenc}

\begin{document}
\maketitle

\thispagestyle{empty}

\begin{problem}{Arranging Letters}
 ABC, ACB, and BCA are three ways to arrange the three letters ABC. How many
 unique ways are there in total? (Include the three listed in the total;
 repeating letters is not allowed.)
\end{problem}

\begin{problem}{Equal Areas}
 A rectangle and a square have the same area. The rectangle's longer side (the
 length) is twice the square's side length. If the square's side length is
 $20~\mathrm{cm}$, calculate the perimeter of the rectangle.
\end{problem}

\begin{problem}{Handshakes}
 Four people attend a meeting. Each shakes the hand of every person (except
 themselves) exactly once. How many handshakes occur?
\end{problem}

\begin{problem}{Guess the Number}
 A number yields a remainder of 1 when divided by 2, a remainder of 2 when
 divided by 3, a remainder of 3 when divided by 4, and a remainder of 4 when
 divided by 5.

 \begin{enumerate}[\hspace{1cm}a.]
  \item What is the smallest positive number to have this property?
  \item Give two other positive numbers to have the above property.
 \end{enumerate}
\end{problem}

\begin{problem}{Garden}
 The area of a rectangular garden is $36~\mathrm{m^2}$. What are its dimensions
 if\ldots

 \begin{enumerate}[\hspace{1cm}a.]
   \item \ldots{}the width is $3~\mathrm{m}$?
   \item \ldots{}the length and width are equal?
   \item \ldots{}the perimeter is $26~\mathrm{m}$?
 \end{enumerate}
\end{problem}

\begin{problem}{Challenge}
 Write the next two terms in each pattern.

 \begin{enumerate}[\hspace{1cm}a.]
  \item 1, 1, 1, 1\ldots
  \item A, B, C, D, E\ldots
  \item 2, 4, 6, 8\ldots
  \item 3, 9, 27, 81\ldots
  \item $\uparrow$, $\leftarrow$, $\downarrow$, $\rightarrow$, $\uparrow$,
  $\leftarrow$ \ldots
  \item O, T, T, F, F, S, S, E\ldots
  \item 1, 2, 6, 24, 120, 720\ldots
  \item 0, 1, 1, 2, 3, 5, 8, 13, 21\ldots
 \end{enumerate}

\end{problem}

\end{document}
